\documentclass[11pt]{amsart}

\usepackage{geometry}
\geometry{a4paper,top=3.2cm,bottom=3.2cm,left=2.5cm,right=2.5cm}
%\usepackage{soul}

%\setcounter{tocdepth}{3}

%\usepackage[pagewise]{lineno}\linenumbers
%\usepackage[inline]{showlabels}

\hyphenpenalty=5000
\tolerance=1000

\usepackage{todonotes}
 \newcommand\liu[1]{\todo[color=green!40]{#1}} %Liu
 \newcommand\liuinline[1]{\todo[inline,color=green!40]{#1}} %Liu inline
 \newcommand\wang[1]{\todo[color=yellow!40]{#1}} %wang
 \newcommand\wanginline[1]{\todo[inline,color=yellow!40]{#1}} %wang inline
\newcommand\chen[1]{\todo[color=pink!40]{#1}} %chen
 \newcommand\cheninline[1]{\todo[inline,color=pink!40]{#1}} %wang inline


\usepackage{amsfonts, adjustbox, amssymb, amscd}
\numberwithin{equation}{section}

%\usepackage[symbol]{footmisc}
%\renewcommand{\thefootnote}{\fnsymbol{footnote}}
\renewcommand{\thepart}{\Roman{part}}

\usepackage{bm}
\usepackage{verbatim}
%\usepackage{amssymb}
\usepackage{mathrsfs}
\usepackage{graphicx}
\usepackage{tikz-cd}
\usepackage{subcaption}
\usepackage{listings}
\usepackage{subfiles}
\usepackage[toc,page]{appendix}
\usepackage{mathtools}
\usepackage{comment}
\usepackage{enumerate}
\usepackage{enumitem}
\usepackage[all]{xy}

\usepackage{graphicx}
\usepackage{appendix}
\usepackage{hyperref}
\hypersetup{
    colorlinks=true,
    citecolor=red,
    linkcolor=blue,
    filecolor=magenta,      
    urlcolor=red,
}
\lstset{
  basicstyle=\ttfamily,
  columns=fullflexible,
  frame=single,
  breaklines=true,
  postbreak=\mbox{\textcolor{red}{$\hookrightarrow$}\space},
}

\newcommand{\bir}{\dashrightarrow}

\newcommand{\bQ}{\mathbb{Q}}
\newcommand{\bP}{\mathbb{P}}
\newcommand{\bA}{\mathbb{A}}
\newcommand{\cA}{\mathcal{A}}
\newcommand{\cO}{\mathcal{O}}
\newcommand{\oE}{\overline{E}}
\newcommand{\cF}{\mathcal{F}}
\newcommand{\LD}{\mathcal{LD}}
\newcommand{\bZ}{\mathbb{Z}}
\newcommand{\bb}{\bm{b}}
\newcommand{\Mm}{{\bf{M}}}
\newcommand{\Nn}{{\bf{N}}}
\newcommand{\PP}{{\bf{P}}}
\newcommand{\Pp}{{\bf{P}}}
\newcommand{\NN}{{\bf{N}}}
\newcommand{\Dd}{{\bf{D}}}
\newcommand{\oY}{\overline{Y}}
\newcommand{\oL}{\overline{L}}
\newcommand{\cI}{\mathcal{I}}
\newcommand{\ind}{\mathrm{ind}}
\newcommand{\Spec}{\mathrm{Spec}}
\newcommand{\id}{\mathrm{id}}
\newcommand{\exc}{\mathrm{exc}}



\newcommand{\Cc}{\mathbb{C}}
\newcommand{\KK}{\mathbb{K}}
\newcommand{\Qq}{\mathbb{Q}}
\newcommand{\QQ}{\mathbb{Q}}
\newcommand{\Rr}{\mathbb{R}}
\newcommand{\RR}{\mathbb{R}}
\newcommand{\Zz}{\mathbb{Z}}
\newcommand{\ZZ}{\mathbb{Z}}





\newcommand{\zz}{\mathbf{z}}
\newcommand{\xx}{\mathbf{x}}
\newcommand{\yy}{\mathbf{y}}
\newcommand{\ww}{\mathbf{w}}
\newcommand{\vv}{\bm{v}}
\newcommand{\uu}{\mathbf{u}}
\newcommand{\kk}{\mathbf{k}}
\newcommand{\Span}{\operatorname{Span}}
\newcommand{\alct}{a\operatorname{LCT}}
\newcommand{\vol}{\operatorname{vol}}
\newcommand{\Center}{\operatorname{center}}
\newcommand{\Cone}{\operatorname{Cone}}
\newcommand{\Exc}{\operatorname{Exc}}
\newcommand{\Ext}{\operatorname{Ext}}
\newcommand{\Fr}{\operatorname{Fr}}
\newcommand{\Conv}{\operatorname{Conv}}
\newcommand{\Bir}{\operatorname{Bir}}
\newcommand{\Gal}{\operatorname{Gal}}
\newcommand{\Aut}{\operatorname{Aut}}
\newcommand{\glct}{\operatorname{glct}}
\newcommand{\ct}{\operatorname{ct}}
\newcommand{\GLCT}{\operatorname{GLCT}}
\newcommand{\HH}{\operatorname{H}}
\newcommand{\Hom}{\operatorname{Hom}}
\newcommand{\rk}{\operatorname{rank}}
\newcommand{\red}{\operatorname{red}}
\newcommand{\Ker}{\operatorname{Ker}}
\newcommand{\Ima}{\operatorname{Im}}
\newcommand{\Nklt}{\operatorname{Nklt}}
\newcommand{\mld}{{\operatorname{mld}}}
\newcommand{\Bs}{{\operatorname{Bs}}}
\newcommand{\Src}{{\operatorname{Src}}}
\newcommand{\Spr}{{\operatorname{Spr}}}
\newcommand{\num}{{\operatorname{num}}}
\newcommand{\tang}{{\operatorname{tang}}}
\newcommand{\pld}{{\operatorname{pld}}}
\newcommand{\tmld}{{\operatorname{tmld}}}
\newcommand{\relin}{\operatorname{relin}}

\newcommand{\loc}{\operatorname{loc }}
\newcommand{\expsing}{\mathrm{exp}}
\newcommand{\lcm}{\operatorname{lcm}}
\newcommand{\Weil}{\operatorname{Weil}}
\newcommand{\lct}{\operatorname{lct}}
\newcommand{\LCT}{\operatorname{LCT}}
\newcommand{\fol}{\operatorname{fol}}
\newcommand{\CR}{\operatorname{CR}}
\newcommand{\proj}{\operatorname{Proj}}
\newcommand{\spec}{\operatorname{Spec}}
\newcommand{\pet}{\operatorname{pet}}
\newcommand{\Supp}{\operatorname{Supp}}
\newcommand{\Ngklt}{\operatorname{Ngklt}}
\newcommand{\Nlc}{\operatorname{Nlc}}
\newcommand{\ld}{\operatorname{ld}}
\newcommand{\Diff}{\operatorname{Diff}}
\newcommand{\codim}{\operatorname{codim}}
\newcommand{\mult}{\operatorname{mult}}
\newcommand{\Rct}{\operatorname{Rct}}
\newcommand{\RCT}{\operatorname{RCT}}
\newcommand{\Div}{\operatorname{Div}}
\newcommand{\cont}{\operatorname{cont}}

\newcommand{\la}{\langle}
\newcommand{\ra}{\rangle}
\newcommand{\lf}{\lfloor}
\newcommand{\rf}{\rfloor}


\newcommand{\Aa}{{\bf{A}}}
\newcommand{\CC}{\mathcal{C}}
\newcommand{\Bb}{{\bf{B}}}
\newcommand{\Ff}{\mathcal{F}}
\newcommand{\Gg}{\mathcal{G}}
\newcommand{\LCP}{\mathcal{LCP}}
\newcommand{\Oo}{\mathcal{O}}
\newcommand{\Ii}{\Gamma}
\newcommand{\Jj}{\mathcal{J}}
\newcommand{\Ee}{\mathcal{E}}
\newcommand{\Hh}{\mathcal{H}}
\newcommand{\Ll}{\mathcal{L}}
\newcommand{\me}{\mathcal{E}}
\newcommand{\mo}{\mathcal{O}}
\newcommand{\nN}{\mathcal{N}}
\newcommand{\anN}{\mathcal{AN}}
\newcommand{\Tt}{\mathcal{T}}
\newcommand{\Ww}{\mathcal{W}}
\newcommand{\Xx}{\mathcal{X}}
\newcommand{\Ss}{\mathcal{S}}
\newcommand{\Yy}{\mathcal{Y}}


\newcommand{\BB}{\mathfrak{B}}
\newcommand{\mm}{\mathfrak{m}}

\newcommand{\NE}{\mathrm{NE}}
\newcommand{\Nef}{\mathrm{Nef}}
\newcommand{\Sing}{\mathrm{Sing}}
\newcommand{\Pic}{\mathrm{Pic}}
\newcommand{\reg}{\mathrm{reg}}
\newcommand{\creg}{\mathrm{creg}}
\newcommand\MLD{{\rm{MLD}}}
\newcommand\FT{{\rm{FT}}}
\newcommand{\crt}{{\rm{crt}}}
\newcommand{\CRT}{{\rm{CRT}}}
\newcommand{\Coeff}{{\rm{Coeff}}}
\newcommand\coeff{{\rm{coeff}}}
\newcommand{\rank}{\mathrm{rank}}

\makeatletter
\newenvironment{subtheorem}[1]{%
  \def\subtheoremcounter{#1}%
  \refstepcounter{#1}%
  \protected@edef\theparentnumber{\csname the#1\endcsname}%
  \setcounter{parentnumber}{\value{#1}}%
  \setcounter{#1}{0}%
  \expandafter\def\csname the#1\endcsname{\theparentnumber.\Alph{#1}}%
  \ignorespaces
}{%
  \setcounter{\subtheoremcounter}{\value{parentnumber}}%
  \ignorespacesafterend
}
\makeatother
\newcounter{parentnumber}

\newtheorem{thm}{Theorem}[subsection]
\newtheorem{conj}[thm]{Conjecture}
\newtheorem{cor}[thm]{Corollary}
\newtheorem{lem}[thm]{Lemma}
\newtheorem{prop}[thm]{Proposition}
\newtheorem{exprop}[thm]{Example-Proposition}
\newtheorem{claim}[thm]{Claim}

\newtheorem{alphthm}{Theorem}
\renewcommand{\thealphthm}{\Alph{alphthm}}
\newtheorem{alphcor}{Corollary}
\renewcommand{\thealphcor}{\Alph{alphthm}}

\newtheorem{innercustomthm}{Theorem}
\newenvironment{mythm}[1]
  {\renewcommand\theinnercustomthm{#1}\innercustomthm}
  {\endinnercustomthm}

  \newtheorem{innercustomcor}{Corollary}
\newenvironment{mycor}[1]
  {\renewcommand\theinnercustomcor{#1}\innercustomcor}
  {\endinnercustomcor}

  
%\newtheorem{theorempart}{Theorem B.}[alphthm]


\theoremstyle{definition}
\newtheorem{defn}[thm]{Definition}
\newtheorem{ques}[thm]{Question}
\theoremstyle{definition}
\newtheorem{rem}[thm]{Remark}
\newtheorem{remdef}[thm]{Remark-Definition}
\newtheorem{deflem}[thm]{Definition-Lemma}
\newtheorem{defthm}[thm]{Definition-Theorem}
\newtheorem{setup}[thm]{Set-up}
\newtheorem{ex}[thm]{Example}
\newtheorem{sce}[thm]{Scenario}
\newtheorem{nota}[thm]{Notation}
\newtheorem{exlem}[thm]{Example-Lemma}
\newtheorem{cons}[thm]{Construction}
\newtheorem{code}[thm]{Code}

\newtheorem{theorem}{Theorem}[section]
\newtheorem{lemma}[theorem]{Lemma}
\newtheorem{proposition}[theorem]{Proposition}
\newtheorem{corollary}[theorem]{Corollary}
\newtheorem*{notation}{Notation ($\star$)}


\theoremstyle{definition}
\newtheorem{definition}[theorem]{Definition}
\newtheorem{example}[theorem]{Example}
\newtheorem{question}[theorem]{Question}
\newtheorem{remark}[theorem]{Remark}
\newtheorem{conjecture}[theorem]{Conjecture}

\begin{document}

\title{Minimal model program for algebraically integrable foliations and generalized pairs}
\author{Guodu Chen, Jingjun Han, Jihao Liu, and Lingyao Xie}

\subjclass[2020]{14E30, 37F75}
\keywords{Algebraically integrable foliations. Generalized pairs. Minimal model program}
\date{\today}

\begin{abstract}
By systematically introducing and studying the structure of algebraically integrable generalized foliated quadruples, we establish the minimal model program for $\mathbb Q$-factorial foliated dlt algebraically integrable foliations and lc generalized pairs by proving their cone theorems, contraction theorems, and the existence of flips. We also provide numerous applications on their birational geometry and resolve a conjecture of Cascini and Spicer.
\end{abstract}
%The concept of generalized foliated quadruple was introduced by the third author, Luo, and Meng in the study of the global ACC for foliations in dimension $\leq 3$. It is a mixed structure of foliated triples and Birkar-Zhang's generalized pairs. In this paper, we systematically study the structure of algebraically integrable generalized foliated quadruples and use this new structure prove results for both foliations and generalized pairs.

%We prove the cone theorem of algebraically integrable generalized foliated quadruples and establish the canonical bundle formula of generalized foliated quadruples in full generality. As an important application to the minimal model program, we prove the cone theorem, contraction theorem, and the existence of flips for $\Qq$-factorial foliated dlt algebraically integrable foliations and lc generalized pairs. In particular, we can run the minimal model program for both cases.

%We also provide other applications to the theory of foliations and generalized pairs. We establish the canonical bundle formula, the subadjunction formula, the Kodaira and Kawamata-Viehweg vanishing theorems, and the base-point-freeness theorem for generalized pairs. We show that lc generalized pairs have Du Bois singularities. We prove the existence of Mori fiber spaces, the base-point-freeness theorem, the precise adjunction formula, and the ACC for lc thresholds for algebraically integrable foliations and algebraically integrable generalized foliated quadruples. Moreover, we show that foliated dlt algebraically integrable foliations are induced by contractions, answering a conjecture of Cascini and Spicer. 

%Finally, we prove the existence of good minimal models for $\Qq$-factorial foliated dlt triples polarized with an ample divisor. As a consequence, we prove a special case of Prokhorov-Shokurov's $\bb$-semi-ampleness conjecture when the boundary is polarized with a horizontal ample divisor.

\address{School of Mathematical Sciences, Shanghai Jiao Tong University, 800 Dongchuan RD Shanghai, Minhang District Shanghai 200240, China}
\email{chenguodu@sjtu.edu.cn}

\address{Shanghai Center for Mathematical Sciences, Fudan University, Shanghai, 200438, China}
\email{hanjingjun@fudan.edu.cn}


\address{Department of Mathematics, Northwestern University, 2033 Sheridan Road, Evanston, IL 60208, USA}
\email{jliu@northwestern.edu}


\address{Department of Mathematics, The University of Utah, Salt Lake City, UT 84112, USA}
\email{lingyao@math.utah.edu}

\maketitle

\pagestyle{myheadings}\markboth{\hfill G. Chen, J. Han, J. Liu, and L. Xie \hfill}{\hfill MMP for algebraically integrable foliations and generalized pairs\hfill}

\tableofcontents

\section{Introduction}\label{sec:Introduction}

We work over the field of complex numbers $\mathbb C$.

\subsection{Main theorems} Algebraically integrable foliations and generalized pairs are two structures that play important roles in modern birational geometry, specifically in the minimal model program. For more details on their backgrounds, we refer the reader to Subsections \ref{subsec: mmp aif} and \ref{subsec: mmp gpair}. The primary objective of this paper is to develop the minimal model program for both structures. The main theorems of the paper are as follows:

\begin{alphthm}\label{thm: main mmp foliation}
Let $(X,\Ff,B)$ be a $\Qq$-factorial projective F-dlt foliated triple such that $\Ff$ is algebraically integrable. Let $A$ be an ample $\Rr$-divisor on $X$. Then:
\begin{enumerate}
    \item The cone theorem, contraction theorem, and the existence of flips hold for $(X,\Ff,B)$. In particular, we can run a $(K_\Ff+B)$-MMP.
    \item  If $K_{\Ff}+B+A$ is nef, then $K_{\Ff}+B+A$ is semi-ample\footnote{\cite[Theorem 1.2]{CD23} provided a proof of some special cases of Theorem \ref{thm: main mmp foliation}(2) and other results that are similar to some results in this paper. However, the current proofs in \cite{CD23} seem to be incomplete, mainly because of the failure of \cite[Lemma 2.4]{CD23} and some gaps of the proof of \cite[Theorem 3.5]{CD23}. In this paper, we will avoid using any results in \cite{CD23}.}.
    \item If $B\geq A\geq 0$, then $(X,\Ff,B)$ has a good minimal model or a Mori fiber space.
    \item If $K_{\Ff}+B+A$ is $\Qq$-Cartier, then the canonical ring of $K_{\Ff}+B+A$,
    $$R(X,K_{\Ff}+B+A)=\bigoplus_{m=0}^{+\infty}H^0(X,\mathcal{O}_X(\lfloor m(K_{\Ff}+B+A)\rfloor)),$$
    is finitely generated.
\end{enumerate}
\end{alphthm}


\begin{alphthm}\label{thm: main mmp gpair}
Let $(X,B,\Mm)$ be a $\Qq$-factorial projective lc generalized pair. Then:
\begin{enumerate}
    \item The cone theorem, contraction theorem, and the existence of flips hold for $(X,B,\Mm)$. In particular, we can run a $(K_X+B+\Mm_X)$-MMP.
    \item If $K_X+B+A+\Mm_X$ is nef for some ample $\Rr$-divisor $A$, then $K_X+B+A+\Mm_X$ is semi-ample.
\end{enumerate}
\end{alphthm}

We refer the reader to Section \ref{sec: statement of main results} for stronger versions of Theorems \ref{thm: main mmp foliation} and \ref{thm: main mmp gpair} and other main results of this paper.

As explained in \cite{CS21, SS22}, F-dlt foliated triples play the same role as dlt pairs in the classical minimal model program, making it a natural class of singularities to study in the theory of foliations. Roughly speaking, Theorem \ref{thm: main mmp foliation} is an establishment of the minimal model program for algebraically integrable foliations with ``klt" singularities in any dimension. In fact, when $\Ff=T_X$ and $\lfloor B\rfloor=0$, Theorem \ref{thm: main mmp foliation} becomes the classical result of the existence of good minimal models of varieties of general type and the finite generation of the canonical ring \cite[Theorem 1.2]{BCHM10}. We remark that Theorem \ref{thm: main mmp foliation} does not hold in general without the polarization of the ample $\Rr$-divisor $A$ (cf. \cite[Example 5.4]{ACSS21}). 

In parallel, Theorem \ref{thm: main mmp gpair} is a full establishment of the minimal model program for generalized pairs, providing a complete answer to a fundamental question posed by Birkar and Zhang when they first introduced the concept of generalized pairs \cite[Before Lemma 4.4]{BZ16} (see \cite[6.1]{Bir21} and \cite[3.1, 3.3]{HL22} for other variations). Hacon suggested us that Theorem \ref{thm: main mmp gpair} might have essential implications on the cone theorem, contraction theorem, and the existence of flips for K\"ahler varieties in higher dimensions; see Scenario \ref{sce: kahler} for details.

We recall some previous results related to Theorems \ref{thm: main mmp foliation} and \ref{thm: main mmp gpair}:
\begin{itemize}
    \item \cite{ACSS21} proved the cone theorem part of Theorem \ref{thm: main mmp foliation}(1) without the $\Qq$-factorial F-dlt condition.
    \item When $\dim X=3$, \cite{CS20,CS21} (see also \cite{SS22}) proved Theorem \ref{thm: main mmp foliation}(1) when $\dim X\leq 3$ without the algebraically integrable condition. If we further assume that $\rk\Ff=2$, then Theorem \ref{thm: main mmp foliation}(3) was implicitly proven in \cite[Proof of Theorem 2.6]{SS22}.
    \item When $\dim X\leq 4$ or assuming the termination of klt flips in dimension $\leq\rk\Ff$, \cite{CS23a} proved Theorem \ref{thm: main mmp foliation}(1) without the F-dlt condition, but required $(X,B)$ to be klt.
    \item When $(X,B,\Mm)$ satisfies the ``NQC" condition (see Definition \ref{defn: b divisors} for details), \cite{HL21a} proved Theorem \ref{thm: main mmp gpair}, and \cite{HL21a,Xie22,CLX23,LX23b} together proved Theorem \ref{thm: main mmp gpair} for NQC generalized pairs without the $\Qq$-factorial condition.
\end{itemize}


\subsection{Ideas of the proofs of Theorems \ref{thm: main mmp foliation} and \ref{thm: main mmp gpair}} The proofs of Theorems \ref{thm: main mmp foliation} and \ref{thm: main mmp gpair} are crucially relied on a larger framework: the theory of \emph{generalized foliated quadruples}. 
\begin{defn}[{cf. \cite[Definition 1.2]{LLM23}}]\label{defn: gfq intro}
A \emph{generalized foliated quadruple} (\emph{gfq} for short) $(X,\Ff,B,\Mm)/U$ consists of a normal quasi-projective variety $X$, a foliation $\Ff$ on $X$, an $\mathbb R$-divisor $B\geq 0$ on $X$, a projective morphism $X\rightarrow U$, and a nef$/U$ $\Rr$-divisor $\Mm_{X'}$ on a high model $X'$ of $X$, such that $K_{\Ff}+B+\Mm_X$ is $\Rr$-Cartier. Here $\Mm_X$ is the image of $\Mm_{X'}$ on $X$.
\end{defn}
The notation $\Mm$ in Definition \ref{defn: gfq intro} is considered as a $\bb$-divisor on $X$. We refer the reader to Definition \ref{defn: b divisors} for the definition of $\bb$-divisors, and to Definition \ref{defn: gfq} for a more detailed definition of generalized foliated quadruples. It is clear that when $\Mm=\bm{0}$ is the trivial $\bb$-divisor, a generalized foliated quadruple is just a foliated triple $(X,\Ff,B)/U$; on the other hand, when $\Ff=T_X$, a generalized foliated quadruple is a generalized pair $(X,B,\Mm)/U$ (\cite[Definition 1.4]{BZ16}). Therefore, generalized foliated quadruples can be considered as a mixture of foliated triples and generalized pairs. We refer the reader to Subsection \ref{sec: reason to consider gfq} for a detailed explanation of why this new structure is vital not only for this paper but also for future studies of foliations and generalized pairs.

Under the framework of generalized foliated quadruples, the proofs of Theorems \ref{thm: main mmp foliation} and \ref{thm: main mmp gpair} proceed simultaneously.

\medskip

The first results to prove are the cone theorems for Theorem \ref{thm: main mmp foliation}(1) and Theorem \ref{thm: main mmp gpair}(1). As a positive beginning, the cone theorem for projective algebraically integrable foliations is already known \cite{ACSS21}. With some adjustments to the details of the proofs, the same approach used in \cite{ACSS21} also works for algebraically integrable generalized foliated quadruples (Theorem \ref{thm: cone theorem gfq}). In particular, the cone theorem for algebraically integrable generalized foliated quadruples implies the cone theorem for generalized pairs by letting $\Ff=T_X$.

Now we move on to prove the rest of Theorem \ref{thm: main mmp foliation}(1). We only need to show that each step of a $(K_{\Ff}+B)$-MMP is also a $(K_X+\Delta)$-MMP for some lc pair $(X,\Delta)$. To do this, we first show that $(X,\Ff,B)$ satisfies a property called ``ACSS" and that this property is preserved under each step of the MMP (Lemma \ref{lem: ACSS mmp can run}). The property "ACSS", named in honor of Ambro-Cascini-Shokurov-Spicer, can be viewed as the analogue of the concept of qdlt (cf. \cite{dFKX17}) for algebraically integrable foliations; see Definition \ref{defn: ACSS f-triple} for details. With this, we prove the termination of MMP with scaling for algebraically integrable foliations satisfying the property ``ACSS" and with very exceptional foliated log canonical divisor (Theorem \ref{thm: mmp very exceptional alg int fol}), which implies that F-dlt foliated triples are always ACSS. This implies the rest of Theorem \ref{thm: main mmp foliation}(1). Note that the same approach to the proof also works for $\Qq$-factorial generalized foliated quadruples with F-dlt singularities.

Our next goal is to establish the rest of Theorem \ref{thm: main mmp gpair}. First, by considering a class of structures larger than the category of generalized pairs (see Lemma \ref{lem: flip reduce special gpair to pair}) and applying some arguments, we can reduce the existence of flips for generalized pairs to the contraction theorem for generalized pairs. The contraction theorem for generalized pairs is an immediate corollary of the base-point-freeness theorem for generalized pairs, so we only need to prove the latter, which is Theorem \ref{thm: main mmp gpair}(2).

A crucial observation is that the base-point-freeness theorem for generalized pairs relies only on the subadjunction formula (Theorem \ref{thm: subadj intro}), which in turn, only depends on the fact that the moduli part of the canonical bundle formula for a generalized pair $f: (X,B,\Mm)\rightarrow Z$ is nef (Theorem \ref{thm: cbf gpair nonnqc}). A key observation is that the moduli part corresponds to the foliated canonical divisor $K_{\Ff}+B+\Mm_X$ (Proposition \ref{prop: weak cbf gfq}), where $\Ff$ is the foliation induced by $f$. With this, the canonical bundle formula for generalized pairs follows from the existence of log minimal models for generalized foliated quadruples with numerical dimension zero (Propositions \ref{prop: weak ss num 0 mmp}, \ref{prop: projective num 0 mmp}). The latter follows from Theorem \ref{thm: main mmp foliation}(1) which we have already established. This concludes the proof of Theorem \ref{thm: main mmp gpair}. It is worth mentioning that some previous literature addresses the canonical bundle formula for generalized pairs. However, these papers only consider generalized pairs with the additional ``NQC" condition (cf. \cite{Fil19,Fil20,JLX22,FS23}) and cannot be applied to our scenario.

Finally, we turn to the proof of Theorem \ref{thm: main mmp foliation}(2-4). Although the Bertini-type theorem fails for foliations, by employing the structure of generalized foliated quadruples, we can, roughly speaking, reduce Theorem \ref{thm: main mmp foliation}(3) and Theorem \ref{thm: main mmp foliation}(4) to Theorem \ref{thm: main mmp foliation}(2) (see Lemma \ref{lem: +a keep under mmp} and Theorem \ref{thm: gmm polarized gfq}). The proof of Theorem \ref{thm: main mmp foliation}(2) is divided into three steps:

In the first step, we use the already-proven contraction theorem for generalized pairs from Theorem \ref{thm: main mmp foliation}(2) to construct a contraction $X\rightarrow T$, where the general fibers of $X\rightarrow Z$ are tangent to $\Ff$.

In the second step, we apply the canonical bundle formula for generalized foliated quadruples to derive a generalized pair structure polarized with an ample divisor on $T$. This canonical bundle formula (Definition-Theorem \ref{defthm: cbf lctrivial morphism}) can be derived from the canonical bundle formula for lc-trivial fibrations of generalized pairs. The latter can be deduced using our approach via the theory of foliations. It is worth noting that the existing literature on the canonical bundle formula for generalized pairs (\cite{Fil19,Fil20,JLX22,FS23}) cannot handle arbitrary lc-trivial fibrations $f: (X,B,\Mm) \rightarrow Z$ as they required that $B\geq 0$ over the generic point of $Z$ or that $\Mm$ is $\bb$-semi-ample. Therefore, those works cannot be applied to deduce the canonical bundle formula for generalized foliated quadruples, which is essential for our purposes.

In the last step, we apply the cone theorem for generalized foliated quadruples to show that the generalized foliated log canonical divisor on $T$ is ample. Hence, the foliated log canonical divisor on $X$ is semi-ample, completing the proof of Theorem \ref{thm: main mmp foliation}(2). This concludes the proof of Theorem \ref{thm: main mmp foliation}.







\subsection{Structure of the Paper} In Section \ref{sec: statement of main results}, we list the main results of this paper and explain the importance of the structure of generalized foliated quadruples. The rest of the paper is divided into four parts. Part \ref{part:prelim} states some preliminary results, Part \ref{part:cone} establishes the cone theorem and the minimal model program for algebraically integrable foliations, Part \ref{part:cbf} establishes the canonical bundle formula and the minimal model program for generalized pairs, and Part \ref{part:gmm} proves the existence of good minimal models.

For the convenience of the reader, we have prepared the following flowchart (Table \ref{tbl: flowchart}) to illustrate the streamlined process involved in the proofs of our main theorems.

\begin{table}[ht]
\caption{Structure of the paper}\label{tbl: flowchart}
\begin{adjustbox}{width=1\textwidth,right}
%\begin{center}
$\xymatrix{
 &  *+[F]\txt{Preliminary results on foliations\\ and generalized pairs\\ (Sections \ref{sec: preliminaries}, \ref{sec: basic property gpair}, \ref{sec: stability gpair}, \ref{sec: acss gfq})} & \\
& *+[F]\txt{Precise adjunction formulas\\ for gfqs (Section \ref{sec: adjunction})}\ar@{->}[d]\ar@{->}[r] & *+[F]\txt{ACC and global ACC\\ for gfqs (Section \ref{sec: acc gfq})}\\
 & *+[F]\txt{Cone theorem for gfqs\\ (Section \ref{sec: cone})}\ar@{->}[d]\ar@{->}[r]\ar@{->}[dr]& *+[F]\txt{Existence of Mori fiber space\\ for gfqs (Subsection \ref{subsec: eomfs})}\ar@{->}[u]\\
*+[F]\txt{Very exceptional MMP\\ (Subsection \ref{subsec: very exceptional})}\ar@/_8pc/[ddddd] & *+[F]\txt{MMP with scaling\\ (Subsection \ref{subsec: eomfs})}\ar@{->}[d]\ar@{->}[l] & *+[F]\txt{Theorem \ref{thm: main mmp gpair}}
 \\
 & *+[F]\txt{$\kappa_{\sigma}=0$ MMP\\ (Subsection \ref{sub: num 0 mmp})}\ar@{->}[d]  &\\
 *+[F]\txt{Stability of gfqs\\ (Subsection \ref{subsec: stability gfq})}\ar@{->}[r] &  *+[F]\txt{Canonical bundle formula\\ for generalized pairs \\(Subsection \ref{subsec: cbf gpair})}\ar@{->}[d]\ar@{->}[dl] & \\
  *+[F]\txt{Canonical bundle formula for gfqs \\ (Subsection \ref{subsec: cbf gfq}, Section \ref{sec: subadj})}\ar@/_2pc/[ddr] & *+[F]\txt{Subadjunction\\ for generalized pairs \\(Section \ref{sec: subadj})}\ar@{->}[d] & \\
 &  *+[F]\txt{Base-point-freeness\\ and Contraction theorem\\ for generalized pairs\\ (Sections \ref{sec: du bois}, \ref{sec: vanishing gpair})}\ar@{->}[r]\ar@/_2pc/[uuuur]\ar@{->}[d] &  *+[F]\txt{Existence of flips\\
for generalized pairs\\ (Section \ref{sec: eof gpair})}\ar@{->}[uuuu]  \\
 *+[F]\txt{$\Qq$-factorial F-dlt\\ $\Rightarrow$ACSS\\ (Theorem \ref{thm: fdlt is acss})}\ar@{->}[r] &  *+[F]\txt{Theorem \ref{thm: main mmp foliation}}&  \\
&  &  \\
}$ 
\end{adjustbox}
\end{table}
\begin{comment}
\noindent\textbf{Part \ref{part:prelim}}. Preliminaries: Sections \ref{sec: preliminaries}, \ref{sec: basic property gpair}, and \ref{sec: stability gpair}. This part contains preliminary results and definitions that will be utilized throughout the remainder of the paper, with few foliation structures involved. In Section \ref{sec: preliminaries}, we introduce some basic definitions, including the concept of generalized foliated quadruples and their singularities. In Section \ref{sec: basic property gpair}, we establish basic properties of generalized pairs. Section \ref{sec: stability gpair} is parallel to \cite[Section 2]{ACSS21}, studying the stability of generalized pairs and introducing the concept of Property $(*)$ for generalized pairs.

\smallskip

\noindent\textbf{Part \ref{part:cone}}. Cone Theorem and the minimal model program for algebraically integrable foliations: Sections \ref{sec: adjunction}, \ref{sec: acss gfq}, \ref{sec: cone}, \ref{sec: mmp gfq}, and \ref{sec: acc gfq}. This part focuses on the cone theorem for algebraically integrable generalized foliated quadruples and its applications. In Section \ref{sec: adjunction}, we prove a precise adjunction formula for algebraically integrable generalized foliated quadruples. Section \ref{sec: acss gfq} defines ACSS generalized foliated quadruples and studies its fundamental behaviors. Section \ref{sec: cone} proves the cone theorem for algebraically integrable generalized foliated quadruples. In Section \ref{sec: mmp gfq}, we prove most results of this paper on the minimal model program for generalized foliated quadruples with $\Qq$-factorial F-dlt singularities, excluding the existence of good minimal models. Section \ref{sec: acc gfq} verifies the ACC, the global ACC, and the existence of uniform rational polytopes for algebraically integrable generalized foliated quadruples.

\smallskip

\noindent\textbf{Part \ref{part:cbf}}. Canonical bundle formula and minimal model program for generalized pairs: Sections \ref{sec: cbf}, \ref{sec: subadj}, \ref{sec: du bois}, \ref{sec: vanishing gpair}, and \ref{sec: eof gpair}. This part presents the canonical bundle formula and applies it to establish the minimal model program for generalized pairs. In Section \ref{sec: cbf}, we state and prove the canonical bundle formula for lc-trivial fibrations of generalized foliated quadruples and, in particular, generalized pairs. Section \ref{sec: subadj} studies lc-trivial morphisms of generalized foliated quadruples and proves the subadjunction formula for generalized pairs. Section \ref{sec: du bois} shows that lc generalized pairs have Du Bois singularities. Section \ref{sec: vanishing gpair} proves the Kodaira vanishing theorem, the Kawamata-Viehweg vanishing theorem, the base-point-freeness theorem, and the contraction theorem for generalized pairs. Section \ref{sec: eof gpair} proves the existence of flips for $\Qq$-factorial generalized pairs.

\smallskip

\noindent\textbf{Part \ref{part:gmm}}. Good minimal model, applications, and proofs of the main theorems: Sections \ref{sec: gmm fdlt} and \ref{sec: proof of the main theorems}. In Section \ref{sec: gmm fdlt}, we prove the existence of good minimal models for $\Qq$-factorial F-dlt generalized foliated quadruples polarized with an ample divisor. Similar arguments imply the base-point-freeness theorem for such quadruples, leading to a special case of the Prokhorov-Shokurov $\bb$-semi-ampleness conjecture. Lastly, in Section \ref{sec: proof of the main theorems}, we prove all the main theorems of the paper that are not covered in the previous sections.
\end{comment}

\medskip


\noindent\textbf{Acknowledgement}. The authors would like to thank Caucher Birkar, Paolo Cascini, Priyankur Chaudhuri, Omprokash Das, Christopher D. Hacon, Chen Jiang, Junpeng Jiao, Jie Liu, Yuchen Liu, Roktim Mascharak, Fanjun Meng, Wenhao Ou, Vyacheslav V. Shokurov, Chenyang Xu, and Qingyuan Xue for fruitful discussions. Part of this work was inspired by discussions that the third author had with Paolo Cascini at the Simons Center at Stony Brook University in May 2023, and at Tsinghua University in August 2023. Portions of this work were completed during visits by the third and fourth authors to Fudan University, and by the last three authors to Tsinghua University in June 2023. The authors extend their gratitude for the warm hospitality received during these visits. The second author is affiliated with LMNS at Fudan University, and has received support from the National Key Research and Development Program of China (Grant No. 2020YFA0713200). The fourth author has been partially supported by NSF research grants no. DMS-1801851 and DMS-1952522, as well as a grant from the Simons Foundation (Award Number: 256202).






\section{Statement of main results}\label{sec: statement of main results}

In this section, we provide the statements of the main results of this paper.

\subsection{Minimal model program for algebraically integrable foliations}\label{subsec: mmp aif} 

The theory of foliations holds a significant place in birational geometry. Most notably, it has played a critical role in Miyaoka's proof of several key cases of the abundance conjecture in dimension three \cite{Miy87}. In recent developments, foliations have been used by Bogomolov and McQuillan to analyze projective varieties which admit a non-trivial fibration with rationally connected fibers \cite{BM16}. Furthermore, foliation theory has strong connections with other areas of algebraic geometry, such as the algebraic geometry in characteristic $p>0$ and number theory as highlighted by the Grothendieck-Katz conjecture and the Ekedahl-Shepherd-Barron-Taylor conjecture. Its importance is also highlighted in hyperbolicity theory, where it was essential in McQuillan's proof of a specific case of the Green-Griffiths-Lang conjecture \cite{McQ98}.

In recent years, it has been discovered that many structures and results in classical birational geometry can be extended to foliations, especially, within the context of the minimal model program. Instead of examining the structures associated with the canonical divisor of the ambient variety $K_X$, the foliations theory concentrates on the structures connected to the foliated canonical divisor $K_{\Ff}$. This approach offers greater flexibility in practice. Specifically, when $\Ff = T_X$, we find that $K_{\Ff}=K_X$, bringing us back to the classical setting.

The foundational work for the minimal model program for foliations has been established for foliated surfaces (cf. \cite{McQ08,Bru15}) and foliated threefolds (cf. \cite{CS20,Spi20,CS21,SS22}). Moreover, several classic questions from the minimal model programs, such as the ascending chain condition (ACC) conjecture for minimal log discrepancies, the ACC conjecture for lc thresholds, the global ACC, and the index theorems, have been adapted to foliations and verified in dimensions 2 and/or 3, as indicated in \cite{Che22,Che23,LLM23,LMX23a,LMX23b}.

Given these developments, it is natural to ask whether the minimal model program for foliations could extend to higher dimensions. Unfortunately, this seems to be a challenging question, with limited information available, even in dimension 4. However, from the perspective of the minimal model program, it seems sufficient to focus on a subset of foliations that have an additional structure: algebraically integrable foliations.

Algebraically integrable foliations are foliations where the general leaves are algebraic varieties; in other words, they are induced by dominant rational maps. These foliations naturally come into play when a fibration structure is established. Notably, Miyaoka's study of the abundance conjecture in dimension $3$ primarily utilized algebraically integrable foliations \cite{Miy87}, as opposed to arbitrary ones. This approach has been reflected in recent research into the abundance conjecture for Kähler threefolds \cite{DO23a,DO23b} and threefolds over fields of characteristic $p>3$ with numerical dimension $2$ \cite{Xu23}. In these studies, the algebraic integrability of foliations is guaranteed; indeed, all the foliations addressed in these papers are induced by MRC fibrations, making them automatically algebraically integrable. Given this, algebraically integrable foliations are expected to be crucial in future research of the minimal model program, particularly in questions related to the abundance conjecture.

The first objective of this paper is to develop the minimal model program for algebraically integrable foliations of arbitrary rank with ``mild" singularities in arbitrary dimensions. Here ``mild" singularity is usually referred to as ``F-dlt" (see Definition \ref{defn: fdlt}). As explained in \cite{CS21,SS22}, F-dlt foliated triples play the same role as dlt pairs in the classical MMP and is a natural class of singularities to study. Moreover, any terminal foliated singularity is F-dlt.

Recall that a foliated triple $(X,\Ff,B)/U$ consists of a normal quasi-projective variety $X$ associated with a projective surjective morphism $X\rightarrow U$, a foliation $\Ff$ on $X$, and an $\Rr$-divisor $B\geq 0$ on $X$, such that $K_{\Ff}+B$ is $\Rr$-Cartier. The first result of this paper shows that we can run a $(K_{\Ff}+B)$-MMP$/U$ provided that it is $\Qq$-factorial F-dlt:

\begin{thm}[Minimal model program]\label{thm: mmp fdlt}
    Let $(X,\Ff,B)/U$ be a $\Qq$-factorial foliated triple. Assume that $\Ff$ is algebraically integrable and $(X,\Ff,B)$ is F-dlt. Then we may run a $(K_{\Ff}+B)$-MMP$/U$.
\end{thm}

We remark that when $\dim X\leq 3$, Theorem \ref{thm: mmp fdlt} is known when $\rk\Ff=2$ (\cite[Corollary 2.3]{SS22}; \cite[Theorem 1.1]{CS21} when $U=\{pt\}$) and when $\rk\Ff=1$ and $U=\{pt\}$ (\cite[Theorems 1.1, 2.36, Section 6]{CS21}), even without the algebraically integrable condition. When assuming the termination of klt flips in dimension $\leq\rk\Ff$, \cite[Theorem 1.1]{CS23a} proves Theorem \ref{thm: mmp fdlt} even without the F-dlt condition, but requires that $(X,B)$ is klt. In particular, when $(X,B)$ is klt and $\dim X\leq 4$, Theorem \ref{thm: mmp fdlt} can be deduced from \cite[Theorem 1.1]{CS23a}.

Proceeding further, we demonstrate the termination of MMP with scaling as well as the existence of good minimal models for algebraically integrable foliations polarized with an ample divisor. The polarization of the ample divisor is a natural condition to add, as can be seen in \cite[Theorem 1.2]{CS21} and \cite[Theorem 1.3]{CS20}. It is worth noting that, even within the framework of the classical MMP, the existence of good minimal models in higher dimensions is only known when polarized with an ample divisor (\cite[Theorem C]{BCHM10}, \cite[Theorem 1.5]{HH20}),  while the general case remains an open conjecture.


\begin{thm}[Good minimal model]\label{thm: +a gmm fdlt}
    Let $(X,\Ff,B)/U$ be a $\Qq$-factorial foliated triple. Assume that $\Ff$ is algebraically integrable, $B\geq A\geq 0$ for some ample$/U$ $\Rr$-divisor $A$, and $(X,\Ff,B)$ is F-dlt. Then we may run a $(K_{\Ff}+B)$-MMP$/U$ with scaling of an ample$/U$ $\Rr$-divisor $H$, and any such MMP terminates 
    \begin{enumerate}
        \item with a Mori fiber space of $(X,\Ff,B)/U$ if $K_{\Ff}+B$ is not pseudo-effective$/U$, and
        \item with a good minimal model of $(X,\Ff,B)/U$ if $K_{\Ff}+B$ is pseudo-effective$/U$.
    \end{enumerate}
\end{thm}


We also have the following result on the abundance of algebraically integrable foliations polarized with an ample divisor.

\begin{thm}[Abundance]\label{thm: +a abundance fdlt}
  Let $(X,\Ff,B)/U$ be a $\Qq$-factorial foliated triple and $A$ an ample$/U$ $\Rr$-divisor on $X$.  Assume that $\Ff$ is algebraically integrable and $(X,\Ff,B)$ is F-dlt. Then 
  $$\kappa_{\sigma}(X/U,K_{\Ff}+B+A)=\kappa_{\iota}(X/U,K_{\Ff}+B+A).$$
\end{thm}

It is important to note that Theorem \ref{thm: +a abundance fdlt} is not a direct consequence of Theorem \ref{thm: +a gmm fdlt}. This is because Bertini type theorems fail for foliations, and it is possible that $(X,\Ff,B+H)$ is not lc for any $H\in |A/U|_{\mathbb R}$ (see \cite[Example 3.4]{DLM23}).

We also prove a base-point-freeness theorem for algebraically integrable foliations.

\begin{thm}[Base-point-freeness]\label{thm: bpf fdlt}
  Let $(X,\Ff,B)/U$ be a $\Qq$-factorial foliated triple. Assume that $\Ff$ is algebraically integrable and $(X,\Ff,B)$ is F-dlt. Let $A$ be an ample$/U$ $\Rr$-divisor on $X$ such that $K_{\Ff}+B+A$ is nef$/U$. Then:
  \begin{enumerate}
      \item $K_{\Ff}+B+A$ is semi-ample$/U$.
      \item Suppose that there exists a positive integer $m$ such that $m(K_{\Ff}+B+A)$ is Cartier. Then
      $$\mathcal{O}_X(mn(K_{\Ff}+B+A))$$
      is globally generated$/U$ for any integer $n\gg 0$.
    \end{enumerate}
\end{thm}
In the literature, the semi-ampleness of $K_\Ff+B+A$ is known when $(X,\Ff,B+A)$ is $\mathbb Q$-factorial F-dlt, $U=\{pt\}$, and $\dim X\leq 3$, even without the algebraically integrable condition (see \cite[Theorem 1.3]{CS20}, \cite[Theorem 1.3]{CS21}). However, there was no result on the base-point-freeness theorem of foliations in dimensions $\geq 3$. It is worth mentioning that the base-point-freeness theorem Theorem \ref{thm: bpf fdlt}(2) is crucial for us to prove a special case of the Prokhorov-Shokurov's $\bb$-semi-ampleness conjecture later in this paper (Theorem \ref{thm: ps intro}).

An important application of Theorem \ref{thm: +a gmm fdlt} is the existence of Mori fiber spaces for foliated triples, even with, at worst, lc singularities. We note that in this paper, Mori fiber spaces and log minimal models are in the sense of Birkar-Shokurov; that is, we allow the extraction of lc centers. See Definitions \ref{defn: models I} and \ref{defn: models ii} for details.

\begin{thm}\label{thm: eomfs}
    Let $(X,\Ff,B)/U$ be an lc foliated triple. Assume that $\Ff$ is algebraically integrable and $K_{\Ff}+B$ is not pseudo-effective$/U$. Then $(X,\Ff,B)/U$ has a Mori fiber space.
\end{thm}

Another interesting type of foliations is the class of foliations with numerical dimension zero. For example, based on \cite[Theorem 1.4]{CS20} and \cite[Theorem 1.7]{CS21}, \cite[Theorem 1.9]{LLM23} has shown the existence of good minimal models for numerical dimension zero foliations in dimension $\leq 3$. In this paper, we obtain the existence of good minimal models for algebraically integrable foliations with numerical dimension zero:
\begin{thm}\label{thm: gmm ai num0}
    Let $(X,\Ff,B)$ be a projective lc foliated triple. Assume that $\Ff$ is algebraically integrable and $\kappa_{\sigma}(K_{\Ff}+B)=0$. Then:
   \begin{enumerate}
       \item $(X,\Ff,B)$ has a good minimal model.
       \item $\kappa_{\iota}(K_{\Ff}+B)=0$.
       \item If $(X,\Ff,B)$ is $\Qq$-factorial dlt, then we may run a $(K_{\Ff}+B)$-MMP with scaling of an ample $\Rr$-divisor, and any such MMP terminates with a good minimal model of $(X,\Ff,B)$.
   \end{enumerate}
\end{thm}

Siu \cite{Siu10} has used Eckl’s construction of numerically trivial foliations \cite{Eck04} to sketch a plan to solve the abundance conjecture. One step of Siu's approach, \cite[(4.1)]{Siu10}, focuses on the abundance conjecture for smooth projective varieties associated with an ``algebraically integrable numerically trivial foliation". Though the concept of ``numerically trivial foliation" in \cite{Siu10}, which was defined analytically in \cite{Eck04}, seems to differ from the concept of ``foliations whose canonical divisor has numerical dimension zero", these two types of foliations are closely connected. Hence, studying the abundance properties of numerical dimension zero algebraically integrable foliations (potentially with singularities that are worse than lc) on smooth projective varieties becomes intriguing, as it may have implications for the abundance conjecture. With this in mind, we prove the following theorem in this paper:

\begin{thm}\label{thm: abundance num 0 no restriction to f}
    Let $(X,\Ff,B)$ be a projective algebraically integrable f-triple such that $\kappa_{\sigma}(K_{\Ff}+B)=0$. Assume that $K_X+B$ is pseudo-effective and $(X,B)$ is lc. Then $\kappa_{\iota}(K_{\Ff}+B)=0$.
\end{thm}

Finally, we recall the following conjecture of Cascini and Spicer:

\begin{conj}[{\cite[Conjecture 4.2]{CS23a}}]\label{conj: cs23 4.2(1)}
Let $(X,\Ff,B)$ be a $\Qq$-factorial projective foliated triple, such that $\Ff$ is algebraically integrable, $B$ is a $\Qq$-divisor, $(X,B)$ is klt, and one of the following cases hold:
\begin{enumerate}
    \item $(X,\Ff,B)$ is F-dlt.
    \item $(X,\Ff,B)$ is canonical. 
\end{enumerate}
 Then there exists a morphism $f: X\rightarrow Y$ which induces $\Ff$. 
\end{conj}
In this paper, we provide a positive answer to Conjecture \ref{conj: cs23 4.2(1)}(1) with weaker assumptions and stronger results:
\begin{thm}\label{thm: cs23 4.2(1)}
    Let $(X,\Ff,B)$ be a $\Qq$-factorial foliated triple such that $\Ff$ is algebraically integrable and $(X,\Ff,B)$ is F-dlt. Then:
    \begin{enumerate}
    \item $(X,B)$ is qdlt (cf. Definition \ref{defn: qdlt}). In particular, if $\lfloor B\rfloor=0$, then $(X,B)$ is klt.
    \item There exists a morphism $f: X\rightarrow Y$ to a smooth variety which induces $\Ff$.
    \end{enumerate}
\end{thm}

We also prove a weaker form of Conjecture \ref{conj: cs23 4.2(1)}(2) without assuming that $(X,B)$ is klt.
\begin{thm}\label{thm: canonical almost holomorphic}
    Let $(X,\Ff,B)$ be a $\Qq$-factorial canonical foliated triple such that $\Ff$ is algebraically integrable. Then $\Ff$ is induced by an almost holomorphic map.
\end{thm}
A very recent result \cite[Theorem 1.4]{CS23b} shows that the algebraic part of a foliation $\Ff$ on a projective variety $X$ is induced by an almost holomorphic map, provided that $X$ is $\Qq$-factorial klt and $\Ff$ is canonical. In particular, \cite[Theorem 1.4]{CS23b} implies Theorem \ref{thm: canonical almost holomorphic} when $X$ is projective klt and $B=0$.

We would like to mention that the results in this subsection are not expected to work over fields of characteristic $p>0$ due to counterexamples in \cite{Ber23}.







\subsection{Minimal model program for generalized pairs}\label{subsec: mmp gpair} In the past few years, there has been significant advancement in the minimal model program for NQC generalized pairs. The cone theorem, as well as the $\mathbb Q$-factorial cases of the contraction theorem and the proof of the existence of flips, were established in \cite{HL21a}. Later, the existence of flips for (potentially non-$\mathbb Q$-factorial) NQC generalized pairs was verified in \cite{LX23a}, while the contraction theorem for these pairs was proven in \cite{Xie22}. Additionally, \cite{CLX23} confirmed the Kodaira and the Kawamata-Viehweg vanishing theorems for NQC generalized pairs, offering an alternative proof for the contraction theorem. These developments form the foundation of the minimal model program for NQC generalized pairs, with numerous corollaries and applications already provided in \cite{LT22,TX23}.

The structure of NQC generalized pairs has naturally arisen in the study of the canonical bundle formulas, making them a fundamental structure in the study in the minimal model program. For a considerable amount of time, it has been presumed that the realm of NQC generalized pairs would be the most extensive category necessary to establish in the minimal model program. This is because the structure of NQC generalized pairs is maintained under the canonical bundle formula, adjunction formula, and each stage of the minimal model program, thereby eliminating the need to consider the minimal model program for non-NQC generalized pairs or other larger categories.

However, recent studies on the minimal model program for K\"ahler varieties \cite{DH23,DHY23} have emphasized the critical role the structure of non-NQC generalized pairs plays in the minimal model program for K\"ahler varieties. In the case of K\"ahler varieties, the selection of ample divisors is restricted, preventing many procedures, such as the minimal model program with scaling and general hyperplane section cuttings. Nevertheless, the associated K\"ahler class $\omega$ on a K\"ahler variety serves as a substitute for ample divisors. Although $\omega$ cannot be categorized as an $\Rr$-divisor, it can be considered as an nef $\Rr$-class and is suitable for the nef part of a generalized pair. As explained in \cite{DHY23}, it is now possible to formally define ``running MMP with scaling of the nef $\Rr$-$(1,1)$-class $\overline{\omega}$". Given that $\omega$ is only an $\Rr$-class and NQC cannot be assured, the study of the structure of non-NQC generalized pairs immediately becomes vital for the minimal model program on K\"ahler vareties.

Although little was known about the minimal model program for non-NQC generalized pairs, we have been able to establish the cone theorem and the contraction theorem for non-NQC generalized pairs, thanks to the cone theorem and the canonical bundle formula for generalized foliated quadruples. With additional effort, we also prove the existence of flips for $\Qq$-factorial non-NQC generalized pairs. These results collectively lay the groundwork for the minimal model program for $\Qq$-factorial generalized pairs. The detailed theorems are as follows:

\begin{thm}[Cone and contraction theorems]\label{thm: cone theorem nonnqc gpair}
Let $(X,B,\Mm)/U$ be a generalized pair and $\pi: X\rightarrow U$ the associated morphism. Let $\{R_j\}_{j\in\Lambda}$ be the set of $(K_{X}+B+\Mm_X)$-negative extremal rays in $\overline{NE}(X/U)$ that are rational. Then:
\begin{enumerate}
    \item $$\overline{NE}(X/U)=\overline{NE}(X/U)_{K_{X}+B+\Mm_X\geq 0}+\overline{NE}(X/U)_{\Nlc(X,B,\Mm)}+\sum_{j\in\Lambda} R_j.$$
    In particular, any $(K_{X}+B+\Mm_X)$-negative extremal ray in $\overline{NE}(X/U)$ is rational.
    \item Each $R_j$ is spanned by a rational curve $C_j$ such that $\pi(C_j)=\{pt\}$ and 
    $$0<-(K_{X}+B+\Mm_X)\cdot C_j\leq 2\dim X.$$
    \item For any ample$/U$ $\Rr$-divisor $A$ on $X$,
    $$\Lambda_A:=\{j\in\Lambda\mid R_j\subset\overline{NE}(X/U)_{K_{X}+B+A+\Mm_X<0}\}$$
    is a finite set. In particular, $\{R_j\}_{j\in\Lambda}$ is countable, and is a discrete subset in $\overline{NE}(X/U)_{K_{X}+B+\Mm_X<0}$. Moreover, we may write
    $$\overline{NE}(X/U)=\overline{NE}(X/U)_{K_{X}+B+A+\Mm_X\geq 0}+\overline{NE}(X/U)_{\Nlc(X,B,\Mm)}+\sum_{j\in\Lambda_A}R_j.$$
    \item Let $F$ be a $(K_X+B+\Mm_X)$-negative extremal face in $\overline{NE}(X/U)$ that relatively ample at infinity (cf. Definition \ref{defn: basics of cone theorem}) with respect to $(X,B,\Mm)$. Then $F$ is a rational extremal face, and there exists a contraction$/U$ $\cont_F: X\rightarrow Z$ of $F$ satisfying the following.
\begin{enumerate}
    \item For any integral curve $C$ on $X$ such that the image of $C$ in $U$ is a closed point, $\cont_F(C)$ is a point if and only if $[C]\in F$.
    \item $\mathcal{O}_Y=(\cont_F)_*\mathcal{O}_X$. In other words, $\cont_F$ is a contraction.
    \item For any Cartier divisor $D$ on $Y$ such that $D\cdot C=0$ for any curve $C$ contracted by $\cont_F$, there exists a Cartier divisor $D_Y$ on $Y$ such that $D=\cont_F^*D_Y$.
\end{enumerate}
\end{enumerate}
\end{thm}
When $\Mm$ is NQC$/U$ and $(X,B,\Mm)$ is lc, Theorem \ref{thm: cone theorem nonnqc gpair}(1-3) was proven in \cite[Theorem 1.3]{HL21a} and Theorem \ref{thm: cone theorem nonnqc gpair}(4) was proven in \cite[Theorem 1.5]{Xie22} (see also \cite[Theorem 1.7]{CLX23}).

\begin{thm}[Existence of flips]\label{thm: eof nonnqc}
Let $(X,B,\Mm)/U$ be a $\Qq$-factorial lc generalized pair and $f: X\rightarrow Z$ a $(K_X+B+\Mm_X)$-flipping contraction$/U$.

Then the flip $f^+: X^+\rightarrow Z$ of $f$ exists. Moreover, $X^+$ is $\Qq$-factorial, and $\rho(X)=\rho(X^+)$.
\end{thm}
When $\Mm$ is NQC$/U$, Theorem \ref{thm: eof nonnqc} was proven in \cite[Theorem 1.2]{HL21a} (see also \cite[Theorem 1.2]{LX23b}). Theorem \ref{thm: cone theorem nonnqc gpair} and Theorem \ref{thm: eof nonnqc} allow us to run the minimal model program for any $\Qq$-factorial lc generalized pair:

\begin{thm}\label{thm: qfact nonnqc mmp can run intro} 
We may run the minimal model program for $\Qq$-factorial lc generalized pairs. More precisely, for any $\mathbb Q$-factorial lc generalized pair $(X,B,\Mm)/U$, there exists a sequence of $(K_X+B+\Mm_X)$-flips and divisorial contractions$/U$. Moreover, any such sequence ends either with a Mori fiber space of $(X,B,\Mm)/U$, or a minimal model of $(X,B,\Mm)/U$, or an infinite sequence of flips over $U$.
\end{thm}

There are several other important results on the structure of generalized lc pairs. The first two are the Kodaira vanishing theorem and the Kawamata-Viehweg vanishing theorem:

\begin{thm}[Kodaira vanishing theorem for lc generalized pairs]\label{thm: kod vanishing gpair intro}
Let $(X,B,\Mm)$ be a projective lc generalized pair, and let $D$ be a Cartier divisor on $X$ such that $D-(K_X+B+\Mm_X)$ is ample. Then $H^i(X,\mathcal{O}_X(D))=0$ for any positive integer $i$.
\end{thm}

\begin{thm}[Relative Kawamata-Viehweg vanishing for lc generalized pairs]\label{thm: kv vanishing gpair intro}
Let $(X,B,\Mm)/U$ be an lc generalized pair associated with morphism $f: X\rightarrow U$, and let $D$ be a Cartier divisor on $X$ such that $D-(K_X+B+\Mm_X)$ is nef$/U$ and log big$/U$ with respect to $(X,B,\Mm)$. Then $R^if_*\mathcal{O}_X(D)=0$ for any positive integer $i$.
\end{thm}

When $\Mm$ is NQC$/U$, Theorem \ref{thm: kod vanishing gpair intro} was proven in \cite[Theorem 1.3]{CLX23} while Theorem \ref{thm: kv vanishing gpair intro} was proven in \cite[Theorem 1.4]{CLX23}.

\smallskip

Next, we have the base-point-freeness theorem and the semi-ampleness theorem for lc generalized pairs:

\begin{thm}[Base-point-freeness theorem]\label{thm:base-point-freeness intro}
Let $(X,B,\Mm)/U$ be an lc generalized pair and $D$ a nef$/U$ Cartier divisor on $X$, such that $aD-(K_X+B+\Mm_X)$ is ample$/U$ for some positive real number $a$. Then $\mathcal{O}_X(mD)$ is globally generated over $U$ for any integer $m\gg 0$.
\end{thm}

\begin{thm}[Semi-ampleness theorem]\label{thm: semi-ampleness intro}
Let $(X,B,\Mm)/U$ be an lc generalized pair and $D$ a nef$/U$ $\mathbb R$-Cartier $\Rr$-divisor on $X$, such that $D-(K_X+B+\Mm_X)$ is ample$/U$. Then $D$ is semi-ample$/U$.
\end{thm}

When $\Mm$ is NQC$/U$, Theorem \ref{thm:base-point-freeness intro} was proven in \cite[Theorem 1.4]{Xie22}, \cite[Theorem 1.5]{CLX23} while Theorem \ref{thm: semi-ampleness intro} was proven in \cite[Theorems 1.2]{Xie22}, \cite[Theorem 1.6]{CLX23}.

We also prove the canonical bundle formula and the subadjunction formula for generalized pairs. As the canonical bundle formula's statement is very technical and is a special case of Theorem \ref{thm: cbf gfq} below (by setting $\Ff=T_X$), we will omit it here and only state the subadjunction formula.

\begin{thm}[Subadjunction formula]\label{thm: subadj intro}
    Let $(X,B,\Mm)/U$ be an lc generalized pair an $V$ an lc center of $(X,B,\Mm)$ such that $\dim V\geq 1$. Let $W$ be the normalization of $V$. Then there exists an lc generalized pair $(W,B_W,\Mm^W)/U$ such that
    $$K_W+B_W+\Mm^W_W\sim_{\mathbb R}(K_X+B+\Mm_X)|_W.$$
    Moreover, the image of any lc center of $(W,B_W,\Mm^W)$ in $X$ is an lc center of $(X,B,\Mm)$.
\end{thm}
The main part of Theorem \ref{thm: subadj intro} was proven in \cite[Theorem 5.1]{HL21b} when $\Mm$ is NQC$/U$.

Finally, we can show that lc generalized pairs are Du Bois:

\begin{thm}\label{thm: glc sings are Du Bois}
Let $(X,B,\Mm)$ be an lc generlaized pair. Then any union of lc centers of $(X,B,\Mm)$ is Du Bois. In particular, $X$ is Du Bois.
\end{thm}
Theorem \ref{thm: glc sings are Du Bois} was proven in \cite[Theorem 1.6]{LX23b} when $\Mm$ is NQC$/X$.



\subsection{Generalized foliated quadruples}

As explained above, to establish the minimal model program for algebraically integrable foliations and generalized pairs, we need to broaden the category of objects for our study and consider the structure of \emph{generalized foliated quadruples} $(X,\Ff,B,\Mm)$, as defined in Definition \ref{defn: gfq intro}.

Most of the main theorems of this paper on algebraically integrable foliations can also be extended to the category of algebraically integrable generalized foliated quadruples. Two results related to this structure that are particularly worth mentioning are the cone theorem and the canonical bundle formula. These two results will be essential in other main theorems of the paper, the statements of most of which do not rely on the language of generalized foliated quadruples.

\subsubsection{Cone theorem} We establish the cone theorem for algebraically integrable generalized foliated quadruples in full generality. 

\begin{thm}[Cone theorem for algebraically integrable generalized foliated quadruples]\label{thm: cone theorem gfq}
Let $(X,\Ff,B,\Mm)/U$ be a generalized foliated quadruple and $\pi: X\rightarrow U$ the associated morphism. Let $\{R_j\}_{j\in\Lambda}$ be the set of $(K_{\Ff}+B+\Mm_X)$-negative extremal rays in $\overline{NE}(X/U)$ that are rational. Assume that $\Ff$ is algebraically integrable. Then:
\begin{enumerate}
    \item $$\overline{NE}(X/U)=\overline{NE}(X/U)_{K_{\Ff}+B+\Mm_X\geq 0}+\overline{NE}(X/U)_{\Nlc(X,\Ff,B,\Mm)}+\sum_{j\in\Lambda} R_j.$$
    Here $\Nlc(X,\Ff,B,\Mm)$ is the non-lc locus of $(X,\Ff,B,\Mm)$ (cf. Definition \ref{defn: gfq singularity}). In particular, any $(K_{\Ff}+B+\Mm_X)$-negative extremal ray in $\overline{NE}(X/U)$ is rational.
    \item Each $R_j$ is spanned by a rational curve $C_j$ such that $\pi(C_j)=\{pt\}$, $C_j$ is tangent to $\Ff$, and 
    $$0<-(K_{\Ff}+B+\Mm_X)\cdot C_j\leq 2\dim X.$$
    \item For any ample$/U$ $\Rr$-divisor $A$ on $X$,
    $$\Lambda_A:=\{j\in\Lambda\mid R_j\subset\overline{NE}(X/U)_{K_{\Ff}+B+A+\Mm_X<0}\}$$
    is a finite set. In particular, $\{R_j\}_{j\in\Lambda}$ is countable, and is a discrete subset in $\overline{NE}(X/U)_{K_{\Ff}+B+\Mm_X<0}$. Moreover, we may write
    $$\overline{NE}(X/U)=\overline{NE}(X/U)_{K_{\Ff}+B+A+\Mm_X\geq 0}+\overline{NE}(X/U)_{\Nlc(X,\Ff,B,\Mm)}+\sum_{j\in\Lambda_A}R_j.$$
    \item Let $F$ be a $(K_X+B+\Mm_X)$-negative extremal face in $\overline{NE}(X/U)$ that relatively ample at infinity (cf. Definition \ref{defn: basics of cone theorem}) with respect to $(X,\Ff,B,\Mm)$. Then $F$ is a rational extremal face.
\end{enumerate}
\end{thm}
 When $\Ff=T_X$ and $\Mm=\bm{0}$, Theorem \ref{thm: cone theorem gfq} follows from \cite[Theorem 5.10]{Amb03} and \cite[Theorem 4.5.2]{Fuj17}. However, whenever either $\Ff\not=T_X$ or $\Mm\not=\bm{0}$, Theorem \ref{thm: cone theorem gfq} becomes new. More precisely, there are two cases that worth to mention:
 \begin{enumerate}
 \item When $\Ff=T_X$, we get the cone theorem for generalized pairs, Theorem \ref{thm: cone theorem nonnqc gpair}, which is new.
 \item When $U=\{pt\}$ and $\Mm=\bm{0}$, (2) and a large part of (1) (the part without considering he rationality of $R_j$) were proven in \cite[Theorem 3.9]{ACSS21}, but the rest parts are missing. Therefore, we cannot directly use \cite[Theorem 3.9]{ACSS21} to prove Theorem \ref{thm: mmp fdlt} and Theorem \ref{thm: cone theorem gfq} becomes necessary.
 \end{enumerate}
We would like to note that the contraction theorem, the existence of flips, and the base-point-freeness theorem are still valid for generalized foliated quadruples that possess nice singularities (e.g. F-dlt). However, since these theorems do not hold the same level of importance in proving our other main theorems as the cone theorem does, we choose to omit them here.


\subsubsection{Canonical bundle formula} The canonical bundle formula for foliated triples, as established in \cite[Theorem 1.3]{LLM23}, plays a crucial role in proving the global ACC for foliated threefolds. However, due to technical challenges, the work presented in \cite{LLM23} could not prove the canonical bundle formula for generalized foliated quadruples $(X,\Ff,B,\Mm)$ unless the nef part $\Mm$ is $\bb$-semi-ample. In this study, we overcome these technical difficulties with innovative approaches, successfully proving the canonical bundle formula for generalized foliated quadruples in a more comprehensive manner.

\begin{thm}\label{thm: cbf gfq}
Let $(X,\Ff,B,\Mm)/U$ be a sub-generalized foliated quadruple and $f: X\rightarrow Z$ a contraction$/U$, such that $f: (X,\Ff,B,\Mm)\rightarrow Z$ is an lc-trivial morphism (see Definition \ref{defn: lc trivial morphism}). Let $B_Z$ and $\Mm^Z$ be the discriminant part and the base moduli part of $f: (X,\Ff,B,\Mm)\rightarrow Z$ (see Definition-Theorem \ref{defthm: cbf lctrivial morphism}). Then $\Mm^Z$ is $\bb$-nef$/U$ and
$$K_{\Ff}+B+\Mm_X\sim_{\mathbb R}f^*(K_{\Ff_Z}+B_Z+\Mm^Z_Z).$$
Moreover, we have the following properties:
\begin{enumerate}
\item $B_Z$ is uniquely determined and $\Mm^Z$ is uniquely determined up to $\Rr$-linear equivalence.
\item If $B\geq 0$, then $B_Z\geq 0$.
\item If $(X,\Ff,B,\Mm)$ is sub-lc, then $(Z,\Ff_Z,B_Z,\Mm^Z)$ is sub-lc.
\item If $(X,\Ff,B,\Mm)$ is lc, then $(Z,\Ff_Z,B_Z,\Mm^Z)$ is lc.
\item If $(X,\Ff,B,\Mm)$ is sub-lc or $f$ has connected fibers, then any lc center of $(Z,\Ff_Z,B_Z,\Mm^Z)$ is the image of an lc center of $(X,\Ff,B,\Mm)$ on $Z$.
\item If $f$ has connected fibers, then the image of any lc center of $(X,\Ff,B,\Mm)$ on $Z$ is an lc center of $(Z,\Ff_Z,B_Z,\Mm^Z)$.
\item If $f$ has connected fibers, then for any prime divisor $D$ on $X$, 
$$\mult_DB_Z=\epsilon(D)-\sup\{t\mid (X,\Ff,B+tf^*D,\Mm)\text{ is lc over the generic point of }D\}$$
where $\epsilon(D)=0$ if $D$ is $\Ff_Z$-invariant, and $\epsilon(D)=1$ otherwise (see Definition \ref{defn: special divisors on foliations}). 
\item The $\Rr$-linear equivalence class of $\Mm^Z$ only depends on $(X,B,\Mm)$ over the generic point of $Z$.
\item If $\Mm$ is NQC$/U$, then $\Mm^Z$ is NQC$/U$.
\end{enumerate}
\end{thm}

We want to emphasize that Theorem \ref{thm: cbf gfq} is applicable to any foliation, not just those that are algebraically integrable. Consequently, we anticipate that Theorem \ref{thm: cbf gfq} will play a significant role in future studies, encompassing both algebraically integrable foliations and those that are not necessarily algebraically integrable.

Next, we revisit the history of partial results that have contributed to the main part of Theorem \ref{thm: cbf gfq}, i.e. the nefness$/U$ of $\Mm^Z$. 
\begin{enumerate}
    \item When $\Ff=T_X$ and $\Mm=\bm{0}$, the main part of Theorem \ref{thm: cbf gfq} is\cite[Theorem 1.2]{JLX22} (or \cite[Theorem 2.23]{JLX22} combined with \cite[Lemma 1.1]{FG12}). For other related references, see \cite{Kod64,Kaw98,Amb05,Kol07,Flo14,FG14}.
    \item When $\Ff=T_X$ and $\Mm$ is NQC$/U$, previously we only knew the cases where either $B\geq 0$ at the generic point of $Z$ or $\Mm$ is $\bb$-semi-ample$/Z$ (\cite[Theorem 2.23]{JLX22}+\cite[Theorem 4.5]{HL21b}). We direct the reader to \cite{Fil19,Fil20,FS23} for other related references. It is worth noting that no results were known when $\Mm$ is not NQC$/U$.
    \item When $\Ff \neq T_X$, we only knew the cases where $f$ is a contraction and $\Mm$ is NQC$/U$ and $\bb$-semi-ample$/Z$ (\cite[Proposition 6.14]{LLM23}).
\end{enumerate}
In this paper, we not only prove Theorem \ref{thm: cbf gfq} in full generality but also clarify why \cite{Kol07} was able to address the horizontal negative coefficients, while \cite{Fil19,Fil20,JLX22,FS23} cannot deal with this issue. Further details on this are provided in Remark \ref{rem: lc trivial fibration definition}. Following this discussion, we refine the definition of lc-trivial fibrations and lc-trivial morphisms, which are elaborated in Definition \ref{defn: lc trivial fibration gfq}.

Finally, we note that the proof of Theorem \ref{thm: cbf gfq} does not depend on the mixed Hodge structure, as opposed to what is done in \cite{Kol07}. Remark \ref{rem: lc trivial fibration definition} also explains why the mixed Hodge structure cannot be applied to our case. Instead, our approach is based on the structure of algebraically integrable foliations, a method similar to the one used in \cite[Proof of Theorem 1.3]{ACSS21}. However, the canonical bundle formula in that reference is not complete as the ``BP stable" condition is required. Moreover, \cite[Theorem 1.3]{ACSS21} also has the additional requirement that $B\geq 0$ over the generic point of $Z$, a condition we aim to avoid.

\subsection{Singularities of algebraically integrable generalized foliated quadruples}

The cone theorem and the canonical bundle formula are primarily concerned with understanding the global behavior of algebraically integrable generalized foliated quadruples. However, it is equally important to examine the local behavior, particularly the singularities of these structures. In this paper, we will concentrate on two key aspects that are tied to the singularity structure of generalized foliated quadruples: the precise adjunction formula and the ACC for lc thresholds.

\subsubsection{Adjunction formulas} \cite[Proposition 3.2]{ACSS21} proves the adjunction formula for algebraically integrable foliations provided that the ambient variety is $\Qq$-factorial and that the foliation is induced by a contraction. In this paper, we remove these two technical conditions and prove the adjunction formula for algebraically integrable foliations in full generality:

\begin{thm}\label{thm: fol adj intro}
    Let $(X,\Ff,B)$ be an foliated triple such that $\Ff$ is algebraically integrable. Let $S$ be a prime divisor on $X$, such that $\mult_SB=0$ if $S$ if $\Ff$-invariant and $\mult_SB=1$ otherwise.  Let $S^\nu$ be the normalization of $S$ and $\Ff_S$ the restricted foliation (see Definition \ref{defn: restricted foliation}) of $\Ff$ on $S^\nu$. Then
    $$K_{\Ff_S}+B_S=(K_{\Ff}+B)|_{S^\nu}$$
    for some $\Rr$-divisor $B_S\geq 0$. Moreover, if $(X,\Ff,B)$ is lc, then $(S^\nu,\Ff_S,B_S)$ is lc.
\end{thm}
We remark that \cite[Theorem 3.16]{CS23b} proves the adjunction formula to non-invariant divisors for any foliation with a minor requirement that the boundary has $\Qq$-coefficients. In particular, when $B$ has rational coefficients and $\mult_SB=1$, Theorem \ref{thm: fol adj intro} is implied by \cite[Theorem 3.16]{CS23b}.


Theorem \ref{thm: fol adj intro} can be extended to the category of algebraically integrable generalized foliated quadruples:

\begin{thm}\label{thm: lc adjunction foliation nonnqc}
  Let $(X,\Ff,B,\Mm)/U$ be a generalized foliated quadruple such that $\Ff$ is algebraically integrable. Let $S$ be a prime divisor on $X$, such that $\mult_SB=0$ if $S$ if $\Ff$-invariant, and $\mult_SB=1$ otherwise. Let $S^\nu$ be the normalization of $S$, $\Mm^S:=\Mm|_{S^\nu}$ (see Definition \ref{defn: restriction b divisor}), and $\Ff_S$ the restricted foliation of $\Ff$ on $S^\nu$. Then
  $$K_{\Ff_S}+B_S+\Mm^S_{S^\nu}:=(K_{\Ff}+B+\Mm_X)|_{S^\nu}$$
  for some $\Rr$-divisor $B_S\geq 0$. Moreover, if $(X,\Ff,B,\Mm)$ is lc, then $(S^\nu,\Ff_S,B_S,\Mm^S)$ is lc.
\end{thm}

In \cite[Theorem 1.6]{DLM23}, a precise adjunction formula was introduced for algebraically integrable foliated triples, playing a crucial role in proving the ACC for lc thresholds and the global ACC for algebraically integrable foliated triples. Building upon this concept, we formulate and establish a precise adjunction formula for algebraically integrable generalized foliated quadruples in this paper. We then apply it to prove the ACC for lc thresholds and the global ACC for algebraically integrable generalized foliated quadruples. We have the following theorem:

\begin{thm}\label{thm: dcc adjunction is dcc}
Let $\Ii\subset [0,+\infty)$ be a set of real numbers. Let $(X,\Ff,B,\Mm)/U$ be an lc generalized foliated quadruple, $S$ a prime divisor on $X$ with normalization $S^\nu$, such that $\mult_SB=0$ if $S$ is $\Ff$-invariant and $\mult_SB=1$ otherwise. Assume that the coefficients of $B$ belong to $\Ii$ and $\Mm$ is a $\Ii$-linear combination of nef$/U$ $\bb$-Cartier $\bb$-divisors. Let
 $$K_{\Ff_S}+B_S+\Mm^S_{S^\nu}:=(K_{\Ff}+B+\Mm_X)|_{S^\nu}$$
where $\Ff_S$ is the restricted foliation of $\Ff$ on $S^\nu$ and $\Mm^S=\Mm|_{S^\nu}$.

Then the coefficients of $B_S$ belong to $D(\Ii)$ (see Definition \ref{defn: derived set}). In particular, if $\Ii$ is a DCC set, then the coefficients of $B_S$ belong to a DCC set.
\end{thm}

We offer a more detailed version of Theorem \ref{thm: dcc adjunction is dcc} in Theorem \ref{thm: precise adj gfq}. Given its highly technical nature, we omit it from the introduction. 

\subsubsection{ACC and the global ACC} Theorem \ref{thm: dcc adjunction is dcc} leads to the proof of the ACC and the global ACC for algebraically integrable generalized foliated quadruples. These results were proven in \cite{HMX14} for usual pairs ($\Ff=T_X$ and $\Mm=\bm{0}$), in \cite{BZ16} for generalized pairs ($\Ff=T_X$), and in \cite{DLM23} for foliated triples ($\Mm=\bm{0}$).


\begin{thm}[ACC for lc thresholds for algebraically integrable generalized foliated quadruples]\label{thm: acc lct alg int gfq}
Let $r$ be a positive integer and $\Ii\subset [0,+\infty)$ a DCC set. Then there exists an ACC set $\Ii'$ depending only on $r$ and $\Ii$ satisfying the following. Let $(X,\Ff,B,\Mm)/X$ be an lc generalized foliated quadruple, such that 
\begin{enumerate}
    \item $\Ff$ is algebraically integrable of rank $r$, 
    \item the coefficients of $B$ belong to $\Ii$, and
    \item  $\Mm$ is a $\Ii$-linear combination of nef$/X$ $\bb$-Cartier $\bb$-divisors.
\end{enumerate}
Then the lc threshold
$$\lct(X,\Ff,B,\Mm;D,\Nn):=\sup\{t\mid t\geq 0, (X,\Ff,B+tD,\Mm+t\Nn)\text{ is lc}\}$$
is contained in $\Ii'$.
\end{thm}

\begin{thm}[Global ACC for algebraically integrable generalized foliated quadruples]\label{thm: global acc alg int gfq}
Let $r$ be a positive integer and $\Ii\subset [0,1]$ a DCC set. Then there exists a finite set $\Ii_0\subset\Ii$ depending only on $r$ and $\Ii$ satisfying the following. Let $(X,\Ff,B,\Mm)$ be a projective lc generalized foliated quadruple such that 
\begin{enumerate}
\item $\Ff$ is algebraically integrable of rank $r$, 
\item the coefficients of $B$ belong to $\Ii$,
\item  $\Mm=\sum\gamma_j\Mm_j$, where each $\gamma_j\in\Ii$ and each $\Mm_j$ is a nef $\bb$-Cartier $\bb$-divisor,
\item $\Mm_j\not\equiv\bm{0}$ if $\gamma_j\not=0$, and
\item $K_{\Ff}+B+\Mm_X\equiv 0$.
\end{enumerate}
Then the coefficients of $B$ belong to $\Ii_0$, and $\gamma_j\in\Ii_0$ for each $j$.
\end{thm}
The proof of Theorem \ref{thm: global acc alg int gfq} is harder than the proof of the global ACC for algebraically integrable foliated triples \cite[Theorem 1.2]{DLM23}. This is because our proof heavily relies on the existence of Mori fiber spaces (Theorem \ref{thm: eomfs}), unlike the proof in \cite[Theorem 1.2]{DLM23}.

As a straightforward corollary of Theorem \ref{thm: global acc alg int gfq}, we obtain the global ACC for rank one generalized foliated quadruples:

\begin{cor}[Global ACC for rank one generalized foliated quadruples]\label{cor: global acc rank 1 gfq}
Let $\Ii\subset [0,1]$ be a DCC set. Then there exists a finite set $\Ii_0\subset\Ii$ depending only on $\Ii$ satisfying the following. Let $(X,\Ff,B,\Mm)$ be an lc generalized foliated quadruple such that
\begin{enumerate}
\item $\rk\Ff=1$,
\item the coefficients of $B$ belong to $\Ii$, 
\item  $\Mm=\sum\gamma_j\Mm_j$, where each $\gamma_j\in\Ii$ and each $\Mm_j$ is a nef $\bb$-Cartier $\bb$-divisor,
\item $\Mm_j\not\equiv\bm{0}$ if $\gamma_j\not=0$, and
\item $K_{\Ff}+B+\Mm_X\equiv 0$.
\end{enumerate}
Then the coefficients of $B$ belong to $\Ii_0$, and $\gamma_j\in\Ii_0$ for each $j$.
\end{cor}

When $\Mm=\bm{0}$, Corollary \ref{cor: global acc rank 1 gfq} was proven in \cite[Corollary 1.3]{DLM23}.

\subsubsection{Uniform rational polytopes} As a direct consequence of Theorems \ref{thm: acc lct alg int gfq} and \ref{thm: global acc alg int gfq}, we establish the existence of uniform lc rational polytopes for algebraically integrable generalized foliated quadruples. Despite their complex nature, these polytopes are powerful tools in birational geometry. They are notably used in several applications on the ACC conjecture for minimal log discrepancies and the boundedness of complements. These polytopes are essential for the formal definition of KSBA moduli spaces \cite[6.27.3, Theorem 11.49]{Kol23}, and play a crucial role in proving the global ACC for foliated threefolds \cite{LMX23b}. In this paper, we prove the existence of uniform lc rational polytopes for algebraically integrable generalized foliated quadruples:

\begin{thm}\label{thm: uniform rational polytope gfq}
Let $r$ be a positive integer, $v_1^0,\dots,v_m^0,u_1^0,\dots,u_n^0$ positive real numbers, $\bm{v}_0:=(v_1^0,\dots,v_m^0)$, and $\bm{u}_0:=(u_1^0,\dots,u_n^0)$. Then there exists an open set $U\ni (\bm{v}_0,\bm{u}_0)$ of the rational envelope of $(\bm{v}_0,\bm{u}_0)$ in $\mathbb R^{m+n}$ depending only on $r$ and $\bm{v}_0$, $\bm{u}_0$ satisfying the following. Let $$\left(X,\Ff,B=\sum_{j=1}^mv_j^0B_j,\Mm=\sum_{k=1}^nu_k^0\Mm_k\right)\Bigg{/}X$$ be an lc generalized foliated quadruple, such that $\Ff$ is algebraically integrable, $\rk\Ff=r$, $B_j\geq 0$ are distinct Weil divisors, and $\Mm_k$ are nef$/X$ $\bb$-Cartier $\bb$-divisors. Then $$\left(X,\Ff,B=\sum_{j=1}^mv_jB_j,\sum_{k=1}^nu_k\Mm_k\right)$$ is lc for any $(v_1,\dots,v_m,u_1,\dots,u_n)\in U$.
\end{thm}


\subsection{Miscellaneous results on the minimal model program and foliations} We also prove several other interesting theorems that can be useful for further applications.

\subsubsection{Analogues of dlt models} We establish the existence of $(*)$-models and ACSS models for algebraically integrable generalized foliated quadruples (see Definitions \ref{defn: ACSS f-triple} and \ref{defn: acss model}). As detailed in \cite{ACSS21,CS23a,DLM23}, these models play the same role as dlt models in the classic MMP. Moreover, $\Qq$-factorial dlt algebraically integrable generalized foliated quadruples always satisfy the property ``ACSS" and the property $(*)$ (see Theorem \ref{thm: fdlt is acss}).

\begin{thm}[Existence of ACSS model]\label{thm:  ACSS model}
    Let $(X,\Ff,B,\Mm)/U$ be an lc generalized foliated quadruple. Assume that $\Ff$ is algebraically integrable. Then $(X,\Ff,B,\Mm)/U$ has an ACSS model which is also a $(*)$-model.
    
    In particular, there exists a birational morphism $h_Y: Y\rightarrow X$ and a contraction $f_Y: Y\rightarrow Z$ satisfying the following. Let $\Ff_Y:=h_Y^{-1}\Ff$ and
    $$K_{\Ff_Y}+B_Y+\Mm_Y=h_Y^*(K_{\Ff}+B+\Mm_X),$$
    then
    \begin{enumerate}
        \item  $(Y,B_Y,\Mm)$ is $\Qq$-factorial qdlt, 
        \item $f_Y$ is equi-dimensional and $\Ff_Y$ is induced by $f_Y$, and
        \item any prime $f$-exceptional divisor is an lc place of $(X,\Ff,B,\Mm)$.
    \end{enumerate}
\end{thm}

\subsubsection{Minimal model program for very exceptional divisors} When running the relative minimal model program, especially the birational minimal model program, we often encounter the minimal model program for very exceptional divisors \cite[Theorem 1.8]{Bir12}. In this paper, we establish the minimal model program for algebraically integrable generalized foliated quadruples whose generalized foliated log canonical divisor is very exceptional. 

\begin{thm}\label{thm: gfq mmp very exceptional intro}
    Let $(X,\Ff,B,\Mm)/U$ be a $\Qq$-factorial F-dlt generalized foliated quadruple and $E\geq 0$ and $\Rr$-divisor on $X$, such that $E$ is very exceptional$/U$ and
    $$K_{\Ff}+B+\Mm_X\sim_{\mathbb R,U}E.$$
    Then we may run a $(K_{\Ff}+B+\Mm_X)$-MMP$/U$ with scaling of an ample$/U$ $\Rr$-divisor, and any such MMP terminates with a good minimal model $(X',\Ff',B',\Mm)/U$ of $(X,\Ff,B,\Mm)/U$, such that $K_{\Ff'}+B'+\Mm_{X'}\sim_{\mathbb R,U}0$. 
\end{thm}
Theorem \ref{thm: gfq mmp very exceptional intro} is vital for proving that $\Qq$-factorial dlt implies ACSS (Theorem \ref{thm: fdlt is acss}). This is essential for proving Theorem \ref{thm: main mmp foliation}.

\subsubsection{A special case of Prokhorov-Shokurov's base-point-freeness conjecture}
Prokhorov-Shokurov's base-point-freeness conjecture \cite[Conjecture 7.13]{PS09} is a major conjecture in birational geometry. It has been verified when the relative dimension of the fibration is $1$ (\cite[Theorem 8.1]{PS09}) or $2$ (\cite[Theorem 1.4]{ABBDILW23}). However, for the relative dimension of the fibration $\geq 3$, the conjecture is still largely open. Through the application of foliation theory, we prove a special case of the Prokhorov-Shokurov base-point-freeness conjecture:
\begin{thm}\label{thm: ps intro}
Let $d$ and $m$ be two positive integers. Then there exists a positive integer $I$ depending only on $d$ and $I$ satisfying the following.
    
Let $(X,B)$ be a projective klt pair, and $f: X\rightarrow Z$ a contraction to a smooth variety $Z$. Let $B^h$ and $B^v$ be the horizontal$/Z$ part of $B$ and the vertical$/Z$ part of $B$ respectively, and let $\Mm$ be the moduli part of $f: (X,B)\rightarrow Z$. Assume that the following conditions hold.
    \begin{enumerate}
        \item (Semi-stability) $(X,B)$ is BP semi-stable$/Z$.
        \item (Klt-trivial) $K_X+B\sim_{\mathbb Q,Z}0$.
        \item (Coefficient control) $mB$ is a Weil divisor.
        \item (Fano type) There exists an ample $\Rr$-divisor $H$ such that $B^h\geq H\geq 0$.
        \item (Snc condition) There exists a reduced divisor $\Sigma_Z$ on $Z$, such that $B^v=f^{-1}(\Sigma_Z)$, $(Z,\Sigma_Z)$ is log smooth, and for any reduced divisor $H\geq 0$ such that $(Z,\Sigma+H)$ is log smooth, $(X,B+f^*H)$ is lc.
    \end{enumerate}
Then $\Mm$ descends to $X$, $I\Mm_X$ is Cartier, and $nI\Mm_X$ is base-point-free for any integer $n\gg 0$.
\end{thm}
While the conditions of Theorem \ref{thm: ps intro} are quite restrictive, mainly because condition (4) is not preserved under birational transformations, there are no requirements on the dimension of the varieties or the relative dimension of $f$. This makes the theorem potentially useful for future applications.

\subsection{Why should we care about generalized foliated quadruples?}\label{sec: reason to consider gfq}

Before we move on to the main part of the paper, we would like to briefly explain why we need to consider the structure of generalized foliated quadruples and why it is essential for us to prove some main theorems of the paper, even if these theorems' statements do not explicitly mention this structure. To clarify this, we present five scenarios where generalized foliated quadruples play a crucial role. Four out of these five scenarios are unavoidable in the proofs of this paper.

\begin{sce}[Canonical bundle formula of foliations]
\cite[Theorem 1.2]{LLM23} established the canonical bundle formula for foliations. More precisely, given a projective lc foliated triple $(X,\Ff,B)$ and a contraction $f: X\rightarrow Z$ such that the general fibers of $f$ are tangent to $\Ff$ and $K_{\Ff}+B\sim_{\mathbb R,Z}0$, we have
$$K_{\Ff}+B\sim_{\mathbb R}f^*(K_{\Ff_Z}+B_Z+\Mm_Z)$$
where $(Z,\Ff_Z,B_Z,\Mm)$ is a projective lc generalized foliated quadruple. Therefore, if we want to study the behavior of $(X,\Ff,B)$, then it is necessary to study the structure of $(Z,\Ff_Z,B_Z,\Mm)$. When $\Ff=T_X$ and $(X,B)$ is klt, the classical approach is to find an $\Rr$-divisor
$$0\leq\Delta_Z\sim_{\mathbb R}B_Z+\Mm_Z$$
such that $(Z,\Delta_Z)$ is klt \cite[Theorem 0.2]{Amb05} and use the structure of $(Z,\Delta_Z)$ instead of $(Z,B_Z,\Mm)$. This approach is essential for the proof of the finite generation of the canonical ring (\cite[Proof of Corollary 1.1.2]{BCHM10}, \cite{FM00}). 

However, when $\Ff\not=T_X$, it is generally not possible for us to combine $B_Z$ and $\Mm_Z$ and get an lc triple structure $(Z,\Ff_Z,\Delta_Z\sim_{\mathbb R}B_Z+\Mm_Z)$. This is because of the following two reasons:
\begin{enumerate}
    \item ``Klt" is almost an empty condition for foliations when $\Ff\not=T_X$. Actually, unless the foliation is purely transcendental, there are always (infitely many) lc centers of $(X,\Ff,B)$ as long as $\Ff\not=T_X$. This will cause trouble when we try to perturb the coefficients of $B_Z+\Mm_Z$ and get $\Delta_Z$. In fact, even when $\Ff=T_X$, if $(X,B)$ is not klt, then we do not know whether there exists such $\Delta_Z$ so that $(Z,\Delta_Z)$ is lc, and we usually need the $\bb$-semi-ampleness of $\Mm$ to show this fact. The $\bb$-semi-ampleness of $\Mm$, on the other hand, is the Prokhorov-Shokurov conjecture \cite[Conjecture 7.13]{PS09} as mentioned above, which is widely open when $\dim X-\dim Z\geq 3$. Indeed, the unconfirmed status of the Prokhorov-Shokurov conjecture is one key reason why Birkar-Zhang introduced the concept of generalized pairs in \cite{BZ16}.
    \item Even if $\Mm$ is semi-ample, the existence of such $\Delta_Z$ is also unknown as Bertini type theorems fail for foliations in general, even for surfaces (cf. \cite[Example 3.4]{DLM23}). In other words, it is possible that $(Z,\Ff_Z,B_Z+G_Z)$ is not lc for any $G_Z\in |\Mm_Z|_{\mathbb R}$. 
\end{enumerate}
Therefore, in many situations, we must analyze the structure of generalized foliated quadruples rather than foliated triples. Furthermore, due to (2), the concept of generalized foliated quadruples becomes essential for studying foliations, even in lower dimensions. This is a key reason why \cite{LLM23,LMX23b} rely on the theory of generalized foliated quadruples to establish the global ACC for foliated triples in dimension $3$.
\end{sce}

\begin{sce}[MMP with scaling]
We recall how we run the minimal model program for with scaling for usual pairs. For simplicity, we only consider the projective case. Given a projective lc pair $(X,B)$ and an  $\Rr$-divisor $A\geq 0$ on $X$ such that $K_X+B+A$ is nef, we consider the scaling numbers
$$\lambda:=\inf\{t\mid t\geq 0, K_X+B+tA\text{ is nef}\}.$$
If $t=0$ then we are done. Otherwise, we contract a $(K_X+B)$-negative extremal ray $R$ such that $(K_X+B+tA)\cdot R=0$, and let $f: (X,B)\dashrightarrow (X',B')$ be a corresponding divisorial contraction, flip, or Mori fiber space associated to the contraction of $R$. We may replace $(X,B)$ with $(X',B')$ and $A$ with $A'$ and continue this process.

Although we only need $K_X+B+A$ to be nef to run the MMP, in practice, we usually also need the additional condition that $(X,B+A)$ is lc. This is helpful in many situations: since we do not know the termination of the MMP, it is likely for us to consider pairs $(X,B+\mu A)$ where $\mu$ is related to the scaling numbers $\lambda$. In this case, we usually need $(X,B+A)$ to be lc in order to guarantee that $(X,B+\mu A)$ is lc. For this reason, for the very first step of the MMP, we usually require $A$ to be a general ample $\Rr$-divisor, or a general base-point-free big and nef $\Rr$-divisor when $(X,B)$ is klt.

Now we consider the minimal model program for foliations. We definitely want to consider the minimal model program with scaling of ample divisors as well. However, as we have explained above, Bertini type theorems fail for foliations in general, even for surfaces (cf. \cite[Example 3.4]{DLM23}). Therefore, it is possible that for any ample $\Rr$-divisor $A\geq 0$ on $X$, $(X,\Ff,B+A)$ is not lc. Now the minimal model program of $(X,\Ff,B)$ with scaling of $A$ becomes weird: we can still run the minimal model program, but it will become difficult to study the intermediate outputs $(X',B'+\lambda A')$ with $\lambda>0$ after each step of the MMP, where $\lambda$ is the scaling number. This causes a lot of inconvenience for the minimal model program of foliations.

The structure of generalized foliated quadruples, however, can easily resolve this issue: if we identify $(X,\Ff,B)$ with the generalized foliated quadruple $(X,\Ff,B,\overline{0})$, then instead of running an MMP with scaling of an ample $\Rr$-divisor $A$, we can let $\Aa:=\overline{A}$ be the nef $\bb$-divisor associated to $A$. Now may run an MMP with scaling of $(0,\Aa)$. That is, although we still consider
$$\lambda:=\inf\{t\mid t\geq 0, K_{\Ff}+B+tA\text{ is nef}\},$$
the output of the first step of the MMP $\phi: X\dashrightarrow X'$ becomes $$(X',\Ff'=\phi_*\Ff,B'=\phi_*B,\lambda\Aa)$$ which is still an lc generalized foliated quadruple. Therefore, by using the theory of generalized foliated quadruples, we can bypass the failure of Bertini type theorems of foliations straightforwardly.
\end{sce}

\begin{sce}[Minimal model program on K\"ahler varieties]\label{sce: kahler}
It is well-known that foliations, especially algebraically integrable foliations, has a tight connection with the minimal model program for K\"ahler varieties. As we have mentioned above, Das and Ou essentially use the structure of algebraically integrable foliations to prove the abundance conjecture for K\"ahler manifolds in dimension $3$ \cite{DO23a,DO23b}. There is no doubt that foliations are expected to be useful in the study of K\"ahler minimal model program in the future.

On the other hand, generalized pairs is also known to have a tight connection with the minimal model program for K\"ahler varieties \cite{DH23,DHY23}. The key reason is due to the minimal model program with scaling of K\"ahler classes. K\"ahler classes cannot be considered as divisors, but by considering K\"ahler classes as $\bb$-nef classes and move it to the nef part of the generalized pair, we can formally define the minimal model program with scaling of K\"ahler classes.

In summary, the study of K\"ahler varieties seems to be a natural place for foliations and generalized pairs to get mixed together. We therefore can expect the structure of generalized foliated quadruples, particularly the algebraically integrable ones, to play a crucial role in the study of K\"ahler varieties in the future. 
\end{sce}


\begin{sce}[Cone theorem and semi-ampleness theorem]
In \cite[Theorem 3.9]{ACSS21}, a version of the cone theorem for algebraically integrable foliated triples $(X,\Ff,B)$ is proved. When $(X,\Ff,B)$ is lc, the main part of the cone theorem, i.e. the formula
$$\overline{NE}(X)=\overline{NE}(X)_{K_{\Ff}+B\geq 0}+\sum R_j$$
was proved in  \cite[Theorem 3.9]{ACSS21}. However, \cite[Theorem 3.9]{ACSS21} did not prove the countableness of $R_j$ nor the finiteness of $R_j$ when polarizing $(X,\Ff,B)$ with an ample divisor $A$. That is, the formula
$$\overline{NE}(X)=\overline{NE}(X)_{K_{\Ff}+B+A\geq 0}+\sum_{\text{finite}} R_j$$
is missing. One key reason for this seems to be the issue that $(X,\Ff,B+A)$ may no longer be lc, and this is, again, due to the failure of the Bertini type theorems for foliations. However, if we consider $(X,\Ff,B,\bar A)$ instead of $(X,\Ff,B+A)$, then $(X,\Ff,B,\bar A)$ becomes an lc generalized pair and we have immediately have more flexibility.

Similar issues appear when we consider the semi-ampleness theorem for foliations. For usual pairs, the semi-ampleness theorem is usually formulated in the following way:
$$(X,B) \text{ lc}, A\text{ ample}, K_X+B+A\text{ nef}\Rightarrow K_X+B+A\text{ semi-ample}.$$
However, for foliations, the semi-ampleness theorem is usually formulated in the following way: (under suitable conditions)
$$(X,\Ff,B+A) \text{ lc}, B\geq 0, A\geq 0\text{ ample}, K_{\Ff}+B+A\text{ nef}\Rightarrow K_{\Ff}+B+A\text{ semi-ample}.$$
This is again due to the failure of Bertini-type theorems. Nevertheless, with the new concept of generalized foliated quadruples, these arguments can now be strengthened back to:  (under suitable conditions)
$$(X,\Ff,B) \text{ lc}, A\text{ ample}, K_{\Ff}+B+A\text{ nef}\Rightarrow K_{\Ff}+B+A\text{ semi-ample}.$$
\end{sce}

\begin{sce}[Canonical bundle formula for generalized pairs]
The final scenario where the structure of generalized foliated quadruples plays a vital role is in obtaining the canonical bundle formula for generalized pairs. To establish this formula for lc-trivial fibrations in cases involving either non-NQC generalized pairs or NQC generalized pairs with potentially negative coefficients, we cannot rely on the structure of mixed Hodge structure (see Remark \ref{rem: lc trivial fibration definition}). Filipazzi's approach \cite{Fil19,Fil20} is also unsuitable due to its requirements for $\Qq$-coefficients and its inability to handle negative coefficients. Therefore, the only viable approach to achieve such a canonical bundle formula is by utilizing the theory of foliations as in  \cite{ACSS21}. Now, since we are dealing with generalized pairs in this context, the introduction of generalized foliated quadruples becomes necessary. For more details, we refer the reader to the proof of Theorem \ref{thm: cbf gpair nonnqc}.
\end{sce}

\part{Preliminaries}\label{part:prelim}

\section{Basic definitions}\label{sec: preliminaries}

 Throughout the paper, we will mainly work with normal quasi-projective varieties to ensure consistency with the references. However, most results should also hold for normal varieties that are not necessarily quasi-projective. Similarly, most results in our paper should hold for any algebraically closed field of characteristic zero. We will adopt the standard notations and definitions in \cite{KM98, BCHM10} and use them freely. For foliations, we will generally follow the notations and definitions in \cite{CS20,ACSS21,CS21}, but there may be minor differences. For generalized pairs, we will follow the notations and definitions in \cite{HL21a}.

\subsection{Special notations}

\begin{nota}
    In this paper, $\mathbb N$ stands for the set of non-negative integers and $\mathbb N^+$ stands for the set of positive integers. 
\end{nota}

\begin{nota}
In this paper, the notation ``$/$" is always considered as a simplified writing of ``over". For example, ``$/Z$" means ``over $Z$".
\end{nota}

\begin{nota}\label{nota: general r divisor}
    Let $X\rightarrow U$ be a projective morphism from a normal variety to a variety, and let $A$ be a semi-ample$/U$ $\Rr$-divisor on $X$. An $\Rr$-divisor $H$ on $X$ is said to be \emph{general} in $|A/U|_{\mathbb R}$ if there exist base-point-free$/U$ divisors $A_1,\dots,A_n$ and real numbers $r_1,\dots,r_n\in (0,1)$, such that $A=\sum_{i=1}^nr_iA_i$ and $H=\sum_{i=1}^nr_iH_i$, where $H_i\in |A_i/U|$ are general elements. A \emph{general ample$/U$ $\Rr$-divisor} on $X$ is an ample$/U$ $\Rr$-divisor $D$ on $X$ such that $D$ is general in $|D/U|_{\mathbb R}$.
\end{nota}

\begin{nota}
    Let $\Ii$ be a set of real numbers, $X$ a normal variety, and $B$ an $\Rr$-divisor on $X$. We write $B\in\Ii$ if the coefficients of $B$ belong to $\Ii$.
\end{nota}

\begin{nota}
    Let $\pi: X\rightarrow U$ be a projective morphism between varieties and $D$ an $\Rr$-divisor on $X$. We denote by $\kappa_{\sigma}(X/Z,D)$ (resp. $\kappa_{\iota}(X/Z,D)$, $\kappa(X/Z,D)$) the relative numerical dimension (resp. relative invariant Iitaka dimension, relative Iitaka dimension) of $D$ over $Z$. When $Z=\{pt\}$, we may drop $X/Z$ and use the notation $\kappa_{\sigma}(D)$ (resp. $\kappa_{\iota}(D)$, $\kappa(D)$) instead. We refer the reader to \cite[Section 2]{HH20} for the formal definitions and basic properties of $\kappa_{\sigma}(X/Z,D)$, $\kappa(X/Z,D)$, and $\kappa(X/Z,D)$.
\end{nota}

\begin{defn}[Log big]\label{defn: log big}
Let $(X,B,\Mm)/U$ be a g-pair and $D$ an $\Rr$-Cartier $\Rr$-divisor $D$ on $X$. We say that $D$ is \emph{log big$/U$ with respect to $(X,B,\Mm)$} if $D|_V$ is big$/U$ for any lc center $V$ of $(X,B,\Mm)$. In particular, $D$ is big$/U$.
\end{defn}

\begin{defn}[{cf. \cite[Definition 5.3]{Amb03}, \cite[Definition 6.7.2]{Fuj11}}]\label{defn: basics of cone theorem}
Let $(X,\Delta)$ be a (not necessarily lc) pair and $\pi: X\rightarrow U$ a projective morphism. Let $F$ be an extremal face of $\overline{NE}(X/U)$.
\begin{enumerate}
    \item A \emph{supporting function} of $F$ is a  $\pi$-nef $\Rr$-divisor $H$ such that $F=\overline{NE}(X/U)\cap H^{\bot}$. If $H$ is a $\Qq$-divisor, we say that $H$ is a \emph{rational supporting function}. Since $F$ is an extremal face of $\overline{NE}(X/U)$, $F$ always has a supporting function.
    \item We say that $F$ is \emph{rational} if $F$ has a rational supporting function.
    \item For any $\Rr$-Cartier $\Rr$-divisor $D$ on $X$, we say that $F$ is $D$-\emph{negative} if $$F\cap\overline{NE}(X/U)_{D\geq 0}=\{0\}.$$
    \item We say that $F$ is \emph{relatively ample at infinity with respect to} $(X,\Delta)$  if $$F\cap\overline{NE}(X/U)_{\Nlc(X,\Delta)}=\{0\}.$$ Equivalently, $H|_{\Nlc(X,\Delta)}$ is $\pi|_{\Nlc(X,\Delta)}$-ample for any supporting function $H$ of $F$.
    \item We say that $F$ is \emph{contractible at infinity with respect to} $(X,\Delta)$ if $F$ has a rational supporting function $H$ and $H|_{\Nlc(X,\Delta)}$ is $\pi|_{\Nlc(X,\Delta)}$-semi-ample.
\end{enumerate}
\end{defn}

\begin{deflem}\label{deflem: exposed ray}
Let $K$ be a convex cone containing no lines. A ray $R$ of $K$ is called \emph{exposed} if there is a hyperplane meeting $K$ exactly along $R$. In particular, any exposed ray of $K$ is extremal in $K$. If $K$ does not contain any line, then $K$ is the closure of the subcone of $K$ spanned by exposed rays (\cite[Corollary 18.7.1]{Roc97}, \cite[Lemma 6.2]{Spi20}).

Let $\pi: X\rightarrow U$ be a projective morphism from a normal quasi-projective variety to a variety. By definition, an extremal ray in $\overline{NE}(X/U)$ is exposed if and only if it has a supporting function (that is not necessarily rational). Moreover, for any sub-cone $V$ of $\overline{NE}(X/U)$, we have
$$\overline{NE}(X/U)=\overline{V+\sum R_i}$$
where $R_i$ are exposed rays that are not contained in $V$.
\end{deflem}




\subsection{Sets}

\begin{defn}\label{defn: DCC and ACC}
Let $\Ii\subset\Rr$ be a set. We say that $\Ii$ satisfies the \emph{descending chain condition} (DCC) if any decreasing sequence in $\Ii$ stabilizes, and $\Ii$ satisfies the \emph{ascending chain condition} (ACC) if any increasing sequence in $\Ii$ stabilizes. 
\end{defn}

\begin{defn}\label{defn: derived set}
    Let $\Ii\subset [0,+\infty)$ be a set. We define 
    $$\Ii_+:=\{0\}\cup\left\{\sum_{i=1}^l\gamma_i\bigg| \gamma_i\in\Ii,l\in\mathbb N^+\right\}\text{ and }D(\Ii):=\left\{\frac{m-1+\gamma}{m}\bigg|\gamma\in\Ii_+,m\in\mathbb N^+\right\}.$$
\end{defn}



\subsection{Foliations} In this subsection, we define foliations and some of its related concepts. For preliminaries regarding algebraically integrable foliations, we refer the reader to Subsection \ref{subsec: ai foliation}.

\begin{defn}[Foliations, {cf. \cite[Section 2.1]{CS21}}]\label{defn: foliation}
Let $X$ be a normal variety. A \emph{foliation} on $X$ is a coherent sheaf $\Ff\subset T_X$ such that
\begin{enumerate}
    \item $\Ff$ is saturated in $T_X$, i.e. $T_X/\Ff$ is torsion free, and
    \item $\Ff$ is closed under the Lie bracket.
\end{enumerate}
The \emph{rank} of the foliation $\Ff$ is the rank of $\Ff$ as a sheaf and is denoted by $\rk\Ff$. The \emph{co-rank} of $\Ff$ is $\dim X-\rk\Ff$. The \emph{canonical divisor} of $\Ff$ is a divisor $K_\Ff$ such that $\mathcal{O}_X(-K_{\mathcal{F}})\cong\mathrm{det}(\Ff)$. We define $N_{\Ff}:=(T_X/\Ff)^{\vee\vee}$ and $N_{\Ff}^*:=N_{\Ff}^{\vee}$.

If $\Ff=0$, then we say that $\Ff$ is a \emph{foliation by points}.
\end{defn}

\begin{defn}[Singular locus]
     Let $X$ be a normal variety and $\Ff$ a rank $r$ foliation on $X$. We can associate to $\Ff$ a morphism $$\phi: \Omega_X^{[r]}\to \mathcal{O}_X(K_{\Ff})$$ defined by taking the double dual of the $r$-wedge product of the map $\Omega^1_X\to \Ff^*$, induced by the inclusion $\Ff\to T_X$. This yields a map $$\phi': (\Omega_X^{[r]}\otimes\mathcal{O}_X(-K_{\Ff}))^{\vee\vee}\to \mathcal{O}_X$$ and we define the singular locus, denoted as $\Sing \Ff$, to be the co-support of the image of $\phi'$.
\end{defn}

\begin{defn}[Pullbacks and pushforwards, {cf. \cite[3.1]{ACSS21}}]\label{defn: pullback}
Let $X$ be a normal variety, $\Ff$ a foliation on $X$, $f: Y\dashrightarrow X$ a dominant map, and $g: X\dashrightarrow X'$ a birational map. We denote $f^{-1}\Ff$ the \emph{pullback} of $\Ff$ on $Y$ as constructed in \cite[3.2]{Dru21}. We also say that $f^{-1}\Ff$ is the \emph{induced foliation} of $\Ff$ on $Y$. If $\Ff=0$, then we say $f^{-1}\Ff$ is induced by $f$. In this case, we say $f^{-1}\Ff$ is \emph{algebraically integrable}.

We define the \emph{pushforward} of $\Ff$ on $X'$ as $(g^{-1})^{-1}\Ff$ and denote it by $g_*\Ff$.
\end{defn}




\begin{defn}[Invariant subvarieties, {cf. \cite[3.1]{ACSS21}}]\label{defn: f-invariant}
Let $X$ be a normal variety, $\Ff$ a foliation on $X$, and $S\subset X$ a subvariety. We say that $S$ is \emph{$\Ff$-invariant} if and only if for any open subset $U\subset X$ and any section $\partial\in H^0(U,\Ff)$, we have $$\partial(\mathcal{I}_{S\cap U})\subset \mathcal{I}_{S\cap U}$$ 
where $\mathcal{I}_{S\cap U}$ is the ideal sheaf of $S\cap U$. Note that if $\Ff$ is the foliation induced by a dominant map $f:X\dashrightarrow Z$, then a divisor $D$ is $\Ff$-invariant if and only if $D$ is vertical with respect to $f$.
\end{defn}



\begin{defn}[Special divisors on foliations, cf. {\cite[Definition 2.2]{CS21}}]\label{defn: special divisors on foliations}
Let $X$ be a normal variety and $\Ff$ a foliation on $X$. For any prime divisor $C$ on $X$, we define $\epsilon_{\Ff}(C):=1$ if $C$ is not $\Ff$-invariant, and  $\epsilon_{\Ff}(C):=0$ if $C$ is $\Ff$-invariant. If $\Ff$ is clear from the context, then we may use $\epsilon(C)$ instead of $\epsilon_{\Ff}(C)$. For any $\Rr$-divisor $D$ on $X$, we define $$D^{\Ff}:=\sum_{C\mid C\text{ is a component of }D}\epsilon_{\Ff}(C)C.$$
Let $E$ be a prime divisor over $X$ and $f: Y\rightarrow X$ a projective birational morphism such that $E$ is on $Y$. We define $\epsilon_{\Ff}(E):=\epsilon_{f^{-1}\Ff}(E)$. It is clear that $\epsilon_{\Ff}(E)$ is independent of the choice of $f$.
\end{defn}

\subsection{Polarized foliations} In this subsection we introduce the concept of generalized foliated quadruples, which was originally introduced by the third author, Luo, and Meng in their study of the global ACC for foliated threefolds \cite{LLM23}.  The category of generalized foliated quadruples is a larger category comparing to generalized pairs, foliated triples, and usual pairs, so we shall not formally define the latter three concepts and only consider them as special generalized foliated quadruples. Since this is a very technical definition and we do not need its full power for some parts of the paper (e.g. we only need the concept of generalized pairs in Sections \ref{sec: basic property gpair} and \ref{sec: stability gpair}), for the reader's convenience, we refer the reader to \cite{KM98,BCHM10} for the definition of pairs and \cite{BZ16,HL21a} for the definition of generalized pairs. We refer the reader to \cite{CS20,CS21} for the definition of foliated pairs $(\Ff,B)$; a foliated pair $(\Ff,B)$ together with its ambient variety $X$ is a foliated triple $(X,\Ff,B)$.


\begin{defn}[$\bb$-divisors]\label{defn: b divisors} Let $X$ be a normal quasi-projective variety. We call $Y$ a \emph{birational model} over $X$ if there exists a projective birational morphism $Y\to X$. 

Let $X\dashrightarrow X'$ be a birational map. For any valuation $\nu$ over $X$, we define $\nu_{X'}$ to be the center of $\nu$ on $X'$. A \emph{$\bb$-divisor} $\Dd$ on $X$ is a formal sum $\Dd=\sum_{\nu} r_{\nu}\nu$ where $\nu$ are valuations over $X$ and $r_{\nu}\in\mathbb R$, such that $\nu_X$ is not a divisor except for finitely many $\nu$. If in addition, $r_{\nu}\in\Qq$ for every $\nu$, then $\Dd$ is called a \emph{$\Qq$-$\bb$-divisor}. The \emph{trace} of $\Dd$ on $X'$ is the $\Rr$-divisor
$$\Dd_{X'}:=\sum_{\nu_{X'}\text{ is a divisor}}r_\nu\nu_{X'}.$$
If $\Dd_{X'}$ is $\Rr$-Cartier and $\Dd_{Y}$ is the pullback of $\Dd_{X'}$ on $Y$ for any birational model $Y$ over $X'$, we say that $\Dd$ \emph{descends} to $X'$ and $\Dd$ is the \emph{closure} of $\Dd_{X'}$, and write $\Dd=\overline{\Dd_{X'}}$. 

Let $X\rightarrow U$ be a projective morphism and assume that $\Dd$ is a $\bb$-divisor on $X$ such that $\Dd$ descends to some birational model $Y$ over $X$. If $\Dd_Y$ is nef$/U$ (resp. base-point-free$/U$, semi-ample$/U$), then we say that $\Dd$ is \emph{nef}$/U$ (resp. \emph{base-point-free}$/U$, \emph{semi-ample}$/U$). If $\Dd_Y$ is a Cartier divisor, then we say that $\Dd$ is \emph{$\bb$-Cartier}. If $\Dd_Y$ is a $\Qq$-Cartier $\Qq$-divisor, then we say that $\Dd$ is \emph{$\Qq$-$\bb$-Cartier}. If $\Dd$ can be written as an $\Rr_{\geq 0}$-linear combination of nef$/U$ $\bb$-Cartier $\bb$-divisors, then we say that $\Dd$ is \emph{NQC}$/U$.

Let $X\rightarrow U$ be a projective morphism and assume that $\Dd$ and $\Dd'$ are two $\bb$-divisors over $X$. We write $\Dd\sim_{\mathbb R,U}\Dd'$ (resp. $\Dd\sim_{\mathbb Q,U}\Dd',\Dd\equiv_{\mathbb Q,U}\Dd'$) if for any birational model $Y$ of $X$, $\Dd_Y\sim_{\mathbb R,U}\Dd'_Y$ (resp. $\Dd_Y\sim_{\mathbb Q,U}\Dd'_Y,\Dd_Y\equiv_{\mathbb Q,U}\Dd_Y'$). 

We let $\bm{0}$ be the $\bb$-divisor $\bar{0}$.
\end{defn}

\begin{defn}\label{defn: restriction b divisor}
We will use two types of restrictions of $\bb$-divisors in this paper. Let $X$ be a normal variety and $\Dd$ a $\bb$-divisor on $X$.
\begin{enumerate}
    \item Let $V$ be a non-empty subset of $X$.  We define the \emph{restricted $\bb$-divisor} of $\Dd$ on $V$, which is denoted by $\Dd|_{V}$, in the following way. 

    For any birational morphism $\pi: W\to V$, there exists a birational morphism $\pi': Y\rightarrow X$ such that $W\subset Y$ and $\pi'|_W=\pi$. We let $(\Dd|_V)_{W}=(\Dd_Y)|_W$. It is easy to see that this definition is independent of the choice of $Y$ and defines a $\bb$-divisor.
    \item Suppose that $\Dd$ descends to a birational model of $X$. Let $S$ be a prime divisor on $X$ and $\nu: S^\nu\rightarrow S$ the normalization of $S$. The \emph{restricted $\bb$-divisor} of $\Dd$ on $S^\nu$, which is denoted by $\Dd|_{S^\nu}$, is defined in the following way. 

     Let $f: Y\rightarrow X$ be a log resolution of $(X,S)$ such that $\Dd$ descends to $Y$. Let $S_Y:=f^{-1}_*S$. Then there exists an induced birational morphism $f_S: S_Y\rightarrow S^\nu$ such that $\nu\circ f_S=f|_{S_Y}$.
     We define $$\Dd|_{S^\nu}:=\overline{\Dd_Y|_{S_Y}}.$$
     It is clear that $\Dd|_{S^\nu}$ is well-defined and is independent of the choice of $Y$.
\end{enumerate}
\end{defn}

\begin{defn}[Generalized foliated quadruples]\label{defn: gfq}
A \emph{generalized foliated sub-quadruple} (\emph{sub-gfq} for short) $(X,\Ff,B,\Mm)/U$ consists of a normal quasi-projective variety $X$, a foliation $\Ff$ on $X$, an $\Rr$-divisor $B$ on $X$, a projective morphism $X\rightarrow U$, and a nef$/U$ $\bb$-divisor $\Mm$ over $X$, such that $K_{\Ff}+B+\Mm_X$ is $\mathbb R$-Cartier. If $\Mm$ is NQC$/U$, then we say that $(X,\Ff,B,\Mm)/U$ is \emph{NQC}. If $B\geq 0$, then we say that $(X,\Ff,B,\Mm)/U$ is a \emph{generalized foliated quadruple} (\emph{gfq} for short). If $U=\{pt\}$, we usually drop $U$ and say that $(X,\Ff,B,\Mm)$ is \emph{projective}. 

Let $(X,\Ff,B,\Mm)/U$ be a (sub-)gfq. If $\Mm=\bm{0}$, then we may denote $(X,\Ff,B,\Mm)/U$ by $(X,\Ff,B)/U$ or $(X,\Ff,B)$, and say that $(X,\Ff,B)$ is a \emph{foliated (sub-)triple} (\emph{f-(sub-)triple} for short). If $\Ff=T_X$, then we may denote $(X,\Ff,B,\Mm)/U$ by $(X,B,\Mm)/U$, and say that $(X,B,\Mm)/U$ is a \emph{generalized (sub-)pair} (\emph{g-(sub-)pair} for short). If $\Mm=\bm{0}$ and $\Ff=T_X$, then we may denote $(X,\Ff,B,\Mm)/U$ by $(X,B)/U$ or $(X,B)$, and say that $(X,B)$ is a \emph{(sub-)pair}. 

A (sub-)gfq (resp. f-(sub-)triple, f-(sub-)pair, g-(sub-)pair, (sub-)pair) $(X,\Ff,B,\Mm)/U$ (resp. $(X,\Ff,B)/U$,$(X,B,\Mm)/U$, $(X,B)/U$) is called a \emph{$\mathbb Q$-(sub-)gfq} (resp. \emph{$\mathbb Q$-f-(sub-)triple, $\mathbb Q$-g-(sub-)pair, $\mathbb Q$-(sub-)pair} if $B$ is a $\mathbb Q$-divisor and $\Mm$ is a $\mathbb Q$-$\bb$-divisor.

It is worth mentioning that our definition of generalized foliated quadruples slightly differs from \cite[Definition 1.2]{LLM23}, as the latter requires $\Mm$ to be NQC$/U$ (see Definition \ref{defn: b divisors}), while we only require it to be nef$/U$.  This will be crucial for us to use this structure to prove Theorem \ref{thm: main mmp gpair}.
\end{defn}

\begin{nota}
In the previous definition, if $U$ is not important, we may also drop $U$. This usually happens when we emphasize the structures of $(X,\Ff,B,\Mm)$ which are independent of the choice of $U$, such as the singularities of $(X,\Ff,B,\Mm)$. In addition, if $B=0$, then we may drop $B$.
\end{nota}



\begin{defn}[Singularities of gfqs]\label{defn: gfq singularity}
Let $(X,\Ff,B,\Mm)$ be a (sub-)gfq. For any prime divisor $E$ over $X$, let $f: Y\rightarrow X$ be a birational morphism such that $E$ is on $Y$, and suppose that
$$K_{\Ff_Y}+B_Y+\Mm_Y:=f^*(K_\Ff+B+\Mm_X)$$
where $\Ff_Y:=f^{-1}\Ff$. We define $a(E,\Ff,B,\Mm):=-\mult_EB_Y$ to be the \emph{discrepancy} of $E$ with respect to $(X,\Ff,B,\Mm)$. It is clear that $a(E,\Ff,B,\Mm)$ is independent of the choice of $Y$. If $\Mm=\bm{0}$, then we let $a(E,\Ff,B):=a(E,\Ff,B,\Mm)$. If $\Ff=T_X$, then we let $a(E,X,B,\Mm):=a(E,\Ff,B,\Mm)$. If $\Mm=\bm{0}$ and $\Ff=T_X$, then we let $a(E,X,B):=a(E,\Ff,B,\Mm)$.

We say that $(X,\Ff,B,\Mm)$ is \emph{(sub-)lc} (resp. \emph{(sub-)klt}) if $a(E,\Ff,B,\Mm)\geq -\epsilon_{\Ff}(E)$ (resp. $>-\epsilon_{\Ff}(E)$) for any prime divisor $E$ over $X$. We say that $(X,\Ff,B,\Mm)$ is \emph{(sub-)canonical} (resp. \emph{(sub-)terminal}) if $a(E,\Ff,B,\Mm)\geq 0$ (resp. $>0$) for any prime divisor $E$ that is exceptional over $X$. An \emph{lc place} of $(X,\Ff,B,\Mm)$ is a prime divisor $E$ over $X$ such that $a(E,\Ff,B,\Mm)=-\epsilon_{\Ff}(E)$. An \emph{lc center} of $(X,\Ff,B,\Mm)$ is a subvariety $W$ of $X$, such that either $W$ is the center of an lc place of $(X,\Ff,B,\Mm)$ on $X$, or $W=X$. A \emph{non-trivial lc center} of $(X,\Ff,B,\Mm)$ is an lc center of $(X,\Ff,B,\Mm)$ that is not $X$. A \emph{non-lc place} of $(X,\Ff,B,\Mm)$ is a prime divisor $E$ over $X$ such that $a(E,\Ff,B,\Mm)<-\epsilon_{\Ff}(E)$. A \emph{non-lc center} of $(X,\Ff,B,\Mm)$ is the center of a non-lc place of $(X,\Ff,B,\Mm)$ on $X$. The union of all non-lc centers of $(X,\Ff,B,\Mm)$ is called the \emph{non-lc locus} of $(X,\Ff,B,\Mm)$ and is denoted by $\Nlc(X,\Ff,B,\Mm)$. The union of all non-lc centers and non-trivial lc centers of $(X,\Ff,B,\Mm)$ is called the \emph{non-klt locus} of $(X,\Ff,B,\Mm)$ and is denoted by $\Nklt(X,\Ff,B,\Mm)$.
\end{defn}

\begin{defn}\label{defn: lct}
  Let $(X,\Ff,B,\Mm)$ be a sub-gfq, $D\geq 0$ an $\Rr$-divisor on $X$ and $\Nn$ a nef$/X$ $\bb$-divisor, such that $D+\Nn_X$ is $\Rr$-Cartier. The \emph{lc threshold} (\emph{lct} for short) of $(D,\Nn)$ with respect to $(X,\Ff,B,\Mm)$ is defined as
  $$\lct(X,\Ff,B,\Mm;D,\Nn):=\sup\{+\infty,t\mid (X,\Ff,B+tD,\Mm+t\Nn)\text{ is sub-lc}\}.$$
  If $\Nn=0$, then we may drop $\Nn$ and denote $\lct(X,\Ff,B,\Mm;D,\Nn)$ by $\lct(X,\Ff,B,\Mm;D)$.
\end{defn}

\begin{defn}[Models, I]\label{defn: models I}
Let $(X,\Ff,B,\Mm)/U$ be an lc gfq, $\phi: X\dashrightarrow X'$ a birational map over $U$, $E:=\Exc(\phi^{-1})$ the reduced $\phi^{-1}$-exceptional divisor, $\Ff':=\phi_*\Ff$, and $B':=\phi_*B+E^{\Ff'}$.
\begin{enumerate}
    \item $(X',\Ff',B',\Mm)/U$ is called a \emph{log birational model} of $(X,\Ff,B,\Mm)/U$. 
    \item $(X',\Ff',B',\Mm)/U$ is called a \emph{weak lc model} of $(X,\Ff,B,\Mm)/U$ if 
\begin{enumerate}
\item $(X',\Ff',B',\Mm)/U$ is a log birational model of $(X,\Ff,B,\Mm)/U$, 
    \item $K_{\Ff'}+B'+\Mm_{X'}$ is nef$/U$, and
    \item for any prime divisor $D$ on $X$ which is exceptional over $X'$, $$a(D,\Ff,B,\Mm)\leq a(D,\Ff',B',\Mm).$$
\end{enumerate}
\item $(X',\Ff',B',\Mm)/U$ is called a \emph{semi-good minimal model} of $(X,\Ff,B,\Mm)/U$ if
\begin{enumerate}
        \item $(X',\Ff',B,\Mm)/U$ is a weak lc model of $(X,\Ff,B,\Mm)/U$, and
        \item $K_{\Ff'}+B'+\Mm_{X'}$ is semi-ample$/U$.
\end{enumerate}
\item  Suppose that there exists a contraction$/U$: $X'\rightarrow Z$. $(X',\Ff',B',\Mm)\rightarrow Z$ is called a \emph{Mori fiber space} of $(X,\Ff,B,\Mm)/U$ if
\begin{enumerate}
    \item  $(X',\Ff',B',\Mm)/U$ is a log birational model of $(X,\Ff,B,\Mm)/U$,
    \item $X'$ is $\Qq$-factorial,
    \item $X'\rightarrow Z$ is a $(K_{\Ff'}+B'+\Mm_{X'})$-Mori fiber space$/U$,
    \item for any prime divisor $D$ on $X$ which is exceptional over $X'$, $$a(D,\Ff,B,\Mm)<a(D,\Ff',B',\Mm).$$
\end{enumerate}
\end{enumerate}
We shall not define ``good minimal models" until Definition \ref{defn: models ii}.
\end{defn}

\begin{nota}
    Let $(X_0,\Ff_0,B_0,\Mm)/U$ be a gfq. When we say the following
\begin{center}$\xymatrix{
(X_0,\Ff_0,B_0,\Mm)\ar@{-->}[r]^{f_0} & (X_1,\Ff_1,B_1,\Mm)\ar@{-->}[r]^{\ \ \ \ \ \ \ \ f_1} & \dots\ar@{-->}[r] & (X_n,\Ff_n,B_n,\Mm)\ar@{-->}[r]^{\ \ \ \ \ \ \ \ \ f_n} & \dots 
}$
\end{center}
is a (possibly infinite) sequence of steps of a $(K_{\Ff_0}+B_0+\Mm_{X_0})$-MMP$/U$, we mean the following: for any $i$, $f_i: X_{i}\dashrightarrow X_{i+1}$ is a step of a $(K_{\Ff_i}+B_i+\Mm_{X_i})$-MMP$/U$ that is not a Mori fiber space, $\Ff_{i+1}:=(f_i)_*\Ff_i$, and $B_{i+1}:=(f_i)_*B_i$.
\end{nota}



\begin{cons}[MMP with scaling]\label{cons: mmp with scaling}
    Let $(X,\Ff,B,\Mm)/U$ be an lc gfq. Let $D\geq 0$ be an $\Rr$-divisor on $X$ and $\Nn$ a nef$/U$ $\bb$-divisor on $X$ such that $D+\Nn_X$ is $\Rr$-Cartier and $K_{\Ff}+B+\Mm_X+t(D+\Nn_X)$ is nef$/U$ for some positive real number $t$. A step of a $(K_{\Ff}+B+\Mm_X)$-MMP$/U$ with scaling of $(D,\Nn)$ is defined in the following way. Let
$$\lambda:=\inf\{s\geq 0\mid K_{\Ff}+B+sD+\Mm_X+s\Nn_X\text{ is nef}/U\}.$$
Assume that the following conditions hold:
\begin{itemize}
  \item There exists an extremal ray $R$ in $\overline{NE}(X/U)$ such that $(K_{\Ff}+B+\lambda D+\Mm_X+\lambda\Nn_X)\cdot C=0$ and $(D+\Nn_X)\cdot C>0$. In particular, $R$ is a $(K_{\Ff}+B+\Mm_X)$-negative extremal ray.
  \item The contraction associated to $R$ exists, and if it is a small contraction, then the corresponding $(K_{\Ff}+B+\Mm_X)$-flip exists.
\end{itemize}
Then for any such $R$, we call the divisorial contraction or the Mori fiber space associated to $R$, or the $(K_{\Ff}+B+\Mm_X)$-flip associated to $R$, as a step of a $(K_{\Ff}+B+\Mm_X)$-MMP$/U$ with scaling of $(D,\Nn)$. 

A sequence of steps of a $(K_{\Ff}+B+\Mm_X)$-MMP$/U$ with scaling of $(D,\Nn)$ is a sequence of steps of a $(K_{\Ff}+B+\Mm_X)$-MMP$/U$
\begin{center}$\xymatrix{
(X_0,\Ff_0,B_0,\Mm)\ar@{-->}[r]^{f_0} & (X_1,\Ff_1,B_1,\Mm)\ar@{-->}[r]^{\ \ \ \ \ \ \ \ f_1} & \dots\ar@{-->}[r] & (X_n,\Ff_n,B_n,\Mm)\ar@{-->}[r]^{\ \ \ \ \ \ \ \ \ f_n} & \dots 
}$
\end{center}
where $(X_0,\Ff_0,B_0,\Mm)=(X,\Ff,B,\Mm)$, each $f_i$ is a step of a $(K_{\Ff_i}+B_i+\Mm_{X_i})$-MMP$/U$ with scaling of $(D_i,\Nn)$, where $D_i$ is the image of $D$ on $X_i$. 
$$\lambda_{i}:=\inf\{t,s\geq 0\mid K_{\Ff_{i}}+B_{i}+sD_{i}+\Mm_{X_{i}}+s\Nn_{X_{i}}\text{ is nef}/U\}$$
are called the \emph{scaling numbers} (of this MMP with scaling of $(D,\Nn)$). Note that $D_i+\Nn_{X_i}$ is $\Rr$-Cartier for any $i$ by our construction, so $\lambda_i$ is well-defined. If this MMP does not terminate, then we call $\lim_{i\rightarrow+\infty}\lambda_i$ the \emph{limit of the scaling numbers}. By definition, we have $\lambda_i\geq\lambda_{i+1}$ for any $i$  (of this MMP with scaling of $(D,\Nn)$), so the limit of the scaling numbers is well-defined. 

If $\Nn=\bm{0}$, then a (sequence of) step(s) of a $(K_{\Ff}+B+\Mm_X)$-MMP$/U$ with scaling of $(D,\Nn)$ is called a (sequence of) step(s) of a $(K_{\Ff}+B+\Mm_X)$-MMP$/U$ with scaling of $D$. 
\end{cons}
We remark that Construction \ref{cons: mmp with scaling} does not require the condition that $(X,\Ff,B+D,\Mm+\Nn)$ is lc.

\begin{defn}
   Let $(X,\Ff,B,\Mm)$ and $(X',\Ff',B',\Mm')$ be two sub-gfqs   
   We say that $(X,\Ff,B,\Mm)$ and $(X',\Ff',B',\Mm')$ are \emph{crepant} to each other if $\Mm=\Mm'$, and there exist two birational morphisms $p: W\rightarrow X$ and $q: W\rightarrow X'$ and a foliation $\Ff_W$ on $W$, such that $\Ff_W=p^{-1}\Ff=q^{-1}\Ff'$, $\Mm=\Mm'$, and
   $$p^*(K_{\Ff}+B+\Mm_X)=q^*(K_{\Ff'}+B'+\Mm'_{X'}).$$
\end{defn}

\section{Basic properties of generalized pairs}\label{sec: basic property gpair}

In this section, we present several results concerning the structure of generalized pairs. Although most of these results, or their analogous forms, have already been established in existing literature, it is somewhat surprising to note that many fundamental results for non-NQC generalized pairs remain unaddressed, despite the significant progress in the field of generalized pairs in recent years. Additionally, there are relatively few references available on this subject. For clarity, for the reader's convenience, and to provide a solid reference for future work, we will present detailed proofs for all the results in this section


\subsection{Dlt modification}

First we recall the definition of dlt for generalized pairs.

\begin{defn}[Dlt, {\cite[Definition 2.3]{HL22}}]\label{defn: dlt}
Let $(X,B,\Mm)/U$ be an lc g-pair. We say that $(X,B,\Mm)$ is \emph{dlt} if there exists an open subset $V\subset X$ satisfying the following.
\begin{enumerate}
    \item $(V,B|_V)$ is log smooth. In particular, $B|_V$ is an snc Weil $\Qq$-divisor.
    \item $V$ contains the generic point of any lc center of $(X,B,\Mm)$.
    \item The generic point of any lc center of $(X,B,\Mm)$ is the generic point of an lc center of $(V,B|_V)$.
\end{enumerate}
If $(X,B,\Mm)$ is dlt and $\lfloor B\rfloor$ is normal, then we say that $(X,B,\Mm)$ is plt.
\end{defn}

The following lemma indicates that the definition of dlt in \cite[Definition 2.3]{HL22} is the same as the definition of dlt in \cite{Bir20,FS23}. 
\begin{lem}\label{lem: equi def dlt 1}
Let $(X,B,\Mm)/U$ be an lc g-pair. Then the following two conditions are equivalent: 
\begin{enumerate}
    \item $(X,B,\Mm)$ is dlt.
    \item For any lc center of $(X,B,\Mm)$ with generic point $\eta$, over a neighborhood of $\eta$, $(V,B|_V)$ is log smooth and $\Mm$ descends to $X$. 
\end{enumerate} 
\end{lem}
\begin{proof}
Since dlt, the property in (2), and log smooth are local properties, we may work over a neighborhood of a generic point $\eta$ of an lc center of $(X,B,\Mm)$ (2)$\Rightarrow$ (1) immediately becomes obvious, so we only need to prove (1)$\Rightarrow$ (2). 

By Definition \ref{defn: dlt}, there exists a neighborhood $V$ of $\eta$ such that $(V,B|_V)$ is log smooth and $\eta$ is an lc center of $(V,B|_V)$. Since $(V,B|_V)$ is log smooth, $\Mm_X|_V$ is $\Rr$-Cartier. We let $\Mm^V:=\Mm|_V$ be the restricted $\bb$-divisor of $\Mm$ on $V$, then $\Mm^V$ is nef$/X$ and $\Mm^V_V=\Mm_X|_V$. Suppose that $h: V'\rightarrow V$ is a resolution of $V$ such that $\Mm^V$ descends to $V'$ and there exists a prime divisor $E$ on $V'$ such that $\Center_{V}E=\bar\eta$ and $E$ is an lc place of $(V,B|_V)$. By the negativity lemma,
    $$\Mm^V_{V'}=h^*\Mm^V_V-F$$
    for some $F\geq 0$, such that either $F=0$ over $\bar\eta$ or $\Supp F=\Supp h^{-1}(\bar\eta)$. Since $(X,B,\Mm)$ is lc, $(V,B|_V,\Mm^V)$ is lc. Thus $F=0$ over $\bar\eta$. Possibly shrinking $V$, we may assume that $\Mm$ descends to $V$. The lemma follows.
\end{proof}

Lemma \ref{lem: equi def dlt 1} implies the following result:

\begin{deflem}[Dlt modification, {\cite[Theorem 2.9]{FS23}}]\label{deflem: dlt model}
    Let $(X,B,\Mm)/U$ be a g-pair. Then there exists a birational morphism $f: Y\rightarrow X$ satisfying the following. Let $E_1,\dots,E_n$ be the prime $f$-exceptional divisors and $B_Y:=f^{-1}_*(B\wedge\Supp B)+\sum_{i=1}^nE_i$, then:
    \begin{enumerate}
        \item $(Y,B_Y,\Mm)$ is $\Qq$-factorial dlt.
        \item $a(E_i,X,B,\Mm)\leq 0$ for any $i$.
    \end{enumerate}
    In particular, if $(X,B,\Mm)$ is lc, then $a(E_i,X,B,\Mm)=0$ for any $i$, and $$K_Y+B_Y+\Mm_Y=f^*(K_X+B+\Mm_X).$$
    
    For any such $f$, we call $f$ a \emph{dlt modification} of $(X,B,\Mm)$, and say that $(Y,B_Y,\Mm)$ is a \emph{dlt model} of $(X,B,\Mm)$. 
\end{deflem}

We conjecture that dlt has another equivalent definition:

\begin{conj}\label{conj: dlt has good resolution}
   Let $(X,B,\Mm)/U$ be an lc g-pair. Then $(X,B,\Mm)$ is dlt if and only if there exists a log resolution  $f: Y\rightarrow X$ of $(X,\Supp B)$ and an open subset $V\subset X$, such that $\Mm$ descends to $Y$, $f$ is an isomorphism over $V$, and $V$ contains the generic point of any lc center of $(X,B,\Mm)$. 
\end{conj}
When $(X,B,\Mm)/U$ is NQC, Conjecture \ref{conj: dlt has good resolution} was proven in \cite[Theorem 6.1]{Has22}.





\subsection{Perturbation and MMP}

\begin{lem}\label{lem: gklt+ample terminate}
    Let $(X,B+A,\Mm)/U$ be a $\Qq$-factorial lc g-pair such that $X$ is klt, $A\geq 0$ is ample$/U$, and $B\geq 0$. Then any $(K_X+B+A+\Mm_X)$-MMP$/U$ with scaling of an ample$/U$ $\Rr$-divisor terminates with either a semi-good minimal model of $(X,B+A,\Mm)/U$ or a Mori fiber space of  $(X,B+A,\Mm)/U$.
\end{lem}
\begin{proof}
   By \cite[Lemma 3.4]{HL22}, there exists $0\leq\Delta\sim_{\mathbb R,U}B+A+\Mm_X$ such that $(X,\Delta)$ is klt. By \cite[Corollary 1.4.2]{BCHM10}. any $(K_X+\Delta)$-MMP$/U$ with scaling of an ample$/U$ $\Rr$-divisor terminates with either a Mori fiber space of $(X,\Delta)/U$ or a semi-good minimal model of $(X,\Delta)/U$. The lemma follows. 
\end{proof}

\begin{lem}\label{lem: scaling number go to 0}
Let $(X,B,\Mm)/U$ be a $\Qq$-factorial lc g-pair such that $X$ is klt, and $A\geq 0$ an ample$/U$ $\Rr$-divisor on $X$. Then we may run a $(K_X+B+\Mm_X)$-MMP$/U$ with scaling of $A$. Moreover, let $$(X,B,\Mm):=(X_1,B_1,\Mm)\dashrightarrow (X_2,B_2,\Mm)\dashrightarrow\dots\dashrightarrow (X_i,B_i,\Mm)\dashrightarrow\dots$$
be any $(K_X+B+\Mm_X)$-MMP$/U$ with scaling of $A$, and let $\lambda_i$ be the $i$-th scaling number of this MMP for each $i$, i.e.
$$\lambda_i:=\inf\{t\mid t\geq 0, K_{X_i}+B_i+tA_i+\Mm_{X_i}\text{ is nef/}U\},$$
where $A_i$ is the strict transform of $A$ on $X_i$ for each $i$. Then one of the followings holds:
\begin{enumerate}
    \item This MMP terminates after finitely many steps.
    \item $\lim_{i\rightarrow +\infty}\lambda_i=0$.
\end{enumerate}
\end{lem}
\begin{proof}
Possibly rescaling $A$ we may assume that $K_X+B+A+\Mm_X$ is nef$/U$. We first prove that we may run this MMP by induction on $i$. Let $\lambda_0:=1$ and suppose that there is already a sequence of steps of a $(K_X+B+\Mm_X)$-MMP$/U$ with scaling of $A$
$$(X,B,\Mm):=(X_1,B_1,\Mm)\dashrightarrow (X_2,B_2,\Mm)\dashrightarrow\dots\dashrightarrow (X_k,B_k,\Mm)$$
for some $k\ge1$, such that $\lambda_i\geq\lambda_{i+1}$ for any $i\leq k-2$. If $K_{X_k}+B_k+\Mm_{X_k}$ is nef$/U$, then we are done, so we may assume that $K_{X_k}+B_k+\Mm_{X_k}$ is not nef$/U$. Since nef$/U$ is a closed condition, $\lambda_k>0$. By construction, $\lambda_{k-1}\geq\lambda_{k}$.

By \cite[Lemma 3.4]{HL22}, there exists a klt pair $(X,\Delta)$ such that 
$$K_X+\Delta\sim_{\mathbb R,U}K_X+B+\Mm_X+\frac{\lambda_k}{2}A.$$ 
Possibly replacing $A$, we may assume that $(X,\Delta+(1-\frac{\lambda_k}{2})A)$ is lc. Then we have an induced sequence of steps of a $(K_X+\Delta)$-MMP$/U$ with scaling of $(1-\frac{\lambda_k}{2})A$
$$(X,\Delta):=(X_1,\Delta_1)\dashrightarrow (X_2,\Delta_2)\dashrightarrow\dots\dashrightarrow (X_k,\Delta_k),$$
such that $K_{X_k}+\Delta_k$ is not nef. Let $(X_k,\Delta_k)\dashrightarrow (X_{k+1},\Delta_{k+1})$ be the next step of the $(K_X+\Delta)$-MMP$/U$ with scaling of $(1-\frac{\lambda_k}{2})A$. Then the induced birational map $X_k\dashrightarrow X_{k+1}$ is a step of a $(K_X+B+\Mm_X)$-MMP$/U$ with scaling of $A$. 

We left to prove that if this MMP does not terminate, then $\lim_{i\rightarrow+\infty}\lambda_i=0$. Suppose that $\lambda:=\lim_{i\rightarrow+\infty}\lambda_i>0$. Then
\begin{align*}
\left(X,B+\frac{\lambda}{2}A,\Mm\right):=&\left(X_1,B_1+\frac{\lambda}{2}A_1,\Mm\right)\dashrightarrow\left(X_2,B_2+\frac{\lambda}{2}A_2,\Mm\right)\dashrightarrow\\
\dots\dashrightarrow&\left(X_i,B_i+\frac{\lambda}{2}A_i,\Mm\right)\dashrightarrow\dots
\end{align*}
is an infinite sequence of steps of a $(K_X+B+\frac{\lambda}{2}A+\Mm_X)$-MMP$/U$, which contradicts Lemma \ref{lem: gklt+ample terminate}.
\end{proof}


The following result seems to be missed in known literature.

\begin{prop}\label{prop: qfact nqc any scaling terminate}
    Let $(X,B+A,\Mm)/U$ be a $\Qq$-factorial NQC lc g-pair such that $A\geq 0$ is ample$/U$ and $B\geq 0$. Then any $(K_X+B+A+\Mm_X)$-MMP$/U$ with scaling of an ample$/U$ $\Rr$-divisor terminates with either a semi-good minimal model of $(X,B+A,\Mm)/U$ or a Mori fiber space of $(X,B+A,\Mm)/U$.
\end{prop}
\begin{proof}
By \cite[Lemma A.5]{LT22}, possibly replacing $A$ with a general element in $|A/U|_{\mathbb R}$, there exists an lc pair $(X,\Delta+\frac{1}{2}A)$ such that $0\leq \Delta\sim_{\mathbb R}B+\frac{1}{2}A+\Mm_X$. By \cite[Theorem 1.5]{HH20} and \cite[Theorem 1.9]{Bir12}, any $(K_X+\Delta+\frac{1}{2}A)$-MMP$/U$ with scaling of an ample$/U$ $\Rr$-divisor terminates. Thus any $(K_X+B+A+\Mm_X)$-MMP$/U$ with scaling of an ample$/U$ $\Rr$-divisor terminates. The rest part of the proposition follows from \cite[Theorem 1.3]{LX23a} and \cite[Lemma 3.9(1)]{HL21a}.
\end{proof}



\begin{lem}[{cf. \cite[Lemma 2.6(3)]{Gon11}}]\label{lem: movable num 0 is 0}
Let $X$ be a normal projective variety and $D$ a movable $\Rr$-Cartier $\Rr$-divisor on $X$ such that $\kappa_{\sigma}(D)=0$. Then $D\equiv 0$.
\end{lem}
\begin{proof}
We let $f: Y\rightarrow X$ be a resolution of $X$. Let $P_Y:=P(Y,f^*D)$ and $N_Y:=N(Y,f^*D)$ be the positive and negative part of the Nakayama-Zariski decomposition of $f^*D$ respectively, and let $P:=P(X,D)$ and $N:=N(X,D)$ be the positive and negative part of the Nakayama-Zariski decomposition of $D$ respectively. Since $D$ is movable, by \cite[Lemma 3.7(3)]{LX23a}, $N=0$. By \cite[Lemma 3.4(3)]{LX23a}, $f_*N_Y=N$, so $N_Y$ is exceptional$/X$. Since $\kappa_\sigma(f^*D)=\kappa_\sigma(D)=0$, by \cite[V 2.7 Proposition(8)]{Nak04}, $P_Y\equiv 0$. Thus $D=f_*D_Y=f_*(P_Y+N_Y)\equiv 0.$
\end{proof}

\begin{prop}
Let $(X,B,\Mm)$ be a projective $\Qq$-factorial lc g-pair such that $\kappa_{\sigma}(K_X+B+\Mm_X)=0$ and $X$ is klt. Let $A$ be an ample $\Rr$-divisor. Then we may run a $(K_X+B+\Mm_X)$-MMP with scaling of $A$, and any such MMP terminates with a model $(X',B',\Mm)$ of $(X,B,\Mm)$ such that $K_{X'}+B'+\Mm_{X'}\equiv 0$. Moreover, if $\kappa_{\iota}(K_X+B+\Mm_X)=0$, then $K_{X'}+B'+\Mm_{X'}\sim_{\mathbb R}0$.
\end{prop}
\begin{proof}
By Lemma \ref{lem: scaling number go to 0}, we may run a $(K_X+B+\Mm_X)$-MMP with scaling of $A$. Let
$$(X,B,\Mm):=(X_1,B_1,\Mm)\dashrightarrow (X_2,B_2,\Mm)\dashrightarrow\dots\dashrightarrow (X_i,B_i,\Mm)\dashrightarrow\dots$$
be any such MMP with scaling numbers $\lambda_i\ge\lambda_{i+1}$. If this MMP does not terminate, then $\lim_{i\rightarrow+\infty}\lambda_i=0$ by Lemma \ref{lem: scaling number go to 0}. There exists a positive integer $m$ such that $X_i\dashrightarrow X_{i+1}$ is a flip for any $i\geq m$. We may denote by $\phi_i: X_m\dashrightarrow X_i$ the induced birational contraction and $A_i$ the strict transform of $A$ on $X_i$ for any $i> m.$ Since $K_{X_i}+B_i+\lambda_iA_i+\Mm_{X_i}$ is nef for each $i$,
$$K_{X_m}+B_m+\Mm_{X_m}=\lim_{i\rightarrow+\infty}(\phi_i^{-1})_*(K_{X_i}+B_i+\lambda_iA_i+\Mm_{X_i})$$
is movable. Moreover, since $\kappa_{\sigma}(K_X+B+\Mm_X)=0$, $\kappa_{\sigma}(K_{X_m}+B_m+\Mm_{X_m})=0$. By Lemma \ref{lem: movable num 0 is 0}, $K_{X_m}+B_m+\Mm_{X_m}\equiv 0$, a contradiction. Thus this MMP terminates with a model $(X',B',\Mm)$ such that $K_{X'}+B'+\Mm_{X'}$ is nef and $\kappa_{\sigma}(K_{X'}+B'+\Mm_{X'})=0$. By Lemma \ref{lem: movable num 0 is 0} again, $K_{X'}+B'+\Mm_{X'}\equiv 0$. Moreover, if $\kappa_{\iota}(K_X+B+\Mm_X)=0$, then $\kappa_{\iota}(K_{X'}+B'+\Mm_{X'})=0$, hence $K_{X'}+B'+\Mm_{X'}\sim_{\mathbb R}0$.
\end{proof}






\subsection{Lc centers of generalized pairs}

 We will discuss the structure of lc centers of lc g-pairs in this section. For NQC generalized pairs, the structure of their lc centers is well-studied in \cite{LX23a} based on the connected principle established in \cite{Bir20,FS23} and the canonical bundle formula \cite{Fil20,HL21b,JLX22,FS23} but little was known for the non-NQC case. 


\begin{defn}[Adjunction for generalized pairs to divisors, cf. {\cite[Definition 4.7]{BZ16}}]\label{defn: adj to lc places}
Let $(X,B,\Mm)/U$ be a g-(sub-)pair and $S$ a component of $B^{=1}$. Let $S^\nu$ be the normalization of $S$. The g-(sub-)pair $(S^\nu,B_S,\Mm^S)/U$ induced by the adjunction
$$K_{S^\nu}+B_S+\Mm^S_S:=(K_X+B+\Mm_X)|_S$$
is given in the following way. Let $f: Y\rightarrow X$ be a log resolution of $(X,\Supp B)$ such that $\Mm$ descends to $Y$, $S_Y$ the strict transform of $S$ on $Y$, and 
$$K_Y+B_Y+\Mm_Y:=f^*(K_X+B+\Mm_X).$$ 
We define $\Mm^S:=\Mm|_{S^\nu}$ and $B_{S_Y}:=(B_Y-S_Y)|_{S_Y}$. We let $f|_{S_Y}: S_Y\rightarrow S^\nu$ be the induced birational morphism and define $B_S:=(f|_{S_Y})_*B_{S_Y}$.
\end{defn}



\begin{lem}[cf. {\cite[Lemma 3.18(2)]{LX23b}}]\label{lem: inversion of adjunction gdlt}
Let $(X,B,\Mm)/U$ be a dlt g-pair, $S$ a component of $\lfloor B\rfloor$, and $(S,B_S,\Mm^S)/U$ the g-pair induced by the adjunction 
$$K_S+B_S+\Mm^S_S:=(K_X+B+\Mm_X)|_S.$$
Then $(S,B_S,\Mm^S)$ is dlt. Moreover:
\begin{enumerate}
    \item Any lc center of $(S,B_S,\Mm^S)$ is an lc center of $(X,B,\Mm)$.
    \item Any lc center of $(X,B,\Mm)$ that is contained in $S$ is an lc center of $(S,B_S,\Mm^S)$.
\end{enumerate}
\end{lem}
\begin{proof}
By \cite[Lemma 2.9]{HL22}, $(S,B_S,\Mm^S)$ is dlt.

(1) Let $f: \tilde X\rightarrow X$ be a log resolution of $(X,\Supp B)$ such that $\Mm$ descends to $\tilde X$. Let $K_{\tilde X}+\tilde B+\Mm_{\tilde X}:=f^*(K_X+B+\Mm_X)$ and let $\tilde S$ be the strict transform of $S$ on $\tilde X$, then
$f|_{\tilde S}$ is a log resolution of $(S,\Supp B_S)$ such that $\Mm^S$ descends to $\tilde S$. We have
$$f|_{\tilde S}^*(K_S+B_S+\Mm^S_S)=K_{\tilde S}+B_{\tilde S}+\Mm^S_{\tilde S}:=(K_{\tilde X}+\tilde B+\Mm_{\tilde X})|_{\tilde S}.$$
Let $W_S$ be an lc center of $(S,B_S,\Mm^S)$. Then $W_S$ is the image of an lc center $W_{\tilde S}$ of $(\tilde S,B_{\tilde S},\Mm^S)$ in $S$. Since $(\tilde X,\tilde B)$ is log smooth and $\Mm$ descends to $\tilde X$, $W_{\tilde S}$ is also an lc center of $(\tilde X,\tilde B,\Mm)$ which is contained in $\tilde S$, so $W:=f(W_{\tilde S})$ is an lc center of $(X,B,\Mm)$ which is contained in $S$. It is clear that $W_S=W$ under the natural inclusion $S\rightarrow X$. This implies (1).

(2) Let $W$ be an lc center of $(X,B,\Mm)$ that is contained in $S$. Since $(X,B,\Mm)$ is dlt, by Lemma \ref{lem: equi def dlt 1}, possibly shrinking $X$ to a neighborhood of the generic point of $W$, we may assume that $(X,B)$ is log smooth and $\Mm$ descends to $X$. Thus $W$ is an lc center of $(X,B)$, $K_S+B_S=(K_X+B)|_S$, and $\Mm^S$ descends to $S$. Since $(X,B)$ is log smooth, $W$ is an lc center of $(S,B_S)$, hence an lc center of $(S,B_S,\Mm^S)$. This implies (2).
\end{proof}

\begin{deflem}
    Let $(X,B,\Mm)/U$ be a dlt g-pair and $V$ an lc center of $(X,B,\Mm)$ such that $\dim V\geq 1$. Then we may construct a dlt g-pair $(V,B_V,\Mm^V)/U$ on $V$ inductively the following way. If $V=X$ then we let $(V,B_V,\Mm^V):=(X,B,\Mm)$. Otherwise, let $S$ be a codimension $1$ lc center of $(X,B,\Mm)$ such that $V\subset S$. By \cite[Lemma 2.9]{HL22}, there exists an dlt g-pair $(S,B_{S},\Mm^S)$ induced by adjunction
    $$K_{S}+B_{S}+\Mm^S_{S}=(K_X+B+\Mm_X)|_{S}.$$
    By Lemma \ref{lem: inversion of adjunction gdlt}, $V$ is an lc center of $(S,B_S,\Mm^S)$. By repeating this process and applying induction on dimension, we get a dlt g-pair $(V,B_V,\Mm^V)/U$ on $V$. $(V,B_V,\Mm^V)/U$ is called the dlt g-pair \emph{induced by repeatedly applying adjunction to codimension $1$ lc centers}:
    $$K_V+B_V+\Mm^V_V:=(K_X+B+\Mm_X)|_V.$$
\end{deflem}

\begin{defn}\label{defn: lc cls}
An \emph{lc crepant log structure} is of the form $f: (X,B,\Mm)\rightarrow Z$, where
\begin{enumerate}
    \item $(X,B,\Mm)/Z$ is an lc g-pair,
    \item $K_X+B+\Mm_X\sim_{\Rr,Z}0$, and
    \item $f$ is a contraction. In particular, $f_*\Oo_X=\Oo_Z$.
\end{enumerate}
In addition, if
\begin{enumerate}
    \item[(4)] $(X,B,\Mm)$ is dlt, 
\end{enumerate}
then we say that $f: (X,B,\Mm)\rightarrow Z$ is a \emph{dlt crepant log structure}. 

For any irreducible subvariety $W\subset Z$, we say that $W$ is an \emph{lc center} of an lc crepant log structure $f: (X,B,\Mm)\rightarrow Z$, if there exists an lc center $W_X$ of $(X,B,\Mm)$ such that $W=f(W_X)$. For any (not necessarily closed) point $z\in Z$, we say that $\bar z$ is an \emph{lc center} of $f: (X,B,\Mm)\rightarrow Z$ if $\bar z$ is an lc center of $f: (X,B,\Mm)\rightarrow Z$.
\end{defn}

\begin{rem} 
In Section \ref{sec: cbf} below we will introduce the concept of lc-trivial fibrations. We will see that an lc crepant log structure is an lc-trivial fibration $f: (X,B,\Mm)\rightarrow Z$ such that $B\geq 0$ (see Definition \ref{defn: lc trivial fibration gfq} below). We will also see that an lc center of an lc crepant log structure $f: (X,B,\Mm)\rightarrow Z$ is indeed an lc center of the induced g-pair $(Z,B_Z,\Mm^Z)$ via the canonical bundle formula (see Theorem \ref{thm: cbf gpair nonnqc} below).
\end{rem}



\begin{defn}[Standard $\mathbb P^1$-link, cf. {\cite[Definition 2.21]{FS23}}]\label{defn: standard p1 link}
Let $X\rightarrow U$ be a projective morphism from a normal quasi-projective variety to a variety. A \emph{standard $\mathbb P^1$-link}$/U$ $f: (X,B,\Mm)\rightarrow T$ is an lc g-pair $(X,B,\Mm)/U$ with a projective morphism $f: X\to T$ over $U$ satisfying the following properties.
\begin{enumerate}
\item $K_X+B+\Mm_X\sim_{\mathbb R,T}0$,
\item there exists a birational morphism $X'\rightarrow X$ such that $\Mm_{X'}\sim_{\mathbb R,T}0$,
\item $\lfloor B\rfloor=D_1+D_2$, where $D_1,D_2$ are prime divisors and $f|_{D_i}: D_i\rightarrow T$ are isomorphisms,
\item $(X,B,\Mm)$ is plt, and
\item every reduced fiber of $f$ is isomorphic to $\mathbb P^1$.
\end{enumerate}
We call $D_1$ and $D_2$ the \emph{horizontal sections} of $(X,B,\Mm)/T$.
\end{defn}


\begin{defn}[$\mathbb P^1$-link, cf. {\cite[Definition 2.23]{FS23}}]\label{defn: p1 link}
Let $(X,B,\Mm)/U$ be a dlt g-pair associated with a projective morphism $f: X\rightarrow U$, such that $K_X+B+\Mm_X\sim_{\mathbb R,Z}0$. Let $Z_1$, $Z_2$ be two lc centers of $(X,B,\Mm)$. We say that $Z_1$ and $Z_2$ are \emph{directly $\mathbb P^1$-linked$/U$} if there exists an lc center $W$ of $(X,B,\Mm)$ satisfying the following. 
\begin{enumerate}
        \item $Z_i\subset W$ for each $i$.
        \item $f(W)=f(Z_1)=f(Z_2)$.
        \item Let $(W,B_W,\Mm^W)/U$ be the dlt g-pair induced by repeatedly applying adjunction to codimension $1$ lc centers
$$K_W+B_W+\Mm^W_W:=(K_X+B+\Mm_X)|_W.$$
        Then there exists a standard $\mathbb P^1$-link$/U$ $h: (W',B_{W'},\Mm^W)\rightarrow T$ such that $(W',B_{W'},\Mm^W)$ is crepant to $(W,B_W,\Mm^W)$, and $Z_1|_{W'}$ and $Z_2|_{W'}$ are the horizontal sections of $(W',B_{W'},\Mm^W)/T$.
\end{enumerate}
We say that $Z_1$ and $Z_2$ are \emph{$\mathbb P^1$-linked$/U$} if either $Z_1=Z_2$, or there exists an integer $n\geq 2$ and lc centers $Z_1',\dots,Z_n'$ of $(X,B,\Mm)$, such that $Z_1'=Z_1,Z_n'=Z_2$, and $Z'_i$ and $Z'_{i+1}$ are directly $\mathbb P^1$-linked$/U$ for any $1\leq i\leq n-1$.
\end{defn}

The following theorem is important when characterizing the structure of lc centers of g-pairs. We emphasize that, in the following theorem, we do not require $(X,B,\Mm)$ to be NQC.

\begin{thm}[{\cite[Theorem 3.5]{Bir20}; \cite[Theorem 1.4]{FS23} for the $\Qq$-coefficient case}]\label{thm: P1 link for gdlt crepant log structure} 
Let $(X,B,\Mm)/U$ be a dlt g-pair associated with a projective morphism $f: X\rightarrow U$, such that $K_X+B+\Mm_X\sim_{\mathbb R,U}0$. Let $s\in U$ be a (not necessarily closed) point such that $f^{-1}(s)$ is connected (as a $k(s)$-scheme). Let
$$\mathcal{S}:=\{V\mid V\text{ is an lc center of }(X,B,\Mm), s\in f(V)\}$$
and $Z,W\in\mathcal{S}$ be two elements such that $Z$ is minimal in $\mathcal{S}$ with respect to the inclusion. Then there exists $Z_W\in\mathcal{S}$ such that $Z_W\subset W$, and $Z$ and $Z_W$ are $\mathbb P^1$-linked$/U$. In particular, any minimal elements in $\mathcal{S}$ with respect to inclusion are $\mathbb P^1$-linked$/U$ to each other.
\end{thm}
\begin{proof}
\noindent\textbf{Step 1}. In this step, we show that the theorem holds over an \'etale neighborhood $(s'\in U')\rightarrow (s\in U)$ such that $k(s)=k(s')$. We use induction on $\dim X$ and on $\dim U$.

If $f^{-1}(s)\cap\lfloor B\rfloor$ is disconnected, then by \cite[Theorem 3.5]{Bir20}, after an \'etale base change, there are exactly two non-trivial lc centers of $(X,B,\Mm)$ intersecting $f^{-1}(s)$, and they are $\mathbb P^1$-linked with each other. We are done in this case.

If $f^{-1}(s)\cap\lfloor B\rfloor$ is connected, then we let $D_1,\dots,D_r$ be the irreducible components of $\lfloor B\rfloor$.  By passing to an \'etale neighborhood of $s\in S$ without changing $k(s)$, we may assume that each $D_i$ has connected fiber over $s$, and every lc center of $(X,B,\Mm)$ intersects $f^{-1}(s)$ (cf. \cite[Claim 4.38.1]{Kol13}). Possibly reordering indices, we may assume that $Z\subset D_1$, $W\subset D_r$, and 
$$f^{-1}(s)\cap D_i\cap D_{i+1}\not=\emptyset$$
for any $1\leq i\leq r-1$. Let $(D_i,B_{D_i},\Mm^{D_i})$ be the g-pair induced by adjunction
$$K_{D_i}+B_{D_i}+\Mm^{D_i}_{D_i}:=(K_X+B+\Mm_X)|_{D_i}$$
for each $i$.
\begin{claim}\label{claim: p1link induction}
Let $Z_1:=Z$. For any $2\leq i\leq r$, there exists an lc center $Z_i\subset D_{i-1}\cap D_{i}$ in $\mathcal{S}$ such that
\begin{enumerate}
    \item $Z_i$ and $Z_{i-1}$ are $\mathbb P^1$-linked$/U$ with each other,
    \item $Z_i$ is minimal in $\mathcal{S}$, and
    \item $Z_i$ is an lc center of $(D_i,B_{D_i},\Mm^{D_i})$ and $(D_{i-1},B_{D_{i-1}},\Mm^{D_{i-1}})$.
\end{enumerate}
\end{claim}
\begin{proof}
Suppose we have already constructed $Z_{i-1}$. By Lemma \ref{lem: inversion of adjunction gdlt}, $Z_{i-1}$ and $D_{i-1}\cap D_i$ are lc centers of $(D_{i-1},B_{D_{i-1}},\Mm^{D_{i-1}})$, and $Z_{i-1}$ is minimal among all lc centers of $(D_{i-1},B_{D_{i-1}},\Mm^{D_{i-1}})$ which dominate $s$. By induction hypothesis of $\dim X$ and $\dim U$, there exists an lc center $Z_i\subset D_{i-1}\cap D_i$ that is minimal in among all lc centers of $(D_{i-1},B_{D_{i-1}},\Mm^{D_{i-1}})$ which dominate $s$, and $Z_i$ and $Z_{i-1}$ are $\mathbb P^1$-linked with each other. By Lemma \ref{lem: inversion of adjunction gdlt}, $Z_i$ is an lc center of $(X,B,\Mm)$, is minimal in $\mathcal{S}$, and is an lc center of $(D_i,B_{D_i},\Mm^{D_i})$. The claim follows by induction on $i$.
\end{proof}
\noindent\textit{Proof of Theorem \ref{thm: P1 link for gdlt crepant log structure} continued}. By Claim \ref{claim: p1link induction} applied to $i=r$, the theorem holds over an \'etale neighborhood $(s'\in U')\rightarrow (s\in U)$ such that $k(s)=k(s')$ under the induction hypothesis of $\dim X$ and $\dim U$.

\medskip

\noindent\textbf{Step 2}. We show that the \'etale base change was not necessary and conclude the proof of the theorem. Let 
$$X\xrightarrow{\tilde f}\tilde U\rightarrow U$$
be the Stein factorization of $f$. Since $f^{-1}(s)$ is connected, there exists a unique pre-image $\tilde s\in\tilde U$ of $s$. Let $Z_i$ be the minimal elements of $\mathcal{S}$. Since lc centers commute with \'etale base change, we see that there is a unique irreducible subvariety 
$$\tilde s\in \tilde V\subset U$$
such that $\tilde V=\tilde f(Z_i)$ for each $i$. 

Let $\tilde v$ be the generic point of $\tilde V$. By \textbf{Step 1} and induction hypothesis, the theorem holds after an \'etale base change 
$$\tilde\pi: (\tilde v'\in \tilde U')\rightarrow (\tilde v\in\tilde U).$$
Since $\tilde f$ has connected fibers, $\tilde\pi$ induces an isomorphism of the fibers
$$\tilde\pi: (\tilde f')^{-1}(\tilde v')\cong\tilde f^{-1}(\tilde v).$$
Thus $Z_i$ canonically lift to $Z_i'\cong Z_i$ and the $\mathbb P^1$-links$/U$ between the $Z_i'$ descend
to $\mathbb P^1$-links$/U$ between the $Z_i$.
\end{proof}




\begin{lem}\label{lem: intersection of lc center gpair}
Let $f: (X,B,\Mm)\rightarrow Z$ be an lc crepant log structure and $z\in Z$ a (not necessarily closed) point. Let $$\mathcal{S}_z:=\{V\mid V\text{ is an lc center of }f: (X,B,\Mm)\rightarrow Z, z\in V\}.$$
Then:
\begin{enumerate}
    \item There exists a unique element $W\in\mathcal{S}_z$ that is minimal with respect to inclusion. 
    \item $W$ is unibranch (\cite[Definition 1.44]{Kol13}) at $z$, i.e.  the completion $\widehat{W}_z$ is irreducible.
    \item Any intersection of lc centers of $f: (X,B,\Mm)\rightarrow Z$  is a union of lc centers.
\end{enumerate}
\end{lem}
\begin{proof}
By Definition-Lemma \ref{deflem: dlt model}, possibly replacing $(X,B,\Mm)$ with a dlt model, we may assume that $(X,B,\Mm)$ is dlt. Since $f$ is a contration, $f^{-1}(z)$ is connected. For any any element $W\in\mathcal{S}_z$ that is minimal with respect to inclusion, there exists an lc center $Z_W$ of $(X,B,\Mm)$ that is minimal among all lc centers whose image on $Z$ is equal to $W$ with respect to inclusion. By Theorem \ref{thm: P1 link for gdlt crepant log structure}, all such $Z_W$ are $\mathbb P^1$-linked$/Z$ to each other, hence their images on $Z$ are the same. This proves (1). (2) follows from (1) by considering every \'etale neighborhood of $z$. 

For any lc centers $W_1,W_2$ on $Z$, let $z\in W_1\cap W_2$ be any point. By (1), there exists a unique element $W_z$ of $\mathcal{S}_z$. Then
$$z\in W_z\subset W_1\cap W_2,$$
so 
$$W_1\cap W_2=\cup_{z\in W_1\cap W_2}z\subset \cup_{z\in W_1\cap W_2}W_z\subset W_1\cap W_2.$$
Therefore, 
$$W_1\cap W_2=\cup_{z\in W_1\cap W_2}W_z$$
is a union of lc centers. We get (3).
\end{proof}

\begin{lem}\label{lem: gdlt crepant log structure is compatible under subadjunction}
Let $f: (X,B,\Mm)\to Z$ be a dlt crepant log structure and $Y\subset X$ an lc center. Let
$$
f|_Y: Y\xrightarrow{f_Y}Z_Y\xrightarrow{\pi} Z
$$
be the Stein factorization of $f|_Y$, and $(Y,B_Y,\Mm^Y)/Z$ the dlt g-pair induced \emph{by repeatedly applying adjunction to codimension $1$ lc centers}
$$K_Y+B_Y+\Mm_Y^Y:=(K_X+B+\Mm_X)|_Y.$$
Then:
\begin{enumerate}
\item $f_Y: (Y,B_Y,\Mm^Y)\rightarrow Z_Y$ is a dlt crepant log structure.
\item For any lc center $W_Y\subset Z_Y$ of $f_Y: (Y,B_Y,\Mm^Y)\rightarrow Z_Y$, $\pi(W_Y)$ is an lc center of $f: (X,B,\Mm)\rightarrow Z$.
\item For any lc center $W\subset Z$ of $f: (X,B,\Mm)\rightarrow Z$, every irreducible component of $\pi^{-1}(W)$ is an lc center of  $f_Y: (Y,B_Y,\Mm^Y)\rightarrow Z_Y$.
\end{enumerate}
\end{lem}

\begin{proof}
(1) We only need to show that $(Y,B_Y,\Mm^Y)$ is dlt, which follows from \cite[Lemma 2.9]{HL22}. 

(2) There exists an lc center $V_Y$ of $(Y,B_Y,\Mm^Y)$ such that $f_Y(V_Y)=W_Y$. By Lemma \ref{lem: inversion of adjunction gdlt}, $V_Y$ is also an lc center of $(X,B,\Mm)$. Thus $\pi(W_Y)=f(V_Y)$ is an lc center of $f: (X,B,\Mm)\rightarrow Z$.

(3) Let $z$ be the generic point of $W$. Since the question is \'etale local, possibly replacing $Z$ by an \'etale neighborhood of $z$ and replacing $Y$ with its irreducible components, we may assume that $f^{-1}(z)\cap Y$ is connnected, and we only need to show that there exists an lc center $V_Y$ of $f_Y: (Y,B_Y,\Mm^Y)\rightarrow Z_Y$ such that $f_Y(V_Y)$ is an irreducible component of $\pi^{-1}(W)$.

Let $V_X$ be a minimal lc center of $(X,B,\Mm)$ which dominates $W$, i.e. $V_X$ is minimal in
$$\{V\mid V\text{ is an lc center of }(X,B,\Mm), V\text{ dominates }W\}$$
with respect to inclusion. Then $f(V_X)=W$. By Theorem \ref{thm: P1 link for gdlt crepant log structure}, there exists an lc center $V_Y\subset Y$ of $(X,B,\Mm)$ that is $\mathbb P^1$-linked$/Z$ to $V_X$. By Lemma \ref{lem: inversion of adjunction gdlt}, $V_Y$ is also an lc center of $(Y,B_Y,\Mm^Y)$. Thus $f_Y(V_Y)\subset Z_Y$ is an lc center of $f_Y: (Y,B_Y,\Mm^Y)\rightarrow Z_Y$. Moreover, since $V_Y$ is $\mathbb P^1$-linked$/Z$ to $V_X$, $(f|_Y)(V_Y)=f(V_X)=W$. Thus $f_Y(V_Y)$ is an irreducible component of $\pi^{-1}(W)$ and we are done.
\end{proof}


\subsection{Inversion of adjunction} In this subsection, we prove the following canonical bundle formula for NQC generalized pairs:

\begin{thm}\label{thm: inversion of adjunction}
Let $(X,B,\Mm)$ be an NQC g-pair and $S$ a component of $B^{=1}$. Let $S^\nu$ be the normalization of $S$, and let $(S^\nu,B_S,\Mm^S)/U$ be the g-pair induced by the adjunction
$$K_{S^\nu}+B_S+\Mm^S_S:=(K_X+B+\Mm_X)|_S.$$
Then $(S^\nu,B_S,\Mm^S)$ is lc if and only if $(X,B,\Mm)$ is lc near $S$.
\end{thm}
\begin{proof}
The if part of the theorem follows from \cite[Definition 4.7]{BZ16} so we only need to prove the only if part. 

First we prove the case when $(X,B,\Mm)$ is a $\Qq$-g-pair. 

By Definition-Lemma \ref{deflem: dlt model}, there exists a birational morphism $f: Y\rightarrow X$ satisfying the following. Let $E$ be the reduced $f$-exceptional divisor and $B_Y:=f^{-1}_*(B\wedge\Supp B)+E$, then
\begin{enumerate}
    \item $(Y,B_Y,\Mm)$ is $\Qq$-factorial dlt,
    \item $a(F,X,B,\Mm)\leq 0$ for any prime $f$-exceptional divisor $F$.
\end{enumerate}
We let
$$K_Y+\bar B_Y+\Mm_Y:=f^*(K_X+B+\Mm_X)$$
and let $S_Y$ be the strict transform of $S$ on $Y$. Let $(S_Y,B_{S_Y},\Mm^S)/U$ and $(S_Y,\bar B_{S_Y},\Mm^S)/U$ be the g-pairs induced by adjunction
$$K_{S_Y}+B_{S_Y}+\Mm^S_{S_Y}=(K_Y+B_Y+\Mm_Y)|_{S_Y}$$
and
$$K_{S_Y}+\bar B_{S_Y}+\Mm^S_{S_Y}=(K_Y+\bar B_Y+\Mm_Y)|_{S_Y}$$
respectively. We let $Q:=\bar B_Y-B_Y$. 

Let $A$ be an ample divisor on $Y$ such that $K_Y+B_Y+\Mm_Y+A$ is nef. We may run a $(K_Y+B_Y+\Mm_Y)$-MMP$/X$ with scaling of $A$
$$(Y,B_Y,\Mm):=(X_0,B_0,\Mm)\dashrightarrow (X_1,B_1,\Mm)\dashrightarrow\dots\dashrightarrow (X_n,B_n,\Mm)\dashrightarrow\dots.$$
Let $S_i,A_i,Q_i,\bar B_i$ be the image of $S_Y,A,Q,\bar B_Y$ on $X_i$ for each $i$, $f_i: X_i\rightarrow X$ the induced birational morphism, and 
$$\lambda_i:=\inf\{t\geq 0\mid K_{X_i}+B_i+tA_i+\Mm_{X_i}\text{ is nef}/X\}$$
the scaling numbers. Then $K_{X_i}+\bar B_i+\Mm_{X_i}=f_i^*(K_X+B+\Mm_X)$ and $\bar B_i=B_i+Q_i$ for any $i$. Let
$$K_{S_i}+B_{S_i}+\Mm^S_{S_i}:=(K_{X_i}+B_i+\Mm_{X_i})|_{S_i}$$
and
$$K_{S_i}+\bar B_{S_i}+\Mm^S_{S_i}:=(K_{X_i}+\bar B_i+\Mm_{X_i})|_{S_i}$$
for any $i$. Then $\bar B_{S_i}=B_{S_i}+Q_i|_{S_i}$. Moreover, there exists a birational morphism $g_i: S_i\rightarrow S$ such that
$$K_{S_i}+\bar B_{S_i}+\Mm^S_{S_i}=g_i^*(K_S+B_S+\Mm^S_S).$$
Thus $(S_i,\bar B_{S_i},\Mm^S)$ is lc. Since $(Y,B_Y,\Mm)$ is dlt, $(X_i,B_i,\Mm)$ is dlt. By Lemma \ref{lem: inversion of adjunction gdlt}, $(S_i,B_{S_i},\Mm^S)$ is dlt. By Lemma \ref{lem: inversion of adjunction gdlt}, any lc center of $(X_i,B_i,\Mm)$ is an lc center of $(S_i,B_{S_i},\Mm^S)$. Since all components of $Q_i$ are lc centers of $(X_i,B_i,\Mm)$ and $(S_i,\bar B_{S_i},\Mm^S)$ is lc, $\Supp Q_i$ does not intersect $S_i$ for any $i$.

We pick a non-negative integer $m$ in the following way. If the $(K_Y+B_Y+\Mm_Y)$-MMP$/X$ terminates, then we let $m$ be the index so that $(X_m,B_m,\Mm)/X$ is a log minimal model of $(Y,B_Y,\Mm)/X$ for some non-negative integer $m$. If the $(K_Y+B_Y+\Mm_Y)$-MMP$/X$ does not terminate, then by Lemma \ref{lem: scaling number go to 0}, $\lim_{i\rightarrow+\infty}\lambda_i=0$, so by special termination (cf. \cite[Lemma 2.18]{LX23a}), we may pick a positive integer $m$, such that $S_i\dashrightarrow S_{i+1}$ is an isomorphism in codimension $1$ for any $i\geq m$. We let $I\geq 2$ be any sufficiently divisible positive integer satisfying the following.
\begin{itemize}
    \item $IQ$ is a Weil divisor.
    \item 
    $$(f_m)_*\mathcal{O}_{X_m}(A_m-IQ_m)\subset (f_m)_*\mathcal{O}_{X_m}(A_m)$$
    are contained in 
    $$\mathcal{I}_{f_m(\Supp Q)}\cdot (f_m)_*\mathcal{O}_{X_m}(A_m).$$
    \item If $(X_m,B_m,\Mm)/X$ is a log minimal model of $(Y,B_Y,\Mm)/X$ and $m\geq 2$, then $\lambda_{m-1}>\frac{1}{I}$.
    \item If the $(K_Y+B_Y+\Mm_Y)$-MMP$/X$ does not terminate, then $\lambda_m>\frac{1}{I}$.
\end{itemize}
Since $S_i\dashrightarrow S_{i+1}$ is an isomorphism in codimension $1$ for any $i\geq m$, for any $j\geq m$, we have
$$(f_j)_*\mathcal{O}_{X_j}(A_j-IQ_j)=(f_m)_*\mathcal{O}_{X_m}(A_m-IQ_m).$$
Since $\Supp Q_i$ does not intersect $S_i$ for any $i$, we  have an induced homomorphism. 
$$(f_i)_*\mathcal{O}_{X_i}(A_i-IQ_i)\rightarrow(f_m|_{S_m})_*\mathcal{O}_{S_m}(A_i)=(f_i|_{S_i})_*\mathcal{O}_{S_I}(A_i)$$
which is not surjective. Therefore,
$$R^1(f_i)_*\mathcal{O}_{X_i}(A_i-IQ_i-S_i)\not=0$$
for any $i\geq m$. 

We let $l:=m$ if $(X_m,B_m,\Mm)/X$ is a log minimal model of $(Y,B_Y,\Mm)/X$, and let $l$ be the unique positive integer such that $\lambda_{l-1}>\frac{1}{I}\geq\lambda_{l}$ if the $(K_Y+B_Y+\Mm_Y)$-MMP$/X$ does not terminate. Then $l\geq m$,
$$X_0\dashrightarrow X_1\dashrightarrow X_{l}$$
is also a sequence of steps of a $(K_{X_0}+B_0+\Mm_{X_0}+\frac{1}{I}A)$-MMP$/X$ with scaling of $A$, and $$K_{X_{l}}+B_{l}+\Mm_{X_{l}}+\frac{1}{I}A_{l}$$ 
is nef$/X$. Since $X_0$ is $\Qq$-factorial klt, we may pick $$0\leq \Delta_0\sim_{\mathbb Q}B_0-S_0+\Mm_{X_0}+\frac{1}{I}A$$ such that $(X_0,\Delta_0)$ is klt and $(X_0,S_0+\Delta_0)$ is plt. We let $\Delta_l$ be the image of $\Delta_0$ on $X_l$, then $(X_l,S_l+\Delta_l)$ is plt, so $(X_l,\Delta_l)$ is klt. Then

$$A_l-IQ_l-S_l\sim_{\mathbb Q,X}K_{X_l}+\Delta_l+(I-1)\left(K_{X_l}+B_l+\Mm_{X_l}+\frac{1}{I}A_l\right),$$
so by the relative Kawamata-Viehweg vanishing \cite[Theorem 1-2-5]{KMM87}, $$R^1(f_{l})_*\mathcal{O}_{X_{l}}(A_{l}-IQ_{l}-S_{l})\not=0,$$
a contradiction. We are done with the case when $(X,B,\Mm)$ is a $\Qq$-g-pair.

\medskip

Now we prove the case when $(X,B,\Mm)$ is not necessarily a $\Qq$-g-pair, hence conclude the proof of the theorem. There exist real numbers $r_1,\dots,r_c$ such that $1,r_1,\dots,r_c$ are linearly independent over $\Qq$, $\bm{r}:=(r_1,\dots,r_c)\in\mathbb R^c$, and $\Qq$-linear functions $s_1,\dots,s_p,t_1,\dots,t_q$, such that
$$B=\sum_{i=1}^ps_i(1,\bm{r})B_i,\Mm=\sum_{i=1}^qt_i(1,\bm{r})\Mm_i,$$
where $B_i\geq 0$ are distinct Weil divisors and $\Mm_i$ are nef$/X$ $\bb$-Cartier $\bb$-divisors. Let 
$$B(\bm{v}):=\sum_{i=1}^ps_i(1,\bm{v})B_i\text{ and }\Mm(\bm{v}):=\sum_{i=1}^qt_i(1,\bm{v})\Mm_i,$$
for any $\bm{v}\in\mathbb R^c$. 

Since the coefficients of divisors under adjunction are transformed via $\Qq$-linear functions, there are $\Qq$-linear functions $s'_1,\dots,s'_{p'},t'_1,\dots,t'_{q'}$, distinct Weil divisors $B_{S_i}\geq 0$, and nef$/X$ $\bb$-Cartier $\bb$-divisors $\Mm^S_i$, 
$$B_S(\bm{v}):=\sum_{i=1}^ps_i(1,\bm{v})B_{S,i},\text{ and }\Mm^S(\bm{v}):=\sum_{i=1}^qt_i(1,\bm{v})\Mm^S_i,$$
such that
$$K_{S^\nu}+B_S(\bm{v})+\Mm^S(\bm{v})_{S^\nu}=(K_X+B(\bm{v})+\Mm(\bm{v})_X)|_{S^\nu}$$
for any $\bm{v}\in\mathbb R^c$. Since $$(S^\nu,B_S=B_S(\bm{r}),\Mm^S=\Mm^S(\bm{r}))$$ is lc, 
there exists an open neighborhood $U\ni\bm{r}$ of $\mathbb R^c$ such that
$$(S^\nu,B_S(\bm{v}),\Mm^S(\bm{v}))$$
is lc for any $\bm{v}\in U$. By the $\Qq$-g-pair case,
$$(X,B(\bm{v}),\Mm(\bm{v}))$$
is lc for any $\bm{v}\in U\cap\mathbb Q$. Thus $$(X,B=B(\bm{r}),\Mm=\Mm(\bm{r}))$$
is lc by continuity of log discrepancies.
\end{proof}

\begin{rem}
    We do not need Theorem \ref{thm: inversion of adjunction} in the rest of the paper but we expect it to be useful for future works. We remark that several alternative versions of Theorem \ref{thm: inversion of adjunction} can be found in \cite[Theorems 1.5, 1.6, 6.7]{Fil20} but we cannot apply them directly to prove Theorem \ref{thm: inversion of adjunction} because of the following reasons:
\begin{enumerate}
\item All these theorems require that $(X,B,\Mm)$ is a $\Qq$-g-pair.
\item \cite[Theorems 1.5]{Fil20} requires $S$ to be a minimal lc center and $S$ is projective.
\item \cite[Theorems 1.6]{Fil20} requires that $X$ is projective and $(X,B,\Mm)$ is a $\Qq$-g-pair. Moreover, the potential g-pair structure constructed on $W^\nu$ \cite[Theorems 1.6]{Fil20} is not known to be identical to the g-pair structure constructed in \cite[Theorem 4.5]{HL21b}.
\item \cite[Theorem 6.7]{Fil20} requires that $X$ is $\Qq$-factorial projective klt.
\end{enumerate}  
\end{rem}




\subsection{Boundedness on the number of components}

\begin{prop}\label{prop: bound number of components}
Let $\gamma_0\le1$ be a positive real number, and $b_1,\dots,b_n\in [\gamma_0,1]$ positive real numbers. Let $(X,B=\sum_{i=1}^nb_iB_i+D,\Mm)/X$ be an lc g-pair and $x\in X$ a point, such that $B_i\geq 0$ is a non-zero $\Qq$-Cartier Weil divisor for each $i$, and $D\geq 0$. Suppose that $\bar x\subset\Supp B_i$ for each $i$. Moreover, assume that one of the followings hold:
    \begin{enumerate}
        \item $\Mm$ is NQC$/X$.
        \item There exist a klt g-pair $(X,B',\Mm')/X$.
        \item $\gamma_0=1$ and each $B_i$ is Cartier.
    \end{enumerate}
Then 
$$n\leq\frac{\dim X-\dim\bar x}{\gamma_0}.$$
\end{prop}
\begin{proof}
When $\dim X=1$ the proposition is trivial, so we may assume that $\dim X\geq 2$. We may also assume that $n\geq 1$, otherwise there is nothing left to prove. 

Let $B_{n+1},\dots,B_{n+\dim\bar X}$ be general hyperplane sections on $X$ and let $b_i:=1$ when $i\geq n+1$. Possibly replacing $x$ with $\bar x\cap\cap_{i=1}^{\dim X}H_i$ and $B$ with $\sum_{i=1}^{n+\dim\bar x}b_iB_i+D$, we may assume that $x$ is a closed point. 

First we prove the proposition under conditions (1) or (2). Possibly adding general hyperplane sections which passes through $x$, we may assume that $x$ is an lc center of $(X,B,\Mm)$. Let $E$ be an lc place of $(X,B,\Mm)$ such that  $\Center_XE=x$.
\begin{claim}\label{claim: extract divisor which is ample}
    There exists a contraction $f: Y\rightarrow X$ of $E$, such that $-E$ is ample$/X$.
\end{claim}
\begin{proof}
    If $\Mm$ is NQC$/X$, then the claim follows from \cite[Theorem 1.7]{LX23b}. Otherwise, the claim follows from \cite[Lemma 2.11]{Bir20}.
\end{proof}
\noindent\textit{Proof of Proposition \ref{prop: bound number of components} continued}. By Claim \ref{claim: extract divisor which is ample}, there exists a contraction $f: Y\rightarrow X$ of $E$, such that $-E$ is ample$/X$. We let $B_{i,Y},D_Y,B_Y$ be the strict transforms of $B_{i},D,B$ on $Y$ respectively. Since $x\in\Supp B_i$ for each $i$, $\mult_EB_i>0$ for each $i$, so $B_{i,Y}$ is ample$/X$ for each $i$. We let $E^\nu$ be the normalization of $E$, $\Mm^E:=\Mm|_{E^\nu}$, and let
$$K_{E^\nu}+B_E+\Mm^E_{E^\nu}=(K_Y+B_Y+E+\Mm_Y)|_{E^\nu}.$$
We let $B_{i,E}:=\Supp(B_{i,Y}|_{E^\nu})$ for each $i$. Then for any component $D_{i,j}$ of $B_{i,E}$, we have
$$\mult_{D_{i,j}}B_E=\frac{n_{i,j}-1+\sum_{k=1}^nb_km_{k,i,j}+\gamma_{i,j}}{n_{i,j}}$$
for some real number $\gamma_{i,j}\geq 0$ and non-negative integers $m_{k,i,j}$, such that $m_{i,i,j}\not=0$. Since $(E,B_E,\Mm^E)$ is lc, $\mult_{D_{i,j}}B_E\leq 1$, so $$B_E\geq\sum_{i=1}^nb_iB_{i,E}.$$
Since $B_{i,Y}$ is ample$/X$, $B_{i,Y}|_{E^\nu}$ is ample, so $B_{i,E}$ is big. The proposition under conditions (1) or (2) follows from \cite[Proposition 5.1]{BZ16}.

Now we prove the proposition under condition (3). Let $S$ be the normalization of an irreducible component of $B_1$ such that $x\in S_1$, and let $(S,B_S,\Mm^S)$ be the g-pair induced by the adjunction
$$K_S+B_S+\Mm^S_S:=(K_X+B+D+\Mm_X)|_S.$$
Since $x\in\Supp B_i$ for each $i$, $B_i|_S\not=0$ for any $i\geq 2$. Since $B_i$ is Cartier and $(S,B_S,\Mm^S)$ is lc, $B_i|_S=\Supp(B_i|_S)$ for any $i\geq 2$, and 
$$B_S\geq\sum_{i=2}^nB_i|_S.$$
Since each $B_i|_S$ is Cartier, by induction on $\dim X$, we have $n\leq\dim X$ and the proposition follows.
\end{proof}


\section{Stability of generalized pairs}\label{sec: stability gpair}

In this section, we discuss the stability properties of g-pairs. We will define the concepts of generically lc, Property $(*)$ BP (semi-)stable, and log stable for g-pairs, and then study the basic properties of g-pairs satisfying these properties. This section is parallel to \cite[Section 2]{ACSS21}.

\subsection{Toroidal generalized pairs}

\begin{defn}[{cf. \cite[Definition 2.1]{ACSS21}}]\label{defn: toroidal g-pairs}
Let $(X,\Sigma_X,\Mm)/U$ be a g-pair. We say that $(X,\Sigma_X,\Mm)$ is \emph{toroidal} if $\Sigma_X$ is a reduced divisor, $\Mm$ descends to $X$, and for any closed point $x\in X$, there exists a toric variety $X_{\sigma}$, a closed point $t\in X_{\sigma}$, and an isomorphism of complete local algebras 
$$\phi_x:\widehat{\mathcal{O}}_{X,x}\cong\widehat{\mathcal{O}}_{X_\sigma,t}$$
such that the ideal of $\Sigma_X$ maps to the invariant ideal of $X_{\sigma}\backslash T_{\sigma}$, where $T_\sigma\subset X_\sigma$ is the maximal torus of $X_{\sigma}$. Any such $(X_\sigma, t)$ will be called as a \emph{local model} of $(X,\Sigma_X,\Mm)$ at $x\in X$.

Let $(X,\Sigma_X,\Mm)/U$ and $(Z,\Sigma_Z,\Mm^Z)/U$ be toroidal g-pairs and $f: X\rightarrow Z$ a surjective morphism$/U$. We say that $f: (X,\Sigma,\Mm)\rightarrow (Z,\Sigma,\Mm^Z)$ is \emph{toroidal}, if for every closed point $x\in X$, there exist a local model $(X_\sigma,t)$ of $(X,\Sigma_X,\Mm)$ at $x$, a local model $(Z_{\tau},s)$ of $(Z,\Sigma_Z,\Mm^Z)$ at $z:=f(x)$, and a toric morphism $g: X_\sigma\to Z_{\tau}$, so that the diagram of algebras commutes.
\begin{center}$\xymatrix{
    \widehat{\mathcal{O}}_{X,x}\ar@{->}[r]^{\cong}  &     \widehat{\mathcal{O}}_{X_{\sigma},t} \\
     \widehat{\mathcal{O}}_{Z,z}\ar@{->}[r]^{\cong}\ar@{->}[u] & \widehat{\mathcal{O}}_{Z_{\tau},s}\ar@{->}[u]
}$
\end{center}
Here the vertical maps are the algebra homomorphisms induced by $f$ and $g$ respectively. 
\end{defn}

\begin{defthm}[{\cite[Definition-Theorem 6.5]{LLM23}, \cite[Theorem 2.2]{ACSS21}}]\label{defthm: weak ss reduction}
Let $X$ be a normal quasi-projective variety, $X\rightarrow U$ a projective morphism, $X\rightarrow Z$ a contraction, $B$ an $\Rr$-divisor on $X$, $\Mm$ a nef$/U$ $\bb$-divisor on $X$, $D_1,\dots,D_m$ prime divisors over $X$, and $D_{Z,1},\dots,D_{Z,n}$ prime divisors over $Z$. Then there exist a toroidal g-pair  $(X',\Sigma_{X'},\Mm)/U$, a log smooth pair $(Z',\Sigma_{Z'})$, and a commutative diagram
 \begin{center}$\xymatrix{
X'\ar@{->}[r]^{h}\ar@{->}[d]_{f'}& X\ar@{->}[d]^{f}\\
Z'\ar@{->}[r]^{h_Z} & Z\\
}$
\end{center}
satisfying the following.
\begin{enumerate}
\item $h$ and $h_Z$ are projective birational morphisms.
\item $f': (X',\Sigma_{X'},\Mm)\rightarrow (Z',\Sigma_{Z'})$ is a toroidal contraction.
\item $\Supp(h^{-1}_*B)\cup\Supp\Exc(h)$ is contained in $\Supp\Sigma_{X'}$.
\item $X'$ has at most toric quotient singularities.
\item $f'$ is equi-dimensional.
\item $\Mm$ descends to $X'$.
\item $X'$ is $\Qq$-factorial klt.
\item The center of each $D_i$ on $X'$ and the center of each $D_{Z,i}$ on $Z'$ are divisors.
\end{enumerate}
We call any such $f': (X',\Sigma_{X'},\Mm)\rightarrow (Z',\Sigma_{Z'})$ (associated with $h$ and $h_Z$) which satisfies (1-7) an \emph{equi-dimensional model} of $f: (X,B,\Mm)\rightarrow Z$. 
\end{defthm}
\begin{proof}
Possibly replacing $X$ and $Z$ with high models, we may assume that $\Mm$ descends to $X$, each $D_i$ is a divisor on $X$, and each $D_{Z,i}$ is a divisor on $Z$. Now the theorem follows from \cite[Theorem 2.2]{ACSS21}, which in turn follows from \cite[Theorem 2.1 and Proposition 4.4]{AK00}. We also refer the reader to \cite[Theorem B.6]{Hu20} for a more detailed explanation.
\end{proof}

\begin{rem}
In Definition-Theorem \ref{defthm: weak ss reduction}, it is important to note that the contraction $X\rightarrow Z$ may not necessarily be over $U$. This kind of phenomenon will appear throughout the rest of the paper.
\end{rem}

\subsection{Discrimiant and moduli parts of generalized pairs}

\begin{defn}[Birationally equivalent morphisms, cf. {\cite[Page 4, Paragraph 2]{ACSS21}}]
Let $f: X\rightarrow Z$ and $f': X'\rightarrow Z'$ be surjective morphisms between normal varieties. We say that $f$ and $f'$ are \emph{birationally equivalent} if there exist birational maps $h: X\dashrightarrow X'$ and $h_Z: Z\dashrightarrow Z'$ such that $f'\circ h=h_Z\circ f$.
\end{defn}

\begin{defn}[Generically lc, cf. {\cite[2.2. Discriminant and Moduli Part]{ACSS21}}]\label{defn: generically lc}
Let  $(X,B,\Mm)/U$ be a g-sub-pair and $f: X\rightarrow Z$ a contraction. We say that $(X,B,\Mm)$ is \emph{generically (sub-)lc$/Z$} if $(X,B,\Mm)$ is (sub-)lc over the generic point of $Z$. Note that $f$ may not be a contraction$/U$. We remark that we will not use the notation ``GLC" for ``generically lc" as in \cite{ACSS21} since GLC also stands for ``generalized lc" in many references.
\end{defn}

\begin{defn}[Crepant generalized pairs, cf. {\cite[Definition 2.3]{ACSS21}}]\label{defn: crepant g-pairs}
Let $(X,B,\Mm)/U$ and $(X',B',\Mm')/U$ be two g-sub-pairs and $f: X\rightarrow Z$, $f': X'\rightarrow Z'$ two contractions. We say that $(X,B,\Mm)$ and $(X',B',\Mm')$ are \emph{crepant over the generic point of $Z$} if we have the following commutative diagram
 \begin{center}$\xymatrix{
& W\ar@{->}[ld]_{p}\ar@{->}[dr]^{q} &\\
X\ar@{.>}[rr]^{h}\ar@{->}[d]_{f}& & X'\ar@{->}[d]^{f'}\\
Z\ar@{.>}[rr]^{h_Z} & & Z'\\
}$
\end{center}
satisfying the following. Let $$K_W+B_W+\Mm_W:=p^*(K_X+B+\Mm_X)$$ and $$K_W+B'_W+\Mm'_W:=q^*(K_{X'}+B'+\Mm'_X).$$ Then:
\begin{enumerate}
    \item $h$ and $h_Z$ are birational maps. In particular, $f$ and $f'$ are birationally equivalent.
    \item $\Mm$ and $\Mm'$ descends to $W$.
    \item $B_W-B'_W$ and $\Mm_W-\Mm'_W$ are vertical$/Z$.
\end{enumerate}
\end{defn}



\begin{defn}[Discrimiant and moduli parts, cf. {\cite[Definition 2.3]{ACSS21}}]
Let $(X,B,\Mm)/U$ be a g-sub-pair and $f: X\rightarrow Z$ a contraction such that $(X,B,\Mm)$ is generically sub-lc$/Z$. In the following, we fix a choice of $K_X$ and a choice of $K_Z$, and suppose that for any birational morphism $g: \bar X\rightarrow X$ and $g_Z: \bar Z\rightarrow Z$, $K_{\bar X}$ and $K_{\bar Z}$ are chosen as the Weil divisors such that $g_*K_{\bar X}=K_X$ and $(g_Z)_*K_{\bar Z}=K_Z$.

Let $f': X'\rightarrow Z'$ be any contraction that is birationally equivalent to $f$ such that the induced birational maps $h: X'\dashrightarrow X$ and  $h_Z: Z'\dashrightarrow Z$ are morphisms and $Z'$ is $\Qq$-factorial. We let 
$$K_{X'}+B'+\Mm_{X'}:=h^*(K_X+B+\Mm_X).$$ 
For any prime divisor $D$ on $Z'$, we define
$$b_D(X',B',\Mm;f):=1-\sup\left\{t\mid \left(X',B'+tf'^*D,\Mm\right)\text{ is sub-lc over the generic point of } D\right\}.$$
Since being sub-lc is a property that is preserved under crepant transformations, $b_D(X,B,\Mm;f)$ is independent of the choices of $X'$ and $Z'$ and is also independent of $U$.

Since $(X,B,\Mm)$ is generically sub-lc$/Z$, $(X',B',\Mm)$ is generically sub-lc$/Z$, so we may define
$$B_{Z'}:=\sum_{D\text{ is a prime divisor on }Z'}b_D(X,B,\Mm;f)D.$$
and $$N_{X'}:=K_{X'}+B'+\Mm_{X'}-f'^*(K_{Z'}+B_{Z'}).$$
We call $B_{Z'}$ and $N_{X'}$ the \emph{discriminant part} and \emph{trace moduli part} of $f': (X',B',\Mm)\rightarrow Z'$ respectively, and call $B_Z:=(h_Z)_*B_{Z'}$ and $N_X:=h_*B$ the \emph{discriminant part} and \emph{trace moduli part} of $f: (X,B,\Mm)\rightarrow Z$ respectively.

By construction, there exist two $\bb$-divisors $\Bb$ on $Z$ and $\Nn$ on $X$, such that for any contraction $f'': X''\rightarrow Z''$ that is birationally equivalent to $f$ such that the induced birational maps $h': X''\dashrightarrow X'$ and  $h_{Z'}: Z''\dashrightarrow Z'$ are morphisms and $Z''$ is $\Qq$-factorial, $\Bb_{Z''}$ is the discriminant part of $f'': (X'',B'',\Mm)\rightarrow Z''$, and $\Nn_{X''}$ is the trace moduli part of  $f'': (X'',B'',\Mm)\rightarrow Z''$, where 
$$K_{X''}+B''+\Mm_{X''}:=h'^*(K_{X'}+B'+\Mm_{X'}).$$ 
We call $\Nn$ the \emph{moduli part} of $f: (X,B,\Mm)\rightarrow Z$ and $\Bb$ the \emph{discriminant $\bb$-divisor} of $f: (X,B,\Mm)\rightarrow Z$. By construction, $\Bb$ is uniquely determined and $\Nn$ is uniquely determined for any fixed choices of $K_X$ and $K_Z$.
\end{defn}

\subsection{BP stability of generalized pairs}

\begin{defn}[BP (semi-)stable, boundary property,  cf. {\cite[Definition 2.5]{ACSS21}}]\label{defn: bpstable}
Let $(X,B,\Mm)/U$ be a g-sub-pair and $f: X\rightarrow Z$ a contraction, such that $(X,B,\Mm)$ is generically sub-lc$/Z$. Let $\Bb$ be the discriminant $\bb$-divisor of $f: (X,B,\Mm)\rightarrow Z$.

We say that $f: (X,B,\Mm)\rightarrow Z$ is BP stable (resp. BP semi-stable) if $K_Z+\Bb_Z$ is $\Rr$-Cartier, and for any birational morphism $h_Z: Z'\rightarrow Z$,
$$h_Z^*(K_Z+\Bb_Z)=\text{(resp. }\geq\text{)} K_{Z'}+\Bb_{Z'}.$$
If $f: (X,B,\Mm)\rightarrow Z$ is BP stable (resp. BP semi-stable), then we say that $(X,B,\Mm)$ is BP stable (resp. BP semi-stable) over $Z$.
\end{defn}

\begin{lem}[cf. {\cite[Remark 2.6(2)]{ACSS21}}]\label{lem: bp stable implies descend}
Let $(X,B,\Mm)/U$ be a g-sub-pair and $f: X\rightarrow Z$ a contraction, such that $f: (X,B,\Mm)\rightarrow Z$ is BP stable. Let $B_Z$ and $\Nn$ be the discriminant part and the moduli part of $f: (X,B,\Mm)\rightarrow Z$ respectively. Then:
\begin{enumerate}
    \item $\Nn_X=K_X+B+\Mm_X-f^*(K_Z+B_Z).$
    \item $\Nn$ descends to $X$.
\end{enumerate}
\end{lem}
\begin{proof}
For any $f': X'\rightarrow Z'$ that is birationally equivalent to $f$, such that the induced birational maps $h: X'\dashrightarrow X$ and $h_Z: Z'\dashrightarrow Z$ are morphisms, we have $K_{Z'}+B_{Z'}=h_Z^*(K_{Z}+B_{Z})$. Thus
\begin{align*}
    \Nn_{X'}&=K_{X'}+B'+\Mm_{X'}-f'^*(K_{Z'}+B_{Z'})=h^*(K_X+B+\Mm_X-f^*(K_Z+B_Z)),
\end{align*}
where $K_{X'}+B'+\Mm_{X'}:=h^*(K_X+B+\Mm_X)$, and $B_{Z'}$ is the discriminant part of $f': (X',B',\Mm)\rightarrow Z'$. The lemma immediately follows.
\end{proof}

\subsection{Property \texorpdfstring{$(*)$}{} generalized pairs}

\begin{lem}[cf. {\cite[Lemma 2.12]{ACSS21}}]\label{lem: dimension of full lc rank}
Let $(X,B,\Mm)/U$ be a g-pair and $f: X\rightarrow Z$ a contraction. Let $d:=\dim X$ and $m:=\dim Z$. Let $z\in Z$ be a closed point, $D_1,\dots,D_m\geq 0$ Cartier divisors on $Z$, such that $z\in\Supp D_i$ for each $i$ and $(X,B+\sum_{i=1}^mf^*D_i,\Mm)$ is lc over $f^{-1}(z)$. Then the dimension of any irreducible component of $f^{-1}(z)$ is $d-m$.
\end{lem}

\begin{proof}
For any irreducible component $G$ of $f^{-1}(z)$, let $H_1,\dots,H_{\dim G}$ be general very ample divisors on $X$, $V:=\cap_{i=1}^{\dim G}H_i$, and $(V,B_V,\Mm^V)/U$ the g-pair induced by the adjunction $$K_V+B_V+\Mm^V_V:=(K_X+B+\Mm_X)|_V.$$ Then $(V,B_V+\sum_{i=1}^mf^*D_i|_V,\Mm^V)$ is lc, $G\cap V$ is a closed point, and $A_i:=f^*D_i|_V$ is Cartier and contains $G\cap V$ for any $i$. By Proposition \ref{prop: bound number of components}, $m\leq\dim V=d-\dim G$. Thus $\dim G\leq d-m$. Therefore, the dimension of  any irreducible component of $f^{-1}(z)$ is $\leq d-m$. By \cite[Exercise II 3.22 (a)]{Har77}, the dimension of  any irreducible component of $f^{-1}(z)$ is $\geq d-m$. The lemma immediately follows.
\end{proof}

\begin{defn}[Property $(*)$ generalized pairs, cf. {\cite[Definition 2.13]{ACSS21}}]\label{defn: property *}
Let $(X,B,\Mm)/U$ be a g-sub-pair and $f: X\rightarrow Z$ a contraction. We say that $f: (X,B,\Mm)\rightarrow Z$ satisfies \emph{Property $(*)$} if there exists a reduced divisor $\Sigma_Z$ on $Z$ satisfying the following.
\begin{enumerate}
\item $(Z,\Sigma_Z)$ is log smooth. In particular, $Z$ is smooth.
\item The vertical$/Z$ part $B^v$ of $B$ is equal to $f^{-1}(\Sigma_Z)$. In particular, $B^v$ is reduced and $\Sigma_Z$ is the image of $B^v$ on $Z$.
\item For any closed point $z\in Z$ and any reduced divisor  $\Sigma\ge \Sigma_Z$ on $Z$ such that  $(Z,\Sigma)$ is log smooth near $z$, $(X,B+f^*(\Sigma-\Sigma_Z),\Mm)$ is sub-lc over a neighborhood of $z$.
\end{enumerate}
%If $f$ is clear from the context and $f: (X,B,\Mm)\rightarrow Z$ satisfies Property $(*)$, then we also say that $(X,B,\Mm)/Z$ satisfies \emph{Property $(*)$}. 
By (2), $\Sigma_Z$ is uniquely determined by $f: (X,B,\Mm)\rightarrow Z$. We will temporarily call $\Sigma_Z$ the \emph{base divisor} associated to $f: (X,B,\Mm)\rightarrow Z$. In Lemma \ref{lem: basic property (*) gpair} below, we will show that $\Sigma_Z$ is actually the discriminant part of $f: (X,B,\Mm)\rightarrow Z$.
\end{defn}



\begin{lem}[cf. {\cite[Lemma 2.14]{ACSS21}}]\label{lem: basic property (*) gpair}
Let $(X,B,\Mm)/U$ be a g-sub-pair and $f: X\rightarrow Z$ a contraction such that $f: (X,B,\Mm)\rightarrow Z$ satisfies Property $(*)$. Let $\Sigma_Z$ be the base divisor associated to $f: (X,B,\Mm)\rightarrow Z$. Then:
\begin{enumerate}
\item $(X,B,\Mm)$ is sub-lc.
\item $\Sigma_Z$ is the discriminant part of $f: (X,B,\Mm)\rightarrow Z$.
\item If $B\geq 0$, then $f$ is equi-dimensional over $Z\backslash\Supp\Sigma_Z$.
\end{enumerate}
\end{lem}
\begin{proof}
 (1) For any closed point $z\in Z$, we pick $\Sigma:=\Sigma_Z$. By Definition \ref{defn: property *}(3), $(X,B,\Mm)$ is sub-lc over a neighborhood of $z$. Thus $(X,B,\Mm)$ is sub-lc.
 
(2) Let $B_Z$ be the discriminant part of $f: (X,B,\Mm)\rightarrow Z$. Since the vertical part of $B$ coincides with $f^{-1}(\Sigma_Z)$, $B_Z\geq\Sigma_Z$.

 Let $P$ be a prime divisor on $Z$ such that $P\not\subset\Supp\Sigma_Z$, and let $z$ be a general closed point in $P$. Then $(Z,\Sigma_Z+P)$ is log smooth at $z$. By Definition \ref{defn: property *}(3), $(X,B+f^*P,\Mm)$ 
is sub-lc over a neighborhood of $z$. Thus
$$\sup\{t\mid (X,B+tf^*P,\Mm)\text{ is sub-lc over the generic point of }P\}=1,$$
so $P\not\subset\Supp B_Z$. Thus $\Sigma_Z=\Supp\Sigma_Z\geq\Supp B_Z$. Since $(X,B,\Mm)$ is sub-lc, $\Supp B_Z\geq B_Z$. This implies (2).

(3) Let $d:=\dim X$ and $m:=\dim Z$. Let $z\in Z\backslash\Supp\Sigma_Z$ be a closed point, and let $\Sigma_1,\dots,\Sigma_m$ be general hyperplane sections on $Z$ such that $z\in \Sigma_i$ for any $i$. Then $(Z,\Sigma_Z+\sum_{i=1}^m\Sigma_i)$ is log smooth at $z$. By Definition \ref{defn: property *}(3), $(X,B+\sum_{i=1}^m f^*\Sigma_i,\Mm)$ is lc over a neighborhood of $z$. By Lemma \ref{lem: dimension of full lc rank}, the dimension of any irreducible component of $f^{-1}(z)$ is $d-m$. This implies (3).
\end{proof}

\begin{lem}[cf. {\cite[Lemma 2.15]{ACSS21}}]\label{lem: lcc property* pullback} 
Let $(X,B,\Mm)/U$ be a g-pair and $f: X\rightarrow Z$ a contraction such that $f: (X,B,\Mm)\rightarrow Z$ satisfies Property $(*)$. Let $\Sigma_Z$ be the discriminant part of $f: (X,B,\Mm)\rightarrow Z$, and let $\Sigma\geq\Sigma_Z$ be a reduced divisor on $Z$, such that $(Z,\Sigma)$ is log smooth.

Consider $\Sigma$ as a reduced subscheme of $Z$. Then for any irreducible stratum $V$ of $\Sigma$, any irreducible component of $f^{-1}(V)$ is an lc center of $(X,B+f^{-1}(\Sigma-\Sigma_Z),\Mm)$.
\end{lem}
\begin{proof} 
Let $k:=\dim Z-\dim V$. Since $(Z,\Sigma)$ is log smooth, there exist irreducible components $\Sigma_1,\dots, \Sigma_k$ of $\Sigma$ such that $V=\bigcap_{i=1}^k \Sigma_i$. By Definition \ref{defn: property *}(3), for any $i$ and any general closed point $z\in\Supp\Sigma_i$, $(X,B+f^*(\Sigma-\Sigma_Z),\Mm)$ is sub-lc over a neighborhood of $z$. Thus any irreducible component of $f^{-1}(\Sigma_i)$ is an lc center of $(X,B+f^*(\Sigma-\Sigma_Z),\Mm)$. Therefore, any irreducible component of $f^{-1}(V)$ is an intersection of lc centers of $(X,B+f^*(\Sigma-\Sigma_Z),\Mm)$. The lemma follows from Lemma \ref{lem: intersection of lc center gpair}.
\end{proof}


\begin{prop}[cf. {\cite[Proposition 2.16]{ACSS21}}]\label{prop: weak ss satisfies *}
Let $(X,\Sigma_X,\Mm)/U$ be a toroidal g-pair, $(Z,\Sigma_Z)$ a log smooth pair, and $f: (X,\Sigma_X,\Mm)\rightarrow (Z,\Sigma_Z)$ a toroidal morphism. Let $(X,B,\Mm)/U$ be a g-sub-pair such that $\Supp B\subset \Supp\Sigma_X$, $(X,B,\Mm)$ is generically sub-lc$/Z$, and the vertical$/Z$ part of $B$ is equal to  $f^{-1}(\Sigma_Z)$. Then $f: (X,B,\Mm)\rightarrow Z$ satisfies Property $(*)$.
\end{prop}
\begin{proof}
Since $(X,B,\Mm)$ is generically sub-lc$/Z$, $\Supp B\subset \Supp\Sigma_X$, and the vertical$/Z$ part of $B$ is equal to  $f^{-1}(\Sigma_Z)$, $(X,B,\Mm)$ is sub-lc. Since $\Mm$ descends to $X$, by \cite[Proposition 2.16]{ACSS21}, $f: (X,B)\rightarrow Z$ satisfies Property $(*)$. By Definition \ref{defn: property *}, $f: (X,B,\Mm)\rightarrow Z$ satisfies Property $(*)$.
\end{proof}

The following result indicates that we can always get Property $(*)$ g-pairs by taking equi-dimensional models.

\begin{prop}[cf. {\cite[Proposition 2.17]{ACSS21}}]\label{prop: weak ss imply *}
Let $(X,B,\Mm)/U$ be a g-sub-pair and $f: X\rightarrow Z$ a contraction, such that $(X,B,\Mm)$ is generically sub-lc$/Z$. Let $f': (X',\Sigma_{X'},\Mm)\rightarrow (Z',\Sigma_{Z'})$ be an equi-dimensional model of $f: (X,B,\Mm)\rightarrow Z$, associated with $h: X'\rightarrow X$ and $h_Z: Z'\rightarrow Z$. Then there exist two $\Rr$-divisors $B'$ and $F$ on $X'$ satisfying the following.
\begin{enumerate}
\item $\Supp B'\subset\Sigma_{X'}$ and $\Supp F\subset\Sigma_{X'}$.
\item $F$ is vertical$/Z'$ and
$$K_{X'}+B'+\Mm_{X'}=h^*(K_X+B+\Mm_X)+F.$$
\item $(X',B',\Mm)$ and $(X,B,\Mm)$ are crepant over the generic point of $Z$.
\item If $(X,B,\Mm)$ is sub-lc, then $F\geq 0$.
\item If $(X,B,\Mm)$ is generically sub-lc$/Z$, then $f': (X',B',\Mm)\rightarrow Z'$ satisfies Property $(*)$.
\end{enumerate}
\end{prop}
\begin{proof} 
Possibly adding components to $\Sigma_{Z'}$, we may assume that $\Sigma_{Z'}$ coincides with the image of the vertical$/Z'$ part of $\Sigma_{X'}$. We let $G:=f^{-1}(\Sigma_{Z'})$ and
$$K_{X'}+\tilde B'+\Mm_{X'}:=h^*(K_X+B+\Mm_X),$$
then $G\subset\Supp\Sigma_{X'}$ and $\Supp\tilde B'\subset\Supp\Sigma_{X'}$. We define $B'$ to be the unique $\Rr$-divisor on $X'$ satisfying the following: for any prime divisor $D$ on $X'$,
\begin{itemize}
\item if $D$ is not a component of $\Supp\tilde B'$ nor $G$, then $\mult_DB'=0$,
\item if $D$ is a component of $G$, then $\mult_DB'=1$, and
\item if $D$ is a component of $\tilde B'$ but is not a component of $G$, then $\mult_DB'=\mult_D\tilde B'$.
\end{itemize}
Since $G$ is the vertical$/Z'$ part of $\Sigma_{X'}$ and
$$\Supp B'\subset\Supp G\cup\Supp\tilde B'\subset\Supp\Sigma_{X'},$$
the vertical$/Z'$ part of $B'$ is equal to $G=f'^{-1}(\Sigma_{Z'})$. 

We define $F:=B'-\tilde B'$. We show that $B'$ and $F$ satisfy our requirements.

(1) holds immediately by our construction. 

(2) For any component $D$ of $\Supp F$, by construction, $\mult_DF\not=0$ only if $D$ is a component of $G$. Thus $F$ is vertical$/Z'$.

(3) By (2), $(X',B',\Mm)$ and $(X,B,\Mm)$ are crepant over the generic point of $Z$.

(4) For any component $D$ of $G$, $\mult_DF=1-\mult_D\tilde B'$. Therefore, if $(X,B,\Mm)$ is sub-lc, then $\mult_D\tilde B'\leq 1$, so $\mult_DF\geq 0$. Thus $F\geq 0$.

(5) Since $(X,B,\Mm)$ is sub-lc over the generic point of $Z$, $(X',B',\Mm)$ is sub-lc over the generic point of $Z$. By Proposition \ref{prop: weak ss satisfies *}, $f': (X',B',\Mm)\rightarrow Z'$ satisfies Property $(*)$.
\end{proof}

The following proposition shows that Property $(*)$ is preserved under any sequence of steps of an MMP.
\begin{prop}[cf. {\cite[Proposition 2.18]{ACSS21}}]\label{prop: MMP preserves *}
Let $(X,B,\Mm)/U$ be an lc g-pair and $f: X\rightarrow Z$ a contraction, such that $f: (X,B,\Mm)\rightarrow Z$ satisfies Property $(*)$. Let $\phi: (X,B,\Mm)\dashrightarrow (Y,B_Y,\Mm)$ be a sequence of steps of a $(K_X+B+\Mm_X)$-MMP$/Z$ and $f_Y: Y\rightarrow Z$ the induced morphism. Assume that $\phi$ is also a sequence of steps of a $(K_X+B+\Mm_X)$-MMP$/U$. Then:
\begin{enumerate}
    \item $f_Y: (Y,B_Y,\Mm)\rightarrow Z$ satisfies Property $(*)$, and the discriminant part of $f_Y: (Y,B_Y,\Mm)\rightarrow Z$ is equal to the discriminant part of $f: (X,B,\Mm)\rightarrow Z$.
    \item For any closed point $z\in Z$, $\phi^{-1}$ is an isomorphism near the generic point of any irreducible component of $f_Y^{-1}(z)$.
    \item If $f$ is equi-dimensional, then $f_Y$ is equi-dimensional.
\end{enumerate}
\end{prop}
\begin{proof}
Without loss of generality, we may assume that $\phi$ is a step of a $(K_X+B+\Mm_X)$-MMP$/Z$. 

(1) Let $\Sigma_Z$ be the discriminant part of $f: (X,B,\Mm)\rightarrow Z$. By definition, $(Z,\Sigma_Z)$ is log smooth. 

Since the vertical$/Z$ part of $B$ is equal to $f^{-1}(\Sigma_Z)$ and $\phi$ does not extract any divisor, the vertical$/Z$ part of $B_Y$ is equal to $\phi\circ f^{-1}(\Sigma_Z)=f_Y^{-1}(\Sigma_Z)$.

For any reduced divisor $\Sigma\geq\Sigma_Z$ on $Z$, $(X,B+f^*(\Sigma-\Sigma_Z),\Mm)/U$ is lc. Since $\phi$ is a step of a $(K_X+B+\Mm_X)$-MMP$/Z$, $\phi$ is also a step of a $(K_X+B+f^*(\Sigma-\Sigma_Z)+\Mm_X)$-MMP$/Z$. Thus 
$$(Y,B_Y+\phi_*f^*(\Sigma-\Sigma_Z)=B_Y+f_Y^*(\Sigma-\Sigma_Z),\Mm)$$
is lc.

Therefore, $f_Y: (Y,B_Y,\Mm)\rightarrow Z$ satisfies Property $(*)$. By Lemma \ref{lem: basic property (*) gpair}(2), $\Sigma_Z$ is the discriminant part of $f_Y: (Y,B_Y,\Mm)\rightarrow Z$.

(2) Possibly shrinking $Z$ to a neighborhood of $z$, there exists a reduced divisor $\Sigma\geq\Sigma_Z$ on $Z$, such that $(Z,\Sigma)$ is log smooth and $z$ is a stratum of $\Sigma$. By Lemma \ref{lem: lcc property* pullback}, any irreducible component of $f_Y^{-1}(z)$ is an lc center of $(Y,B_Y+f_Y^*(\Sigma-\Sigma_Z),\Mm)$. By Definition \ref{defn: property *}(3), $(X,B+f^*(\Sigma-\Sigma_Z),\Mm)$ is lc. For any irreducible component $G$ of $f^{-1}(z)$, let $D_G$ be an lc place of $(Y,B_Y+f_Y^*(\Sigma-\Sigma_Z),\Mm)$ over the generic point of $G$. Then
$$0\leq a(D_G,X,B+f^*(\Sigma-\Sigma_Z),\Mm)\leq a(D_G,Y,B_Y+f_Y^*(\Sigma-\Sigma_Z),\Mm)=0.$$
Thus
$$a(D_G,X,B+f^*(\Sigma-\Sigma_Z),\Mm)=a(D_G,Y,B_Y+f_Y^*(\Sigma-\Sigma_Z)=0,$$
so $\phi^{-1}$ is an isomorphism near the generic point of $G$.

(3) It immediately follows from (2).
\end{proof}




\part{Cone theorem and MMP for algebraically integrable foliations}\label{part:cone}

\section{Precise adjunction formula for algebraically integrable foliations}\label{sec: adjunction}

In this section, we will establish a \emph{precise} adjunction formula for foliations that are induced by a morphism. By saying ``precise", we mean that the adjunction formulas we provide not only preserve the log canonicity of the the generalized foliated quadruple, but also give a nice characterization of the coefficients of the boundary. More precisely, in this section we will prove the following theorem under the additional assumption that $\Ff$ is induced by a contraction:


\begin{thm}[Precise adjunction formula for generalized foliated quadruples]\label{thm: precise adj gfq}
Let $m$ an $n$ be two non-negative integers, and $b_1,\dots,b_m,r_1,\dots,r_n$ non-negative real numbers. Let $(X,\Ff,B,\Mm)/U$ be a generalized foliated quadruple such that $\Ff$ is algebraically integrable. Let $S,B_1,\dots,B_m$ be distinct prime divisors on $X$, and let $\Mm_1,\dots,\Mm_n$ be nef$/U$ $\bb$-Cartier $\bb$-divisors on $X$. Suppose that
$$B=\epsilon_{\Ff}(S)S+\sum_{j=1}^mb_jB_j\text{ and }\Mm=\sum_{k=1}^n r_k\Mm_k.$$
Let $S^\nu\rightarrow S$ the normalization of $S$, $\Ff_S$ the restricted foliation of $\Ff$ on $S^\nu$ (see Definition \ref{defn: restricted foliation}), and $\Mm_{k}^S:=\Mm_k|_{S^\nu}$ for any $k$. Then there exist prime divisors $T_1,\dots,T_l,C_1,\dots,C_q$ on $S^\nu$, positive integers $w_1,\dots,w_q$, and non-negative integers $\{w_{i,j}\}_{1\leq i\leq q,1\leq j\leq m}$ and $\{v_{i,k}\}_{1\leq i\leq q, 1\leq k\leq n}$, such that for any  real numbers $b_1',\dots,b_m'$ and $r_1',\dots,r_n'$, we have the following. 

Let $B':=\epsilon_{\Ff}(S)S+\sum_{j=1}^mb_j'B_j$ and $\Mm':=\sum_{k=1}^nr_k'\Mm_{k}$. Then:
\begin{enumerate}
    \item $$K_{\Ff_S}+B'_{S}+\Mm'^{S}_{S^\nu}=\left(K_\Ff+B'+\Mm'_X\right)|_{S^\nu},$$
    where 
    $$B'_{S}:=\sum_{i=1}^lT_i+\sum_{i=1}^q\frac{w_i-1+\sum_{j=1}^mw_{i,j}b_j'+\sum_{k=1}^nv_{i,k}r_k'}{w_i}C_i$$
and
$$\Mm'^{S}:=\sum_{k=1}^nr_k'\Mm_k^S=\Mm'|_{S^\nu}.$$
\item If $(X,\Ff,B',\Mm')$ is lc near $S$, then $(S^\nu,\Ff_S,B'_{S},\Mm'^{S})$ is lc.
\end{enumerate}
\end{thm}

The complete proof of Theorem \ref{thm: precise adj gfq} will be provided in Section \ref{sec: cone} as a consequence of the cone theorem and the existence of ACSS modifications.


\subsection{Preliminaries for algebraically integrable foliations}\label{subsec: ai foliation}

In this subsection, we recall some basic knowledge of the theory of algebraically integrable foliations that will be used in the rest part of the paper.

\begin{defn}[Algebraically integrable foliations, {cf. \cite[3.1]{ACSS21}}]\label{defn: algebraically integrable}
Let $X$ be a normal quasi-projective variety and $\Ff$ a foliation on $X$. We say that $\Ff$ is an \emph{algebraically integrable foliation} if there exists a dominant map $f: X\dashrightarrow Y$ to a quasi-projective variety $Y$ such that $\Ff=f^{-1}\Ff_Y$, where $\Ff_Y$ is a foliation by points. In this case, we say that $\Ff$ is \emph{induced by $f$}.
\end{defn}


\begin{defn}[Transverse]
  Let $X$ be a normal variety, $\Ff$ a foliation on $X$, and $V\subset X$ a subvariety. For any point $x\in V$, we say that $V$ is \emph{transverse} to $\Ff$ at $x$ if $x\not\in\Sing(X)\cup\Sing(\Ff)\cup\Sing(V)$, and for any analytic neighborhood $U$ of $x$, $T_{V}|_U\rightarrow T_X|_U$ does not factor through $T_{\Ff}|_U$. We say that $V$ is \emph{everywhere transverse} to $\Ff$ if $V$ is transverse to $\Ff$ at $x$ for any $x\in V$ (in particular, $V$ is smooth and $V$ does not intersect $\Sing(X)$ or $\Sing(\Ff)$). We say that $V$ is \emph{generically transverse} to $\Ff$ if $V$ is transverse to $\Ff$ at the generic point $\eta_V$ of $V$.
\end{defn}

\begin{defn}[Tangent, {cf. \cite[Section 3.4]{ACSS21}}]\label{defn: tangent to foliation}
 Let $X$ be a normal variety, $\Ff$ a foliation on $X$, and $V\subset X$ a subvariety. Suppose that $\Ff$ is a foliation induced by a dominant rational map $X\dashrightarrow Z$. We say that $V$ is \emph{tangent} to $\Ff$ if there exists a birational morphism $\mu: X'\rightarrow X$, an equi-dimensional contraction $f': X'\rightarrow Z$, and a subvariety $V'\subset X'$, such that
    \begin{enumerate}
    \item $\mu^{-1}\Ff$ is induced by $f'$, and
        \item $V'$ is contained in a fiber of $f'$ and $\mu(V')=V$.
    \end{enumerate}
\end{defn}

\begin{defn}[Tangency of general fibers]
 Let $X$ be a normal variety, $\Ff$ a foliation on $X$, and $f: X\dashrightarrow Z$ a dominant map. We say that the general fibers of $f$ are \emph{tangent to $\Ff$} if for any general closed point $x$ on a general fiber $F$ of $f$, the linear subspace $\Ff_x\subset T_{X,x}$ determined by the inclusion $\Ff\subset T_X$ contains $T_{F,x}$.
\end{defn}


\begin{defn}[Restricted foliation]\label{defn: restricted foliation}
Let $X$ be a normal variety, $\Ff$ a foliation on $X$, $S$ a prime divisor on $X$, and $\nu: S^\nu\rightarrow S$ the normalization of $S$. The \emph{restricted foliation} of $\Ff$ on $S^\nu$ is defined in the following way. 
\begin{enumerate}
  \item If $S$ is $\Ff$-invariant, then we let $U\subset X$ be the largest open subset which does not contain $\Sing(\Ff)\cup\Sing(X)\cup\Sing(S)$ and let $S':=S\cap U$. The natural inclusion of sheaves
  $$\Ff|_{S'}\rightarrow T_X|_{S'}$$
  factors through $T_{S'}$ over $U$, which defines a foliation $\Ff_{S'}$ on $S'$. $\Ff_{S'}$ extends to a foliation $\Ff_S$ on $S^\nu$ (cf. \cite[Lemma 2.2]{CS23b}), and we call $\Ff_S$ the  \emph{restricted foliation} of $\Ff$ on $S^\nu$.
  \item If $S$ is not $\Ff$-invariant, then we let $U\subset X$ be the largest open subset which does not contain $\Sing(\Ff)\cup\Sing(X)\cup\Sing(S)$ and $S$ is transverse to $\Ff$ everywhere in $U$. We let $S':=S\cap U$. Then natural inclusion of sheaves $$\Ff|_{S'}\rightarrow T_X|_{S'}$$ induces an inclusion of sheaves $\Ff|_{S'}\cap T_{S'}\rightarrow T_{S'}$. Since $\Ff$ is saturated in $T_X$, $\Ff|_{S'}\cap T_{S'}$ is saturated in $T_{S'}$. Since $\Ff$ is closed under the Lie bracket, $\Ff|_{S'}\cap T_{S'}\subset\Ff$ is closed under the Lie bracket. Thus $\Ff_{S'}:=\Ff|_{S'}\cap T_{S'}$ is a foliation on $S'$. $\Ff_{S'}$ extends to a foliation $\Ff_S$ on $S^\nu$ (cf. \cite[Lemma 2.2]{CS23b}), and we call $\Ff_S$ the  \emph{restricted foliation} of $\Ff$ on $S^\nu$.
\end{enumerate}
\end{defn}

\begin{defn}[Almost holomorphic]
    Let $f: X\dashrightarrow Z$ be a dominant rational map. We say that $f$ is \emph{almost holomorphic} if there exist non-empty open subsets $U\subset X$ and $V\subset Z$ such that $f|_U: U\rightarrow V$ is a morphism. 
\end{defn}

The following several results are useful when applying the canonical bundle formula and adjunction formula for algebraically integrable foliations.


\begin{lem}[{cf. \cite[Lemma 2.7]{DLM23}}]\label{lem:foliation-invariant}
    Let $f: X'\to X$ be birational morphism between normal varieties, $\Ff$ is a foliation on $X$, and $\Ff':=f^{-1}\Ff$ the pullback foliation on $X'$. Then $\Ff'$ is algebraically integrable if and only if $\Ff$ is algebraically integrable.
\end{lem}

\begin{lem}\label{lem: gen fiber tangent mean induce}
Let $X$ be a normal quasi-projective variety, $\Ff$ a foliation on $X$, and $f: X\rightarrow Z$ a contraction. Suppose that the general fibers of $f$ are tangent to $\Ff$. Then there exists a foliation $\Ff_Z$ on $Z$, such that $\Ff=f^{-1}\Ff_Z$.
\end{lem}
\begin{proof}
By Definition-Lemma \ref{defthm: weak ss reduction}, there exists an equi-dimensional model $f': (X',\Sigma_{X'},\Mm)\rightarrow (Z',\Sigma_{Z'})$ of $f: X\rightarrow Z$ associated with $h: X'\rightarrow X$ and $h_Z: Z'\rightarrow Z$. By \cite[Lemma 6.7]{AD13}, there exists a foliation $\Ff_{Z'}$ on $Z'$ such that $(f')^{-1}\Ff_{Z'}=h^{-1}\Ff$. We may let $\Ff_Z:=(h_Z)_*\Ff_{Z'}$.
\end{proof}

\begin{lem}\label{lem: stein induce same foliation}
Let $f: X\rightarrow Z$ be a projective surjective morphism from a normal variety to a variety and let $X\xrightarrow{\sigma}Y\xrightarrow{\tau}Z$ be the Stein factorization of $f$. Let $\Ff$ be the foliation on $X$ induced by $f$. Then $\Ff$ is also induced by $\sigma$.
\end{lem}
\begin{proof}
Let $\Ff_Z$ be the foliation by points on $Z$. Then $\Ff_Y:=\tau^{-1}\Ff_Z$ is the foliation by points on $Y$. Since
$$\Ff=(\tau\circ \sigma)^{-1}\Ff_Z=\sigma^{-1}\Ff_Y,$$
 $\Ff$ is induced by $\sigma$.
\end{proof}

\begin{prop}[cf. {\cite[Proposition 3.2]{DLM23}}]\label{prop: a.i preserved adjunction}
Let $\Ff$ be an algebraically integrable foliation on a normal variety $X$, $S$ a prime divisor on $X$, and $S^\nu\rightarrow S$ the normalization of $S$. Let $\Ff_S$ be the restricted foliation of $\Ff$ on $S^\nu$. Then $\Ff_S$ is algebraically integrable and $\rk\Ff_S=\rk\Ff-\epsilon_{\Ff}(S)$.
\end{prop}

Finally, we recall the following theorem, which was essentially proven in \cite[Theorem 1.1]{CP19}.

\begin{thm}[{\cite[Theorem 3.1]{LLM23},\cite[Theorem 1.1]{CP19}}]\label{thm: subfoliation algebraic integrable}
Let $\Ff$ be a foliation on a normal projective variety $X$ such that $K_{\Ff}$ is not pseudo-effective. Then there exists an algebraically integrable foliation $\Ee$ such that $0\not=\Ee\subset\Ff$.
\end{thm}


\subsection{Foliated log resolution and adjunction formula}

\begin{defn}[{cf. \cite[\S 3.2]{ACSS21}}]\label{defn: foliated log smooth}
Let $(X,\Ff,B,\Mm)/U$ be a sub-gfq such that $\Ff$ is algebraically integrable. We say that $(X,\Ff,B,\Mm)$ is \emph{foliated log smooth} if there exists a contraction $f: X\rightarrow Z$ satisfying the following.
\begin{enumerate}

  \item $X$ has at most quotient toric singularities.
  \item $\Ff$ is induced by $f$.
  \item $(X,\Sigma_X)$ is toroidal for some reduced divisor $\Sigma_X$ such that $\Supp B\subset\Sigma_X$.  In particular, $(X,\Supp B)$ is toroidal, and $X$ is $\Qq$-factorial klt.
  \item There exists a log smooth pair $(Z,\Sigma_Z)$ such that $$f: (X,\Sigma_X,\Mm)\rightarrow (Z,\Sigma_Z)$$ is an equi-dimensional toroidal contraction.
  \item $\Mm$ descends to $X$.
\end{enumerate}
We say that $f: (X,\Sigma_X,\Mm)\rightarrow (Z,\Sigma_Z)$ is \emph{associated with} $(X,\Ff,B,\Mm)$, and also say that $f$ is \emph{associated with} $(X,\Ff,B,\Mm)$. It is important to remark that $f$ may not be a contraction$/U$. In particular, $\Mm$ may not be nef$/Z$.
\end{defn}


\begin{lem}[cf. {\cite[Lemma 3.1]{ACSS21}}]\label{lem: foliated log smooth imply lc}
Let $(X,\Ff,B,\Mm)$ be a sub-gfq such that $\Ff$ is algebraically integrable and $(X,\Ff,B,\Mm)$ is foliated log smooth. Then $(X,\Ff,B^\Ff,\Mm)$ is lc.
\end{lem}
\begin{proof}
By \cite[Lemma 3.1]{ACSS21}, $(X,\Ff,B^\Ff)$ is lc. Since $\Mm$ descends to $X$, $(X,\Ff,B^\Ff,\Mm)$ is lc.
\end{proof}

\begin{defn}\label{defn: log resolution}
Let $X$ be a normal quasi-projective variety, $B$ an $\Rr$-divisor on $X$, $\Mm$ a nef$/X$ $\bb$-divisor on $X$, and $\Ff$ an algebraically integrable foliation on $X$. A \emph{foliated log resolution} of $(X,\Ff,B,\Mm)$ is a birational morphism $h: X'\rightarrow X$ such that 
$$(X',\Ff':=h^{-1}\Ff,B':=h^{-1}_*B+\Exc(h),\Mm)$$ 
is foliated log smooth, where $\Exc(h)$ is the reduced $h$-exceptional divisor. 

We remark that we do not require $K_{\Ff}+B+\Mm_X$ to be $\Rr$-Cartier.
\end{defn}

\begin{lem}\label{lem: existence foliated log resolution}
Let $X$ be a normal quasi-projective variety, $B$ an $\Rr$-divisor on $X$, $\Mm$ a nef$/X$ $\bb$-divisor on $X$, and $\Ff$ a foliation on $X$ that is induced by a dominant map $f: X\dashrightarrow Z$. Then:
\begin{enumerate}
    \item If $f$ is a contraction, then for any equi-dimensional model $f': (X',\Sigma_{X'},\Mm)\rightarrow (Z',\Sigma_{Z'})$ of $f: (X,B,\Mm)\rightarrow Z$ associated with $h: X'\rightarrow X$ and $h_Z: Z'\rightarrow Z$, $h$ is a foliated log resolution of $(X,\Ff,B,\Mm)$ and $h^{-1}\Ff$ is induced by $f'$.
    \item $(X,\Ff,B,\Mm)$ has a foliated log resolution.
\end{enumerate}
\end{lem}
\begin{proof}
(1) It immediately follows from the definition of equi-dimensional models.

(2) Possibly compacifying $X$ and $Z$ and applying \cite[Lemma 2.2]{CS23b}, we may assume that $X$ and $Z$ are projective. Let $g: X''\rightarrow X$ be a birational morphism such that $f\circ g: X''\rightarrow Z$ is a morphism, $\Ff'':=g^{-1}\Ff''$, and $B'':=g^{-1}_*B+\Exc(g)$, where  $\Exc(g)$ is the reduced $g$-exceptional divisor. Possibly replacing $(X,\Ff,B,\Mm)$ with $(X',\Ff',B',\Mm)$, we may assume that $f$ is a morphism. Since $X$ and $Z$ are projective, $f$ is a projective surjective morphism. By Lemma \ref{lem: stein induce same foliation}, we may assume that $f$ is a contraction. (2) follows from (1) and Definition-Theorem \ref{defthm: weak ss reduction}.
\end{proof}

Next, we prove a simple version adjunction formula for algebraically integrable generalized foliated quadruples. The detailed version of this formula, with specific coefficient control, will be discussed later. In particular, we cannot show that the boundary coefficient after adjunction is non-negative, so we can only get ``sub-lc" instead of ``lc".

\begin{thm}\label{thm: not precise adjunction}
    Let $(X,\Ff,B,\Mm)/U$ be an lc gfq such that $\Ff$ is algebraically integrable. Let $S$ be a prime divisor on $X$ such that $\mult_SB=\epsilon_{\Ff}(S)$, $\nu: S^\nu\rightarrow S$ the normalization of $S$, $\Mm^S:=\Mm|_S$, $\Ff_S$ the restricted foliation of $\Ff$ on $S^\nu$, and
    $$K_{\Ff_S}+B_S+\Mm^S_{S^\nu}:=(K_X+B+\Mm_X)|_{S^\nu}.$$
    Then $(S^\nu,\Ff_S,B_S,\Mm^S)$ is sub-lc.
\end{thm}
\begin{proof}
By Lemma \ref{lem: existence foliated log resolution}, there exists a foliated log resolution $h: X'\rightarrow X$ of $(X,\Ff,B+S,\Mm)$. By Lemma \ref{lem: foliated log smooth imply lc}, 
$$(X',\Ff':=h^{-1}\Ff,\tilde B':=(B')^{\Ff'},\Mm)$$ is lc. Let
$$K_{\Ff'}+{B}'+\Mm_{X'}:=h^*(K_\Ff+B+\Mm_X),$$
and let $\tilde B':=B'^{\geq 0}$.
Since $(X,\Ff,B,\Mm)$ is lc, 
$$B'^{\Ff'}\geq \tilde B'\geq B'.$$
Therefore, $(X',\Ff',\tilde B',\Mm)$ is lc. In particular, $(X',\Ff',\tilde B')$ is lc.

Let $S':=h^{-1}_*S$. Then there a birational morphism $h_S: S'\rightarrow S^\nu$ such that $\nu\circ h_S=h|_{S'}$. Let $\Ff_{S'}$ be the restricted foliation of $\Ff$ on $S'$, then $\Ff_{S'}=h_S^{-1}\Ff_S$. Let
$$K_{\Ff_{S'}}+\tilde B_{S'}:=(K_{X'}+\tilde B')|_{S'}$$
and
$$K_{\Ff_{S'}}+B_{S'}+\Mm^S_{S'}:=(K_{\Ff'}+B'+\Mm_{X'})|_{S'}.$$
By \cite[Proposition 3.2]{ACSS21}, $(S',\Ff_{S'},\tilde B_{S'})$ is lc. Since $\Mm$ descends to $X'$, $\Mm^S$ descends to $S'$, so $(S',\Ff_{S'},\tilde B_{S'},\Mm^S)$ is lc.
Since $\tilde B'\geq B'$, $\tilde B_{S'}\geq B_{S'}$. Thus
$$\left(S',\Ff_{S'},\tilde B_{S'},\Mm^S\right)$$ 
is sub-lc. Since
\begin{align*}
K_{\Ff_{S'}}+B_{S'}+\Mm^S_{S'}&=(K_{\Ff'}+B'+\Mm_{X'})|_{S'}=h^*(K_{\Ff}+B+\Mm_X)|_{S'}\\
&=h_S^*((K_{\Ff}+B+\Mm_X)|_{S^\nu})=h_S^*(K_{\Ff_S}+B_S+\Mm^S_{S^\nu}),
\end{align*}
$(S^\nu,\Ff_S,B_S,\Mm^S)$ is sub-lc and we are done.
\end{proof}

Finally, we recall the following definition of F-dlt.

\begin{defn}[F-dlt]\label{defn: fdlt}
    Let $(X,\Ff,B,\Mm)/U$ be an lc gfq such that $\Ff$ is algebraically integrable. We say that $(X,\Ff,B,\Mm)$ is \emph{F-dlt} if there exists a foliated log resolution $f: Y\rightarrow X$ of  $(X,\Ff,B,\Mm)$ such that $a(D,\Ff,B,\Mm)>-\epsilon_{\Ff}(D)$ for any prime $f$-exceptional divisor $D$.
\end{defn}



\subsection{Cutting foliations by general hyperplane sections} By Theorem \ref{thm: not precise adjunction}, to prove the precise adjunction formulas, we need to control the coefficients of the boundary divisors on the restricted foliation. We achieve this by cutting the foliations using general hyperplane sections until we reach the surface case. Then, we use the structure of surface singularities to achieve our result. In this subsection, we tackle the first issue: cutting foliations by general hyperplane sections. It is important to note that general hyperplane sections for foliations behave very differently comparing to usual varieties. For example, log canonicity is often not preserved \cite[Example 3.4]{ACSS21}. On the other hand, we can use the methods introduced in \cite[Section 3.2]{DLM23} to resolve this issue.

\begin{lem}\label{lem: general hyperplane cut keep m}
Let $X$ be a normal quasi-projective variety and $H$ a prime divisor on $X$, such that $H$ is base-point-free and is a general member of $|H|$. Let $\Mm$ be a $\bb$-divisor on $X$ such that $\Mm$ descends to a birational model $X'$ of $X$ and $\Mm^H:=\Mm|_H$. Then $\Mm^H_H=\Mm_X|_H$.
\end{lem}
\begin{proof}
We may assume that the induced birational map $f: X'\dashrightarrow X$ is a morphism. We let $$V:=f(\Supp(\Mm_{X'}-f^{-1}_*\Mm_X)),$$
then $\dim X-\dim V\geq 2$. Since $H$ is general, $\dim H-\dim (V\cap H)\geq 2$. Therefore, for any prime divisor $D$ on $H$ and identifying $D$ with its image in $X$, we have that $\Mm$ descends to $X$ near the generic point of $D$. The lemma follows immediately.
\end{proof}

\subsubsection{Cutting by invariant hyperplane sections}

First, we show that we can cut foliations by invariant base-point-free linear systems freely.

\begin{prop}\label{prop: general hyperplane invariant}
Let $(X,\Ff,B,\Mm)/U$ be a sub-gfq and $W$ a proper subvariety of $X$. Suppose that $\Ff$ is induced by a morphism $f: X\rightarrow Z$, $\dim Z>0$, and $W$ is transverse to $\Ff$. Let $H_Z\subset Z$ be a general hyperplane section. Let $H:=f^*H_Z$, $\Mm^H:=\Mm|_H$, and 
$$K_{\Ff_H}+B_H+\Mm_H^H:=(K_{\Ff}+B+\Mm_X)|_H,$$ 
where $\Ff_H$ is the restricted foliation of $\Ff$ on $H$. Then:
\begin{enumerate}
    \item $H$ intersects $W$.
    \item For any component $D$ of $\Supp B$ such that $D$ intersects $H$ and any component $C$ of $D\cap H$, $\mult_CB_H=\mult_DB$.
    \item If $(X,\Ff,B,\Mm)$ is (sub-)lc, then $(H,\Ff_H,B_H,\Mm^H)$ is (sub-)lc.
    \item $\Ff_H$ is induced by $f|_H: H\rightarrow H_Z$.
\end{enumerate}
\end{prop}
\begin{proof}
By Definition-Theorem \ref{defthm: weak ss reduction} and Lemma \ref{lem: existence foliated log resolution}, there exists an equi-dimensional model $f': (X',\Sigma_{X'},\Mm)\rightarrow (Z',\Sigma_{Z'})$ of $f: (X,B,\Mm)\rightarrow Z$ associated with $h: X'\rightarrow X$ and $h_Z: Z'\rightarrow Z$, such that $h$ is a foliated log resolution of $(X,\Ff,B,\Mm)$ and $\Ff':=h^{-1}\Ff$ is induced by $f'$. We let $H':=h^*H$,
$$K_{\Ff'}+B'+\Mm_{X'}:=h^*(K_{\Ff}+B+\Mm_X),$$
and
$$K_{\Ff_{H'}}+B_{H'}+\Mm^H_{H'}:=\left(K_{\Ff'}+B'+\Mm_{X'}\right)|_{H'}.$$ 
First we show that $B_{H'}=B'|_{H'}$. Let
$$R(f'):=\sum_{D\mid D\text{ is a prime divisor on }Z'}(f'^*D-f'^{-1}(D))$$
be the ramification divisor of $f'$, then
$$R(f')=\sum_{D\subset\Supp\Sigma_{Z'}}(f'^*D-f^{-1}(D)).$$
Since $H'$ and $H_{Z'}$ are general, by \cite[Proposition 3.2]{AK00}, $f'|_{H'}:(H',\Sigma_{X'}|_{H'},\Mm|_{H'})\rightarrow (H_{Z'},\Sigma_{Z'}|_{H_{Z'}})$ is an equi-dimensional toroidal contraction. Therefore, for any prime divisor $D_Z$ on $H_{Z'}$, $(f'|_{H'})^*D\not=f'^{-1}(D)$ only if $D_Z$ is a component of $\Sigma_{Z'}|_{H_{Z'}}$. Therefore,
\begin{align*}
R(f')|_{H'}=R(f')|_{f'^*H_{Z'}}&=\sum_{D\subset\Supp\Sigma_{Z'}}((f'|_{H'})^*D|_{H_{Z'}}-(f'|_{H'})^{-1}(D|_{H_{Z'}}))\\
&=\sum_{D_Z\subset\Sigma_{Z'}|_{H_{Z'}}}((f'|_{H'})^*D_Z-(f'|_{H'})^{-1}(D_Z))\\
&=\sum_{D_Z\mid D_Z\text{ is a prime divisor on }H_{Z'}}((f'|_{H'})^*D_Z-(f'|_{H'})^{-1}(D_Z)):=R(f'|_{H'})
\end{align*}
is the ramification divisor of $f'|_{H'}$ Thus
$$K_{\Ff'}|_{H'}=(K_{X'/Z'}-R(f'))|_{H'}=K_{H'/H_{Z'}}-R(f'|_{H'})=K_{\Ff_{H'}}.$$
Since  $\Mm^H_H=\Mm_X|_H$, we have $B_{H'}=B'|_{H'}$. 

(1) Since  $W$ is transverse to $\Ff$,  $W':=h^{-1}(W)$ is not tangent to $\Ff'$. Thus $\dim g(W')\geq 1$, so $H_{Z'}:=h_Z^*H_Z$ intersects $g(W')$ and $H_Z$ intersects $h_Z(g(W'))=f(W)$. Hence $H$ intersects $W$.

(2) Since $H$ is general, near the generic point $\eta_C$ of $C$, $h$ is an isomorphism. Since $B_{H'}=B'|_{H'}$, $B|_H=B_H$ near $\eta_C$. We may write $B=\sum b_iB_i$ where $B_i$ are the irreducible components of $B$, then 
$$B_H=B|_H=\sum b_i(B_i\cap H)$$ 
near $\eta_C$. Since $H$ is general, there exists a unique index $i$ such that $B_i\cap H\not=0$ at $\eta_C$. Then $B_i\cap H=C$, $B_i=D$, and hence $\mult_CB_H=b_i=\mult_DB$.

(3) By Lemma \ref{lem: foliated log smooth imply lc}, $(X',\Ff',\tilde B':=(B')^{\geq 0},\Mm)$ is lc. Let $K_{\Ff_{H'}}+\tilde B_{H'}:=(K_{\Ff'}+\tilde B')|_{H'}$. By \cite[Proposition 3.2]{ACSS21}, $(H',\Ff_{H'},\tilde B_{H'})$ is lc. Since $\Mm$ descends to $X'$, 
$$K_{\Ff_{H'}}+\tilde B_{H'}+\Mm^H_{H'}=(K_{\Ff'}+B+\Mm_X)|_H,$$
and $\Mm^H$ descends to $H'$. Thus $(H',\Ff_{H'},\tilde B_{H'},\Mm^H)$ is lc. Since $\tilde B'\geq B'$, $\tilde B_{H'}\geq B_{H'}$, so $(H',\Ff_{H'},B_{H'},\Mm^H)$ is sub-lc. Since
$$K_{\Ff_{H'}}+B_{H'}+\Mm^H_{H'}=(h|_{H'})^*\left(K_{\Ff_H}+B_H+\Mm^H_H\right),$$
$(H,\Ff_H,B_H,\Mm^H)$ is sub-lc. 

If $(X,\Ff,B,\Mm)$ is lc, then $B\geq 0$. By (2), $B_H\geq 0$. Thus $(H,\Ff_H,B_H,\Mm^H)$ is lc.

(4) It immediately follows from the definition of restricted foliations and the the condition that $H_Z$ is a general hyperplane section of $Z$.
\end{proof}


\subsubsection{Cutting by non-invariant hyperplanes}

Next we show that, if we only consider the local property of foliations, then we can cut foliation by non-invariant hyperplane sections. 

\begin{lem}\label{lem: toroidal cut general hyperplane still lc}
    Let $f: (X,\Sigma,\Mm)\rightarrow (Z,\Sigma_Z)$ be a toroidal morphism and $z\in Z$ a closed point. Let $\Ff$ be the foliation induced by $f$ and let $B$ be the horizontal$/Z$ part of $\Sigma$. Let $H$ be a general member of a base-point-free linear system on $X$, such that $H$ dominates $Z$.  Then $(X,\Ff,B+H,\Mm)$ is lc over a neighborhood of $z$.
\end{lem}
\begin{proof}
By \cite[Lemma 3.6]{DLM23}, $(X,\Ff,B+H)$ is lc over a neighborhood of $z$. Since $\Mm$ descends to $X$, $(X,\Ff,B+H,\Mm)$ is lc over a neighborhood of $z$.
\end{proof}

\begin{prop}\label{prop: general hyperplane non-invariant}
Let $(X,\Ff,B,\Mm)$ be a sub-gfq and $W$ a proper subvariety of $X$. Suppose that $\Ff$ is algebraically integrable, $W$ is tangent to $\Ff$, and $\dim W\geq 1$. Let $H\subset X$ be a general hyperplane section. Let $\Mm^H:=\Mm|_H$ and $$K_{\Ff_H}+B_H+\Mm^H_H:=(K_{\Ff}+B+H+\Mm_X)|_H,$$ where $\Ff_H$ is the restricted foliation of $\Ff$ on $H$. Then:
\begin{enumerate}
    \item $H$ intersects $W$.
    \item For any component $D$ of $\Supp B$ such that $D$ intersects $H$ and any component $C$ of $D\cap H$, $\mult_CB_H=\mult_DB$.
    \item  If $(X,\Ff,B,\Mm)$ is (sub-)lc near $W$, then $(H,\Ff_H,B_H,\Mm^H)$ is (sub-)lc near $W|_H$.
    \item If $f$ is induced by a morphism $f: X\rightarrow Z$, then $\Ff_H$ is induced by $f|_H: H\rightarrow Z$.
\end{enumerate}
\end{prop}
\begin{proof}
(1) is obvious. 

(2) By \cite[Proposition 3.6]{Dru21}, $K_{\Ff_H}=(K_{\Ff}+H)|_H$, so $B_H+\Mm^H_H=B|_H+\Mm_X|_H$.  We remark that \cite[Proposition 3.6]{Dru21} requires that $\rk\Ff\geq 2$, but the same lines of the proof works for the case when $\rk\Ff=1$ as well. We may write $B=\sum b_iB_i$ where $B_i$ are the irreducible components of $B$. Since $H$ is general, 
$$B_H=B|_H=\sum b_i(B_i\cap H),$$ 
and there exists a unique index $i$ such that $B_i\cap H\not=0$ at the generic point of $C$. Then $B_i\cap H=C$, $B_i=D$, and hence $\mult_CB_H=b_i=\mult_DB$.

(3) By Definition-Theorem \ref{defthm: weak ss reduction} and Lemma \ref{lem: existence foliated log resolution}, there exists an equi-dimensional model $f': (X',\Sigma_{X'},\Mm)\rightarrow (Z,\Sigma_Z)$ of $f: (X,B,\Mm)\rightarrow Z$ associated with $h: X'\rightarrow X$ and $h_Z: Z'\rightarrow Z$, such that $h$ is a foliated log resolution of $(X,\Ff,B,\Mm)$ and $\Ff':=h^{-1}\Ff$ is induced by $f'$. We let $$K_{\Ff'}+B'+\Mm_{X'}:=h^*(K_{\Ff}+B+\Mm_X),$$ $H':=h^*H$, $W':=h^{-1}(W)$, and $\tilde B':=(B')^{\geq 0}$. We let $z$ be the image of $W'$ on $Z'$. Since $(X,\Ff,B,\Mm)$ is lc, by Lemma \ref{lem: foliated log smooth imply lc}, $(X',\Ff',\tilde B',\Mm)$ is lc. Moreover, all components of $\tilde B'$ are horizontal$/Z$. By Lemma \ref{lem: toroidal cut general hyperplane still lc}, $(X',\Ff',\tilde B'+H',\Mm)$ is lc over a neighborhood of $z'$. In  particular, $(X',\Ff',\tilde B',\Mm)$ is lc near $W'|_{H'}$. Let
$$K_{\Ff_{H'}}+\tilde B_{H'}:=(K_{\Ff'}+\tilde B')|_{H'}$$
and
$$K_{\Ff_{H'}}+B_{H'}:=(K_{\Ff'}+B')|_{H'}.$$ 
By \cite[Proposition 3.2]{ACSS21}, $(H',\Ff_{H'},\tilde B_{H'})$ is lc near $W'|_{H'}$. Since $\tilde B'\geq B'$, $\tilde B_{H'}\geq B_{H'}$. Thus $(H',\Ff_{H'},B_{H'})$ is sub-lc near $W'|_{H'}$. Since $\Mm$ descends to $X'$, $\Mm^H$ descends to $H'$, so $(H',\Ff_{H'},B_{H'},\Mm^H)$ is sub-lc near $W'|_{H'}$. Since $$K_{\Ff_{H'}}+B_{H'}+\Mm^H_{H'}=h|_{H'}^*(K_{\Ff_H}+B_H+\Mm^H_H),$$ 
$(H,\Ff_H,B_H,\Mm^H)$ is sub-lc near $W|_H$. 

If $(X,\Ff,B,\Mm)$ is lc, then $B\geq 0$. By (2), $B_H\geq 0$. Thus $(H,\Ff_H,B_H,\Mm^H)$ is lc near $W|_H$.

(4) It immediately follows from the definition of restricted foliations and the the condition that $H$ is a general hyperplane section of $X$.
\end{proof}

\subsection{Basic properties of foliated surfaces}
In this subsection, we recall some basic properties of foliated surfaces. Moreover, we introduce the concept of \emph{surface numerical gfqs} and study their basic properties. This is crucial for the proof of adjunction formulas.


\begin{defn}
Let $X$ be a normal surface, $\Ff$ a foliation on $X$, and $x\in X$ a closed point such that $x\not\in\Sing(X)$ and $x\in\Sing(\Ff)$. Let $v$ be a vector field generating $\Ff$ near $x$. By \cite[Page 2, Line 17-18]{Bru15}, $v(x)=0$ and $(Dv)|_x$ has exactly two eigenvalues $\lambda_1$ and $\lambda_2$. 

We say that $x$ is a \emph{reduced singularity} of $\Ff$ if at least one of $\lambda_1$ and $\lambda_2$ is not $0$ (say, $\lambda_2$) and $\frac{\lambda_1}{\lambda_2}\not\in\mathbb Q^+$. We say that $\Ff$ has \emph{at most reduced singularities} if for any closed point $p\in X$, $\Ff$ is either non-singular at $p$ or $p$ is a reduced singularity of $\Ff$. 
\end{defn}


\begin{defn}[Minimal resolution]\label{defn: minimal resolution foliation}
Let $X$ be a normal surface, $\Ff$ a foliation on $X$, $f: Y\rightarrow X$ a projective birational morphism, and $\Ff_Y:=f^{-1}\Ff$.


We say that $f$ is a \emph{resolution} of $\Ff$ if $Y$ is smooth and $\Ff_Y$ has at most reduced singularities. By \cite{Sei68} (we refer to \cite[Pages 908--912]{Can04} for a detailed explanation), resolution of $\Ff$ always exists.

We say that  $f$ is the \emph{minimal resolution} of $\Ff$ if for any resolution $g: W\rightarrow X$ of $\Ff$, $g$ factors through $f$, i.e. there exists a projective birational morphism $h: W\rightarrow Y$ such that $g=f\circ h$. By definition, the minimal resolution of $\Ff$ is unique, and by \cite[Proposition 1.17]{Che23}, the minimal resolution of $\Ff$ exists.
\end{defn}

\begin{defn}
Let $X$ be a normal surface with at most cyclic quotient singularities, $\Ff$ a foliation on $X$, and $C$ a reduced curve on $X$ such that no component of $C$ is $\Ff$-invariant. For any closed point $x\in X$, we define $\tang(\Ff,C,x)$ in the following way. 
\begin{itemize}
\item If $x\notin \Sing(X)$, then we let $v$ be a vector field generating $\Ff$ around $x$, and $f$ a holomorphic function defining $C$ around $x$. We define 
$$\tang(\Ff,C,x):=\dim_{\mathbb{C}}\frac{\mathcal{O}_{X,x}}{\langle f, v(f)\rangle}.$$
\item If $x\in\Sing(X)$, then $x$ is a cyclic quotient singularity of index $r$ for some integer $r\geq 2$. Let $\rho:\tilde X\rightarrow X$ be an index $1$ cover of $X\ni x$, $\tilde x:=\rho^{-1}(x)$, $\widetilde C:=\rho^*C$, and $\tilde\Ff$ the foliation induced by the sheaf $\rho^*\Ff$ near $\tilde x$. Then $\tilde x$ is a smooth point of $\tilde X$, and we define
$$\tang(\Ff,C,x):=\frac{1}{r}\tang(\tilde\Ff,\tilde C,\tilde x).$$
\end{itemize}
We define
$$\tang(\Ff,C):=\sum_{x\in X}\tang(\Ff,C,x).$$
By \cite[Section 2]{Bru02}, $\tang(\Ff,C)$ is well-defined.
\end{defn}

\begin{defn}
Let $X$ be a normal surface with at most cyclic quotient singularities, $\Ff$ a foliation on $X$, and $C$ a reduced curve on $X$ such that all components of $C$ are $\Ff$-invariant. For any closed point $x\in X$, we define $Z(\Ff,C,x)$ in the following way.
\begin{itemize}
\item If $x\notin \Sing(X)$, then we let $\omega$ be a $1$-form generating $\Ff$ around $x$, and $f$ a holomorphic function generating $C$ around $x$. Then there are uniquely determined holomorphic functions $g,h$ and a holomorphic $1$-form $\eta$ on $X$ near $x$, such that $g\omega=hdf+f\eta$ and $f,h$ are coprime. We define 
$$Z(\Ff,C,x):= \text{ the vanishing order of }\frac{h}{g}\biggm|_C\text{ at }x.$$
By [Chapter 2, Page 15]{Bru15}, $Z(\Ff,C,x)$ is independent of the choice of $\omega$.
\item If $x\in C\cap \Sing(X)$, then we define 
$Z(\Ff,C,x):=0.$
\end{itemize}
We define $$Z(\Ff,C):=\sum_{x\in C}Z(\Ff,C,x).$$
By \cite[Section 2]{Bru02}, $Z(\Ff,C)$ is well-defined.
\end{defn}


\begin{defn}[Dual graph]\label{defn: dual graph}
Let $n$ be a positive integer, and $C=\cup_{i=1}^nC_i$ be a collection of irreducible curves contained in the non-singular locus of a normal surface $X$. We define the \emph{dual graph} $\mathcal{D}(C)$ of $C$ as follows.
\begin{enumerate}
    \item The vertices $v_i=v_i(C_i)$ of $\mathcal{D}(C)$ correspond to the curves $C_i$.
    \item For any $i\neq j$, the vertices $v_i$ and $v_j$ are connected by $C_i\cdot C_j$ edges.
    \item Each vertex $v_i$ is labeled by $w(C_i):=-C_i^2$. The integer $w(C_i)$ is called the \emph{weight} of $C_i$.
\end{enumerate}
For any projective birational morphism $f: Y\rightarrow X$ between surfaces, let $E=\cup_{i=1}^nE_i$ be the reduced exceptional divisor for some non-negative integer $n$. Suppose that $E$ is not contained in the non-singular locus of $Y$. Then we define $\mathcal{D}(f):=\mathcal{D}(E)$.
\end{defn}


\begin{defn}\label{defn: num foliated sing}
A \emph{surface numerical sub-gfq} (\emph{surface num-sub-gfq} for short) $(X,\Ff,B,\Mm)/U$ consists of a normal surface $X$, a rank $1$ foliation $\Ff$ on $X$, an $\Rr$-divisor $B$ on $X$, and a nef$/U$ $\bb$-divisor $\Mm$. We say that $(X,\Ff,B,\Mm)$ is a \emph{surface numerical gfq} (\emph{surface num-gfq} for short) if  $(X,\Ff,B)$ is a surface num-sub-gfq and $B\geq 0$.

Let $(X,\Ff,B,\Mm)$ be a surface num-sub-gfq. Let $f: Y\rightarrow X$ be a resolution of $X$ with prime $f$-exceptional divisors $E_1,\dots,E_n$ for some non-negative integer $n$. Since $\{(E_i\cdot E_j)\}_{n\times n}$ is negative definite, the equation 
$$\begin{pmatrix}
  (E_1\cdot E_1) &\cdots & (E_1\cdot E_n) \\
  \vdots&\ddots & \vdots         \\
  (E_n\cdot E_1) &\cdots & (E_n\cdot E_n) \\
\end{pmatrix} 
\begin{pmatrix}
  a_1 \\
  \vdots \\
  a_n \\
\end{pmatrix} 
 =
 \begin{pmatrix}
-(K_{\Ff_Y}+B_Y+\Mm_Y)\cdot E_1\\
\vdots\\
-(K_{\Ff_Y}+B_Y+\Mm_Y)\cdot E_n\\
\end{pmatrix}$$
has a unique solution $(a_1,\dots,a_n)$, where $\Ff_Y:=f^{-1}\Ff$ and $B_Y:=f^{-1}_*B$. For any prime divisor $E$ on $Y$, we define 
$$a_{\num,f}(E,\Ff,B,\Mm):=-\mult_E\left(B_Y+\sum_{i=1}^n a_iE_i\right).$$
\end{defn}


\begin{lem}\label{lem: anum same as a}
Let $(X,\Ff,B,\Mm)$ be a sub-gfq such that $\dim X=2$ and $\rk\Ff=1$. Let $f: Y\rightarrow X$ be a resolution of $X$ and $E$ a prime divisor on $Y$. Then $a_{\num,f}(E,\Ff,B,\Mm)=a(E,\Ff,B,\Mm)$.
\end{lem}
\begin{proof}
If $E$ is not exceptional over $X$, then 
$$a_{\num,f}(E,\Ff,B,\Mm)=-\mult_EB=a(E,\Ff,B,\Mm)$$ 
and we are done. Thus we may assume that $E$ is exceptional over $X$. Let $E_1,\dots,E_n$ be all the $f$-exceptional prime divisors and let
$$K_{\Ff_Y}+\sum_{j=1}^na_jE_j+B_Y+\Mm_Y=f^*(K_{\Ff}+B+\Mm_X),$$
where $\Ff_Y:=f^{-1}\Ff$ and $B_Y:=f^{-1}_*B$. Then
$$\left(K_{\Ff_Y}+\sum_{j=1}^na_jE_j+B_Y+\Mm_Y\right)\cdot E_i=0$$
for any $i$. Therefore,
$$a_{\num,f}(E_i,\Ff,B,\Mm)=-a_i=a(E_i,\Ff,B,\Mm)$$ for any $i$. Since $E=E_j$ for some $j$, $$a_{\num,f}(E,\Ff,B,\Mm)=a(E,\Ff,B,\Mm)$$ and we are done.
\end{proof}

\begin{lem}\label{lem: anum not depend on resolution}
Let $(X,\Ff,B,\Mm)$ be a surface num-sub-gfq and $f: Y\rightarrow X$, $f': Y'\rightarrow X$ two resolutions of $X$. Let $E$ be a prime divisor over $X$ such that $\Center_YE$ and $\Center_{Y'}E$ are divisors. Then
$$a_{\num,f}(E,\Ff,B,\Mm)=a_{\num,f'}(E,\Ff,B,\Mm).$$
\end{lem}
\begin{proof}
If $E$ is on $X$ then $$a_{\num,f}(E,\Ff,B,\Mm)=-\mult_EB=a_{\num,f'}(E,\Ff,B,\Mm),$$ so we may assume that $E$ is exceptional over $X$.

Let $g: W\rightarrow Y$ and $g': W\rightarrow Y'$ be a common resolution, and $h: W\rightarrow X$ the induced birational morphism. Possibly replacing $f'$ with $h$, we may assume that there exists a morphism $g: Y'\rightarrow Y$. Let $E_i$ be the prime $f'$-exceptional divisors,
$$B_{Y'}:=f'^{-1}_*B-\sum_ia_{\num,f'}(E_i,\Ff,B,\Mm)E_i,$$
and $B_Y:=g_*B_{Y'}$.
Then $(K_{\Ff_{Y'}}+B_{Y'}+\Mm_{Y'})\cdot E_i=0$ for any $E_i$. Since $Y$ is smooth, $K_{\Ff_Y}+B_Y+\Mm_Y$ is $\Rr$-Cartier. By applying the negativity lemma twice, we have 
$$K_{\Ff_{Y'}}+B_{Y'}+\Mm_{Y'}=g^*(K_{\Ff_Y}+B_Y+\Mm_Y).$$ 
Thus $(K_{\Ff_Y}+B_Y+\Mm_Y)\cdot g_*E_i=0$ for any $E_i$, so 
$$a_{\num,f}(E_i,\Ff,B,\Mm)=-\mult_{g_*E_i}B_Y=\mult_{E_i}B_{Y'}=a_{\num,f'}(E_i,\Ff,B,\Mm)$$
for any $E_i$ such that $g_*E_i\not=0$. In particular, $a_{\num,f}(E,\Ff,B,\Mm)=a_{\num,f'}(E,\Ff,B,\Mm)$.
\end{proof}

\begin{defn}
Let $(X,\Ff,B)$ be a surface num-sub-gfq. We define $a(E,\Ff,B,\Mm):=a_{\num,f}(E,\Ff,B,\Mm)$ for an arbitrary resolution $f: Y\rightarrow X$ of $X$ such that $E$ is a divisor on $Y$. Lemmas \ref{lem: anum same as a} and \ref{lem: anum not depend on resolution} guarantee that there is no abuse of notations.

Let $(X,\Ff,B,\Mm)$ be a surface num-gfq. We say that $(X,\Ff,B,\Mm)$ is \emph{num-lc} if $a(E,\Ff,B,\Mm)\geq-\epsilon_{\Ff}(E)$ for any prime divisor $E$ over $X$.
\end{defn}

\begin{lem}\label{lem: add component worse sing}
Let $(X,\Ff,B,\Mm)$ be a surface num-gfq and $x\in X$ a closed point. Then for any prime divisor $E$ over $X\ni x$, 
$$a(E,\Ff,B,\Mm)\leq a(E,\Ff,B),$$ 
and 
$$a(E,\Ff,B,\Mm)=a(E,\Ff,B)$$
if and only if $\Mm$ descends to $X$ over a neighborhood of $x$. In particular, if $(X,\Ff,B,\Mm)$ is num-lc, then $(X,\Ff,B)$ is num-lc.
\end{lem}
\begin{proof} 
It follows from \cite[Lemma 3.41]{KM98}.
\end{proof}

\begin{lem}\label{lem: r cartier b m lc gfq}
    Let $(X,\Ff,B,\Mm)$ be an lc gfq such that $\dim X=2$ and $\rk\Ff=1$. Then $K_{\Ff}$, $\Mm_X$, and all components of $B$ are $\Rr$-Cartier.
\end{lem}
\begin{proof}
We only need to show that $\Mm_X$ and all components of $B$ are $\Rr$-Cartier near $x$ for any closed point $x\in X$. If $\Ff$ is num-terminal near $x$, then by \cite[Theorem 3.19]{LMX23a}, $\Ff$ is terminal near $x$, and $X$ is $\Qq$-factorial klt near $x$. Therefore, $\Mm_X$ and all components of $B$ are $\Rr$-Cartier near $x$. If $\Ff$ is not num-terminal near $x$, then by Lemma \ref{lem: add component worse sing}, $\Mm$ descends to $X$ over a neighborhood of $x$, and  $(X,\Ff,B)$ is num-lc. By \cite[Theorem 3.19]{LMX23a}, $x\not\in\Supp B$. In particular, $\Mm_X$ and all components of $B$  are $\Rr$-Cartier near $x$. 
\end{proof}


\subsection{Adjunction formula for surface generalized foliated quadruples}\label{subsec: adj gfq surface} In this subsection we establish the adjunction formula for surface generalized foliated quadruples based on the classification of foliated surface singularities. Depending on whether the foliation itself is terminal, we establish two adjunction formulas.

\begin{lem}\label{lem: surface pia not terminal}
     Let $(X,\Ff,B,\Mm)$ be an lc gfq such that $\dim X=2$, $\rk\Ff=1$, and $B_j$ are the irreducible components of $B$. Let $C$ be an $\Ff$-invariant curve with normalization $\nu: C^\nu\rightarrow C$. Let $x\in C$ be a closed point, such that $\Ff$ is not terminal near $x$. Then:
     \begin{enumerate}
         \item $x\not\in\Supp B$ and $\Mm$ descends to $X$ over a neighborhood of $x$.
         \item For any closed point $y\in\nu^{-1}(x)$, the vanishing order of $K_{\Ff}|_{C^\nu}$ at $y$ is a non-negative integer.
     \end{enumerate}
\end{lem}
\begin{proof}
  (1) Since $\Ff$ is not terminal near $x$ and $(X,\Ff,B,\Mm)$ is lc, by Lemma \ref{lem: r cartier b m lc gfq} and \cite[Theorem 3.19]{LMX23a}, $B=0$ near $x$. By Lemma \ref{lem: add component worse sing}, $\Mm$ descends to $X$ over a neighborhood of $x$. 

  (2) By considering a local analytic neighborhood of $x$ and separate $C$ into different analytic irreducible components, we may assume that $y=\nu^{-1}(x)$. (2) follows from \cite[Theorem 3.19]{LMX23a}. More precisely, we let $h: Y\rightarrow X$ be the minimal resolution of $\Ff$ near $x$ and let $C_Y:=h^{-1}_*C$, then we only need to show that
  $$K_{\Ff}\cdot C-K_{C^\nu}=h^*K_{\Ff}\cdot C_Y-K_{C_Y}$$
  is a positive integer over a neighborhood of $x$. (2) follows by checking all cases of \cite[Theorem 3.19]{LMX23a} and apply \cite[Proposition 2.16(3)]{CS20} to $C_Y$ for each case.
\end{proof}

\begin{lem}\label{lem: surface pia terminal}
    Let $(X,\Ff,B=\sum_{j=1}^mb_jB_j,\Mm)$ be a gfq such that $\dim X=2$, $\rk\Ff=1$, and $B_j$ are the irreducible components of $B$. Let $C$ be an $\Ff$-invariant curve with normalization $C^\nu$. Let $x\in C$ be a closed point such that $\Ff$ is terminal near $x$,  $I$ the order of the local fundamental group $\pi_1(X\ni x)$, and $\Mm^C:=\Mm|_{C^\nu}$. Then there exists a positive integer $I$ and non-negative integers $w_1,\dots,w_m$ satisfying the following.
    \begin{enumerate}
        \item $X$ is $\Qq$-factorial klt near $x$ and $C$ is non-singular near $x$. 
        \item $\mu:=\mult_x(\Mm_X|_{C^\nu}-\Mm^C_{C^\nu})\geq 0$.
        \item For any real numbers $b_1',\dots,b_m'$, the vanishing order of
$$\left(K_{\Ff}+\sum_{j=1}^mb_j'B_j+\Mm_X\right)\Biggm|_{C^\nu}-\Mm^C_{C^\nu}$$
        at $x$ is 
        $$\frac{I-1+\sum_{j=1}^mw_jb_j'}{I}+\mu.$$
        Moreover, if $(X,\Ff,\sum_{j=1}^mb_j'B_j,\Mm)$ is lc, then       $$0\leq \frac{I-1+\sum_{j=1}^mw_jb_j'}{I}+\mu\leq 1.$$
        \item Suppose that $\Mm=\sum_{k=1}^mr_k\Mm_k$ where each $\Mm_k$ is a nef$/X$ $\bb$-Cartier $\bb$-divisor. Let $\Mm^C_k:=\Mm_k|_{C^\nu}$ for each $k$. Then there exist non-negative integers $v_1,\dots,v_n$, such that for any real numbers $b_1',\dots,b_m', r_1',\dots,r_n'$, the vanishing order of
$$\left(K_{\Ff}+\sum_{j=1}^mb_j'B_j+\sum_{k=1}^nr_k'\Mm_{i,X}\right)\Biggm|_{C^\nu}-\sum_{k=1}^nr_k'\Mm^C_{k,C^\nu}$$
at $x$ is
$$\frac{I-1+\sum_{j=1}^mw_jb_j'+\sum_{k=1}^nv_kr_k'}{I}.$$
Moreover, if $(X,\Ff,\sum_{j=1}^mb_j'B_j,\sum_{k=1}^nr_k'\Mm_k)$ is lc, then
$$0\leq\frac{I-1+\sum_{j=1}^mw_jb_j'+\sum_{k=1}^nv_kr_k'}{I}\leq 1.$$
    \end{enumerate}
\end{lem}
 We note that $\Mm_X|_{C^\nu}$ in (2), $$\left(K_{\Ff}+\sum_{j=1}^mb_j'B_j+\Mm_X\right)\Biggm|_{C^\nu}$$ in (3), and $$\left(K_{\Ff}+\sum_{j=1}^mb_j'B_j+\sum_{k=1}^nr_k'\Mm_{k,X}\right)\Biggm|_{C^\nu}$$ in (4) may not be well-defined, but they are at least well-defined near $x$ so there is no confusion for the statements of the lemma.
\begin{proof}
    (1) It follows from \cite[Theorem 3.19]{LMX23a}.

    Since all statements in the lemma are local near $x$, possibly shrinking $X$ to a neighborhood of $x$, in the following, we may assume that $X$ is $\Qq$-factorial klt, $\Ff$ is terminal, and $C$ is non-singular. In particular, we will identify $C$ with $C^\nu$ in the following arguments.
    
    (2) Let $h: Y\rightarrow X$ be a birational morphism such that $\Mm$ descends to $Y$. Let $C_Y:=h^{-1}_*C$. Since $\Mm$ is nef$/X$ and $\Mm_X$ is $\Rr$-Cartier, by the negativity lemma, $h^*\Mm_X-\Mm_Y\geq 0$. Thus
    $$\mu=\mult_x((h|_{C_Y})_*(h^*\Mm_X-\Mm_Y)|_{C_Y})\geq 0.$$
    
    (3) Since $\Ff$ is terminal, by \cite[Theorem 3.2]{LMX23b}, there exist non-negative integers $w_1,\dots,w_m$, such that the vanishing order of $$\left(K_{\Ff}+\sum_{j=1}^mb_j'B_j+\Mm_X\right)\Biggm|_{C}-\Mm^C_{C}$$
        at $x$ is 
        $$q:=\frac{I-1+\sum_{j=1}^mw_jb_j'}{I}+\mu$$
        for any real numbers $b_1',\dots,b_m'$. If $(X,\Ff,\sum_{j=1}^mb_j'B_j,\Mm)$ is lc, then $b_j'\geq 0$ for each $j$ 
 and $\mu\geq 0$ by (2). Thus $q\geq 0$. By Theorem \ref{thm: not precise adjunction}, $q\leq 1$. (3) follows.

 (5) Since $\Mm_{k,X}$ is an integral divisor for each $k$, $I\mult_x\Mm_{k,X}|_C$ is an integer. We let
 $$v_k:=I(\mult_x\Mm_{k,X}|_C-\mult_x\Mm_{k,C}^C),$$
 then each $v_k$ is an integer. Possibly replacing $Y$ with a high model, we may assume that $\Mm_k$ descends to $Y$ for each $k$. By the negativity lemma, $h^*\Mm_{k,X}-\Mm_{k,Y}\geq 0$. Thus
    $$v_k=I\mult_x((h|_{C_Y})_*(h^*\Mm_{k,X}-\Mm_{k,Y})|_{C_Y})\geq 0,$$
 so each $v_k$ is a non-negative integer. By (4), the vanishing order of
$$\left(K_{\Ff}+\sum_{j=1}^mb_j'B_j+\sum_{k=1}^nr_k'\Mm_{k,X}\right)\Biggm|_{C}-\sum_{k=1}^nr_k'\Mm_{k,C}^C$$
at $x$ is
$$l:=\frac{I-1+\sum_{j=1}^mw_jb_j'+\sum_{k=1}^nv_kr_k'}{I}$$
for any real numbers $b_1,\dots,b_m',\dots,r_1',\dots,r_n'$.
 If $(X,\Ff,\sum_{j=1}^mb_j'B_j,\sum_{k=1}^nr_k'\Mm_{k,X})$ is lc, then $b_j'\geq 0$ for each $j$, and $\sum_{k=1}^nv_kr_k'\geq 0$ by (2). Thus $l\geq 0$. By Theorem \ref{thm: not precise adjunction}, $l\leq 1$. (5) follows.
\end{proof}




\subsection{Precise adjunction formula when the foliation is induced by a morphism}
\begin{thm}\label{thm: adjunction foliation nonnqc}
   Let $(X,\Ff,B,\Mm)/U$ be an lc gfq such that $\Ff$ is induced by a contraction $X\rightarrow Z$. Let $S$ be a prime divisor on $X$ such that $\mult_SB=\epsilon_{\Ff}(S)$, $\nu: S^\nu\rightarrow S$ the normalization of $S$, $\Mm^S:=\Mm|_S$, $\Ff_S$ the restricted foliation of $\Ff$ on $S^\nu$, and
    $$K_{\Ff_S}+B_S+\Mm^S_{S^\nu}:=(K_X+B+\Mm_X)|_{S^\nu}.$$
    Then $(S^\nu,\Ff_S,B_S,\Mm^S)$ is lc.    
\end{thm}
\begin{proof}
    By Theorem \ref{thm: not precise adjunction}, $(S^\nu,\Ff_S,B_S,\Mm^S)$ is sub-lc.  The rest part of Theorem \ref{thm: adjunction foliation nonnqc} is only about the coefficients of divisors on $S^\nu$, which is a codimension $2$ property on $X$. Since $\Ff$ is induced by a contraction $X\rightarrow Z$, by Propositions \ref{prop: general hyperplane invariant} and \ref{prop: general hyperplane non-invariant}, we may cut $X$ by general elements in base-point-free linear systems and assume that $\dim X=2$.

If $\rk\Ff=0$, then since $(X,\Ff,B,\Mm)$ is lc, $B=0$ and $\Mm$ descends to $X$, and the theorem is trivial. If $\rk\Ff=2$ then the theorem follows from \cite[Definition 4.7]{BZ16}. Thus we may assume that $\rk\Ff=1$. 

Let $\Ff_S$ be the restricted foliation of $\Ff$ on $S^\nu$. If $S$ is not $\Ff$-invariant, then $\Ff_S$ is a foliation by points. By Lemma \ref{lem: r cartier b m lc gfq}, $K_{\Ff}+S+B$ is $\Rr$-Cartier. By \cite[Proposition 3.4]{Spi20}, there exists an $\Rr$-divisor $B_S\geq 0$ on $S^\nu$ such that
$$(K_{\Ff}+S+B)|_{S^\nu}=K_{\Ff_S}+B_S.$$ 
By Theorem \ref{thm: not precise adjunction}, $(S^\nu,\Ff_S,B_S)$ is lc, so $B_S=0$. The theorem follows in this case. Thus we may assume that $S$ is $\Ff$-invariant.

We only need to check the coefficient near any closed point $y$ on $S^\nu$. Let $x$ be the image of $y$ in $S$. If $\Ff$ is terminal at $x$, then the theorem follows from Lemma \ref{lem: surface pia terminal}. If $\Ff$ is not terminal at $x$, then the theorem follows from Lemma \ref{lem: surface pia not terminal}.
\end{proof}


\begin{thm}\label{thm: precise adjunction when induced}
   Theorem \ref{thm: precise adj gfq} holds when $\Ff$ is induced by a contraction $X\rightarrow Z$. 
\end{thm}
\begin{proof}
When $\Ff$ is induced by a contraction $X\rightarrow Z$, Theorem \ref{thm: precise adj gfq}(2) follows from Theorem \ref{thm: precise adj gfq}(1) and Theorem \ref{thm: adjunction foliation nonnqc}, so we only need to prove Theorem \ref{thm: precise adj gfq}(1). Since Theorem \ref{thm: precise adj gfq}(1) is only about the coefficients of divisors on $S^\nu$, which is a codimension $2$ property on $X$, by Propositions \ref{prop: general hyperplane invariant} and \ref{prop: general hyperplane non-invariant}, we may cut $X$ by general elements in base-point-free linear systems and assume that $\dim X=2$.

If $\rk\Ff=0$, then since $(X,\Ff,B,\Mm)$ is lc, $B=0$ and $\Mm$ descends to $X$, and the theorem is trivial. If $\rk\Ff=2$ then the theorem follows from the usual precise adjunction formula for lc g-pairs \cite[Page 306, Line 30]{BZ16}. Thus we may assume that $\rk\Ff=1$. 

Let $\Ff_S$ be the restricted foliation of $\Ff$ on $S^\nu$. If $S$ is not $\Ff$-invariant, then $\Ff_S$ is a foliation by points. By Lemma \ref{lem: r cartier b m lc gfq}, $K_{\Ff}+S+B$ is $\Rr$-Cartier. By \cite[Proposition 3.4]{Spi20}, there exists an $\Rr$-divisor $B_S\geq 0$ on $S^\nu$ such that
$$(K_{\Ff}+S)|_{S^\nu}=K_{\Ff_S}+B_S.$$ 
By Theorem \ref{thm: not precise adjunction}, $(S^\nu,\Ff_S,B_S)$ is lc, so $B_S=0$. The theorem follows in this case. Thus we may assume that $S$ is $\Ff$-invariant.

We only need to check the coefficient near any closed point $y$ on $S^\nu$. Let $x$ be the image of $y$ in $S$. If $\Ff$ is terminal at $x$, then the theorem follows from Lemma \ref{lem: surface pia terminal}. If $\Ff$ is not terminal at $x$, then the theorem follows from Lemma \ref{lem: surface pia not terminal}.
\end{proof}

\begin{rem}
    Theorem \ref{thm: precise adjunction when induced}, even without the control on the coefficients and with $\Mm=\bm{0}$, is already stronger than \cite[Proposition 3.2]{ACSS21} as the latter requires that $X$ is $\Qq$-factorial. 

    The complete versions of Theorem \ref{thm: precise adj gfq} will be proven after we establish the existence of ACSS modifications in Section \ref{sec: cone}.
\end{rem}





\section{Property \texorpdfstring{$(*)$}{} and ACSS generalized foliated quadruples}\label{sec: acss gfq}

In this section, we introduce the concepts of Property $(*)$ and ACSS generalized foliated quadruples and study their basic properties.



\subsection{Qdlt generalized pairs}
\begin{defn}[Qdlt]\label{defn: qdlt}
Let $(X,B,\Mm)/U$ be an lc g-pair. We say that $(X,B,\Mm)$ is \emph{qdlt} if there exists an open (possibly empty) subset $V\subset X$ satisfying the following.
\begin{enumerate}
    \item $(V,B|_V)$ is $\Qq$-factorial toroidal. In particular, $B|_V$ is a reduced divisor. 
    \item $V$ contains the generic point of any lc center of $(X,B,\Mm)$.
    \item The generic point of any lc center of $(X,B,\Mm)$ is the generic point of an lc center of $(V,B|_V)$.
\end{enumerate}
\end{defn}

\begin{lem}\label{lem: equi def qdlt}
    Let $(X,B,\Mm)/U$ be a lc g-pair. Then the following conditions are equivalent:
    \begin{enumerate}
        \item $(X,B,\Mm)$ is qdlt.
        \item For any lc center of $(X,B,\Mm)$ with generic point $\eta$, near $\eta$, $(X,B)$ is $\Qq$-factorial toroidal and $\Mm$ descends to $X$.
    \end{enumerate}
\end{lem}
\begin{proof}
It is clear that (2) implies (1). Thus we only need to prove (1) implies (2).
    
Let $\eta$ be the generic point of an lc center of $(X,B,\Mm)$. Since $(X,B,\Mm)$ is qdlt, there exists an open subset $V\subset X$ which satisfies Definition \ref{defn: qdlt}. In particular, $\eta$ is an lc center of $(V,B|_V)$ and $\Mm_X|_V$ is $\Rr$-Cartier. We let $\Mm^V:=\Mm|_V$ be the restricted $\bb$-divisor of $\Mm$ on $V$, then $\Mm^V$ is nef$/V$ and $\Mm^V_V=\Mm_X|_V$. Suppose that $h: V'\rightarrow V$ is a resolution of $V$ such that $\Mm^V$ descends to $V'$, and there exists a prime divisor $E$ on $V'$ such that $\Center_{V}E=\bar\eta$ and $E$ is an lc place of $(V,B|_V)$. By the negativity lemma,
$$\Mm^V_{V'}=h^*\Mm^V_V-F$$
for some $F\geq 0$. Moreover, we have either $F=0$ over $\eta$ or $\Supp F=\Supp h^{-1}(\bar\eta)$. Since $(X,B,\Mm)$ is lc, $(V,B|_V,\Mm^V)$ is lc. Thus  $F=0$ over $\eta$. Possibly shrinking $V$, we may assume that $\Mm$ descends to $V$. The lemma follows.
\end{proof}

\begin{lem}\label{lem: qdlt equivalent definition}
Let $(X,B,\Mm)$ be an lc g-pair and $x$ a (not necessarily closed) point of $X$ such that $\bar x$ is an lc center of $(X,B,\Mm)$. Let $d:=\dim X-\dim \bar x$. Then the following conditions are equivalent:
\begin{enumerate}
  \item $(X,B,\Mm)$ is qdlt near $x$.
  \item There exist components $D_1,\dots,D_{d}$ of $\lfloor B\rfloor$, such that 
  \begin{enumerate}
      \item $K_X$ and each $D_i$ is $\Qq$-Cartier near $x$, and
      \item $x\in\Supp D_i$ for each $i$.
  \end{enumerate}
\end{enumerate}
\end{lem}

\begin{proof}
(1)$\Rightarrow$(2) follows from the definition of qdlt, which in turn follows from the definition of toroidal pairs. 
    
We prove (2)$\Rightarrow$(1). Possibly shrinking $X$ to a neighborhood of $x$, we may assume that $(X,\sum_{i=1}^{d}D_i)$ is a pair. Since $B\geq\sum_{i=1}^{d}D_i$, $(X,D)$ is lc near $x$. By \cite[Proposition 34]{dFKX17}, $B=\sum_{i=1}^dD_i$ near $x$, $(X,B)$ is qdlt near $x$, and $\bar x$ is an lc center of $(X,B)$. Since $(X,B,\Mm)$ is lc, $\bar x$ is an lc center of $(X,B,\Mm)$, and $(X,B,\Mm)$ is qdlt near $x$.
\end{proof}

\begin{lem}\label{lem: qdlt perturbation}
Let $(X,B,\Mm)$ be a qdlt g-pair and $D\geq 0$ an $\Rr$-Cartier $\Rr$-divisor on $X$ such that $D\subset\Supp\{B\}$. Then there exists a positive real number $\delta$ such that $(X,B+\delta D,\Mm)$ is qdlt.
\end{lem}
\begin{proof}
By the definition, $\Supp\{B\}$ does not contain any lc center of $(X,B,\Mm)$. Thus $(X,B+\epsilon D,\Mm)$ is lc for some positive real number $\epsilon$. Let $\delta:=\frac{\epsilon}{2}$, then $(X,B+\delta D,\Mm)$ is lc, and any lc center of $(X,B+\delta D,\Mm)$ is an lc center of $(X,B,\Mm)$. By the definition, $(X,B+\delta D,\Mm)$ is qdlt.
\end{proof}



\begin{lem}\label{lem: mmp preserves qdlt}
Let $(X,B,\Mm)/U$ be an lc g-pair and $\phi: (X,B,\Mm)\dashrightarrow (X',B',\Mm)$ a sequence of steps of a $(K_X+B+\Mm_X)$-MMP. Suppose that $(X,B,\Mm)$ is qdlt. Then $(X',B',\Mm)$ is qdlt. 
\end{lem}
We remark here that $\phi$ may not be an MMP$/U$ so $(X',B',\Mm)/U$ may not be a g-pair, but $(X',B',\Mm)/X'$ is a g-pair.
\begin{proof}
Let $S'$ be an lc center of $(X',B',\Mm)$ with generic point $\eta_{S'}$. Let $E$ be an lc place of $(X',B',\Mm)$ such that $\Center_{X'}E=S'$. Since $\phi$ is a sequence of steps of a $(K_X+B+\Mm_X)$-MMP,
$$0\leq a(E,X,B,\Mm)\leq a(E,X',B',\Mm)\leq 0,$$
so $E$ is an lc place of $(X,B,\Mm)$, and $\phi^{-1}$ is an isomorphism near $\eta_{S'}$. 

Let $S:=\Center_XE$. Then near the generic point of $S$, $(X,B)$ is $\Qq$-factorial toroidal and $S$ is an lc center of $(X,B)$. Thus near the generic point of $S'$, $(X',B')$ is $\Qq$-factorial toroidal and $S'$ is an lc center of $(X',B')$. By Lemma \ref{lem: qdlt equivalent definition}, $(X',B',\Mm)$ is qdlt.
\end{proof}

\subsection{Definition of Property \texorpdfstring{$(*)$}{} and ACSS generalized foliated quadruples}

\begin{defn}
Let $f: X\rightarrow Z$ be a projective morphism between normal quasi-projective varieties and $G$ an $\Rr$-divisor on $X$. We say that $G$ is \emph{super$/Z$} if either $Z$ is a point, or there exist ample Cartier divisors $H_1,\dots,H_{2\dim X+1}$ on $Z$ such that $G\geq\sum_{i=1}^{2\dim X+1}f^*H_i.$
\end{defn}

\begin{defn}[Property $(*)$ gfq]\label{defn: foliation property *}
Let $(X,\Ff,B,\Mm)/U$ be a sub-gfq. Let $G\geq 0$ be a reduced divisor on $X$ and let $f: X\rightarrow Z$ be a projective morphism. We say that $(X,\Ff,B,\Mm;G)/Z$ \emph{satisfies Property $(*)$} if the following conditions hold:
\begin{enumerate}
  \item $f: (X,B+G,\Mm)\rightarrow Z$ satisfies Property $(*)$ (See Definition \ref{defn: property *}). In particular, $\pi$ is a contraction.
  \item $\Ff$ is induced by $f$.
  \item $G$ is an $\Ff$-invariant divisor.
\end{enumerate}

If $(X,\Ff,B,\Mm;G)/Z$ satisfies Property $(*)$, then we say that $(X,\Ff,B,\Mm)$ satisfy Property $(*)$, and say that $f$, $Z$, and $G$ are \emph{associated} with $(X,\Ff,B,\Mm)$.

It is clear that property $(*)$ is independent of the choice of $U$. We remark that the choice of $f$ and $G$ may not be unique. We also remark that $f$ may not be a morphism$/U$.
\end{defn}


\begin{defn}[ACSS gfq, {cf. \cite[Definition 4.3]{DLM23}}]\label{defn: ACSS f-triple}
Let $(X,\Ff,B,\Mm)/U$ be a gfq, $G\geq 0$ a reduced divisor on $X$, and $f: X\rightarrow Z$ a projective morphism. We say that $(X,\Ff,B,\Mm;G)/Z$ is \emph{weak ACSS} if 
\begin{enumerate}
    \item $(X,\Ff,B,\Mm;G)/Z$ satisfies Property $(*)$ and $(X,\Ff,B,\Mm)$ is lc, and
    \item $f$ is equi-dimensional.
\end{enumerate}
We say that $(X,\Ff,B,\Mm;G)/Z$ is \emph{ACSS} if the following additional conditions are satisfied:
\begin{enumerate}
  \item[(3)]  There exist an $\Rr$-divisor $D\geq 0$ on $X$ and a nef$/X$ $\bb$-divisor $\Nn$ such that
    \begin{enumerate}
      \item  $\Supp\{B\}\subset\Supp D$,
      \item $\Nn-\alpha \Mm$ is nef$/X$ for some $\alpha>1$, and
      \item for any reduced divisor $\Sigma\geq f(G)$ such that $(Z,\Sigma)$ is log smooth, $$(X,B+D+G+f^*(\Sigma-f(G)),\Nn)$$ 
      is qdlt. In particular, $D+\Nn_X-\Mm_X$ is $\Rr$-Cartier, 
    \end{enumerate}
  \item[(4)] For any lc center of $(X,\Ff,B,\Mm)$ with generic point $\eta$, over a neighborhood of $\eta,$
    \begin{enumerate}
      \item $\Mm$ descends to $X$,
      \item $\eta$ is the generic point of an lc center of $(X,\Ff,\lfloor B\rfloor)$, and
       \item $f: (X,B+G)\rightarrow (Z,f(G))$ is a toroidal morphism, in particular, $(X,B)$ is toroidal and $B=\lfloor B\rfloor$.
    \end{enumerate}
\end{enumerate}

If $(X,\Ff,B,\Mm;G)/Z$ is ACSS, then we say that $f$, $Z$, and $G$ are \emph{properly associated with} $(X,\Ff,B,\Mm)$. If $(X,\Ff,B,\Mm;G)/Z$ is ACSS and $G$ is super$/Z$, then we say that $(X,\Ff,B,\Mm;G)/Z$ is \emph{super ACSS}.

If $(X,\Ff,B,\Mm;G)/Z$ is ACSS \emph{weak ACSS} (resp. \emph{ACSS}, \emph{super ACSS}), then we say that $(X,\Ff,B,\Mm)/Z$ and $(X,\Ff,B,\Mm)$ are \emph{weak ACSS} (resp. \emph{ACSS}, \emph{super ACSS}).
\end{defn}

\begin{rem}
It is possible that $(X,\Ff,B,\Mm;G)/Z$ and $(X,\Ff,B,\Mm;G')/Z$ both satisfy Property $(*)$, but $(X,\Ff,B,\Mm;G)/Z$ is ACSS while $(X,\Ff,B,\Mm;G')/Z$ is not. On the other hand, by definition, if $(X,\Ff,B,\Mm;G)/Z$ and $(X,\Ff,B,\Mm;G')/Z$ both satisfy Property $(*)$, then $(X,\Ff,B,\Mm;G)/Z$ is weak ACSS if and only if $(X,\Ff,B,\Mm;G')/Z$ is weak ACSS.
\end{rem}

\begin{rem}
    The key reason why we define the technical concept ``ACSS" is because of the following two reasons, one from the classical minimal model program point of view, and the other from the foliation point of view.
    
    From the classical minimal model program point of view, ACSS foliated triples behave more similar to qdlt pairs than Property $(*)$ foliated triples. In fact, when $\Ff=T_X$, ``Property $(*)$" is equivalent to ``lc", while ``ACSS" is equivalent to ``qdlt".
    
    From the foliation point of view, ACSS foliated triples are very close to F-dlt foliated triples. In fact, we will show that $\Qq$-factorial F-dlt foliated triples are always ACSS (Theorem \ref{thm: fdlt is acss}). We conjecture that the condition ACSS is equivalent to the condition F-dlt.
\end{rem}

\begin{conj}\label{conj: fdlt equivalent to acss}
    Let $(X,\Ff,B,\Mm)$ be a generalized foliated quadruple. Then $(X,\Ff,B,\Mm)$ is F-dlt if and only if it is ACSS.
\end{conj}
An interesting case of Conjecture \ref{conj: fdlt equivalent to acss} is when $\Ff=T_X$ and $\Mm=\bm{0}$, when it says that a pair $(X,B)$ is qdlt if and only there exists a log toroidal modification $f: Y\rightarrow X$ which only extracts divisors $E$ such that $a(E,X,B)>-1$. We cannot find any literature even on this simplified version of the conjecture. In fact, the dlt version of this conjecture, which indicates that different definitions of dlt coincides, is not a trivial result, and is only proven by Szab\'o \cite{Sza94} based on a complicated resolution lemma.



\subsection{Basic properties of Property \texorpdfstring{$(*)$}{} and ACSS generalized foliated quadruples} In this subsection, we prove several lemmas that will be very useful when applying to the minimal model program for algebraically integrable foliations.


\begin{lem}\label{lem: weak acss can be super}
    Let $(X,\Ff,B,\Mm)/U$ be a gfq and $f: X\rightarrow Z$ a contraction such that $(X,\Ff,B,\Mm)/Z$ satisfies Property $(*)$ (resp. is weak ACSS). Then there exists a super$/Z$ divisor $G$ on $X$ such that if $(X,\Ff,B,\Mm;G)/Z$ satisfies Property $(*)$ (resp. is weak ACSS).
\end{lem}
\begin{proof}
If $(X,\Ff,B,\Mm)/Z$ satisfies Property $(*)$ (resp. is weak ACSS), then there exists a divisor $G_0\geq 0$ on $X$ such that $(X,\Ff,B,\Mm;G_0)/Z$ satisfies Property $(*)$ (resp. is weak ACSS). We let $H_1,\dots,H_{2\dim X+1}$ be general elements of a very ample linear system on $Z$ and let $$G:=G_0+\sum_{i=1}^{2\dim X+1}f^*H_i.$$ 
Then $(X,\Ff,B,\Mm;G)/Z$ satisfies Property $(*)$ (resp. is weak ACSS).
\end{proof}


\begin{lem}\label{lem: acss smaller coefficient}
Assume that $(X,\Ff,B,\Mm)/U$ and $(X,\Ff,B',\Mm')/U$ are two gfqs such that $B\geq B'$ and $\Mm-\Mm'$ is nef$/X$, and all components of $B$ are horizontal$/Z$.

Let $f: X\rightarrow Z$ be a contraction and $G$ a divisor on $X$ such that $(X,\Ff,B,\Mm;G)/Z$ satisfies Property $(*)$ (resp. is weak ACSS, is ACSS, is super ACSS). Then $(X,\Ff,B',\Mm';G)/Z$ satisfies Property $(*)$ (resp. is weak ACSS, is ACSS, is super ACSS). 
\end{lem}
\begin{proof}
The proof of this lemma is straightforward by checking the definitions. However, for the sake of clarity and to assist the reader, we offer a detailed proof below.

\medskip

\noindent\textbf{Step 1}. Suppose that $(X,\Ff,B,\Mm;G)/Z$ satisfies Property $(*)$. Since $(X,\Ff,B,\Mm;G)/Z$ satisfies Property $(*)$, we have the following:
\begin{itemize}
\item $f: (X,B+G,\Mm)\rightarrow Z$ satisfies Property $(*)$. Let $\Sigma_Z:=f(G)$. Then we have the following:
\begin{itemize}
    \item $(Z,\Sigma_Z)$ is log smooth.
    \item Since all components of $B$ are horizontal$/Z$, $G=f^{-1}(\Sigma_Z)$. Since $B\geq B'\geq 0$, all components of $B'$ are horizontal$/Z$. Thus the vertical$/Z$ part of $B'+G$ is equal to $G$.
    \item For any closed point $z\in Z$ and any reduced divisor $\Sigma\geq\Sigma_Z$ such that $(Z,\Sigma)$ is log smooth near $z$, $(X,B+f^*(\Sigma-\Sigma_Z),\Mm)$ is sub-lc over a neighborhood of $z$. Since $B\geq B'$ and $\Mm-\Mm'$ is nef$/U$. $(X,B'+f^*(\Sigma-\Sigma_Z),\Mm')$ is sub-lc over a neighborhood of $z$.
\end{itemize}
\item $\Ff$ is induced by $f$.
\item $G$ is an $\Ff$-invariant divisor.
\end{itemize}
Therefore, $f: (X,B'+G,\Mm')\rightarrow Z$ satisfies Property $(*)$.

\medskip

\noindent\textbf{Step 2}. Suppose that $(X,\Ff,B,\Mm;G)/Z$ is weak ACSS. Then:
\begin{itemize}
    \item $(X,\Ff,B,\Mm;G)/Z$ satisfies Property $(*)$ and $(X,\Ff,B,\Mm)$ is lc. By \textbf{Step 1}, $(X,\Ff,B',\Mm';G)/Z$ satisfies Property $(*)$. Since $B\geq B'$ and $\Mm-\Mm'$ is nef$/U$,  $(X,\Ff,B',\Mm')$ is lc. 
    \item $f$ is equi-dimensional.
\end{itemize}
Thus $(X,\Ff,B',\Mm';G)/Z$ is weak ACSS.

\medskip

\noindent\textbf{Step 3}.  Suppose that $(X,\Ff,B,\Mm;G)/Z$ is ACSS. Then:
\begin{itemize}
\item $(X,\Ff,B,\Mm;G)/Z$ is weak ACSS. By \textbf{Step 2}, $(X,\Ff,B',\Mm';G)/Z$ is weak ACSS.
\item There exists an $\Rr$-divisor $D\geq 0$ on $X$ and a nef$/X$ $\bb$-divisor $\Nn$ satisfying the following. Let $D':=B-B'+D$. Then:
\begin{itemize}
    \item $\Supp\{B\}\subset\Supp D$.  Since $B\geq B'\geq 0$, $\Supp\{B'\}\subset\Supp D'$.
    \item $\Nn-\alpha\Mm$ is nef$/X$ for any $\alpha\geq 1$. Since $\Mm-\Mm'$ is nef$/U$, 
    $$\Nn-\alpha\Mm'=(\Nn-\alpha\Mm)+\alpha(\Mm-\Mm')$$
    is nef$/X$.
    \item For any reduced divisor $\Sigma\geq f(G)$ such that $(Z,\Sigma)$ is log smooth,
    $$(X,B+D+G+f^*(\Sigma-f(G)),\Nn)=(X,B'+D'+G+f^*(\Sigma-f(G)),\Nn)$$
    is qdlt.
\end{itemize}
\item For any lc center $W$ of $(X,\Ff,B',\Mm')$ with generic point $\eta_W$, since $(X,\Ff,B,\Mm)$ is lc, $B\geq B'$, and $\Mm-\Mm'$ is nef$/U$, $W$ is an lc center of $(X,\Ff,B,\Mm)$. Moreover,  over a neighborhood of $\eta_W$, $B=B'$ and $\Mm=\Mm'$. Therefore, over a neighborhood of $\eta_W$, we have the following:
\begin{itemize}
\item $\Mm$ descends to $X$, so $\Mm'$ descends to $X$.
\item $W$ is an lc center of $(X,\Ff,\lfloor B\rfloor)$. Since $B=B'$, $W$ is an lc center of $(X,\Ff,\lfloor B'\rfloor)$.
\item $f: (X,B+G)\rightarrow (Z,f(G))$ is a toroidal morphism.  Since $B=B'$, $f: (X,B'+G)\rightarrow (Z,f(G))$ is a toroidal morphism. 
\end{itemize}
\end{itemize}
Thus $(X,\Ff,B',\Mm';G)/Z$ is ACSS.

\medskip

\noindent\textbf{Step 4}. Suppose that $(X,\Ff,B,\Mm;G)/Z$ is super ACSS. Then $G$ is super$/Z$. By \textbf{Step 3}, $(X,\Ff,B',\Mm';G)/Z$ is ACSS. Thus $(X,\Ff,B',\Mm';G)/Z$ is super ACSS.
\end{proof}




\begin{lem}\label{lem: fls imply acss}
Let $(X,\Ff,B,\Mm)$ be foliated log smooth gfq such that $\Ff$ is algebraically integrable, $f: (X,\Sigma_X,\Mm)\rightarrow (Z,\Sigma_Z)$ a contraction associated to $(X,\Ff,B,\Mm)$, and $G$ the vertical$/Z$ part of $\Sigma_X$. Then $(X,\Ff,B^\Ff,\Mm;G)/Z$ is $\Qq$-factorial ACSS, and $(X,\Ff,B^\Ff,\Mm;G')/Z$ is $\Qq$-factorial super ACSS for some $G'\geq G$.
\end{lem}
\begin{proof}
The proof of this lemma is straightforward by checking the definitions and applying \cite[Proposition 3.2]{AK00}. However, for the sake of clarity and to assist the reader, we offer a detailed proof below.

First we show that $(X,\Ff,B^\Ff,\Mm;G)/Z$ is $\Qq$-factorial ACSS. By assumption, $X$ is $\Qq$-factorial. By Lemma \ref{lem: acss smaller coefficient}, we only need to show that $(X,\Ff,\Sigma_X-G,\Mm;G)/Z$ is ACSS, and we may assume that $B=B^{\Ff}=\Sigma_X-G$.

 By Proposition \ref{prop: weak ss satisfies *}, $(X,\Ff,B,\Mm;G)/Z$ satisfies Property $(*)$. By Lemma \ref{lem: foliated log smooth imply lc}, $(X,\Ff,B,\Mm)$ is lc, so  $(X,\Ff,B,\Mm;G)/Z$ is weak ACSS.

Let $D:=0$ and $\Nn:=\bm{0}$. Then:
 \begin{itemize}
     \item Since $\{B\}=0$, $\Supp\{B\}\subset\Supp D$.
     \item Since $\Mm$ descends to $X$, $\Nn-2\Mm$ is nef$/X$.
     \item For any reduced divisor $\Sigma\geq f(G)$, by \cite[Proposition 3.2]{AK00}, 
$$f: (X,B+D+G+f^*(\Sigma-f(G)),\Mm)\rightarrow (Z,\Sigma)$$
is toroidal.
 \end{itemize}
 For any lc center $W$ of $(X,\Ff,B,\Mm)$ with generic point $\eta_W$, near $\eta_W$, we have the following:
 \begin{itemize}
     \item $\Mm$ descends to $X$.
     \item Since $B=\lfloor B\rfloor$ and $\Mm$ descends to $X$, $W$ is an lc center of $(X,\Ff,\lfloor B\rfloor)$.
     \item Since $f: (X,B+G,\Mm)\rightarrow (Z,f(G))$ is a toroidal morphism, $f: (X,B+G)\rightarrow (Z,f(G))$ is a toroidal morphism.
 \end{itemize}
 Therefore, $(X,\Ff,B,\Mm;G)/Z$ is ACSS. 
 
 
 Let $H_1,\dots, H_{2\dim X+1}$ be ample Cartier divisors on $Z$. By \cite[Proposition 3.2]{AK00}, $$f: (X,\Sigma_X+\sum_{i=1}^{2\dim X+1}f^*H_i,\Mm)\rightarrow (Z,\Sigma_Z+\sum_{i=1}^{2\dim X+1}H_i)$$ is associated with $(X,\Ff,B,\Mm)$. Thus
 $$\left(X,\Ff,B,\Mm;G':=G+\sum_{i=1}^{2\dim X+1}f^*H_i\right)/Z$$
 is $\Qq$-factorial ACSS. Since $G'$ is super$/Z$, we are done.
\end{proof}

\begin{lem}\label{lem: alg int foliation lct achieved}
Let $(X,\Ff,B,\Mm)/U$ be a sub-gfq, $D$ an $\Rr$-divisor on $X$, and $\Nn$ a $\bb$-divisor on $X$ such that $D+\Nn_X$ is $\Rr$-Cartier and $\Nn$ descends to a birational model of $X$. Suppose that $\Ff$ is algebraically integrable. Let
$$t:=\sup\{s\mid s\geq 0, \Mm+s\Nn\text{ is nef}/U,\text{ and } (X,\Ff,B+sD,\Mm+s\Nn)/X\text{ is sub-lc}\}.$$
Then either $t=+\infty$, or 
$$t=\max\{s\mid s\geq 0, \Mm+s\Nn\text{ is nef}/U,\text{ and } (X,\Ff,B+sD,\Mm+s\Nn)/X\text{ is sub-lc}\}.$$
Moreover, one of the following cases hold:
\begin{enumerate}
    \item $t=+\infty$.
    \item $t<+\infty$, and $\Mm+(t+\delta)\Nn$ is not nef$/U$ for any $\delta>0$.
    \item $t<+\infty$, $\Mm+(t+\delta_0)\Nn$ is nef$/U$ for some $\delta_0>0$, and there exists a prime divisor $E$ over $X$, such that
    $$a(E,X,\Ff,B+tD,\Mm+t\Nn)=-\epsilon_{\Ff}(E)$$
    and
    $$a(E,X,\Ff,B+sD,\Mm+s\Nn)<-\epsilon_{\Ff}(E)$$
    for any $s>t$.
\end{enumerate}


In particular, $(X,\Ff,B+tD)$ is \text{sub-lc} and $\Mm+t\Nn$ is nef$/U$ if $t<+\infty$.
\end{lem}
\begin{proof}
We may assume that $t<+\infty$. Since discrepancies of divisors are preserved under crepant pullbacks, by Definition-Theorem \ref{defthm: weak ss reduction} and Lemma \ref{lem: existence foliated log resolution}, we may assume that $\Mm$ and $\Nn$ descend to $X$ and $(X,\Ff,\Supp B\cup\Supp D)$ is foliated log smooth. Then
$$t=\min\{\sup\{s\mid s\geq 0, \Mm_X+s\Nn_X\text{ is nef}/U\},\sup\{s\mid s\geq 0, (X,\Ff,B+sD)/X\text{ is sub-lc}\}\}.$$
Since nef is a closed condition, 
$$\sup\{s\mid s\geq 0, \Mm_X+s\Nn_X\text{ is nef}/U\}=\max\{s\mid s\geq 0, \Mm_X+s\Nn_X\text{ is nef}/U\}\text{ or }+\infty.$$
Thus we may assume that 
$$t=\sup\{s\mid s\geq 0, (X,\Ff,B+sD)/X\text{ is sub-lc}\}<+\infty$$ 
and $\Mm_X+t\Nn_X$ is nef$/U$. By Lemma \ref{lem: foliated log smooth imply lc},
\begin{align*}
    t&=\sup\{s\mid 0\leq s\leq l, a(E,X,\Ff,B+sD)\geq-\epsilon_{\Ff}(E)\text{ for any prime divisor } E\text{ on }X\}\\
    &=\sup\{s\mid 0\leq s\leq l, a(E,X,\Ff,B+sD)\geq-\epsilon_{\Ff}(E)\text{ for any prime divisor } E\subset\Supp D\}.
\end{align*}%\chen{$E$ on $\bar B$}
Since there are only finitely many components of $\Supp D$ and $t<+\infty$, 
$$t=\max\{s\mid 0\leq s\leq l, a(E,X,\Ff,B+sD)\geq-\epsilon_{\Ff}(E)\text{ for any prime divisor } E\subset\Supp D\}.$$
and there exists a component $E$ of $\Supp D$, such that $a(E,X,\Ff,B+tD)=-\epsilon_{\Ff}(E)$ and $\mult_ED>0$. The lemma follows.
\end{proof}


\begin{lem}\label{lem: acss f-triple perturb coefficient}
Let $(X,\Ff,B,\Mm)/U$ be a gfq, $f: X\rightarrow Z$ a contraction, and $G$ a divisor on $X$, such that $(X,\Ff,B,\Mm;G)/Z$ is ACSS. Let $D$ be an $\Rr$-divisor on $X$ and $\Nn$ a $\bb$-divisor on $X$ satisfying the following:
\begin{enumerate}
  \item $D$ and $\Nn_X$ are $\Rr$-Cartier.
  \item $\Supp D\subset\Supp\{B\}$ and $\Nn$ descends to a birational model of $X$.
  \item $\Mm+\Nn$ is nef$/U$, and $\Mm-\delta\Nn$ is nef$/U$ for some $\delta>0$.
\end{enumerate}
Then there is a positive real number $\gamma$ such that $(X,\Ff,B+\alpha D,\Mm+\beta\Nn;G)/Z$ is ACSS for any $\alpha,\beta\in [0,\gamma]$.
\end{lem}
\begin{proof}
Possibly replacing $\delta$ with $\min\{1,\delta\}$ and then replacing $\Nn$ with $\delta\Nn$, we may assume that $\delta=1$ and $\Mm-\Nn$ is nef$/U$.

By assumption, $\Supp D$ does not contain any lc center of $(X,\Ff,B,\Mm)$, and $\Mm$ descends to $X$ near the generic point of any lc center of $(X,\Ff,B,\Mm)$. Since $\Mm-\Nn$ is nef$/X$ and $\Mm+\Nn$ is nef$/X$, near the generic point of any lc center of $(X,\Ff,B,\Mm)$, $\Nn$ is nef$/X$ and $-\Nn$ is nef$/X$. Thus $\Nn$ descends to $X$ near the generic point of any lc center of $(X,\Ff,B,\Mm)$.

Since $\Mm+\Nn$ is nef$/U$, by Lemma \ref{lem: alg int foliation lct achieved}, there exists a real number $\gamma_0\in (0,1)$ such that $(X,\Ff,B+\gamma_0 D,\Mm+\gamma_0\Nn)$ is lc. Possibly replacing $\gamma_0$ with $\frac{1}{2}\gamma_0$, we may assume that $(X,\Ff,B+\gamma_0 D,\Mm+\gamma_0\Nn)$ and $(X,\Ff,B,\Mm)$ have the same lc centers. 

Since $(X,\Ff,B,\Mm;G)/Z$ is ACSS, there exists an $\Rr$-divisor $D'\geq 0$ on $X$ and a nef$/X$ $\bb$-divisor $\Nn'$ on $X$, such that $\Supp\{B\}\subset\Supp D'$, $\Nn'-\alpha'\Mm$ is nef$/X$ for some $\alpha'>1$, and for any $\Sigma\geq f(G)$ such that $(Z,\Sigma)$ is log smooth, 
$$(X,B+D'+G+\pi^*(\Sigma-f(G)),\Nn')$$ 
is qdlt. Possibly replacing $\alpha'$, we may assume that $D'\geq (\alpha'-1)\Supp D'$.

In the following, we show that
$$\gamma:=\min\left\{\gamma_0,\frac{\alpha'-1}{2}\right\}$$ 
satisfies our requirements. By Lemma \ref{lem: acss smaller coefficient}, we only need to show that $(X,\Ff,B+\gamma D,\Mm+\gamma\Nn;G)/Z$ is ACSS.
\begin{itemize}
    \item (Definition \ref{defn: ACSS f-triple}(3.a)) Since
$$\Supp D\subset\Supp\{B\}\subset\Supp D',$$
we have
$$D'-\gamma D\geq 2\gamma\Supp D'-\gamma\Supp D=\gamma(\Supp D'-\Supp D)+\gamma\Supp D'\geq\gamma\Supp D',$$
hence
$$\Supp\{B+\alpha D\}\subset\Supp D'=\Supp(D'-\gamma D).$$
\item  (Definition \ref{defn: ACSS f-triple}(3.b)) Let $\alpha'':=\frac{\alpha'}{1+\gamma}$. Then $\alpha''>1$, and
$$\Nn'-\alpha''(\Mm+\gamma\Nn)=\Nn'-\alpha'\Mm+\alpha''\gamma(\Mm-\Nn)$$
is nef$/X$.
\item  (Definition \ref{defn: ACSS f-triple}(3.c)) For any reduced divisor $\Sigma\geq f(G)$ such that $(Z,\Sigma)$ is log smooth,
$$(X,B+\gamma D+(D'-\gamma D)+f^*(\Sigma-f(G)),\Nn')=(X,B+D'+f^*(\Sigma-f(G)),\Nn')$$
is qdlt. In particular, $(X,B+\gamma D+f^*(\Sigma-f(G)),\Mm+\beta\Nn)$ is lc.
\item (Definition \ref{defn: ACSS f-triple}(1-2)) Since $(X,\Ff,B+\gamma_0D,\Mm+\gamma_0\Nn)$ is lc, $(X,\Ff,B+\gamma D,\Mm+\gamma \Nn)$ is lc. Since $(X,\Ff,B,\Mm;G)/Z$ is ACSS, $(Z,\Sigma_Z:=f(G))$ is log smooth, $G=f^{-1}(\Sigma_Z)$, $B$ is  horizontal$/Z$, $\Ff$ is induced by $f$, $G$ is $\Ff$-invariant, and $f$ is equi-dimensional. Since $\Supp D\subset\Supp\{B\}$, $B+\gamma D$ is horizontal$/Z$, so the horizontal$/Z$ part of $B+\gamma D+G$ is $G$.
\item Definition \ref{defn: ACSS f-triple}(4)) Let $W$ be an lc center of $(X,\Ff,B+\gamma D,\Mm+\gamma \Nn)$. Since $(X,\Ff,B+\gamma_0 D,\Mm+\gamma_0\Nn)$ and $(X,\Ff,B,\Mm)$ have the same lc centers, $W$ is an lc center of $(X,\Ff,B,\Mm)$ and an lc center of $(X,\Ff,B+\gamma D,\Mm+\gamma \Nn)$. In particular, $\Nn$ descends to $X$ near the generic point of $X$ and $D=0$ near the generic point of $X$. Since $(X,\Ff,B,\Mm)$ is ACSS, near the generic point $\eta$ of any lc center of $(X,\Ff,B+\gamma D,\Mm+\gamma \Nn)$,
\begin{itemize}
    \item $\Mm+\gamma\Nn$ descends to $X$,
    \item $\eta$ is the generic point of an lc center of $(X,\Ff,\lfloor B\rfloor)=(X,\Ff,\lfloor B+\gamma D\rfloor)$, and
    \item $f: (X,B+G)\rightarrow (Z,\Sigma_Z)$ is a toroidal morphism, so $f: (X,B+\gamma D+G)\rightarrow (Z,\Sigma_Z)$ is a toroidal morphism.
\end{itemize}
\end{itemize}
\end{proof}

Finally, we recall the following proposition which shows that the numerical property of the foliated log canonical divisor and the log canonical divisor are related with each other for generalized foliated quadruples satisfying Property $(*)$.

\begin{prop}[cf. {\cite[Proposition 3.6]{ACSS21}}]\label{prop: weak cbf gfq}
Let $(X,B+G,\Mm)$ be a g-sub-pair and $f: X\rightarrow Z$ an equi-dimensional contraction, such that $f: (X,B+G,\Mm)\rightarrow Z$ satisfies Property $(*)$. Assume that $B$ is horizontal$/Z$ and $G$ is vertical$/Z$. Let $\Ff$ be the foliation induced by $f$ and let $\Nn$ be the moduli part of $f: (X,B+G,\Mm)\rightarrow Z$. Then:
\begin{enumerate}
  \item $K_{\Ff}+B+\Mm_X\sim \Nn_X$.
  \item $K_{\Ff}+B+\Mm_X\sim_{Z}K_X+B+G+\Mm_X.$
\end{enumerate}
In particular, $K_{\Ff}+B+\Mm_X$ is $\Rr$-Cartier.
\end{prop}
\begin{proof}
Since $f: (X,B+G,\Mm)\rightarrow Z$ satisfies Property $(*)$, $Z$ is smooth. Let
$$R:=\sum_{D\mid D\text{ is a prime divisor on }Z}(f^*D-f^{-1}(D)).$$
Since $f$ is equi-dimensional, we have
$$K_{\Ff}=K_{X/Z}-R.$$
Let $B_Z$ be the discriminant part of $f: (X,B+G,\Mm)\rightarrow Z$. By Lemma \ref{lem: basic property (*) gpair}, $B_Z$ is reduced. Since $B$ is horizontal$/Z$, $B_Z=f(G)$.
\begin{claim}\label{claim: f*B_Z=R+G}
    $f^*B_Z=R+G$.
\end{claim}
\begin{proof}
    We let $D$ be a prime divisor on $X$ such that $D$ is vertical$/Z$. Since $f$ is equi-dimensional, $D_Z:=f(D)$ is a divisor. 

    If $D_Z$ is a component of $B_Z$, then $D$ is a component of the vertical$/Z$ part of $B+G$. Since $B$ is horizontal$/Z$, $D$ is a component of $G$. Thus $\mult_DG=1$. Therefore, 
    \begin{align*}
    \mult_Df^*B_Z&=\mult_Df^*D_Z=\mult_Df^{-1}(D_Z)+\mult_D(f^*D_Z-f^{-1}(D_Z))\\
    &=\mult_DG+\mult_DR.
\end{align*}
If $D_Z$ is not a component of $B_Z$, then $\mult_Df^*B_Z=0$. Since $f(G)=B_Z$, $\mult_DG=0$. Since $B_Z$ is the discriminant part of $f: (X,B+G,\Mm)\rightarrow Z$,
$$1=\sup\{t\mid (X,B+G+tf^*D_Z,\Mm)\text{ is sub-lc over the generic point of }D_Z\}.$$
Thus $f^*D_Z$ is a reduced divisor, hence $\mult_DR=0$. 

Since $f^*B_Z$ and $R+G$ are both vertical$/Z$, the claim follows.
\end{proof}
\noindent\textit{Proof of Proposition \ref{prop: weak cbf gfq} continued}. By Claim \ref{claim: f*B_Z=R+G}, $f^*B_Z=R+G$. Thus
\begin{align*}
    \Nn_X&\sim K_X+B+G+\Mm_X-f^*(K_Z+B_Z)=K_{X/Z}+B+G+\Mm_X-f^*B_Z\\
    &=K_{\Ff}+R+B+G-f^*B_Z=K_{\Ff}+B+\Mm_X.
\end{align*}
(1) immediately follows. Since $Z$ is smooth and $B_Z$ is reduced, $K_Z+B_Z$ is Cartier. Thus
$$\Nn_X\sim K_X+B+G+\Mm_X-f^*(K_Z+B_Z)\sim_Z K_X+B+G+\Mm_X.$$
\end{proof}

\subsection{\texorpdfstring{$(*)$-}{ }models and ACSS models}

\begin{defn}\label{defn: acss model}
Let $(X,\Ff,B,\Mm)/U$ be a gfq such that $\Ff$ is algebraically integrable. A \emph{$(*)$-modification} (resp. \emph{$\Qq$-factorial $(*)$-modification, ACSS modification, super ACSS modification}) of $(X,\Ff,B,\Mm)$ is a birational morphism $h: X'\rightarrow X$ such that
    \begin{enumerate}
\item $$\left(X',\Ff':=h^{-1}\Ff,B':=h^{-1}_*(B\wedge\Supp B)+(\Supp\Exc(h))^{\Ff'},\Mm\right)$$ is weak ACSS (resp. $\Qq$-factorial weak ACSS, ACSS, super ACSS),
\item $X'$ is klt, and
\item for any $h$-exceptional prime divisor $E$,
$$a(E,\Ff,B,\Mm)\leq-\epsilon_{\Ff}(E).$$
In particular, if $(X,\Ff,B,\Mm)$ is lc, then $a(E,\Ff,B,\Mm)=-\epsilon_{\Ff}(E)$ for any $h$-exceptional prime divisor $E$.
\end{enumerate}
We say that $(X',\Ff',B',\Mm)$ is a $(*)$-model (resp. $\Qq$-factorial $(*)$-model, ACSS model, super ACSS model) of $(X,\Ff,B,\Mm)$. Moreover, for any divisor $G$ on $X'$ and contraction $f: X'\rightarrow Z$ such that $(X',\Ff',B',\Mm;G)/Z$ satisfies Property $(*)$ (resp. satisfies Property $(*)$, is ACSS, is super ACSS), we say that $(X',\Ff',B',\Mm;G)/Z$  is a $(*)$-model (resp. $\Qq$-factorial $(*)$-model, ACSS model, super ACSS model) of $(X,\Ff,B,\Mm)$. In addition, if
\begin{enumerate}
\item[(4)] $D\subset\Supp G$ for any $h$-exceptional $\Ff'$-invariant divisor,
\end{enumerate}
then we say that $h: X'\rightarrow X$ is a \emph{proper \emph{$(*)$-modification}} (resp. \emph{proper $\Qq$-factorial $(*)$-modification, proper ACSS modification, great ACSS modification}) of $(X,\Ff,B,\Mm)$, and say that $(X',\Ff',B',\Mm)$ is a \emph{proper $(*)$-model} (resp. \emph{$\Qq$-factorial proper $(*)$-model, proper ACSS model, great ACSS model}) of $(X,\Ff,B,\Mm)$.
\end{defn}


\begin{nota}
Let $(X_0,\Ff_0,B_0,\Mm)/U$ be a gfq satisfying Property $(*)$ and is associated with $X\rightarrow Z$ and $G$. When we say the following
\begin{center}$\xymatrix{
(X_0,\Ff_0,B_0,\Mm;G_0)\ar@{-->}[r]^{f_0} & (X_1,\Ff_1,B_1,\Mm;G_1)\ar@{-->}[r]^{\ \ \ \ \ \ \ \ \ \ f_1} & \dots\ar@{-->}[r] & (X_n,\Ff_n,B_n,\Mm;G_n)\ar@{-->}[r]^{\ \ \ \ \ \ \ \ \ \ f_n} & \dots 
}$
\end{center}
is a (possibly infinite) sequence of steps of a $(K_{\Ff_0}+B_0+\Mm_{X_0})$-MMP$/U$, we mean the following: for any $i$, $f_i: X_{i}\dashrightarrow X_{i+1}$ is a step of a $(K_{\Ff_i}+B_i+\Mm_{X_i})$-MMP$/U$ that is not a Mori fiber space, $\Ff_{i+1}:=(f_i)_*\Ff_i$, $B_{i+1}:=(f_i)_*B_i$, and $G_{i+1}:=(f_i)_*G_i$.
\end{nota}





\section{Cone theorem and ACSS modifications}\label{sec: cone}

In this section we prove the cone theorem (Theorem \ref{thm: cone theorem gfq}) and the existence of ACSS modifications (Theorem \ref{thm:  ACSS model}). As an immediate corollary, we will prove the precise adjunction formula (Theorem \ref{thm: precise adj gfq}) in full generality, without assuming that $\Ff$ is induced by a contraction.

\subsection{Bend and break} It is important to notice that we will work under the relative setting, so the following relative bend and break theorem is crucial for our proofs.

\begin{thm}[Relative bend and break]\label{thm: relative bb}
Let $d$ be a positive integer, $\pi: X\rightarrow U$ a contraction from a normal quasi-projective variety to a variety such that $\dim X-\dim U=d$, $M,D_1,\dots,D_d$ $\Rr$-divisors on $X$ that are nef along general fibers of $\pi$, $B\geq 0$ an $\Rr$-divisor on $X$, and $\Ff$ a foliation on $X$. Suppose that for any general fiber $F$ of $\pi$,
\begin{enumerate}
    \item $(D_1|_F)\cdot (D_2|_F)\cdot\dots\cdot (D_d|_F)=0$, and
    \item $-(K_{\Ff}+B)|_F\cdot (D_2|_F)\cdot\dots\cdot (D_d|_F)>0$.
\end{enumerate}
Then for any general closed point $x\in X$, there exists a rational curve $C_x$ satisfying the following.
\begin{enumerate}
    \item $x\in C_x$, 
    \item $\pi(C_x)$ is a point, and
    \item $D_1\cdot C_x=0$ and
    $$M\cdot C_x\leq 2d\frac{M|_F\cdot  (D_2|_F)\cdot\dots\cdot (D_d|_F)}{-K_{\Ff}|_F\cdot  (D_2|_F)\cdot\dots\cdot (D_d|_F)}.$$
\end{enumerate}
\end{thm}
\begin{proof}
Since  (3) is a closed condition and $M$ is a limit of $\Qq$-divisors that are nef along general fibers of $\pi$, we may assume that $M$ is a $\Qq$-divisor. Possibly replacing $M$ with a multiple, we may assume that $M$ is a Weil divisor.

We let $X^c$ and $U^c$ be compactifications of $X$ and $U$, such that $X^c$ and $U^c$ are normal projective, $X$ is a non-empty open subset of $X^c$, $U$ is a non-empty open subset of $U^c$, and there exists a contraction $\pi^c: X^c\rightarrow U^c$ such that $\pi^C|_{X}=\pi$. Let $M^c,D^c_1,\dots,D^c_d,B^c$ be the closures of $M,D_1,\dots,D_d,B$ in $X^c$ respectively, and let $\Ff^c$ be the natural extension of $\Ff$ in $X^c$ \cite[Lemma 2.2]{CS23b}. Then the general fibers of $\pi^c$ are general fibers of $\pi$, and $M^c,D^c_1,\dots,D^c_d$ are $\Rr$-divisors that are nef along general fibers of $\pi$. Since we only care about properties about general fibers of $\pi$ and properties near a general closed point $x\in X$, we may replace $\pi: X\rightarrow U$ with $\pi^c: X^c\rightarrow U^c$, $M,D_1,\dots,D_d,B$ with $M^c,D^c_1,\dots,D^c_d,B^c$, and $\Ff$ with $\Ff^c$, and assume that $\pi$ is a projective morphism between normal projective varieties.

Let $x\in X$ be a general closed point. Then $x$ is contained in a general fiber $F$ of $\pi$. Let $q:=\dim U$. Then there exist general hyperplane sections $H_1,\dots,H_q$ with $A_i:=\pi^*H_i$, such that $F=\cap_{i=1}^q\pi^*A_i$. Let $V_k:=X\cap\cap_{i=1}^kA_i$ and $W_k:=U\cap\cap_{i=1}^kH_i$ for each $0\leq k\leq q$, then
$$F=V_q\subset V_{q-1}\subset\dots\subset V_0=X$$
and
$$z:=W_q\subset W_{q-1}\subset\dots\subset W_0=U,$$
where $z$ is a general closed point. We may inductively define $\Ff_{k}$ to be the restricted foliation of $\Ff$ on $V_k$ for each $k$, and let $\Ff_F:=\Ff_{q}$. We let $M_k:=M|_{V_k}$, $B_k:=B|_{V_k}$, $M_F:=M|_F$, and $B_F:=B|_F$. Then it is clear that $M_k|_F=M|_F$, $B_k|_F=B_F$ for each $k$, and $B_{V_k}\geq 0$ for each $k$. Moreover, since $H_1,\dots,H_q$ are general hyperplane sections, $M_k$ is a Weil divisor for each $k$.

\begin{claim}\label{claim: induction bend and break}
There exists a rational curve $C_x$, such that $x\in C_x$, $\pi(C_x)$ is a closed point, $D_1\cdot C_x=0$, and
$$M|_F\cdot C_x\leq2d\frac{M|_F\cdot  (D_2|_F)\cdot\dots\cdot (D_d|_F)}{-K_{\Ff_k}|_F\cdot  (D_2|_F)\cdot\dots\cdot (D_d|_F)}$$
for each $k$.
\end{claim}
\begin{proof}
    We apply induction on $q-k$. When $q-k=0$, the existence of $C_x$ follows from \cite[Corollary 2.28]{Spi20}. We will show that this $C_x$ satisfies our requirement for all $q-k$ as well. In the following, we may assume that $q>k$.

    We let $\pi_k: V_k\rightarrow W_k$ be the restricted contraction of $\pi$ to $V_k$ for each $k$. We consider $W_{k+1}$ as a divisor on $W_k$ and $V_{k+1}$ as a divisor on $V_k$. There are two possibilities.

\medskip

\noindent\textbf{Case 1}. $V_{k+1}$ is $\Ff_k$-invariant. In this case, the general fibers of $\pi_k$ are tangent to $\Ff_k$, so 
$$K_F=K_{\Ff_F}=K_{\Ff_k}|_F.$$
Thus by the $q-k=0$ case,
$$M|_F\cdot C_x\leq2d\frac{M|_F\cdot  (D_2|_F)\cdot\dots\cdot (D_d|_F)}{-K_{\Ff_F}\cdot  (D_2|_F)\cdot\dots\cdot (D_d|_F)}=2d\frac{M|_F\cdot  (D_2|_F)\cdot\dots\cdot (D_d|_F)}{-K_{\Ff_k}|_F\cdot  (D_2|_F)\cdot\dots\cdot (D_d|_F)}.$$

\medskip

\noindent\textbf{Case 2}. $V_{k+1}$ is not $\Ff_k$-invariant. In this case, by \cite[Proposition 3.6(1)]{Dru21}, we have
$$(K_{\Ff_k}+V_{k+1})|_{V_{k+1}}\sim K_{\Ff_{k+1}}+D_{k+1}$$
for some $\Qq$-divisor $D_{k+1}\geq 0$.  We remark that \cite[Proposition 3.6(1)]{Dru21} requires that $2\leq \rk\Ff\dim X-1$, but the same lines of the proof works for the case when $\rk\Ff=1$ as well, and the $\rk\Ff=\dim X$ case is the classical adjunction formula.

Since $H_{k+1}$ is a general hyperplane section, there exists $H_{k+1}'\sim H_{k+1}$ such that $H_{k+1}'$ does not contain $z$. Thus
$$V_{k+1}|_F=(H_{k+1}|_{V_k})|_F=H_{k+1}|_F\sim H'_{k+1}|_F=0.$$
Since $H_{k+2},\dots,H_q$ are general hyperplane sections, $D_{k+1}|_F\geq 0$. Therefore,
\begin{align*}
    &-K_{\Ff_k}|_F\cdot  (D_2|_F)\cdot\dots\cdot (D_d|_F)\\
    =&-(K_{\Ff_k}+V_{k+1})|_F\cdot  (D_2|_F)\cdot\dots\cdot (D_d|_F)\\
    =&-((K_{\Ff_k}+V_{k+1})|_{V_{k+1}})|_F\cdot  (D_2|_F)\cdot\dots\cdot (D_d|_F)\\
    =&-(K_{\Ff_{k+1}}+D_{k+1})|_F\cdot  (D_2|_F)\cdot\dots\cdot (D_d|_F)\\
    \leq&-K_{\Ff_{k+1}}|_F\cdot  (D_2|_F)\cdot\dots\cdot (D_d|_F).
\end{align*}
By induction hypothesis,
$$M|_F\cdot C_x\leq2d\frac{M|_F\cdot  (D_2|_F)\cdot\dots\cdot (D_d|_F)}{-K_{\Ff_{k+1}}|_F\cdot  (D_2|_F)\cdot\dots\cdot (D_d|_F)}=2d\frac{M|_F\cdot  (D_2|_F)\cdot\dots\cdot (D_d|_F)}{-K_{\Ff_k}|_F\cdot  (D_2|_F)\cdot\dots\cdot (D_d|_F)}.$$
\end{proof}
\noindent\textit{Proof of Lemma \ref{thm: relative bb} continued}. It immediately follows from Claim \ref{claim: induction bend and break} by letting $k=0$.
\end{proof}

\subsection{Inductive statements to cone theorem}

Similar to \cite[Theorems 3.9, 3.10]{ACSS21}, the cone theorem for generalized foliated quadruples is closely related to the existence of $(*)$-models for generalized foliated quadruples, and their proofs are done inductively. For applications to the rest of the paper as well as future works, we shall establish a much stronger version of the existence of $(*)$-models: the existence of great ACSS models with controlled extraction of divisors. This kind of model is more technically constructed, but is also more useful in practice.

\begin{thm}[Cone theorem for induction, cf. {\cite[Theorem 3.9]{ACSS21}}]\label{thm: cone theorem induction}
Let $d$ be a positive integer. Let $(X,\Ff,B,\Mm)/U$ be a gfq of dimension $d$ such that $\Ff$ is algebraically integrable. Let $\{R_j\}_{j\in\Lambda}$ be the set of all $(K_{\Ff}+B+\Mm_X)$-negative extremal rays$/U$ that are not contained in the non-lc locus of $(X,\Ff,B,\Mm)$. Then
$$\overline{NE}(X/U)=\overline{NE}(X/U)_{K_{\Ff}+B+\Mm_X\geq 0}+\overline{NE}(X/U)_{\Nlc(X,\Ff,B,\Mm)}+\sum_{j\in\Lambda} R_j,$$
and for any $j\in\Lambda$, $R_j$ is exposed and is spanned by a rational curve $C_j$, such that $C_j$ is tangent to $\Ff$ and $$0<-(K_{\Ff}+B+\Mm_X)\cdot C_j\leq 2d.$$
\end{thm}

\begin{thm}[Existence of ACSS models, cf. {\cite[Theorem 3.10]{ACSS21}, \cite[Proposition 4.14]{DLM23}}]\label{thm: property * induction}
Let $d$ be a positive integer and $s$ a non-negative integer. Let $(X,\Ff,B,\Mm)/U$ be a gfq of dimension $d$ such that $\Ff$ is algebraically integrable, and $E_1,\dots,E_s$ lc places of $(X,\Ff,B,\Mm)$, such that $(X,\Ff,B,\Mm)$ is lc near the generic point of $\Center_X{E_i}$ for each $i$. Then $(X,\Ff,B,\Mm)$ has a great ACSS model $(Y,\Ff_Y,B_Y,\Mm)$, such that $E_1,\dots,E_s$ are on $Y$ if $(X,\Ff,B,\Mm)$ is lc.
\end{thm}

In the following, we will prove Theorems \ref{thm: cone theorem induction} and \ref{thm: property * induction} by induction on $d$. We will often use the following useful lemma:

\begin{lem}\label{lem: extremal ray under morphism}
    Let $X\rightarrow U$ be a projective morphism from a normal quasi-projective variety to a variety and $R$ an extremal ray in $\overline{NE}(X/U)$. Let $h: Y\rightarrow X$ be a projective morphism such that $R$ is contained in the image of the induced map $\iota: \overline{NE}(Y/U)\rightarrow\overline{NE}(X/U)$. Then there exists an extremal ray $R_Y$ in $\overline{NE}(Y/U)$ such that $\iota(R_Y)=R$.
\end{lem}
\begin{proof}
Since $R$ is contained in the image of $\iota$, there exists a ray $R'$ in $\overline{NE}(Y/U)$ such that $\iota(R')=R$. Then there exist extremal rays $R_i'$ in $\overline{NE}(S/U)$ such that $R'=\sum a_iR_i'$ for some $a_i>0$. Thus $R=\sum a_i\iota(R_i')$. Since $R$ is extremal$/U$, for each $i$, either $\iota(R_i')=R$ or $\iota(R_i')=0$. Since $R\not=0$, there exists $j$ such that $\iota(R_j')\not=0$. We may take $R_Y=R_j'$.
\end{proof}

\begin{rem}
    We remark that our proofs of Theorems \ref{thm: cone theorem induction} and \ref{thm: property * induction} generally follows from the same ideas of \cite[Theorems 3.9, 3.10]{ACSS21} but the proofs are much lengthier. This is mainly because we work in the relative setting, and include all details of the proofs. For example, we provide detailed statements when proving the exposedness of extremal rays (Propositions \ref{prop: cone finiteness rays} and \ref{prop: * to cone final part}), and provide a detailed statement on why a certain minimal model program can be run (Claim \ref{claim: induction run mmp with scaling}). It is also worth to mention that we need to deal with the $\Qq$-factorial case first, and then deal with the non-$\Qq$-factorial case due to Claim \ref{claim: induction run mmp with scaling}(4).
\end{rem}

\subsection{Cone theorem to ACSS models}

In this subsection, we prove Theorem \ref{thm: property * induction} in dimension $d$ provided that Theorem \ref{thm: cone theorem induction} holds in dimension $\leq d-1$ and some $\Qq$-factorial properties are satisfied.

\begin{lem}\label{lem: induction cone 1}
Let $d$ be a positive integer. Assume that Theorem \ref{thm: cone theorem induction} holds in dimension $\leq d-1$.

Let $(X,\Ff,B,\Mm)/U$ be an lc gfq of dimension $d$ satisfying Property $(*)$ associated with $f: X\rightarrow Z$. Suppose that for any $(K_{\Ff}+B+\Mm_X)$-negative extremal ray$/U$ $R$, there exists a prime divisor $E$ on $X$, such that $R$ is contained in the image of $\overline{NE}(E/U)\rightarrow\overline{NE}(X/U)$ and $\mult_EB=\epsilon_{\Ff}(E)$. Let $\{R_j\}_{j\in\Lambda}$ be the set of $(K_{\Ff}+B+\Mm_X)$-negative extremal rays$/U$. Then:
\begin{enumerate}
    \item $$\overline{NE}(X/U)=\overline{NE}(X/U)_{K_{\Ff}+B+\Mm_X\geq 0}+\sum_{j\in\Lambda} R_j.$$
  \item Each $R_j$ is spanned by a rational curve $C_j$, such that $C_j$ is tangent to $\Ff$ and $$0\leq -(K_{\Ff}+B+\Mm_X)\cdot C_j\leq 2(d-1).$$
  \item For any curve $C_j'$ such that $[C_j']\in R_i$, $C_j'$ is contracted by $f$.
  \item Assume that $f$ is equi-dimensional, and either $X$ is $\Qq$-factorial klt or $\Mm$ is NQC$/U$. Let $G$ be any divisor associated with $(X,\Ff,B,\Mm)/Z$. Then:
\begin{enumerate}
    \item $\Lambda$ is a countable set.
    \item For any ample$/U$ $\Rr$-divisor $A$ on $X$, there exists a finite set $\Lambda_A\subset\Lambda$, such that
    $$\overline{NE}(X/U)=\overline{NE}(X/U)_{K_{\Ff}+B+A+\Mm_X\geq 0}+\sum_{j\in\Lambda_A}R_j.$$
    \item For any $j\in\Lambda$, there exists a contraction $\phi_j: X\rightarrow X_j'$ of $R_j$, such that 
    \begin{enumerate}
    \item $\phi_j$ is a contraction$/U$ as well as a contraction$/Z$, and
    \item if $\phi_j$ is small, then there exists a small contraction $\phi_j^+: X_j^+\rightarrow X_j'$ such that the induced birational map $\psi_j: X\dashrightarrow X_j^+$ is both a $(K_{\Ff}+B+\Mm_X)$-flip$/U$ and a $(K_{\Ff}+B+\Mm_X)$-flip$/Z$.
    \end{enumerate}
    \item For any $j$,
    $$(K_{\Ff}+B+\Mm_X)\cdot R_j=(K_X+B+G+\Mm_X)\cdot R_j.$$
    In particular,
    \begin{enumerate}
        \item each $R_j$ is a $(K_X+B+G+\Mm_X)$-negative extremal ray, and
        \item $\phi_j$ is a $(K_X+B+G+\Mm_X)$-negative extremal contraction, and if $\phi_j$ is small, then $\psi_j$ is a $(K_X+B+G+\Mm_X)$-flip.
    \end{enumerate}
\end{enumerate}
\end{enumerate}
\end{lem}
\begin{proof}
(1) is obvious.

Pick a $(K_{\Ff}+B+\Mm_X)$-negative extremal ray $R$. By our assumption, there exists a prime divisor $E$ on $X$, such that $R$ is contained in the image of $\overline{NE}(E/U)\rightarrow\overline{NE}(X/U)$ and $\mult_EB=\epsilon_{\Ff}(E)$. We let $S$ be the normalization of $E$, then there exists a natural surjection
$$\overline{NE}(S/U)\rightarrow\overline{NE}(E/U).$$ 
Thus $R$ is contained in the image of 
$$\iota: \overline{NE}(S/U)\rightarrow\overline{NE}(E/U)\rightarrow\overline{NE}(X/U).$$ 
By Lemma \ref{lem: extremal ray under morphism}, there exists an extremal ray $R_S$ in $\overline{NE}(S/U)$ such that $R=\iota(R_S)$. 
Let $\Ff_S$ be the restricted foliation of $\Ff$ on $S$ which is algebraically integrable by Proposition \ref{prop: a.i preserved adjunction}, $\Mm^S:=\Mm|_S$, and
$$K_{\Ff_S}+B_S+\Mm^S_S:=(K_{\Ff}+B+\Mm_X)|_S.$$
Then $R_S$ is a $(K_{\Ff}+B+\Mm_X)|_S$-negative extremal ray. By Theorem \ref{thm: adjunction foliation nonnqc}, $(S,\Ff_S,B_S,\Mm^S)/U$ is an lc gfq. Since we assume Theorem \ref{thm: cone theorem induction} in dimension $\leq d-1$, $R_S$ is spanned by a rational curve $C$ such that $C$ is tangent to $\Ff_S$ and
$$0<-\left(K_{\Ff_S}+B_S+\Mm^S_S\right)\cdot C\leq 2(d-1).$$
We identify $C$ with its image in $X$ under the natural inclusion $S\rightarrow E\rightarrow X$. Then $C$ spans $R$ and
$$0<-\left(K_{\Ff_S}+B_S+\Mm^S_S\right)\cdot C=-\left(K_{\Ff}+B+\Mm_X\right)\cdot C\leq 2(d-1).$$
Moreover, by \cite[Lemma 3.3(4)]{ACSS21}, $C$ is tangent to $\Ff$ and is contracted by $f$. This implies (2). 

By \cite[Lemma 3.3(3)]{ACSS21}, $C$ is contained in a fiber of $f$, so $C$ is contracted by $f$. Let $C'$ be an irreducible curve on $X$ such that $[C']\in R$. If $f(C')$ is not a closed point, then there exists a general ample divisor $H$ on $Z$ such that $H$ intersects $f(C')$ transversally. Thus $f^*H$ intersects $C'$ transversally, so $f^*H\cdot C'>0$. Since $C$ is contracted by $f$, $f^*H\cdot C=0$. This is not possible as $C\equiv\lambda C'$ for some positive rational number $\lambda$. Therefore, $f(C')$ is a closed point, so $C'$ is contracted by $f$.

For any curve $C''$ such that $[C'']\in R$, we let $C''_i$ be the irreducible components of $C''$. Since $R$ is extremal, $[C''_i]\in R$ for each $i$, so $C''_i$ is contracted by $f$ for each $i$. Thus $C''$ is contracted by $f$, and we get (3).

We left to prove (4). We may assume that $f$ is equi-dimensional from now on. We let $G$ be any divisor associated with $(X,\Ff,B,\Mm)/Z$. Since $(X,\Ff,B,\Mm)$ is lc, all components of $B$ are horizontal$/Z$. By Proposition \ref{prop: weak cbf gfq}, 
\begin{equation}\label{equ: f=x+g}
    K_{\Ff}+B+\Mm_X\sim_{\mathbb R,Z}K_{X}+B+G+\Mm_X.
\end{equation}
By (3), for any $j\in\Lambda$, we have
$$(K_{X}+B+G+\Mm_X)\cdot R_j=(K_{\Ff}+B+\Mm_X)\cdot R_j<0,$$
so $R_j$ is a $(K_{X}+B+G+\Mm_X)$-negative extremal ray$/U$. Moreover, for any ample$/U$ $\Rr$-divisor $A$, we have
$$(K_{X}+B+G+A+\Mm_X)\cdot R_j=(K_{\Ff}+B+A+\Mm_X)\cdot R_j<0.$$

By Lemma \ref{lem: basic property (*) gpair}, $(X,B+G,\Mm)$ is lc. If $\Mm$ is NQC$/U$, then by \cite[Theorem 1.3(3)]{HL21a},
$$\Lambda_A:=\{j\in\Lambda|(K_{\Ff}+B+A+\Mm_X)\cdot R_j<0\}$$
is a finite set, hence $\Lambda$ is a countable set. This implies (4.a) and (4.b). (4.c.i) follows from \cite[Theorem 1.5]{Xie22} (see also \cite[Theorem 1.7]{CLX23}), and (4.c.ii) follows from \cite[Theorem 1.2]{LX23b}. 

If $X$ is $\Qq$-factorial klt, then by \cite[Lemma 3.4]{HL22}, for any ample $\Rr$-divisor $A$ on $X$, there exists an $\Rr$-divisor $0\leq\Delta_A\sim_{\mathbb R}B+G+\frac{1}{2}A+\Mm_X$, such that $(X,\Delta_A)$ is klt. Thus
$$\Lambda_A=\left\{j\in\Lambda\Biggm|\left(K_{\Ff}+\Delta+\frac{1}{2}A\right)\cdot R_j<0\right\}$$
is a finite set by the classical cone theorem (cf. \cite[Theorem 4-2-1]{KMM87}, \cite[Theorem 4.5.2]{Fuj17}), and $\Lambda=\cup_{n=1}^{+\infty}\Lambda_{\frac{1}{n}A}$ is a countable set. This implies (4.a) and (4.b). For any $j$, we take an ample $\Rr$-divisor $A$ on $X$, such that $R_j$ is also a $(K_{X}+B+G+A+\Mm_X)$-negative extremal ray$/U$. Then $R_j$ is a $(K_X+\Delta_A)$-negative extremal ray$/U$, so (4.c.i) follows from the classical contraction theorem (cf. \cite[Theorem 3-2-1]{KMM87}, \cite[Theorem 4.5.2]{Fuj17}) and (4.c.ii) follows from the the existence of flips \cite[Corollary 1.4.1]{BCHM10}.

(4.d) follows immediately from (4.c) and (\ref{equ: f=x+g})
\end{proof}

\begin{prop}\label{prop: cone d-1 imply * dim d part 1}
Let $d$ be a positive integer. Assume that Theorem \ref{thm: cone theorem induction} holds in dimension $\leq d-1$. Let $(X,\Ff,B,\Mm)/U$ a gfq of dimension $d$ such that $\Ff$ is algebraically integrable. Let $E_1,\dots,E_s$ be lc places of $(X,\Ff,B,\Mm)$ and $T$ a reduced $\Ff$-invariant divisor on $X$. Further assume that
\begin{itemize}
    \item either $X$ is $\Qq$-factorial, or 
    \item Theorem \ref{thm: cone theorem induction} holds for $\Qq$-factorial varieties in dimension $d$. 
\end{itemize}
Then $(X,\Ff,B,\Mm)$ has a great ACSS model $(Y,\Ff_Y,B_Y,\Mm;G_Y)$ such that
\begin{enumerate}
\item $G_Y$ contains the strict transform of $T$ on $Y$, and
\item $E_1,\dots,E_s$ are on $Y$ if $(X,\Ff,B,\Mm)$ is lc.
\end{enumerate}
\end{prop}
\begin{proof}
By Definition-Theorem \ref{defthm: weak ss reduction} and Lemma \ref{lem: existence foliated log resolution}, there exists a foliated log resolution $h: X'\rightarrow X$ of $(X,\Ff,\Supp B+\Supp T,\Mm)$ such that $E_1,\dots,E_s$ are on $X'$. Then there exists a toroidal contraction $f': (X',\Sigma_{X'},\Mm)\rightarrow (Z,\Sigma_Z)$ such that $(Z,\Sigma_Z)$ is log smooth,
$$\Exc(h)\cup\Supp(h^{-1}_*B)\cup\Supp(h^{-1}_*T)\subset\Sigma_{X'},$$
and $\Ff':=h^{-1}\Ff$ is induced by $f'$. We define
$$B':=h^{-1}_*(B\wedge\Supp B)+(\Supp\Exc(h))^{\Ff'}.$$
By Lemma \ref{lem: fls imply acss}, $(X',\Ff',B',\Mm;G')/Z$ is $\Qq$-factorial super ACSS for some divisor $G'$, such that $G'\geq h^{-1}_*T$ and any $\Ff'$-invariant $h$-exceptional divisor is contained in $G'$.

\begin{claim}\label{claim: induction run mmp with scaling}
Let $A$ be an ample  $\Rr$-divisor on $X$. Then we may run a $(K_{\Ff'}+B'+\Mm_{X'})$-MMP$/X$ 
  \begin{center}
 $\xymatrix{(X_0,\Ff_0,B_0,\Mm;G_0)\ar@{-->}[r]^{\psi_0} & (X_1,\Ff_1,B_1,\Mm;G_1)\ar@{-->}[r]^{\ \ \ \ \ \ \ \ \ \ \psi_1} & \dots\ar@{-->}[r] & (X_n,\Ff_n,B_n,\Mm;G_n)\ar@{-->}[r]^{\ \ \ \ \ \ \ \ \ \ \psi_n} & \dots 
}$
  \end{center}
where $(X_0,\Ff_0,B_0,\Mm;G_0):=(X',\Ff',B',\Mm;G')$, so that the following conditions are satisfied for each $i$. Let $A_i$ be the strict transform of $A$ on $X_i$.
\begin{enumerate}
    \item There exists an contraction $f_i: X_i\rightarrow Z$ such that $f_{i+1}=f_i\circ\psi_i$.
    \item There exists an contraction $h_i: X_i\rightarrow X$ such that $h_{i+1}=h_i\circ\psi_i$.
    \item $(X_i,\Ff_i,B_i,\Mm;G_i)/Z$ is $\Qq$-factorial 
 super ACSS.
    \item If $X$ is $\Qq$-factorial, then for any $(K_{\Ff_i}+B_i+\Mm_{X_i})$-negative extremal ray$/X$ $R$, there exists a prime divisor $F$ on $X_i$, such that $R$ is contained in the image of $\overline{NE}(F/U)\rightarrow\overline{NE}(X/U)$ and $\mult_FB_i=\epsilon_{\Ff_i}(F)$.
    \item For any  extremal ray$/X$ $R$ on $X_i$ such that $R$ is either a $(K_{\Ff_i}+B_i+\Mm_{X_i})$-negative extremal ray or a $(K_{X_i}+B_i+G_i+\Mm_{X_i})$-negative extremal ray, 
    \begin{enumerate}
      \item  $R$ is an extremal ray$/Z$,
      \item  $$(K_{\Ff_i}+B_i+\Mm_{X_i})\cdot R=(K_{X_i}+B_i+G_i+\Mm_{X_i})\cdot R,$$ and
      \item  $R$ is a $(K_{\Ff_i}+B_i+\Mm_{X_i})$-negative extremal ray if and only if $R$ is a $(K_{X_i}+B_i+G_i+\Mm_{X_i})$-negative extremal ray.
    \end{enumerate}
    \item $\psi_i$ is a step of a $(K_{X_i}+B_i+G_i+\Mm_{X_i})$-MMP$/X$ with scaling of $A_i$ as well as a $(K_{\Ff_i}+B_i+\Mm_{X_i})$-MMP$/X$ with scaling of $A_i$.
    \item $\psi_i$ is a step of a $(K_{\Ff_i}+B_i+\Mm_{X_i})$-MMP$/Z$ as well as a step of a $(K_{X_i}+B_i+G_i+\Mm_{X_i})$-MMP$/Z$.
\end{enumerate}
Moreover, there exists a positive integer $m$ satisfying the following.
\begin{enumerate}
    \item[(8)] The induced birational map $X_0\dashrightarrow X_m$ contracts any $h$-exceptional prime divisor $F$ such that $a(F,\Ff,B,\Mm)>-\epsilon_{\Ff}(F)$. 
    \item[(9)] If $(X,\Ff,B,\Mm)$ is lc, then any divisor $F$ contracted by $X_0\dashrightarrow X_m$ satisfies that $a(F,\Ff,B,\Mm)>-\epsilon_{\Ff}(F)$. 
\end{enumerate}
    \end{claim}
\begin{proof}
 
\noindent\textbf{Step 1}. In this step we prove (1-4) for $i=0$. (1) We have $f_0:=f$. (2) We have $h_0:=h$. (3) It follows from our construction. (4) The image of $R$ on $X$ is a closed point, so $R$ is contained in an $h$-exceptional divisor $F$. By our construction, $\mult_FB_0=\epsilon_{\Ff_0}(F)$. 

\medskip

\noindent\textbf{Step 2}. In this step we prove that (1-4) for $i=n$ implies (5) for $i=n$.

First we prove (5.a). Assume that $R$ is a $(K_{\Ff_n}+B_n+\Mm_{X_n})$-negative extremal ray$/X$. If $X$ is $\Qq$-factorial, then by (4) and Lemma \ref{lem: induction cone 1}(2), $R$ is a $(K_{\Ff_n}+B_n+\Mm_{X_n})$-negative extremal ray$/Z$. If Theorem \ref{thm: cone theorem induction} holds for $\Qq$-factorial varieties in dimension $d$, then by (3) and Theorem \ref{thm: cone theorem induction},  $R$ is a $(K_{\Ff_n}+B_n+\Mm_{X_n})$-negative extremal ray$/Z$.

Now assume that $R$ is a $(K_{X_n}+B_n+G_n+\Mm_{X_n})$-negative extremal ray$/X$. Since $G_n$ is super, $G_n\geq\sum_{j=1}^{2d+1}f_n^*H_j$ for some ample Cartier divisors $H_j$ on $Z$. Let $L_n:=G_n-\sum_{j=1}^{2d+1}f_n^*H_j$. By (3), $(X_n,B_n+G_n,\Mm)$ is $\Qq$-factorial lc and $X$ is klt, so $(X_n,B_n+L_n,\Mm)$ is $\Qq$-factorial lc. By the length of extremal rays for lc g-pairs over $\Qq$-factorial klt varieties (cf. \cite[Proposition 3.17]{HL22}), $R$ is spanned by a rational curve $C$ such that
$$0>(K_{X_n}+B_n+G_n+\Mm_{X_n})\cdot C=(K_{X_n}+B_n+L_n+\Mm_{X_n})\cdot C+\left(\sum_{j=1}^{2d+1}f_n^*H_j\right)\cdot C\geq -2d.$$
Therefore, $f_n^*H_j\cdot C=0$ for each $j$, so $R$ is an extremal ray$/Z$. This implies (5.a). 

(5.b) follows from (5.a) and Proposition \ref{prop: weak cbf gfq}, and (5.c) follows from (5.b). Thus (5) holds.

\medskip

\noindent\textbf{Step 4}. In this step we prove that (1-5) for $i=n$ and (1-7) for $i\leq n-1$ imply (6) and (7) for $i=n$, and also imply (1)(2) for $i=n+1$.

By induction hypothesis, the induced birational map $X_0\dashrightarrow X_n$ is a sequence of steps of a $(K_{X_0}+B_0+G_0+\Mm_{X_0})$-MMP$/X$ with scaling of $A$. By Lemma \ref{lem: scaling number go to 0}, either this MMP already terminates at $X_n$ and we are done, or we may run the next step of this $(K_{X_0}+B_0+G_0+\Mm_{X_0})$-MMP$/X$ with scaling of $A$, which is a step of a $(K_{X_n}+B_n+G_n+\Mm_{X_n})$-MMP$/X$ with scaling of $A_n$. (6)  and (7) for $i=n$ now follow from (5) for $i=n$. (1) for $i=n+1$ follows from (7) for $i=n$, and (2) for $i=n+1$ follows from (6) for $i=n$.

\medskip

\noindent\textbf{Step 5}. In this step we prove that (1-7) for $i\leq n-1$ and (1)(2) for $i=n$ imply (3) for $i=n$. 

By (3)(7) for $i=n-1$, $X_n$ is $\Qq$-factorial. By (1) for $i=n$ and (3) for $i=n-1$, $G_n$ is super$/Z$. So we only need to show that $(X_n,\Ff_n,B_n,\Mm;G_n)/Z$ is ACSS. We check conditions (1-4) of Definition \ref{defn: ACSS f-triple} for $(X_n,\Ff_n,B_n,\Mm;G_n)/Z$. 

Definition \ref{defn: ACSS f-triple}(1) for $(X_n,\Ff_n,B_n,\Mm;G_n)/Z$: By (6) for $i=n-1$ and Proposition \ref{prop: MMP preserves *}, $(X_n,B_n+G_n,\Mm)/Z$ satisfies Property $(*)$. Since $\Ff_{n-1}$ is induced by $f_{n-1}$, $\Ff_n$ is induced by $f_n$.  Since $G_{n-1}\geq 0$ is $\Ff_{n-1}$-invariant, $G_n\geq 0$ if $\Ff_n$-invariant. Thus  $(X_n,\Ff_n,B_n,\Mm;G_n)/Z$ satisfies Property $(*)$. By (3)(7) for $i=n-1$, $(X_n,\Ff_n,B_n,\Mm)$ is lc, so Definition \ref{defn: ACSS f-triple}(1) holds for $(X_n,\Ff_n,B_n,\Mm;G_n)/Z.$

Definition \ref{defn: ACSS f-triple}(2) for $(X_n,\Ff_n,B_n,\Mm;G_n)/Z$: It immediately follows from (3)(6) for $i=n-1$ and  Proposition \ref{prop: MMP preserves *}.

Definition \ref{defn: ACSS f-triple}(3) for $(X_n,\Ff_n,B_n,\Mm;G_n)/Z$: By (3) for $i=n-1$, there exist an $\Rr$-divisor $D$ and a $\bb$-divisor $\Nn$ on $X_{n-1}$, such that
\begin{itemize}
    \item $\Supp\{B_{n-1}\}\subset\Supp D$,
    \item $\Nn-\alpha\Mm$ is nef$/X_{n-1}$ for some $\alpha>1$, and
    \item For any divisor $\Sigma$ on $Z$ such that $\Sigma\geq f_{n-1}(G_{n-1})$ and $(Z,\Sigma)$ is log smooth,
$$(X_{n-1},B_{n-1}+G_{n-1}+D+f_{n-1}^*(\Sigma-f_{n-1}(G_{n-1})),\Nn)$$ is qdlt, 
\end{itemize}
Let $\Pp:=\Nn-\Mm$. By (7) for $i=n-1$, $\psi_{n-1}$ is also a step of a $$(K_{\Ff_{n-1}}+B_{n-1}+f_{n-1}^*(\Sigma-f_{n-1}(G_{n-1}))+\Mm_{X_{n-1}})\text{-MMP}/Z,$$ hence a step of a 
$$(K_{\Ff_{n-1}}+B_{n-1}+\delta D+f_{n-1}^*(\Sigma-f_{n-1}(G_{n-1}))+\Mm_{X_{n-1}}+\delta\Pp_{X_{n-1}})\text{-MMP}/Z$$ 
for some $0<\delta\ll 1$. By (1) for $i=n$, $f_{n-1}(G_{n-1})=f_n(G_n)$, so 
$$(X_{n-1},B_{n-1}+G_{n-1}+\delta D+f_{n-1}^*(\Sigma-f_n(G_n)),\Mm+\delta\Pp)$$ is qdlt. By Lemma \ref{lem: mmp preserves qdlt}, $$(X_{n},B_{n}+\delta(\psi_{n-1})_*D+G_n+f_{n}^*(\Sigma-f_{n}(G_{n})),\Mm+\delta\Pp)$$ is qdlt. Since $(\psi_{n-1})_*D\subset\Supp\{B_n\}$ and $(\Mm+\delta\Pp)-\Mm$ is nef$/X_n$, we verify Definition \ref{defn: ACSS f-triple}(3).

Definition \ref{defn: ACSS f-triple}(4) for $(X_n,\Ff_n,B_n,\Mm;G_n)/Z$: For any lc place $S$ of $(X_n,\Ff_n,B_n,\Mm)$, we have
      $$-\epsilon_{\Ff}(S)=a(S,\Ff_n,B_n,\Mm)\geq a(S,\Ff_{n-1},B_{n-1},\Mm)\geq -\epsilon_{\Ff}(S).$$
      Therefore, $S$ is an lc place of $(X_{n-1},\Ff_{n-1},B_{n-1},\Mm)$, and $\psi_{n-1}$ is an isomorphism near the generic point of $\Center_{X_{n-1}}S$. Since Definition \ref{defn: ACSS f-triple}(4) is a property near the generic point of lc places,       
Definition \ref{defn: ACSS f-triple}(4) holds for $(X_n,\Ff_n,B_n,\Mm;G_n)/Z$.

Therefore, $(X_n,\Ff_n,B_n,\Mm;G_n)/Z$ is $\Qq$-factorial super ACSS.

\medskip

\noindent\textbf{Step 6}. In this step we prove (4) for $i=n$ assuming that (1-7) hold for $i=n-1$, hence conclude the proof of (1-7). Since $X$ is $\Qq$-factorial, $\Exc(h_n)$ is of pure dimension, so there exists a prime $h_n$-exceptional divisor $F$ such that $R$ is contained in $F$. Let $F'$ be the strict transform of $F$ on $X'$, then $F'$ is a prime $h$-exceptional divisor, so 
$$\mult_FB_n=\mult_{F'}B_0=\epsilon_{\Ff'}(E)=\epsilon_{\Ff_n}(B_n).$$
This implies (4). 

By induction, (1-7) hold.

\medskip

\noindent\textbf{Step 7}. In this step we prove (8) and (9) and conclude the proof of the claim.

If this MMP terminates, then we let $m$ be the index such that $(X_m,\Ff_m,B_m,\Mm;G_m)$ is the last output of this MMP. In particular, $K_{\Ff_m}+B_m+\Mm_{X_m}$ is nef$/X$, hence it is movable$/X$. If this MMP does not terminate, then we let $m$ be the index such that $\psi_i$ is a flip for any $i\geq m$. We let $\mu_i: X_m\dashrightarrow X_i$ be the induced birational map and let 
$$\lambda_i:=\inf\{t\geq 0\mid K_{\Ff_i}+B_i+\Mm_{X_i}+tA_i\text{ is nef}/X\}$$
be the scaling numbers for each $i$. By (5),
$$\lambda_i=\inf\{t\geq 0\mid K_{X_i}+B_i+G_i+\Mm_{X_i}+tA_i\text{ is nef}/X\}$$
for each $i$. By Lemma \ref{lem: scaling number go to 0}, $\lim_{i\rightarrow+\infty}\lambda_i=0$. Therefore,
$$K_{\Ff_m}+B_m+\Mm_{X_m}=\lim_{i\rightarrow+\infty}(\mu_i)^{-1}_*(K_{\Ff_i}+B_i+\Mm_{X_i}+\lambda_iA_i)$$
is a movable$/X$.

Since $K_{\Ff_m}+B_m+\Mm_{X_m}$ is movable$/X$, for any prime divisor $S$ on $X_m$ any very general curve $C$ on $S$ over $X$, $(K_{\Ff_m}+B_m+\Mm_{X_m})\cdot C\geq 0$. Let $F_1,\dots,F_l$ be the $h_m$-exceptional prime divisors and let $a_k:=a(F_k,\Ff,B,\Mm)+\epsilon_{\Ff_m}(F_k)$ for each $k$, then
\begin{align*}
   &\ \ \ \ K_{\Ff_m}+B_m+\Mm_{X_m}\\
   &=\phi_m^*(K_{\Ff}+B+\Mm_X)+\sum_{a_k>0}a_kF_k-\left(\sum_{\mult_DB>1}(\mult_DB-1)(\phi_m^{-1})_*D+\sum_{a_k<0}(-a_k)F_k\right)\\
    &\sim_{\mathbb R,X}\sum_{a_k>0}a_kF_k-\left(\sum_{\mult_DB>1}(\mult_DB-1)(\phi_m^{-1})_*D+\sum_{a_k<0}(-a_k)F_k\right).\\
\end{align*}
Since each $F_k$ is exceptional$/X$, by \cite[Lemma 3.3]{Bir12}, $a_k\leq 0$ for any $k$. This implies (8). Finally, if $(X,\Ff,B,\Mm)$ is lc, then
$$K_{\Ff'}+B'+\Mm_{X'}\sim_{\mathbb R,X}\sum_{F\mid F\subset\Exc(\phi)}(\epsilon_{\Ff}(F)+a(F,\Ff,X,\Mm))F\geq 0,$$
so any divisor contracted by any $(K_{\Ff'}+B'+\Mm_{X'})$-MMP$/X$ is contained in 
$$\Supp\sum_{F\mid F\subset\Exc(\phi)}(\epsilon_{\Ff}(F)+a(F,\Ff,X,\Mm))F=\Supp\sum_{F\mid F\subset\Exc(\phi),a(F,\Ff,B,\Mm)>-\epsilon_{\Ff}(F)}F.$$
We get (9). The proof of the claim is concluded.
\end{proof}
\noindent\textit{Proof of Proposition \ref{prop: cone d-1 imply * dim d part 1} continued}. 
  We let
        \begin{center}$\xymatrix{
(X_0,\Ff_0,B_0,\Mm;G_0)\ar@{-->}[r]^{\psi_0} & (X_1,\Ff_1,B_1,\Mm;G_1)\ar@{-->}[r]^{\ \ \ \ \ \ \ \ \ \ \psi_1} & \dots\ar@{-->}[r] & (X_n,\Ff_n,B_n,\Mm;G_n)\ar@{-->}[r]^{\ \ \ \ \ \ \ \ \ \ \psi_n} & \dots 
}$
\end{center}
and $m$ be as in Claim \ref{claim: induction run mmp with scaling}. Then Claim \ref{claim: induction run mmp with scaling}(3)(8) guarantee that $(X_m,\Ff_m,B_m,\Mm;G_m)/Z$ is a super ACSS model of $(X,\Ff,B,\Mm)$, and (9) guarantees that if $(X,\Ff,B,\Mm)$ is lc then $E_1,\dots,E_s$ are on $X_m$. Thus $(X_m,\Ff_m,B_m,\Mm;G_m)/Z$ is a super ACSS model of $(X,\Ff,B,\Mm)$ such that $E_1,\dots,E_s$ are on $X_m$ if $(X,\Ff,B,\Mm)$ is lc. Since any $\Ff'$-invariant exceptional$/X$ prime divisor is contained in $G'$, any $\Ff_m$-invariant exceptional$/X$ prime divisor is contained in $G_m$. Therefore, $(X_m,\Ff_m,B_m,\Mm;G_m)/Z$ is a great ACSS model of $(X,\Ff,B,\Mm)$. Finally, since the strict transform of $T$ on $X'$ is contained in $G'$, the strict transform of $T$ on $X_m$ is contained in $G_m$. The proposition follows by taking
$$(Y,\Ff_Y,B_Y,\Mm;G_Y):=(X_m,\Ff_m,B_m,\Mm;G_m).$$
\end{proof}


\begin{prop}\label{prop: cone d-1 imply * dim d part 2}
Let $d$ be a positive integer. Assume that Theorem \ref{thm: cone theorem induction} holds in dimension $\leq d-1$. Then:
\begin{enumerate}
    \item Theorem \ref{thm: property * induction} holds for $\Qq$-factorial varieties in dimension $d$.
    \item If Theorem \ref{thm: cone theorem induction} holds for $\Qq$-factorial varieties in dimension $d$, then Theorem \ref{thm: property * induction} holds in dimension $d$.
\end{enumerate}
\end{prop}
\begin{proof}
 Notations and conditions as in Theorem \ref{thm: property * induction}. Further assume that either $X$ is $\Qq$-factorial, or Theorem \ref{thm: cone theorem induction} holds for $\Qq$-factorial varieties in dimension $d$.
 
 By Proposition \ref{prop: cone d-1 imply * dim d part 2}, $(X,\Ff,B,\Mm)$ has a great ACSS model $(Y',\Ff_{Y'},B_{Y'},\Mm;G_{Y'})/Z$. We let $g: Y'\rightarrow X$ be the induced birational morphism and let 
  $$F:=\Supp(K_{\Ff_{Y'}}+B_{Y'}+\Mm_{Y'}-g^*(K_{\Ff}+B+\Mm_X)).$$
  
  Consider $F$ as a reduced subscheme of $Y'$. Then for any irreducible closed subvariety $V\subset X$ such that $V\subset f(F)$, $V$ is a non-lc center of $(X,\Ff,B,\Mm)$. Therefore, the generic point of $\Center_{X}E_i$ is not contained in $f(F)$ for each $i$, so $E_1,\dots,E_s$ are also lc places of  $(Y',\Ff_{Y'},B_{Y'},\Mm)$. Since $(Y',\Ff_{Y'},B_{Y'},\Mm)$ is lc, by Proposition \ref{prop: cone d-1 imply * dim d part 2} again, $(Y',\Ff_{Y'},B_{Y'},\Mm)$ has a great ACSS model $(Y,\Ff_{Y},B_{Y},\Mm;G_Y)$ such that $E_1,\dots,E_s$ are on $Y$, and $G_Y$ contains the strict transform of $G_{Y'}$ on $Y$. Therefore, $G_Y$ contains all $\Ff_Y$-exceptional prime divisors. Since 
 $$g^*(K_{\Ff}+B+\Mm_X)\geq  K_{\Ff_{Y'}}+B_{Y'}+\Mm_{Y'},$$
 the induced birational morphism $Y\rightarrow X$ is a great ACSS modification $(X,\Ff,B,\Mm)$.
\end{proof}


\subsection{ACSS models to cone theorem}

In this subsection, we prove Theorem \ref{thm: cone theorem induction} in dimension $d$ provided that Theorem \ref{thm: property * induction} holds in dimension $\leq d$ and some $\Qq$-factorial properties are satisfied.

The following lemma is well-known to experts. For the reader's convenience, we conclude a proof here.

\begin{lem}\label{lem: supporting function are +A}
    Let $X\rightarrow U$ be a projective morphism from a normal quasi-projective variety to a variety. Let $D$ be an $\Rr$-Cartier $\Rr$-divisor on $X$ and $R$ a $D$-negative exposed ray in $\overline{NE}(X/U)$. Then there exists an ample$/U$ $\Rr$-divisor $A$ on $X$ such that $H:=D+A$ is the supporting function of $R$.
\end{lem}
\begin{proof}
   Let $H_R$ be a supporting function of $R$, whose existence follows from the assumption that $R$ is exposed. Then $H_R\cdot R=0$ and $H_R\cdot R'>0$ for any $R'\not=R$ in $\overline{NE}(X/U)$. Let 
   $$C:=\left\{D\in N^1(X/U)\mid D\cdot z\geq 0 \text{ for any }z\in\overline{NE}(X/U)_{D\geq0}\right\}.$$
   Then $C$ is the dual cone of $\overline{NE}(X/U)_{D\geq 0}$ and is generated by nef$/U$ divisors and $D$. Since $H_R$ is positive on $\overline{NE}(X/U)_{D\geq 0}\backslash\{0\}$, $H_R$ is contained in the interior of $C$. Thus there exists an ample$/U$ $\Rr$-divisor $\tilde A$ such that $H_R-\tilde A=L+pD$ in $N^1(X/U)$, where $L$ is a nef$/U$ $\Rr$-divisor and $p$ is a non-negative real number. Let $A':=\tilde A+L$, then $A'$ is ample$/U$. We may let $H:=\frac{1}{p}H_R=\frac{1}{p}\tilde A'+D$ and $A:=\frac{1}{p} A'$.
\end{proof}

\begin{prop}\label{prop: * to cone}
    Let $d$ be a positive integer. Assume that Theorem \ref{thm: cone theorem induction} holds in dimension $\leq d-1$ and Theorem \ref{thm: property * induction} holds for $\Qq$-factorial varieties in dimension $d$. 

    Let $(X,\Ff,B,\Mm)/U$ be a gfq of dimension $d$ such that $\Ff$ is algebraically integrable. Let $R$ be a $(K_{\Ff}+B+\Mm_X)$-negative exposed ray$/U$ that is not contained in $\overline{NE}(X/U)_{\Nlc(X,\Ff,B,\Mm)}$. Assume that
    \begin{itemize}
        \item either $X$ is $\Qq$-factorial, or
        \item Theorem \ref{thm: property * induction} holds in dimension $d$. 
    \end{itemize}
Then $R$ is spanned by a rational curve $C_j$, such that $C_j$ is tangent to $\Ff$ and $$0<-(K_{\Ff}+B+\Mm_X)\cdot C_j\leq 2d.$$
\end{prop}
\begin{proof}
\medskip

By Lemma \ref{lem: supporting function are +A}, there exists an ample$/U$ $\Rr$-divisor $A$ on $X$ such that
$$H_R:=K_{\Ff}+B+A+\Mm_X$$
is the supporting function$/U$ of $R$. In particular, $H_R$ is nef, $H_R\cdot R=0$, and $H_R\cdot R'>0$ for any $R'\in\overline{NE}(X/U)\backslash R$. In particular, $H_R\not\equiv_U 0$. 

\medskip

\noindent\textbf{Step 1}. In this step we deal with the case when $H_R$ is not big$/U$. 

Let $\pi: X\rightarrow U$ be the induced projective morphism and $X\rightarrow U'\rightarrow U$ the Stein factorization of $\pi$. Since $\Mm$ is nef$/U$, $\Mm$ is nef$/U'$. Possibly replacing $U$ by $U'$, we may assume that $\pi$ is a contraction.

Let $F$ be a general fiber of $\pi$. Then $H_F:=H_R|_F$ is nef, not big, and is not numerically trivial. Let $q:=\dim F$ and $A_F:=A|_F$, then there exists an integer $1\leq k\leq q-1$ such that
$$H_F^k\cdot A_F^{q-k}>H_F^{k+1}\cdot A_F^{q-k-1}=0.$$
%(Note that $k$ is defined as the numerical dimension of $H_F$ in some references, but since different definitions of numerical dimensions do not coincide, we do not use this notation.)
Let $D_i:=H_R$ for any $1\leq i\leq k+1$, and let $D_i:=A$ for any $k+2\leq i\leq q$. Then
$$(D_1|_F)\cdot (D_2|_F)\cdots\dots\cdot (D_q|_F)=H_F^{k+1}\cdot A_F^{q-k-1}=0$$
and
$$-(K_{\Ff}+B)|_F\cdot (D_2|_F)\cdots\dots\cdot (D_q|_F)=(A_F-H_F)\cdot H_F^{k}\cdot A_F^{q-k-1}=H_F^{k}\cdot A_F^{q-k}>0.$$
Let $M:=H_R+A=K_{\Ff}+B+2A+\Mm_X$. Then $M$ is ample$/U$. By Theorem \ref{thm: relative bb}, for any general closed point $x\in X$, there exists a rational curve $C_x$ such that $x\in C_x$, $\pi(C_x)$ is a closed point, $0=D_1\cdot C_x=H_R\cdot C_x,$ and
\begin{align*}
    0<&-(K_{\Ff}+B+\Mm_X)\cdot C_x=M\cdot C_x\\
    \leq& 2d\cdot\frac{M|_F\cdot  (D_2|_F)\cdot\dots\cdot (D_q|_F)}{-K_{\Ff}|_F\cdot  (D_2|_F)\cdot\dots\cdot (D_q|_F)}=2d\cdot\frac{-(K_{\Ff}+B+\Mm_X)|_F\cdot H_F^k\cdot A_F^{q-k-1}}{-K_{\Ff}|_F\cdot H_F^k\cdot A_F^{q-k-1}}.
\end{align*}
Let $\Mm^F:=\Mm|_F$ and $B_F:=B|_F$. Since $F$ is a general fiber of $\pi$, $B_F\geq 0$ and $\Mm^F$ is nef. Thus $\Mm^F_F$ is pseudo-effective and $(B+\Mm_X)|_F\cdot H_F^k\cdot A_F^{q-k-1}\geq 0.$ Therefore 
$$0<-(K_{\Ff}+B+\Mm_X)\cdot C_x\leq 2d.$$


\medskip

\noindent\textbf{Step 2}. In this step we deal with the case when $H_R$ is not big$/U$. Let $F$ be the Stein factorization of a general fiber of the $\pi$ and let $q:=\dim F$. Then $H_F:=H_R|_F$ is nef, not big, and is not numerically trivial. Let $A_F:=A|_F$, then there exists an integer $1\leq k\leq q-1$ such that
$$H_F^k\cdot A_F^{q-k}>0$$
and
$$H_F^{k+1}\cdot A_F^{q-k-1}=0.$$
(Note that $k$ is defined as the numerical dimension of $H_F$ in some references, but since different definitions of numerical dimensions do not coincide, we do not use this notation.)
Let $D_i:=H_R$ for any $1\leq i\leq k+1$, and let $D_i:=A$ for any $k+2\leq i\leq q$. Then
$$(D_1|_F)\cdot (D_2|_F)\cdots\dots\cdot (D_q|_F)=H_F^{k+1}\cdot A_F^{q-k-1}=0$$
and
$$-(K_{\Ff}+B)|_F\cdot (D_2|_F)\cdots\dots\cdot (D_q|_F)=(A_F-H_F)\cdot H_F^{k}\cdot A_F^{q-k-1}=H_F^{k}\cdot A_F^{q-k}>0.$$
We let $M:=H_R+A=K_{\Ff}+B+2A+\Mm_X$. Since $H_R$ is nef$/U$ and $A$ is ample$/U$, $M$ is ample$/U$. By Theorem \ref{thm: relative bb}, for any general closed point $x\in X$, there exists a rational curve $C_x$ such that $x\in C_x$, $\pi(C_x)$ is a closed point, $$0=D_1\cdot C_x=H_R\cdot C_x,$$
and
\begin{align*}
    0<&-(K_{\Ff}+B+\Mm_X)\cdot C_x=M\cdot C_x\\
    \leq& 2d\frac{M|_F\cdot  (D_2|_F)\cdot\dots\cdot (D_q|_F)}{-K_{\Ff}|_F\cdot  (D_2|_F)\cdot\dots\cdot (D_q|_F)}=2d\frac{-(K_{\Ff}+B+\Mm_X)|_F\cdot H_F^k\cdot A_F^{q-k-1}}{-K_{\Ff}|_F\cdot H_F^k\cdot A_F^{q-k-1}}.
\end{align*}
Let $\Mm^F:=\Mm|_F$ and $B_F:=B|_F$. Since $F$ is a general fiber of $\pi$, $B_F\geq 0$ and $\Mm^F$ is nef. Thus $\Mm^F_F$ is pseudo-effective, so 
$$(B+\Mm_X)|_F\cdot H_F^k\cdot A_F^{q-k-1}\geq 0.$$
Therefore, 
$$0<-(K_{\Ff}+B+\Mm_X)\cdot C_x\leq 2d.$$

\medskip

\noindent\textbf{Step 3}. From now on we may assume that $H_R$ is big$/U$. In this step we construct a set $\Ii$ of tuples $(W,\lambda)$ and show that it contains a minimal element. Since $H_R$ is big$/U$,
$$H_R\sim_{\mathbb R,U}A'+P$$
for some ample$/U$ $\Rr$-divisor $A'$ and $\Rr$-divisor $P\geq 0$. In particular, $P$ is $\Rr$-Cartier and $P\cdot R<0$. Let $S$ be the normalization of $\Supp P$, then $R$ is contained in the image of $\overline{NE}(S/U)\rightarrow\overline{NE}(X/U)$ induced by the natural inclusion 
$$S\rightarrow \Supp P\rightarrow X.$$ 
We let $\Ii$ be he set of all $(W,\lambda)$, such that
\begin{enumerate}
    \item $\lambda$ is a non-negative real number,
    \item $W$ is an lc center of $(X,\Ff,B+\lambda P,\Mm)$ with normalization $W^\nu$ and
    \item $R$ is contained in the image of $\overline{NE}(W^\nu/U)\rightarrow\overline{NE}(X/U)$  induced by the natural inclusion 
$$W^\nu\rightarrow W\rightarrow X.$$
\end{enumerate}
By construction, there exists a component $L$ of $S$ such that $(L,1)\in\Ii$. Thus $\Ii\not=\emptyset$. 

In the rest of this step, we show that there exists $(W_0,\lambda_0)\in\Ii$ that is minimal in the following way: for any $(W,\lambda)\in\Ii$, one of the following cases hold.
\begin{itemize}
\item $\lambda_0<\lambda$.
\item $\lambda_0=\lambda$ and $W_0\subsetneq W$.
\item  $(W,\lambda)=(W_0,\lambda_0)$.
\end{itemize}
By Lemma \ref{lem: existence foliated log resolution}, there exists a foliated log resolution $h: X'\rightarrow X$ of $(X,\Ff,\Supp B\cup\Supp P,\Mm)$. Then there exists a toroidal morphism $f': (X',\Sigma_X)\rightarrow (Z,\Sigma_Z)$ such that $h^{-1}_*(\Supp B\cup\Supp P)\cup\Supp\Exc(h)$ is contained in $\Sigma_X$. By Lemma \ref{lem: foliated log smooth imply lc}, for any $(W,\lambda)\in\Ii$, either $W$ is the image of a stratum of $(X',\Sigma)$ on $X$, or $\lambda=0$. Therefore, the set
\begin{align*}
    \Ii':=\{\lambda\mid& \text{ there exists an lc center of }(X,\Ff,B+\lambda P,\Mm)\\
    & \text{ that is not an lc center of }(X,\Ff,B+(\lambda-\delta)P,\Mm) \text{ for any }0<\delta\ll 1\}
\end{align*}
is a finite, so we may let
$$\lambda_0:=\min\{\lambda\mid \text{ there exists }(W,\lambda)\in\Ii\}.$$
Now by noetherian property, there exists $(W_0,\lambda_0)\in\Ii$ such that $W_0\subset W$ for any $(W,\lambda_0)\in\Ii$. 

\medskip

\noindent\textbf{Step 4}. In this step we construct an $\Rr$-divisor $\bar B$ on $X$ and a $\Qq$-factorial ACSS model $(Y,\Ff_Y,\bar B_Y,\Mm)$ of $(X,\Ff,B,\Mm)$, so that $R$ is the image of a $(K_{\Ff_Y}+\bar B_Y+\Mm_Y)$-negative extremal ray$/U$ in $X$.

Let $\bar B:=B+\lambda_0P$ and let $E$ be an lc place of $(X,\Ff,\bar B,\Mm)$ such that $\Center_XE=W_0$. By our assumption, there exists an ACSS model $(Y,\Ff_Y,\bar B_Y,\Mm;G)$ of $(X,\Ff,\bar B,\Mm)$ such that $E$ is on $Y$ with induced birational morphism $g: Y\rightarrow X$. We have
$$K_{\Ff_Y}+\bar B_Y+\Mm_Y+F=g^*(K_{\Ff}+\bar B+\Mm_X)$$
for some $F\geq 0$. Let $V:=g(\Supp F)$, then $V\subset\Nlc(X,\Ff,\bar B,\Mm)$ is a reduced subscheme of $X$.

By Lemma \ref{lem: extremal ray under morphism}, there exists an extremal ray $R_Y$ in $Y$ such that $g(R_Y)=R$. Then there exist $C_{Y,i}\in NE(Y/U)$ such that $R_Y=[\lim_{i\rightarrow+\infty} C_{Y,i}]$. We let $C_i:=g(C_{Y,i})$, then $R=[\lim_{i\rightarrow+\infty} C_{i}]$. By the projection formula,
$$\lim (K_{\Ff_Y}+\bar B_Y+F+\Mm_Y)\cdot C_{Y,i}=\lim (K_{\Ff}+\bar B+\Mm_X)\cdot C_i,$$
so
$$(K_{\Ff_Y}+\bar B_Y+F+\Mm_Y)\cdot R_Y=(K_{\Ff}+\bar B+\Mm_X)\cdot R<0.$$
Thus $R_Y$ is a $(K_{\Ff_Y}+\bar B_Y+F+\Mm_Y)$-negative extremal ray. 

    If $F\cdot R_Y<0$, then $R_Y$ is contained in the image of $\overline{NE}(\Supp F/U)\rightarrow\overline{NE}(Y/U)$. Then $R=g(R_Y)$ is contained in the image of $\overline{NE}(V/U)\rightarrow\overline{NE}(X/U)$. Thus there exists an irreducible component $V_0$ of $V$ such that $R$ is contained in the image of $\overline{NE}(V_0/U)\rightarrow\overline{NE}(X/U)$.  Since $R$ is not contained in $\overline{NE}(X/U)_{\Nlc(X,\Ff,B,\Mm)}$, $V_0$ is not an lc center of $(X,\Ff,B,\Mm)$. Since $V_0\subset V=f(F)\subset\Nlc(X,\Ff,\bar B,\Mm)$ and $\bar B=B+\lambda_0P$, there exists a real number $0<\lambda_1<\lambda_0$ such that $V_0$ is an lc center of $\Nlc(X,\Ff,B+\lambda_1P,\Mm)$. This contradicts the minimality of $(W_0,\lambda_0)$ as $(V_0,\lambda_1)\in\Ii$ and $\lambda_1<\lambda_0$. Therefore, $F\cdot R_Y\geq 0$, so $R_Y$ is a  $(K_{\Ff_Y}+\bar B_Y+\Mm_Y)$-negative extremal ray.

\medskip

\noindent\textbf{Step 5}. In this step we prove the proposition under the additional condition that $X$ is $\Qq$-factorial. 

Assume that $X$ is $\Qq$-factorial. By \cite[Lemma 3.6.2]{BCHM10}, $\Exc(f)$ is a divisor, so $g^{-1}(W_0)$ is a divisor. Since $R$ is contained in the image of $\overline{NE}(W/U)\rightarrow\overline{NE}(X/U)$ and $g(R_Y)=R$, there exists a divisor $E_0$ on $Y$ such that $R_Y$ is contained in the image of $\overline{NE}(E_0/U)\rightarrow\overline{NE}(Y/U)$. Since $g$ is an ACSS modification of $(X,\Ff,\bar B,\Mm)$, $E_0$ is an lc place of $(X,\Ff,\bar B,\Mm)$ and an lc place of $(Y,\Ff_Y,\bar B_Y,\Mm)$. 

Let $T$ be the normalization of $E_0$, $\Ff_{T}:=\Ff_Y|_{T}$ be the restricted foliation, $\Mm^{T}:=\Mm|_{T}$, and
$$K_{\Ff_T}+\bar B_T+\Mm^T_T:=(K_{\Ff_Y}+\bar B_Y+\Mm_Y)|_T.$$
Since  $R_Y$ is contained in the image of $\overline{NE}(E_0/U)\rightarrow\overline{NE}(Y/U)$,  $R_Y$ is contained in the image of 
$$\iota: \overline{NE}(T/U)\rightarrow \overline{NE}(E_0/U)\rightarrow\overline{NE}(Y/U).$$ 
By Lemma \ref{lem: extremal ray under morphism}, there exists any extremal ray $R_T\in\overline{NE}(T/U)$ such that $\iota(\tilde R)=R_Y$. Then $R_T$ is a $(K_{\Ff_T}+B_T+\Mm^T_T)$-negative extremal ray$/U$. By Theorem \ref{thm: adjunction foliation nonnqc} and Theorem \ref{thm: cone theorem induction} in dimension $\leq d-1$, $R_T$ is spanned by a rational curve $C_T$, such that $C_T$ is tangent to $\Ff_T$ and
$$0<-(K_{\Ff_T}+\bar B_T+\Mm^T_T)\cdot C_T\leq 2(d-1).$$
Let $C_Y$ be the image of $C_T$ in $Y$, then $C_Y$ spans $R_Y$,
$$0<-(K_{\Ff_T}+\bar B_T+\Mm^T_T)\cdot C_T=-(K_{\Ff_Y}+\bar B_Y+\Mm_Y)\cdot C_Y\leq 2(d-1),$$
and by \cite[Lemma 3.3(4)]{ACSS21}, $C_Y$ is tangent to $\Ff_Y$. Let $C:=g(C_Y)$, then $C$ is tangent to $\Ff$. By \textbf{Step 4}, $F\cdot C_Y\geq 0$, so
$$2d\geq -(K_{\Ff_Y}+\bar B_Y+\Mm_Y)\cdot C_Y\geq -(K_{\Ff_Y}+\bar B_Y+F+\Mm_Y)\cdot C_Y=-(K_{\Ff}+B+\Mm_X)\cdot C>0.$$
We are done for the case when $X$ is $\Qq$-factorial.

\medskip

\noindent\textbf{Step 6}. In this step we conclude the proof of the theorem. Since $Y$ is $\Qq$-factorial and $R_Y$ is a $(K_{\Ff_Y}+\bar B_Y+\Mm_Y)$-negative extremal ray, by the $\Qq$-factorial case proved in \textbf{Step 5}, $R_Y$ is spanned by a rational curve $C_Y$ that is tangent to $\Ff_Y$ and 
$$0<-(K_{\Ff_Y}+\bar B_Y+\Mm_Y)\cdot C_Y\leq 2d.$$
 Let $C:=g(C_Y)$, then $C$ is tangent to $\Ff$. Since $F\cdot C_Y\geq 0$,
$$2d\geq -(K_{\Ff_Y}+\bar B_Y+\Mm_Y)\cdot C_Y\geq -(K_{\Ff_Y}+\bar B_Y+F+\Mm_Y)\cdot C_Y=-(K_{\Ff}+B+\Mm_X)\cdot C>0.$$
This concludes the proof of the proposition.
\end{proof}

\begin{prop}\label{prop: cone finiteness rays}
Let $d$ be a positive integer. Assume that Theorem \ref{thm: cone theorem induction} holds in dimension $\leq d-1$ and Theorem \ref{thm: property * induction} holds for $\Qq$-factorial varieties in dimension $d$. 
       
Let $(X,\Ff,B,\Mm)/U$ be a gfq of dimension $d$ such that $\Ff$ is algebraically integrable, and let $A$ be an ample$/U$ $\Rr$-divisor on $X$. Assume that 
       \begin{itemize}
        \item either $X$ is $\Qq$-factorial, or
        \item Theorem \ref{thm: property * induction} holds in dimension $d$. 
    \end{itemize}
Then there are finitely many $(K_{\Ff}+B+A+\Mm_X)$-negative extremal rays$/U$ that are not contained in $\overline{NE}(X/U)_{\Nlc(X,\Ff,B,\Mm)}$.
\end{prop}
\begin{proof}
Let $d:=\dim X$ and let $\omega:=K_{\Ff}+B+\Mm_X$, $\rho:=\rho(X/U)$, and let $A_1,\dots,A_{\rho-1}$ be ample$/U$ Cartier divisors on $X$, such that $\omega,A_1,\dots,A_{\rho-1}$ form a basis of $N^1_{\mathbb R}(X/U)$. Let $0<\epsilon\ll 1$ be a rational number such that $A-\epsilon\sum_{i=1}^{\rho-1}A_i$ is ample$/U$. Then we only need to show that there are finitely many $(K_{\Ff}+B+\epsilon\sum_{i=1}^{\rho-1}A_i+\Mm_X)$-negative extremal rays$/U$ that are not contained in $\overline{NE}(X/U)_{\Nlc(X,\Ff,B,\Mm)}$. Possibly replacing $A$, we may assume that $A=\epsilon\sum_{i=1}^{\rho-1}A_i$.

Suppose that the proposition does not hold. Then there exist an infinite set $\Lambda$ and an infinite set $\{R_j\}_{j\in\Lambda}$ of $(K_{\Ff}+B+A+\Mm_X)$-negative extremal rays$/U$ that are not contained in $\overline{NE}(X/U)_{\Nlc(X,\Ff,B,\Mm)}$. By Definition-Lemma \ref{deflem: exposed ray}, possibly replacing $\Lambda$ with a smaller infinite subset, we may assume that each $R_j$ is a  $(K_{\Ff}+B+A+\Mm_X)$-negative exposed ray$/U$. By Proposition \ref{prop: * to cone}, for any $j\in\Lambda$, there exists a rational curve $C_j$ on $X$ that is tangent to $\Ff$ and $R_j=[C_j]$, such that 
$$-2d\leq \omega\cdot C_j<0.$$ 
For each $j\in\Lambda$, by Lemma \ref{lem: supporting function are +A}, there exists an ample$/U$ $\Rr$-divisor $L_j$ and a nef$/U$ $\Rr$-divisor $H_j$, such that
$$H_j=L_j+(K_{\Ff}+B+A+\Mm_X)=L_j+\epsilon\sum_{i=1}^{\rho-1}A_i+\omega$$
and $H_j$ is the supporting function of $R_j$. We have
$$0=H_j\cdot C_j=L_j\cdot C_j+\epsilon\sum_{i=1}^{\rho-1}A_i\cdot C_j+\omega\cdot C_j\geq-2d+\epsilon\sum_{i=1}^{\rho-1}A_i\cdot C_j.$$
Therefore, $A_i\cdot C_j\leq\frac{2d}{\epsilon}$ for any $i,j$. Since $A_i\cdot C_j\in\mathbb N^+$, there are finitely many possibilities of $A_i\cdot C_j$. Possibly replacing $\Lambda$ with an infinite subset, we may assume that $A_i\cdot C_j=A_i\cdot C_{j'}$ for any $i$ and any $j,j'\in\Lambda$.

We may write $\omega=\sum_{i=1}^c r_iD_i$ such that $r_1,\dots,r_c$ are linearly independent over $\Qq$ and $D_i$ are Weil divisors.  By \cite[Lemma 5.3]{HLS19}, each $D_i$ is a $\Qq$-Cartier divisor. Thus there exist real numbers $a_{i,k}$ and $b_i$, such that
$$D_i\equiv_U\sum_{k=1}^{\rho-1}a_{i,k}A_k+b_i\omega$$
for each $i$. 
    
We let $\delta_1,\dots,\delta_c$ be real numbers such that $\sum_{i=1}^cb_i\delta_i>-1$ and 
$$r_i':=\delta_i+r_i\in\mathbb Q.$$ 
Let $\omega':=\sum_{i=1}^cr_i'D_i$. Then 
$$\omega'=\omega+\sum_{i=1}^c\delta_iD_i=\left(\sum_{k=1}^{\rho-1}\left(\sum_{i=1}^c\delta_ia_{i,k}\right)A_k\right)+\left(1+\sum_{i=1}^c\delta_ib_i\right)\omega.$$
Since $\sum_{i=1}^cb_i\delta_i>-1$, $\omega'$ and $A_1,\dots,A_{\rho-1}$ form a basis of $N^1_\Rr(X/U)$. Moreover,
$$\omega'\cdot C_j=\left(\sum_{i=1}^c\sum_{k=1}^{\rho}\delta_ia_{i,k}\cdot(A_k\cdot C_j)\right)+\left(1+\sum_{i=1}^c\delta_ib_i\right)(\omega\cdot C_j).$$
By our assumptions,
$$\alpha:=\sum_{i=1}^c\sum_{k=1}^{\rho}\delta_ia_{i,k}\cdot(A_k\cdot C_j)$$
and
$$\beta:=1+\sum_{i=1}^c\delta_ib_i>0$$
are constants which do not depend on $j$, and $\omega\cdot C_j\in [-2d,0)$. Therefore,
$$\omega'\cdot C_j\in [-2d\beta+\alpha,\alpha)$$
for any $j$.
Since $r_i'\in\mathbb Q$ for any $i$, $\omega'$ is a $\Qq$-Cartier $\Qq$-divisor. Let $I$ be the Cartier index of $\omega'$, then 
$$\omega'\cdot C_j\in [-2d\beta+\alpha,\alpha)\cap \frac{1}{I}\mathbb Z$$
for any $j$. Therefore, there are only finitely many possibilities of $\omega'\cdot C_j$. Possibly replacing $\Lambda$ with an infinite subset, we may assume that $\omega'\cdot C_j=\omega'\cdot C_{j'}$ for any $j,j'\in\Lambda$. Since $\omega',A_1,\dots,A_{\rho-1}$ form a basis of $N^1_\Rr(X/U)$, $C_j\equiv_U C_{j'}$, which is not possible as $R_j$ and $R_j'$ are different rays in $\overline{NE}(X/U)$.
\end{proof}


\begin{prop}\label{prop: * to cone final part}
    Let $d$ be a positive integer. Assume that Theorem \ref{thm: cone theorem induction} holds in dimension $\leq d-1$ and Theorem \ref{thm: property * induction} holds for $\Qq$-factorial varieties in dimension $d$. Then:
    \begin{enumerate}
    \item Theorem \ref{thm: cone theorem induction} holds for $\Qq$-factorial varieties in dimension $d$.
    \item If Theorem \ref{thm: property * induction} holds in dimension $d$, then Theorem \ref{thm: cone theorem induction} holds in dimension $d$.
    \end{enumerate}
\end{prop}
\begin{proof}
First we show that $\overline{NE}(X/U)=V$, where
$$V:=\overline{NE}(X/U)_{K_{\Ff}+B+\Mm_X\geq 0}+\overline{NE}(X/U)_{\Nlc(X,\Ff,B,\Mm)}+\sum_{j\in\Lambda}R_j.$$
By Definition-Lemma \ref{deflem: exposed ray}, $\overline{NE}(X/U)=\overline{V}$. Suppose that $V\not=\overline{V}$, then there exists an extremal ray $R$ in $\overline{NE}(X/U)$ such that $R\not\in V$. Since $\overline{NE}(X/U)_{K_{\Ff}+B+\Mm_X\geq 0}$ and $\overline{NE}(X/U)_{\Nlc(X,\Ff,B,\Mm)}$ are closed, $R\not\in \overline{NE}(X/U)_{K_{\Ff}+B+\Mm_X\geq 0}$ and $R\not\in \overline{NE}(X/U)_{\Nlc(X,\Ff,B,\Mm)}$, so $R$ is a $(K_{\Ff}+B+\Mm_X)$-negative extremal ray $R$ that is not contained in $\overline{NE}(X/U)_{\Nlc(X,\Ff,B,\Mm)}$. Thus $R=R_j$ for some $j$, a contradiction. Therefore, $\overline{NE}(X/U)=V$.

Next we show that any each $R_j$ is exposed. For any fixed $j$, There exists an ample$/U$ $\Rr$-divisor $A$ such that $R_j$ is a $(K_{\Ff}+B+A+\Mm_X)$-negative extremal ray$/U$. Suppose that $R_j$ is not exposed. By Definition-Lemma \ref{deflem: exposed ray}, $R_j=\lim_{i\rightarrow+\infty}R_{j,i}$ for some exposed rays $R_{j,i}\in\overline{NE}(X/U)$. Since $(K_{\Ff}+B+A+\Mm_X)\cdot R_j<0$, possibly passing to a subsequence, we have $(K_{\Ff}+B+A+\Mm_X)\cdot R_{j,i}<0$ for any $i$.  By Proposition \ref{prop: cone finiteness rays}, there are only finitely many $(K_{\Ff}+B+A+\Mm_X)$-negative extremal rays that are not contained in $\overline{NE}(X/U)_{\Nlc(X,\Ff,B,\Mm)}$, for any $i\gg 0$, $R_{j,i}$ is contained in $\overline{NE}(X/U)_{\Nlc(X,\Ff,B,\Mm)}$. Since $\overline{NE}(X/U)_{\Nlc(X,\Ff,B,\Mm)}$ is a closed sub-cone of $\overline{NE}(X/U)$, $R_j$ is contained in $\overline{NE}(X/U)_{\Nlc(X,\Ff,B,\Mm)}$, a contradiction.

By Proposition \ref{prop: * to cone}, for any $j\in\Lambda$, $R_j$ is spanned by a rational curve $C_j$ such that $C_j$ is tangent to $\Ff$ and
$$0<-(K_{\Ff}+B+\Mm_X)\cdot C_j\leq 2d.$$
\end{proof}





\subsection{Proofs of  Theorems \ref{thm:  ACSS model}, \ref{thm: precise adj gfq}, \ref{thm: dcc adjunction is dcc}, and \ref{thm: lc adjunction foliation nonnqc}}\label{subsec: proof of adj}


\begin{proof}[Proofs of Theorems \ref{thm: cone theorem induction} and \ref{thm: property * induction}]
Theorems \ref{thm: cone theorem induction} and \ref{thm: property * induction} hold when $d=1$ trivially. Therefore, we may assume that $d\geq 2$ and Theorems \ref{thm: cone theorem induction} and \ref{thm: property * induction} hold in dimension $\leq d-1$.

By Proposition \ref{prop: cone d-1 imply * dim d part 2}(1), Theorem \ref{thm: property * induction} holds for $\Qq$-factorial varieties in dimension $d$. By Proposition \ref{prop: * to cone final part}(1), Theorem \ref{thm: cone theorem induction} holds for $\Qq$-factorial varieties in dimension $d$. By Proposition \ref{prop: cone d-1 imply * dim d part 2}(2), Theorem \ref{thm: property * induction} holds in dimension $d$. By  Proposition \ref{prop: * to cone final part}(2), Theorem \ref{thm: cone theorem induction} holds in dimension $d$.

By induction on $d$, Theorems \ref{thm: cone theorem induction} and \ref{thm: property * induction} hold.
\end{proof}


\begin{proof}[Proof of Theorem \ref{thm:  ACSS model}]
    It is a special case of Theorem \ref{thm: property * induction}.
\end{proof}


\begin{proof}[Proof of Theorem \ref{thm: lc adjunction foliation nonnqc}]
 By Theorem \ref{thm:  ACSS model}, there exists a $\Qq$-factorial ACSS model $(X',\Ff',B',\Mm)$ of $(X,\Ff,B,\Mm)$ with induced birational morphism $h: X'\rightarrow X$ and associated with $f: X'\rightarrow Z$. Since  $(X,\Ff,B,\Mm)$ is lc,
 $$K_{\Ff'}+B'+\Mm_{X'}=f^*(K_{\Ff}+B+\Mm_X).$$
 Let $S'$ be the normalization of $h^{-1}_*S$, $\Ff_{S'}$ the restricted foliation of $\Ff$ on $S'$, and $\Mm^S:=\Mm|_{S^\nu}$. Then there exists an induced birational morphism $h_S: S'\rightarrow S^\nu$. Let
 $$K_{\Ff_{S'}}+B_{S'}+\Mm^S_{S'}:=(K_{\Ff'}+B'+\Mm_{X'})|_{S'},$$
 then by Theorem \ref{thm: adjunction foliation nonnqc}, $(S',\Ff_{S'},B_{S'},\Mm^S)$ is lc. Since
 $$K_{\Ff_{S'}}+B_{S'}+\Mm^S_{S'}=h_S^*(K_{\Ff_S}+B_S+\Mm^S_{S^\nu}),$$
 $(S^\nu,\Ff_S,B_S,\Mm^S)$ is lc. 
  \end{proof}

\begin{proof}[Proof of Theorem \ref{thm: precise adj gfq}]
By Theorem \ref{thm:  ACSS model}, there exists a $\Qq$-factorial ACSS model $(X',\Ff',B',\Mm)$ of $(X,\Ff,B,\Mm)$ with induced birational morphism $h: X'\rightarrow X$ and associated with $f: X'\rightarrow Z$. Let $S'$ be the normalization of $h^{-1}_*S$ and $E:=(\Supp\Exc(h))^{\Ff_Y}$.  Then there exists an induced birational morphism $h_S: S'\rightarrow S^\nu$.
    
We let $\Ff_{S'}$ be the restricted foliation of $\Ff'$ on $S'$. By Theorem \ref{thm: precise adjunction when induced}, there exist prime divisors $C_1',\dots,C_q',T_1',\dots,T_l'$ on $S'$, positive integers $w_1,\dots,w_q$, and non-negative integers $\{w_{i,j}\}_{1\leq i\leq q,1\leq j\leq m}$ and $\{v_{i,k}\}_{1\leq i\leq q, 1\leq k\leq n}$ satisfying the following. For any real numbers $b_1',\dots,b_m'$ and $r_1',\dots,r_n'$,
     \begin{align*}
&\left(K_{\Ff'}+\epsilon_{\Ff}(S)S'+\sum_{j=1}^mb_j'B_j'+\sum_{k=1}^nr_k'\Mm_{k,X'}\right)\Bigg|_{S'}\\
=&K_{\Ff_{S'}}+\sum_{i=1}^q\frac{w_i-1+\sum_{j=1}^mw_{i,j}b_j'+\sum_{k=1}^nv_{i,k}r_k'}{w_i}C_i'+\sum_{i=1}^lT_i'+\sum_{k=1}^nr_k'\Mm_{k,S'}^S\\
:=&K_{\Ff_{S'}}+B'_{S'}+\Mm'^S_{S'}.
\end{align*}
Let
$$K_{\Ff_S}+B'_S+\Mm'^S_{S^\nu}:=\left(K_{\Ff}+\sum_{j=1}^mb_j'B_j+\sum_{k=1}^nr_k'\Mm_{k,X'}\right)\Biggm|_{S^\nu},$$
then
 $$K_{\Ff_{S'}}+B'_{S'}+\Mm'^S_{S'}=(f_S)_*(K_{\Ff_{S}}+B'_{S}+\Mm'^S_{S^\nu}).$$
 Theorem \ref{thm: precise adj gfq}(1) follows.  Theorem \ref{thm: precise adj gfq}(2) follows from Theorem \ref{thm: lc adjunction foliation nonnqc}.
\end{proof}

\begin{proof}[Proof of Theorem \ref{thm: dcc adjunction is dcc}]
    It is an immediate corollary of Theorem \ref{thm: precise adj gfq}.
\end{proof}

\subsection{Proof of Theorem \ref{thm: cone theorem gfq}}\label{subsec: proof of cone}

Finally, we prove the full version of the cone theorem for generalized foliated quadruples, Theorem \ref{thm: cone theorem gfq}.


\begin{lem}\label{lem: gfq extremal ray rational}
Let $(X,\Ff,B,\Mm)/U$ be a gfq such that $\Ff$ is algebraically integrable. Assume that $R$ is a $(K_{\Ff}+B+\Mm_X)$-negative extremal ray in $\overline{NE}(X/U)$ that is not contained in $\overline{NE}(X/U)_{\Nlc(X,\Ff,B,\Mm)}$. Then $R$ is a rational extremal ray in $\overline{NE}(X/U)$.
\end{lem}
\begin{proof}
By Lemma \ref{lem: supporting function are +A}, there exists an ample$/U$ $\Rr$-divisor $A$ such that $H_R:=K_{\Ff}+B+A+\Mm_X$ is a supporting function of $R$. We let $\delta\in (0,1)$ be a rational number such that $R$ is a $(K_{\Ff}+B+\delta A+\Mm_X)$-negative extremal ray$/U$ that is not contained in $\overline{NE}(X/U)_{\Nlc(X,\Ff,B,\Mm)}$. Let $\Lambda$ be the set of all $(K_{\Ff}+B+\delta A+\Mm_X)$-negative extremal ray$/U$. By Proposition \ref{prop: cone finiteness rays}, $\Lambda$ is a finite set, and we may write $\Lambda=\{R,R_1,\dots,R_l\}$ for some non-negative integer $l$. Then 
$$V:=\overline{NE}(X/U)_{K_{\Ff}+B+\delta A+\Mm_X\geq 0}+\overline{NE}(X/U)_{\Nlc(X,\Ff,B,\Mm)}+\sum_{i=1}^lR_i$$
is a closed sub-cone of $\overline{NE}(X/U)$ and $R\not\in V$. Let $C$ be the dual cone of $V$ in $N^1(X/U)$, then since $H_R\cdot R'>0$ for any $R'\in V$, $H_R$ is contained in the interior of $C$. Therefore, there exists a real number $\epsilon\in (0,1)$ such that $H_R-\epsilon A$ is contained in the interior of $C$. In particular, $(H_R-\epsilon A)\cdot R'>0$ for any $R'\in V$.

We write $H_R=\sum_{i=1}^cr_iD_i$, where $r_1,\dots,r_c$ are real numbers that are linearly independent over $\Qq$ and $D_i$ are Weil divisors. By \cite[Lemma 5.3]{HLS19}, each $D_i$ is a $\Qq$-Cartier $\Qq$-divisor. Moreover, by Theorem \ref{thm: cone theorem induction}, $R$ is spanned by a rational curve $L$. Since $H_R\cdot L=0$, $D_i\cdot L=0$ for each $i$.

There exist rational numbers $r_1',\dots,r_c'$ such that $\sum_{i=1}^c(r_i'-r_i)D_i+\epsilon A$ is ample$/U$. We let $H_R':=\sum_{i=1}^cr_i'D_i$. Then $H_R'\cdot R=0$. For any extremal ray $R'\in\overline{NE}(X/U)$ such that $R'\not=R$, $R'\in V$. Thus
$$H_R'\cdot R'=H_R\cdot R'+\sum_{i=1}^c(r_i'-r_i)D_i\cdot R'=(H_R-\epsilon A)\cdot R'+\left(\sum_{i=1}^c(r_i'-r_i)D_i+\epsilon A\right)\cdot R'>0.$$
Thus $H_R'$ is a supporting function of $R$. Since $H_R'$ is a $\Qq$-divisor, it is a rational supporting function of $R$, so $R$ is a rational extremal ray in $\overline{NE}(X/U)$.
\end{proof}

\begin{proof}[Proof of Theorem \ref{thm: cone theorem gfq}]
    Theorem \ref{thm: cone theorem gfq}(1) follows from Lemma \ref{lem: gfq extremal ray rational} and Theorem \ref{thm: cone theorem induction}. Theorem \ref{thm: cone theorem gfq}(2) follows from Theorem \ref{thm: cone theorem induction}. Theorem \ref{thm: cone theorem gfq}(3) follows from Proposition \ref{prop: cone finiteness rays} and that $\Lambda=\cup_{n=1}^{+\infty}\Lambda_{\frac{1}{n}A}$ for any ample$/U$ $\Rr$-divisor $A$. We left to prove (4).

     For any $(K_{\Ff}+B+\Mm_X)$-negative extremal face $F$ in $\overline{NE}(X/U)$ that is relatively ample at infinity with respect to $(X,\Ff,B,\Mm)$, $F$ is also a $(K_{\Ff}+B+\Mm_X+A)$-negative extremal face for some ample$/U$ $\Rr$-divisor $A$ on $X$. Let $V:=F^\bot\subset N^1(X/U)$. By (1), $F$ is spanned by a subset of $\{R_j\}_{j\in\Lambda_A}$ and $R_j$ is rational, so $V$ is defined over $\Qq$. We let
$$W_F:=\overline{NE}(X/U)_{K_X+B+\Mm_X+A\geq 0}+\overline{NE}(X/U)_{\Nlc(X,\Ff,B,\Mm)}+\sum_{j\mid j\in\Lambda_A,R_j\not\subset F}R_j.$$
Then $W_F$ is a closed cone, $\overline{NE}(X/U)=W_F+F$, and $W_F\cap F=\{0\}$. The supporting functions of $F$ are the elements in $V$ that are positive on $W_F\backslash\{0\}$, which is a non-empty open subset of $V$, and hence contains a rational element $H$. In particular, $F=H^\bot\cap \overline{NE}(X/U)$, hence $F$ is rational, and we get (4). This concludes the proof of Theorem \ref{thm: cone theorem gfq}.
\end{proof}


\section{Minimal model program for ACSS generalized foliated quadruples}\label{sec: mmp gfq}

With the establishment of the cone theorem, we are ready to study the minimal model program for algebraically integrable generalized foliated quadruples. Unfortunately for us, we cannot prove the contraction theorem and the cone theorem for the time being due to technical reasons. However, we are able to run some special types of the minimal model program for foliations.

\subsection{Models} With the definition of ACSS singularities, we are able to define the concept of \emph{log minimal models} and \emph{good minimal models} for algebraically integrable foliations.

\begin{defn}[Models, II]\label{defn: models ii}
Let $(X,\Ff,B,\Mm)/U$ be an lc gfq and $(X',\Ff',B',\Mm)/U$ a log birational model of $(X,\Ff,B,\Mm)/U$. We say that $(X',\Ff',B',\Mm)/U$ is a \emph{log minimal model} of $(X,\Ff,B,\Mm)/U$ if 
\begin{enumerate}
    \item $(X',\Ff',B',\Mm)/U$ is a weak lc model of $(X,\Ff,B)/U$,
    \item $(X',\Ff',B',\Mm)$ is $\Qq$-factorial ACSS, and
    \item for any prime divisor $D$ on $X$ which is exceptional over $X'$, 
    $$a(D,\Ff,B,\Mm)<a(D,\Ff',B',\Mm).$$
\end{enumerate}
We say that $(X',\Ff',B',\Mm)/U$ is a \emph{good minimal model} of $(X,\Ff,B,\Mm)/U$ if $(X',\Ff',B',\Mm)/U$ is a log minimal model of $(X,\Ff,B,\Mm)/U$ and a semi-good minimal model of $(X,\Ff,B,\Mm)/U$.
\end{defn}

\begin{rem}
    It is important to note that, the concept of ``log minimal model" or ``good minimal model" defined in Definition \ref{defn: models ii} does not coincide with the concept of  ``log minimal model" or ``good minimal model" when $\Ff=T_X$ in the classical setting (\cite[Definition 2.1]{Bir12}, \cite[Definition 3.2]{HL21a}). This is because ``ACSS" is equivalent to ``qdlt" when $\Ff=T_X$, while the classical definition of log minimal models requires the (generalized) pair to be ``dlt". This difference will not cause trouble, mainly because the existence of log (resp. good) minimal models is equivalent to the existence of weak lc (resp. semi-good) minimal models, at least for NQC generalized pairs (cf. \cite[Theorem 2.7]{TX23}).
\end{rem}

The following lemma is straightforward but also convenient for us to apply in some scenarios.
\begin{lem}\label{lem: acss model is gmm}
Let $(X,\Ff,B,\Mm)/U$ be an lc gfq and $(X',\Ff',B',\Mm)$ a $\Qq$-factorial ACSS model of $(X,\Ff,B,\Mm)$. Then $(X',\Ff',B',\Mm)/X$ is a good minimal model of $(X,\Ff,B,\Mm)/X$.
\end{lem}
\begin{proof}
    It immediately follows from the definitions.
\end{proof}

\begin{lem}\label{lem: ACSS mmp can run}
Let $(X,\Ff,B,\Mm)/U$ be an lc gfq, $\Delta\geq 0$ an $\Rr$-divisor on $X$, and $\Nn$ a nef$/U$ $\bb$-divisor on $X$. Assume that $\Ff$ is induced by a contraction $f: X\rightarrow Z$ and 
$$K_{\Ff}+B+\Mm_X\sim_{\mathbb R,Z}K_X+\Delta+\Nn_X.$$
 Then the followings hold.
\begin{enumerate}
    \item Any $(K_{\Ff}+B+\Mm_X)$-negative extremal ray$/U$ $R$ is a $(K_X+\Delta+\Nn_X)$-negative extremal ray$/Z$, and $(K_{\Ff}+B+\Mm_X)\cdot R=(K_X+\Delta+\Nn_X)\cdot R.$
    \item Any step of a $(K_{\Ff}+B+\Mm_X)$-MMP$/U$ is a step of a $(K_X+\Delta+\Nn_X)$-MMP$/Z$. Moreover, assume that $(X,\Delta,\Nn)$ is lc and either $X$ is $\Qq$-factorial klt or $\Nn$ is NQC$/U$, then we may run a step of a $(K_{\Ff}+B+\Mm_X)$-MMP$/U$.
    \item Assume that $(X,\Ff,B,\Mm;G)/Z$ is weak ACSS for some divisor $G$, $\Delta=B+G$, and $\Mm=\Nn$. For any sequence of steps $$\phi: (X,\Ff,B,\Mm;G)\dashrightarrow (X',\Ff',B',\Mm;G')$$ 
    of a $(K_{\Ff}+B+\Mm_X)$-MMP$/U$, we have the following.
    \begin{enumerate}
    \item $(X',\Ff',B',\Mm;G')/Z$ is weak ACSS.
        \item If $(X,\Ff,B,\Mm;G)/Z$ is ACSS, then $(X',\Ff',B',\Mm;G')/Z$ is ACSS. 
        \item If $G$ is super$/Z$, then $G'$ is super$/Z$.
        \item If $X$ is $\Qq$-factorial klt, then $X'$ is $\Qq$-factorial klt. 
        \item If $X$ is $\Qq$-factorial and $\Mm$ is NQC$/U$, then $X'$ is $\Qq$-factorial.
        \item If $X$ is $\Qq$-factorial and $(X,\Ff,B,\Mm;G)/Z$ is (super) ACSS, then $X'$ is $\Qq$-factorial and $(X',\Ff',B',\Mm;G')/Z$ is (super) ACSS.
    \end{enumerate}
    \item Any sequence of steps of a $(K_{\Ff}+B+\Mm_X)$-MMP$/U$ is a sequence of steps of a $(K_X+\Delta+\Nn_X)$-MMP$/Z$.
\end{enumerate}
\end{lem}
\begin{proof}
(1) By Theorem \ref{thm: cone theorem gfq}, any $(K_{\Ff}+B+\Mm_X)$-negative extremal ray$/U$ is tangent to $\Ff$, hence is an extremal ray$/Z$. We get (1).

(2) By (1), any $(K_{\Ff}+B+\Mm_X)$-negative extremal ray$/U$ $R$ is a $(K_X+\Delta+\Nn_X)$-negative extremal ray$/U$. If $X$ is $\Qq$-factorial klt, then by \cite[Lemma 3.4]{HL22} and the cone theorem, contraction theorem, and the existence of flips for usual klt pairs, we get a step of a $(K_X+\Delta+\Nn_X)$-MMP$/U$ associated to $R$, which is also a step of a  $(K_{\Ff}+B+\Mm_X)$-MMP$/U$ associated to $R$. If $\Nn$ is NQC$/U$, then by the cone theorem (\cite[Theorem 1.3]{HL21a}, Theorem \ref{thm: cone theorem gfq}), the contraction theorem (\cite[Theorem 1.5]{Xie22}, \cite[Theorem 1.7]{CLX23}), and the existence of flips (\cite[Theorem 1.2]{LX23b}), we get a step of a $(K_{X}+\Delta+\Nn_X)$-MMP$/U$ associated to $R$, which is also a step of a  $(K_{\Ff}+B+\Mm_X)$-MMP$/U$ associated to $R$. Moreover, by (1), $R$ is a negative extremal ray$/Z$, so this step of the MMP is also a step of an MMP$/Z$.

(3) Without loss of generality, we may assume that $\phi$ is a single step of a $(K_{\Ff}+B+\Mm_X)$-MMP$/U$. By Proposition \ref{prop: MMP preserves *}, we get (3.a). (3.c) is obvious because $G'=\phi_*G$.

If $(X,\Ff,B,\Mm;G)/Z$ is ACSS, then there exist an $\Rr$-divisor $D\geq 0$ on $X$ and a nef$/X$ $\bb$-divisor $\Mm'$, such that $\Supp\{B\}\subset\Supp D$, $\Mm'-\alpha\Mm$ is nef$/X$ for some $\alpha>1$, and for any reduced divisor $\Sigma$ on $Z$ such that $\Sigma\geq f(G)$ and $(Z,\Sigma)$ is log smooth, $$(X,B+D+G+f^*(\Sigma-f(G)),\Mm')$$ is qdlt. Let $\Pp:=\Mm'-\Mm$, then $$(X,B+\delta D+G+f^*(\Sigma-f(G)),\Mm+\delta\Pp)$$ is qdlt for any $0\leq \delta\leq 1$, and 
$$\Mm+\delta\Pp-(1+\delta(\alpha-1))\Mm=\delta(\Mm'-\alpha\Mm)$$ 
is nef$/X$. By (2), $\phi$ is a step of a $(K_X+B+G+\Mm_X)$-MMP$/Z$, hence a step of a  
$$(K_X+B+\delta D+G+f^*(\Sigma-f(G))+\Mm_X+\delta\Pp_X)\text{-MMP}/Z$$
for any $0<\delta\ll 1$.
Thereforem
$$(K_{X'}+B'+\delta\phi_*D+G'+f'^*(\Sigma-f(G)),\Mm_{X}+\delta\Pp)$$ 
is qdlt, where $f': X'\rightarrow Z$ is the induced contraction. Moreover, for any lc place $E$ of $(X',\Ff',B',\Mm)$, since $$-\epsilon_{\Ff}(E)\leq a(E,\Ff,B,\Mm)\leq a(E,\Ff',B',\Mm)\leq -\epsilon_{\Ff'}(E)=-\epsilon_{\Ff}(E),$$
$E$ is also an lc place of $(X,\Ff,B,\Mm)$ and $\phi$ is an isomorphism near the generic point of $\Center_XE$. By (3.a) $(X',\Ff',B',\Mm;G')/Z$ is ACSS. This implies (3.b).

Assume that $X$ is $\Qq$-factorial. By (2),  $\phi$ is a step of a $(K_X+B+G+\Mm_X)$-MMP$/U$, hence a step of a $(K_X+B+G+\Mm_X+A)$-MMP$/U$ for some ample$/U$ $\Rr$-divisor $A$. If $X$ is klt, then by \cite[Lemma 3.4]{HL22}, there exists a klt pair $(X,\Delta)$ such that $0\leq\Delta\sim_{\mathbb R}B+G+\Mm_X+A$, so $\phi$ is a step of a $(K_X+\Delta)$-MMP, and (3.d) follows from \cite[Corollaries 3.17, 3.18]{KM98}. If $\Mm$ is NQC$/U$, then by \cite[Corollary 5.20, Theorem 6.3]{HL21a}, $X'$ is $\Qq$-factorial, and we get (3.e). Since $\Qq$-factorial qdlt implies that the ambient variety is klt, (3.f)
 follows from (3.b), (3.c) and (3.d).

(4) Since the birational transforms of $(X,\Ff,B,\Mm)$ are lc under any sequence of steps of a $(K_{\Ff}+B+\Mm_X)$-MMP, (4) follows from (2).
\end{proof}

\subsection{MMP with super divisors}

\begin{lem}\label{lem: super mmp with scaling}
Let $(X,\Ff,B,\Mm)/U$ be an lc gfq, $(X,\Delta,\Nn)/U$ an lc g-pair, and $f: X\rightarrow Z$ a contraction, such that $\Ff$ is induced by $f$, $\Delta$ is super$/Z$, and
$$K_{\Ff}+B+\Mm_X\sim_{\mathbb R,Z}K_X+\Delta+\Nn_X.$$
Then the followings hold.
\begin{enumerate}
  \item Any $(K_X+\Delta+\Nn_X)$-negative extremal ray$/U$ $R$ is a $(K_{\Ff}+B+\Mm_X)$-negative extremal ray$/Z$ and $(K_{\Ff}+B+\Mm_X)\cdot R=(K_X+\Delta+\Nn_X)\cdot R.$
  \item A step of a $(K_X+\Delta+\Nn_X)$-MMP$/U$ is a step of a $(K_{\Ff}+B+\Mm_X)$-MMP$/Z$.
  \item Any sequence of steps of a $(K_X+\Delta+\Nn_X)$-MMP$/U$ is a sequence of steps of a $(K_{\Ff}+B+\Mm_X)$-MMP$/Z$.
  \item Let $D\geq 0$ be an $\Rr$-divisor on $X$ and $\Nn'$ a nef$/U$ $\bb$-divisor on $X$ such that $D+\Nn'_X$ is $\Rr$-Cartier. Then any sequence of steps of a $(K_X+\Delta+\Nn_X)$-MMP$/U$ with scaling of $(D,\Nn')$ is a sequence of steps of a $(K_{\Ff}+B+\Mm_X)$-MMP$/U$ with scaling of $(D,\Nn')$, and any sequence of steps of a $(K_{\Ff}+B+\Mm_X)$-MMP$/U$ with scaling of $(D,\Nn')$ is a sequence of steps of a $(K_X+\Delta+\Nn_X)$-MMP$/U$ with scaling of $(D,\Nn')$.
\end{enumerate}
\end{lem}
\begin{proof}
(1) Let $d:=\dim X$. Since $\Delta$ is super, $\Delta\geq\sum_{i=1}^{2d+1}f^*H_i$ for some ample Cartier divisors $H_i$ on $Z$. Let $L:=\Delta-\sum_{i=1}^{2d+1}f^*H_i$, then $(X,L,\Mm)$ is lc and $R$ is a $(K_X+L+\Nn_X)$-negative extremal ray. By Theorem \ref{thm: cone theorem nonnqc gpair} (applied to $(X,T_X,L,\Nn)/U$), there exists a rational curve $C$ on $X$ such that $C$ spans $R$ and $$-2d\leq (K_X+L+\Nn_X)\cdot C<0.$$ Therefore,
$$0>(K_X+\Delta+\Nn_X)\cdot C=(K_X+L+\Nn_X)\cdot C+\left(\sum_{i=1}^{2d+1}f^*H_i\cdot C\right)\geq -2d+\left(\sum_{i=1}^{2d+1}f^*H_i\cdot C\right).$$
Thus $f(C)$ is a point, so $R$ is an extremal ray$/Z$. (1) follows from our assumption.

(2) immediately follows from (1). By (2), the birational transforms of $(X,\Ff,B,\Mm)$ and $(X,\Delta,\Nn)$ are lc after any sequences of steps of a $(K_X+\Delta+\Nn_X)$-MMP, and (3) follows from (2). (4) follows from (1), (3) and Lemma \ref{lem: ACSS mmp can run}(1).
\end{proof}

\begin{lem}\label{lem: equivalence over bases}
Let $(X,\Ff,B,\Mm)/U$ be an lc gfq and $(X,\Delta,\Nn)/U$ an lc g-pair such that $\Ff$ is induced by a contraction $X\to Z$ and
$$K_\Ff+B+\Mm_X\sim_{\Rr,Z}K_X+\Delta+\Nn_X.$$
Then the followings hold.
\begin{enumerate}
    \item $K_{\Ff}+B+\Mm_X$ is nef$/Z$ if and only if $K_X+\Delta+\Nn_X$ is nef$/Z$.
    \item If $K_X+\Delta+\Nn_X$ is either nef$/Z$ or nef$/U$, then $K_{\Ff}+B+\Mm_X$ is nef$/U$.
    \item If $\Delta$ is super$/Z$ and $K_{\Ff}+B+\Mm_X$ is either nef$/Z$ or nef$/U$, then $K_X+\Delta+\Nn_X$ is nef$/U$.
\end{enumerate}
\end{lem}
\begin{proof}
(1) is obvious. (2) follows from Lemma \ref{lem: ACSS mmp can run}(1). (3) follows from Lemma \ref{lem: super mmp with scaling}(1).
\end{proof}

\subsection{MMP with scaling and existence of Mori fiber spaces}\label{subsec: eomfs}

\begin{nota}
In the subsequent discussions, it is important to differentiate between ``one specific MMP that adheres to certain properties" and ``all MMPs that adhere to certain properties." For instance, some argument apply to ``all MMPs with scaling of an ample divisor," whereas some only apply to ``a specific MMP with scaling of an ample divisor." Given this nuance, we will regard ``MMPs" as entities, and typically represent them using symbols like $\mathcal{P}$ or similar notations.
\end{nota}


\begin{prop}\label{prop: run mmp with scaling gfq}
Let $(X,\Ff,B,\Mm)/U$ be an lc gfq and $f: X\rightarrow Z$ a contraction, such that 
$$K_{\Ff}+B+\Mm_X\sim_{\mathbb R,Z}K_X+\Delta+\Nn_X$$
for some lc g-pair $(X,\Delta,\Nn)/U$. Assume that either $X$ is $\Qq$-factorial klt or $\Nn$ is NQC$/U$. Then for any ample$/U$ $\Rr$-divisor $A$, we can run a $(K_{\Ff}+B+\Mm_X)$-MMP$/U$ with scaling of $A$. 

Moreover, there exists a $(K_{\Ff}+B+\Mm_X)$-MMP$/U$ with scaling of $A$, say $\mathcal{P}_0$, satisfying the following. Let $\mathcal{P}=\mathcal{P}_0$ if $X$ is not $\Qq$-factorial, and let $\mathcal{P}$ be any $(K_{\Ff}+B+\Mm_X)$-MMP$/U$ with scaling of $A$ if $X$ is $\Qq$-factorial. Then the followings hold.
\begin{enumerate}
  \item Suppose that there exists an lc gfq $(X,\tilde\Delta,\tilde\Nn)/U$ and an ample$/U$ $\Rr$-divisor $H$, such that either $X$ is $\Qq$-factorial klt or $\tilde\Nn$ is NQC$/U$, and $\Delta+\Nn_X\sim_{\mathbb R,U}\tilde\Delta+\tilde\Nn_X+H$. Then $\mathcal{P}$ terminates at a model $(X',\Ff',B',\Mm)/U$ of $(X,\Ff,B,\Mm)/U$, such that
  \begin{enumerate}
    \item either there exists a $(K_{\Ff'}+B'+\Mm_{X'})$-Mori fiber space$/U$ which is also a $(K_{\Ff'}+B'+\Mm_{X'})$-Mori fiber space$/Z$, or 
    \item $$K_{\Ff'}+B'+\Mm_{X'}\sim_{\mathbb R,Z}D$$ 
    for some semi-ample$/U$ $\Rr$-divisor $D$.
  \end{enumerate}
  \item  Either $\mathcal{P}$ terminates, or the limit of the scaling numbers of $\mathcal{P}$ is $0$. 
\end{enumerate}
\end{prop}
\begin{proof}
We first construct $\mathcal{P}_0$. Possibly replacing $\Delta$, we may assume that $\Delta$ is super$/Z$. By Lemma \ref{lem: scaling number go to 0} and \cite[Lemma 2.17]{TX23}, we may run a $(K_X+\Delta+\Nn_X)$-MMP$/U$ with scaling of $A$. By Lemmas \ref{lem: super mmp with scaling} and \ref{lem: equivalence over bases}, this MMP is also a $(K_{\Ff}+B+\Mm_{X})$-MMP$/U$ with scaling of $A$. This shows the existence of $\mathcal{P}_0$. 

Suppose that $X$ is not $\Qq$-factorial. Then $\Nn$ is NQC$/U$. By \cite[Theorem A, Theorem F, Lemma 4.3]{TX23}, there is a choice of $\mathcal{P}_0$ satisfying the following.
\begin{itemize}
    \item Either $\mathcal{P}_0$ terminates, or the limit of the scaling numbers of $\mathcal{P}_0$ is $0$.
    \item Suppose that there exists an lc gfq $(X,\tilde\Delta,\tilde\Nn)/U$ and an ample$/U$ $\Rr$-divisor $H$, such that either $X$ is $\Qq$-factorial klt or $\tilde\Nn$ is NQC$/U$, and $\Delta+\Nn_X\sim_{\mathbb R,U}\tilde\Delta+\tilde\Nn_X+H$. Then $\mathcal{P}$ terminates at
    \begin{itemize}
        \item either a semi-good minimal model $(X',\Delta',\Nn)/U$ of $(X,\Delta,\Nn)/U$, or
        \item a Mori fiber space $(X',\Delta',\Nn)\rightarrow T$ of $(X,\Delta,\Nn)/U$. Moreover, by Lemmas \ref{lem: super mmp with scaling} and \ref{lem: equivalence over bases}, $X'\rightarrow T$ is a contraction$/Z$. 
    \end{itemize}
\end{itemize}
This implies the proposition when $X$ is not $\Qq$-factorial. In the following, we may assume that $X$ is $\Qq$-factorial. 

We prove (1). Possibly replacing $H$ with a general element in $|H/U|_{\mathbb R}$, $\Delta$ with $\tilde\Delta+H$, and $\Nn$ with $\tilde N$, we may assume that $\Delta\geq H\geq 0$. The super$/Z$ property of $\Delta$ is lost here, but we may replace $\Delta$ again and re-assume that $\Delta$ is super$/Z$. By Lemma \ref{lem: super mmp with scaling}(4), any $(K_{\Ff}+B+\Mm_X)$-MMP$/U$ with scaling of $A$ is a $(K_X+\Delta+\Nn_X)$-MMP$/U$ with scaling of $A$. By Lemma \ref{lem: gklt+ample terminate} and Proposition \ref{prop: qfact nqc any scaling terminate}, $\mathcal{P}$ terminates. Let $(X',\Ff',B',\Mm)/U$ be the output of $\mathcal{P}$ and let $\Delta'$ be the image of $\Delta$ on $X'$. Then $\Delta'$ is super$/Z$. By Lemma \ref{lem: gklt+ample terminate} and Proposition \ref{prop: qfact nqc any scaling terminate}, 
\begin{itemize}
    \item either $K_{X'}+\Delta'+\Nn_{X'}$ is semi-ample$/U$ and we get (1.b), or
    \item there exists a $(K_{X'}+\Delta'+\Nn_{X'})$-Mori fiber space $X'\rightarrow T$ over $U$. By Lemma \ref{lem: super mmp with scaling}(1), $X'\rightarrow T$ is a $(K_{\Ff'}+\Delta'+\Nn_{X'})$-Mori fiber space$/Z$  and we get (1.a).
\end{itemize}

We prove (2). Suppose that $\mathcal{P}$ does not terminate and $\lambda$ is the limit of the scaling numbers of $\mathcal{P}$. Then $\mathcal{P}$ is an infinite sequence of steps of a $(K_{\Ff}+B+\frac{\lambda}{2}A+\Mm_X)$-MMP$/U$. Since $(X,\Ff,B,\Mm)$ is lc, $(X,\Ff,B,\Mm+\frac{1}{2}\bar A)$ is lc, and
$$K_{\Ff}+B+\frac{\lambda}{2}\bar A_X+\Mm_X=K_{\Ff}+B+\frac{1}{2}A+\Mm_X\sim_{\mathbb R,Z}K_X+\Delta+\frac{1}{2}A+\Nn_X.$$
(2) follows from (1).
\end{proof}

\begin{prop}\label{prop: run mmp get mfs}
Let $(X,\Ff,B,\Mm)/U$ be a weak ACSS gfq. %Let $A$ be an ample$/U$ $\Rr$-Cartier $\Rr$-divisor on $X$. 
Assume that
\begin{itemize}
    \item either $X$ is $\Qq$-factorial klt or $\Mm$ is NQC$/U$, and
    \item $K_{\Ff}+B+\Mm_X$ is not pseudo-effective$/U$.
\end{itemize}
Then there exists $\mathcal{P}_0$, a $(K_{\Ff}+B+\Mm_X)$-MMP$/U$ with scaling of an ample$/U$ $\Rr$-divisor $A$, which satisfies the following. Let $\mathcal{P}:=\mathcal{P}_0$ if $X$ is not $\Qq$-factorial, and let $\mathcal{P}$ be any $(K_{\Ff}+B+\Mm_X)$-MMP$/U$ with scaling of $A$ if  $X$ is $\Qq$-factorial. Then $\mathcal{P}$ terminates with a Mori fiber space of $(X,\Ff,B,\Mm)/U$.
\end{prop}
\begin{proof}
Let $(X_0,\Ff_0,B_0,\Mm):=(X,\Ff,B,\Mm)$. By Proposition \ref{prop: run mmp with scaling gfq}, we may suppose that $\mathcal{P}$ is an MMP with scaling of $A$
     \begin{center}$\xymatrix{
(X_0,\Ff_0,B_0,\Mm)\ar@{-->}[r] & (X_1,\Ff_1,B_1,\Mm)\ar@{-->}[r] & \dots\ar@{-->}[r] & (X_n,\Ff_n,B_n,\Mm)\ar@{-->}[r] & \dots 
}$
\end{center}
such that either this MMP terminates, or $\lim_{i\rightarrow+\infty}\lambda_i=0$, where
$$\lambda_i:=\inf\{t\geq 0\mid K_{\Ff_i}+B_i+tA_i+\Mm_{X_i}\text{ is nef}/U\}$$
are the scaling numbers and $A_i$ is the strict transform of $A$ on $X_i$. 

First we assume that $\mathcal{P}$ does not terminate.
Let $0<\epsilon\ll 1$ be a real number such that $K_{\Ff}+B+\epsilon A+\Mm_X$ is not pseudo-effective$/U$. Since $\lim_{i\rightarrow+\infty}\lambda_i=0$, there exists an integer $m$ such that $\lambda_m<\epsilon$. Thus $K_{\Ff_m}+B_m+\lambda_mA_m+\Mm_{X_m}$ is nef$/U$ but $K_{\Ff_m}+B_m+\epsilon A_m+\Mm_{X_m}$ is not pseudo-effective$/U$, which is not possible. Thus $\mathcal{P}$ terminates. 

Suppose that $\mathcal{P}$ terminates at $(X_m,\Ff_m,B_m,\Mm)$ for some $m\geq 0$. Since $K_{\Ff}+B+\Mm_X$ is not pseudo-effective$/U$, $K_{\Ff_m}+B_m+\Mm_{X_m}$ is not pseudo-effective$/U$. Thus $K_{\Ff_m}+B_m+\Mm_{X_m}$ is not nef$/U$, so there exists a $(K_{\Ff_m}+B_m+\Mm_{X_m})$-Mori fiber space$/U$. The proposition follows.
\end{proof}



\begin{thm}\label{thm: existence mfs}
Let $(X,\Ff,B,\Mm)/U$ be an lc gfq. Assume that $\Ff$ is algebraically integerable and $K_{\Ff}+B+\Mm_X$ is not pseudo-effective$/U$. Then:
\begin{enumerate}
  \item $(X,\Ff,B,\Mm)/U$ has a Mori fiber space.
  \item Suppose that $(X,\Ff,B,\Mm)$ is weak ACSS, and either $X$ is $\Qq$-factorial klt or $\Mm$ is NQC$/U$. Then:
  \begin{enumerate}
    \item We may run a $(K_{\Ff}+B+\Mm_X)$-MMP$/U$ with scaling of an ample$/U$ $\Rr$-divisor, which terminates with a Mori fiber space$/U$.
    \item If $X$ is $\Qq$-factorial, then any $(K_{\Ff}+B+\Mm_X)$-MMP$/U$ with scaling of an ample$/U$ $\Rr$-divisor terminates with a Mori fiber space$/U$.
   \end{enumerate}
   \end{enumerate}
\end{thm}
\begin{proof}
(2) follows from Proposition \ref{prop: run mmp get mfs} so we only need to show (1). 

By Theorem \ref{thm:  ACSS model}, $(X,\Ff,B,\Mm)$ has a $\Qq$-factorial ACSS model $(Y,\Ff_Y,B_Y,\Mm)$. Let $g: Y\rightarrow X$ be the induced birational morphism, then $g$ only extracts divisors $E$ such that $-\epsilon_{\Ff}(E)=a(E,\Ff,B,\Mm)$, and
$$K_{\Ff_Y}+B_Y+\Mm_Y=g^*(K_{\Ff}+B+\Mm_X)$$
is not pseudo-effective$/U$. By Proposition \ref{prop: run mmp get mfs}, we may run a $(K_{\Ff_Y}+B_Y+\Mm_Y)$-MMP$/U$ which terminates with a Mori fiber space $(Y',\Ff_{Y'},B_{Y'},\Mm)\rightarrow T$ of $(Y,\Ff_Y,B_Y,\Mm)$. Then $(Y',\Ff_{Y'},B_{Y'},\Mm)\rightarrow T$ is a Mori fiber space of $(X,\Ff,B,\Mm)/U$.
\end{proof}


\subsection{MMP for very exceptional divisors}\label{subsec: very exceptional}
%[MMP for very exceptional divisors]
\begin{thm}\label{thm: mmp very exceptional alg int fol}
Let $(X,\Ff,B,\Mm)/U$ be a weak ACSS gfq. Let $E_1,E_2\geq 0$ be two $\Rr$-divisors on $X$ such that $E_1\wedge E_2=0$, $E_1$ is very exceptional$/U$, and
$$K_{\Ff}+B+\Mm_X\sim_{\mathbb R,U}\text{(resp. }\equiv_U,\sim_{\mathbb Q,U}\text{) }E_1-E_2.$$ 
Assume that either $X$ is $\Qq$-factorial klt or $\Mm$ is NQC$/U$. Let $A$ be an ample$/U$ $\Rr$-divisor. Then:
\begin{enumerate}
    \item We may run a $(K_{\Ff}+B+\Mm_X)$-MMP$/U$ with scaling  of $A$.
    \item Let $\mathcal{P}$ be the $(K_{\Ff}+B+\Mm_X)$-MMP$/U$ constructed in (1) if $X$ is not $\Qq$-factorial, and let $\mathcal{P}$ be any $(K_{\Ff}+B+\Mm_X)$-MMP$/U$ with scaling of $A$ if $X$ is $\Qq$-factorial. Then:
    \begin{enumerate}
        \item  Either $\mathcal{P}$ terminates with a Mori fiber space, or $\mathcal{P}$ contracts $E_1$ after finitely many steps.
        \item Suppose that $E_2=0$. Then:
        \begin{enumerate}
        \item $\mathcal{P}$ terminates with a weak lc model $(X',\Ff',B',\Mm)/U$ of $(X,\Ff,B,\Mm)/U$. In particular,
        $K_{\Ff'}+B'+\Mm_{X'}\sim_{\mathbb R,U}\text{(resp. }\equiv_U,\sim_{\mathbb Q,U}\text{) }0.$
         \item The divisors contracted by the induced birational map $X\dashrightarrow X'$ are exactly $\Supp E_1$.
        \item If $(X,\Ff,B,\Mm)$ is $\Qq$-factorial ACSS, then $(X',\Ff',B',\Mm)/U$ is a good minimal model of $(X,\Ff,B,\Mm)/U$. 
        \end{enumerate}
    \end{enumerate}
\end{enumerate}
\end{thm}
\begin{proof}
 (1) is a direct corollary of Proposition \ref{prop: run mmp with scaling gfq}. Moreover, by Proposition \ref{prop: run mmp with scaling gfq}, we may suppose that $\mathcal{P}$ is an MMP$/U$ with scaling of $A$
     \begin{center}$\xymatrix{
(X_0,\Ff_0,B_0,\Mm)\ar@{-->}[r] & (X_1,\Ff_1,B_1,\Mm)\ar@{-->}[r] & \dots\ar@{-->}[r] & (X_n,\Ff_n,B_n,\Mm)\ar@{-->}[r] & \dots 
}$
\end{center}
such that either this MMP terminates, or $\lim_{i\rightarrow+\infty}\lambda_i=0$, where
$$\lambda_i:=\inf\{t\geq 0\mid K_{\Ff_i}+B_i+tA_i+\Mm_{X_i}\text{ is nef}/U\}$$
are the scaling numbers and $A_i$ is the strict transform of $A$ on $X_i$.

(2.a) We let $m$ be the integer satisfying the following: if $\mathcal{P}$ terminates, then $(X_m,\Ff_m,B_m,\Mm)$ is the output of $\mathcal{P}$. If we already get a $(K_{\Ff_m}+B_m+\Mm_{X_m})$-Mori fiber space$/U$ then we are done, so we may assume that $K_{\Ff_m}+B_m+\Mm_{X_m}$ is nef$/U$. In particular, $K_{\Ff_m}+B_m+\Mm_{X_m}$ is movable$/U$.

Otherwise, we let $m$ be a positive integer such that $f_i$ is small for any $i\geq m$. Let $\psi_i: X_m\dashrightarrow X_i$ be the induced birational maps. Since $K_{\Ff_i}+B_i+\lambda_iA_i+\Mm_{X_i}$ is nef$/U$ for any $i$,
$$K_{\Ff_m}+B_m+\Mm_{X_m}=\lim_{i\rightarrow+\infty}(\psi_i^{-1})_*(K_{\Ff_i}+B_i+\lambda_iA_i+\Mm_{X_i})$$
is movable$/U$.

Since $K_{\Ff_m}+B_m+\Mm_{X_m}$ is movable$/U$, for any prime divisor $S$ on $X_m$ and any very general curve $C$ on $S$ over $U$, $(K_{\Ff_m}+B_m+\Mm_{X_m})\cdot C\geq 0$. Let $E_{1,m}$ and $E_{2,m}$ be the images of $E_1$ and $E_2$ on $X_m$ respectively. Then $E_{1,m}$ is very exceptional$/U$ and 
$$K_{\Ff_m}+B_m+\Mm_{X_m}\sim_{\mathbb R,U}\text{(resp. }\equiv_U,\sim_{\mathbb Q,U}\text{) }E_{1,m}-E_{2,m}.$$
By \cite[Lemma 3.3]{Bir12}, $E_{1,m}=0$. This implies (2.a).

(2.b) Now we assume that $E_2=0$. Then $K_{\Ff}+B+\Mm_X\equiv_U E_1\geq 0$, so $\mathcal{P}$ does not terminate with a Mori fiber space. By (2.a), $\mathcal{P}$ contracts $E_1$ and achieves a log birational model $(X',\Ff',B',\Mm)/U$ of $(X,\Ff,B,\Mm)$ after finitely many steps. Since the image of $E_1$ on $X'$ is $0$,
$$K_{\Ff_m}+B_m+\Mm_{X_m}\sim_{\mathbb R,U}\text{(resp. }\equiv_U,\sim_{\mathbb Q,U}\text{) }0.$$
In particular, $\mathcal{P}$ terminates at $X'$. Since the induced birational map $X\dashrightarrow X'$ does not extract any divisor,  $(X',\Ff',B',\Mm)/U$ is a weak lc model of $(X,\Ff,B,\Mm)/U$, which implies  (2.b.i). Since $\mathcal{P}$ is also an $E_1$-MMP$/U$, we get (2.b.ii). (2.b.iii) follows from Lemma \ref{lem: ACSS mmp can run}(3.f).
\end{proof}


\section{ACC for lc thresholds and the global ACC}\label{sec: acc gfq}

\subsection{The global ACC}\label{subsec: global acc}

\begin{lem}\label{lem: trivial trace nef imply trivial}
    Let $X$ be a normal projective variety and $\Mm$ a nef $\bb$-divisor on $X$. If $\Mm_X\equiv 0$, then $\Mm\equiv\bm{0}$.
\end{lem}
\begin{proof}
%Suppose that $\Mm_X\equiv 0$. Then $\Mm_X$ is $\Rr$-Cartier. 
Let $f: Y\rightarrow X$ be a birational morphism such that $\Mm$ descends to $Y$. By the negativity lemma, 
$\Mm_Y=f^*\Mm_X-E\equiv -E$ for some $E\geq 0$. Since $\Mm_Y$ is nef, $\Mm_Y$ is pseudo-effective, so $-E$ is pseudo-effective. Thus $E=0$ and $\Mm_Y\equiv 0$, so $\Mm\equiv\bm{0}$.
\end{proof}

\begin{proof}[Proof of Theorem \ref{thm: global acc alg int gfq}]
 By Theorem \ref{thm:  ACSS model}, possibly replacing $\Ii$ with $\Ii\cup\{1\}$ and replacing $(X,\Ff,B,\Mm)$ with an ACSS model, we may assume that there exists a contraction $f: X\rightarrow Z$ such that $(X,\Ff,B,\Mm)/Z$ is $\Qq$-factorial ACSS. Let $F$ be a general fiber of $f$, $B_F:=B|_F$, $\Mm^F:=\Mm|_F$, and $\Mm^F_j:=\Mm_j|_F$ for each $j$. Since $K_F=K_X|_F=K_{\Ff}|_F$,
$$\left(F,B_F,\Mm^F=\sum\gamma_j\Mm_j^F\right)$$
    is an lc g-pair of dimension $r$ such that $K_F+B_F+\Mm^F_F\equiv 0$. Moreover, $B_F\in\Ii$. By \cite[Theorem 1.6]{BZ16}, there exists a finite set $\Ii_1\subset\Ii$ depending only on $r$ and $\Ii$ such that $B_F\in\Ii_1$. Since $(X,\Ff,B,\Mm)$ is lc, $B$ is horizontal$/Z$. Thus $B\in\Ii_1$.

    Possibly rewrite $\Mm$, we may assume that $\Mm_j\not\equiv\bm{0}$ and $\gamma_j>0$ for any $j$. By Lemma \ref{lem: trivial trace nef imply trivial}, $\Mm_{j,X}\not\equiv 0$ for each $j$. For any $j$, we let $\delta_j\in (0,\gamma_j)$ be a real number and run a 
    $$(K_{\Ff}+B+\Mm_X-\delta_j\Mm_{j,X})\text{-MMP}/Z.$$
    Since $\Mm_{j,X}\not\equiv 0$, $K_{\Ff}+B+\Mm_X-\delta_j\Mm_{j,X}\equiv -\delta_j\Mm_{j,X}$ is not pseudo-effective$/Z$. By Theorem \ref{thm: existence mfs}, this MMP terminates with a Mori fiber space $\pi_j: (X_j,\Ff_j,B_j,\Mm-\delta_j\Mm_j)\rightarrow T_j$ of $(X,\Ff,B,\Mm-\delta_j\Mm_j)$.
    
    Since $K_{\Ff}+B+\Mm_X\equiv 0$, $(X,\Ff,B,\Mm)$ and $(X_j,\Ff_j,B_j,\Mm)$ are crepant, so $(X_j,\Ff_j,B_j,\Mm)$ is lc and $K_{\Ff_j}+B_j+\Mm_{X_j}\equiv 0$. Since $K_{\Ff_j}+B_j+\Mm_{X_j}-\delta_j\Mm_{j,X_j}$ is anti-ample$/T_j$, $\Mm_{j,X_j}$ is ample$/T_j$.
    Let $F_j$ be a general fiber of $\pi_j$, $r_j:=\dim F_j$, $B_{F_j}:=B_j|_{F_j}$, $\Mm^j:=\Mm|_{F_j}$, and $\Mm^j_i:=\Mm_i|_{F_j}$. Then $r_j\leq r$. Since $K_{F_j}=K_{X_j}|_{F_j}=K_{\Ff_j}|_{F_j}$,
    $$\left(F_j,B_{F_j},\Mm^j=\sum_i\gamma_i\Mm^j_i\right)$$
    is an lc g-pair of dimension $r_j$, and $B_{F_j}\in\Ii$. Moreover, since $\Mm_{j,X_j}$ is ample$/T_j$, $\Mm^j_{j,X_j}$ is ample. Thus $\Mm^j_j\not\equiv\bm{0}$. By \cite[Theorem 1.6]{BZ16}, there exists a finite set $\Ii_2\subset\Ii$ depending only on $\Ii$ such that $\gamma_j\in\Ii_2$. Since $j$ can be any index, we may take $\Ii_0:=\Ii_1\cup\Ii_2$.
\end{proof}


\subsection{ACC for lc thresholds}\label{subsec: acc}

\begin{lem}\label{lem: find nontrivial divisor on ACSS model}
Let $(X,\Ff,B,\Mm)/X$ be an lc gfq, $D$ an $\Rr$-divisor on $X$, and $\Nn$ a $\bb$-divisor on $X$ satisfying the following.
\begin{itemize}
    \item[(i)] $\Ff$ is algebraically integrable.
    \item[(ii)] $\Supp B=\Supp(B+D)$.
    \item[(iii)] $\Mm+\Nn$ and $\Mm-\delta\Nn$ are nef$/X$ for some $\delta\in(0,1)$.
    \item[(iv)] $(X,\Ff,B+D,\Mm+\Nn)/X$ is an lc gfq. In particular, $D+\Nn_X$ is $\Rr$-Cartier.
    \item[(v)] $(X,\Ff,B+(1+\epsilon)D,\Mm+(1+\epsilon)\Nn)$ is not lc for any positive real number $\epsilon$.
    \item[(vi)]  For any prime divisor $P$ on $X$ with $a(P,\Ff,B+D,\Mm+\Nn)=-\epsilon_{\Ff}(D)$, $\mult_PD=0$.
\end{itemize}
Then for any real number $t\in (0,1)$, there exist two projective birational morphisms $h: X'\rightarrow X$ and $g: Y'\rightarrow X'$ satisfying the following. 
\begin{enumerate}
    \item $h$ is an ACSS modification of $(X,\Ff,B+tD,\Mm+t\Nn)$.
    \item For any prime $h$-exceptional divisor $P$, $a(P,\Ff,B,\Mm)=-\epsilon_{\Ff}(P)$. In particular, $$a(D,\Ff,B+sD,\Mm+s\Nn)=-\epsilon_{\Ff}(D)$$ 
    for any real number $s$.
    \item $g$ extracts a unique prime divisor $E$. In particular, $-E$ is ample over $X'$.
    \item $a(E,\Ff,B+D,\Mm+\Nn)=-\epsilon_{\Ff}(E)$ and $a(E,\Ff,B,\Mm)>-\epsilon_{\Ff}(E)$. In particular,  $$a(E,\Ff,B+sD,\Mm+s\Nn)>-\epsilon_{\Ff}(E)$$ for any real number $s<1$.
    \item Let $B_{Y'},D_{Y'}$ be the strict transforms of $B,D$ on $Y'$ respectively, $\Ff_{Y'}:=(h\circ g)^{-1}\Ff$, and $F_{Y'}:=(\Supp\Exc(h\circ g))^{\Ff_{Y'}}.$ Then 
    $$(Y',\Ff_{Y'},B_{Y'}+tD_{Y'}+F_{Y'};\Mm+t\Nn)$$ 
    is $\Qq$-factorial ACSS.
\end{enumerate}
\begin{center}$\xymatrix{
Y\ar@{-->}[r]\ar@{->}_f[d] & Y'\ar@{->}^g[d] \\
X & X'\ar@{->}[l]^{h}.
}$
\end{center}
\end{lem}
\begin{proof}
By condition (v), there exists a prime divisor $P_0$ over $X$ such that 
$$a(P_0,\Ff,B+D,\Mm+\Nn)=-\epsilon_{\Ff}(P_0)$$
and
$$a(P_0,\Ff,B+\alpha D,\Mm+\alpha\Nn)<-\epsilon_{\Ff}(P_0)$$
for any $\alpha>1$. In particular, 
$$a(P_0,\Ff,B+tD,\Mm+t\Nn)>-\epsilon_{\Ff}(P_0).$$

By condition (vi), $P_0$ is exceptional$/X$. By Theorem \ref{thm: property * induction}, there exists a proper ACSS model $(Y,\Ff_Y,\tilde B_Y,\Mm+\Nn;G_Y)/Z$ of $(X,\Ff,B+D,\Mm+\Nn)$ such that $P_0$ is on $Y$. Let $f: Y\rightarrow X$ be the induced birational morphism, $B_Y,D_Y$ the strict transforms of $B,D$ on $Y$ respectively, and $F_{Y}:=(\Supp\Exc(f))^{\Ff_{Y}}$. Then $\tilde B_Y=B_Y+D_Y+F_Y$. 

By conditions (ii) and (iii) and Lemma \ref{lem: acss smaller coefficient},
$$(Y,\Ff_Y,B_Y+tD_Y+F_Y,\Mm+t\Nn;G_Y)/Z$$
is ACSS. Let $E_1,\dots,E_n$ be the prime $f$-exceptional divisors, then
$$K_{\Ff_Y}+B_Y+tD_Y+F_Y+\Mm_Y+t\Nn_Y\sim_{\mathbb R,X}\sum_{i=1}^n\left(\epsilon_{\Ff}(E_i)+a(E_i,\Ff,B+tD,\Mm+t\Nn)\right)E_i\geq 0.$$
By Theorem \ref{thm: mmp very exceptional alg int fol}, we may run a $(K_{\Ff_Y}+B_Y+tD_Y+\Mm_Y+t\Nn_Y+F_Y)$-MMP$/X$ which terminates with a good minimal model $(X',\Ff',B'+tD'+F',\Mm+t\Nn)/X$ of $(Y,\Ff_Y,B_Y+tD_Y+F_Y,\Mm+t\Nn)$, such that
$$K_{\Ff'}+B'+tD'+F'+\Mm_{X'}+t\Nn_{X'}\sim_{\mathbb R,X}0,$$ 
where $B',D',F'$ are the strict transforms of $B_Y,D_Y,F_Y$ on $X'$ respectively. Let $G'$ be the image of $G_Y$ on $X'$. By Lemma \ref{lem: ACSS mmp can run},  $(X',\Ff',B'+tD'+F',\Mm+t\Nn;G')/Z$ is $\Qq$-factorial ACSS. In particular, the induced morphism $h: X'\rightarrow X$ is an ACSS modification of $(X,\Ff,B+tD,\Mm+t\Nn)$.

By construction, the divisors contracted by the induced birational map $Y\dashrightarrow X'$ are all divisors $E_i$ satisfying the inequality
$$a(E_i,\Ff,B_Y+tD_Y,\Mm+t\Nn)>-\epsilon_{\Ff}(E_i).$$ 
Thus $Y\dashrightarrow X'$ contracts $P_0$. Therefore, $Y\dashrightarrow X'$ contains a divisorial contraction, so it is not the identity morphism. We let $g: Y'\dashrightarrow X'$ be the last step of this MMP. Since $X'$ is $\Qq$-factorial and $$K_{\Ff'}+B'+tD'+F'+\Mm_{X'}+t\Nn_{X'}\sim_{\mathbb R,X}0,$$ 
$g$ is a divisorial contraction of a prime divisor $E$. 

We show that $h,$ $g$, and $t$ satisfy our requirements. (1) and (5) immediately follow from our construction. (3) follows from our construction and the negativity lemma. %By Lemma \ref{lem: ACSS mmp can run}, any model constructed in the$$(K_{\Ff_Y}+B_Y+tD_Y+\Mm_Y+t\Nn_Y+F_Y)\text{-MMP}/X$$ is $\Qq$-factorial ACSS. In particular, $(Y',\Ff_{Y'},B_{Y'}+tD_{Y'}+F_{Y'},\Mm+t\Nn)$ is $\Qq$-factorial ACSS. This implies (5).
For any prime divisor $Q$ on $X'$ that is exceptional over $X$, 
\begin{align*}
   a(Q,\Ff,B+D,\Mm+\Nn)&=-\epsilon_{\Ff}(Q) && \text{($D$ is also on $Y$ and is exceptional$/X$)}\\
                    &=a(Q,\Ff,B+tD,\Mm+t\Nn) &&  \text{($D$ is on $X'$ and is exceptional$/X$)},
\end{align*}
so $a(Q,\Ff,B+sD,\Mm+s\Nn)=-\epsilon_{\Ff}(Q)$ for any real number $s$. This implies (2). 

Since $g$ is a divisorial contraction of the prime divisor $E$,
\begin{align*}
a(E,\Ff,B+tD,\Mm+t\Nn)&=a(E,\Ff',B'+tD'+F',\Mm+t\Nn)\\
&>a(E,\Ff_Y,B_Y+tD_Y+F_Y,\Mm+t\Nn)=-\epsilon_{\Ff}(E),
\end{align*}
so $a(E,\Ff,B,\Mm)>-\epsilon_{\Ff}(E)$. This implies (4) and completes the proof. 
\end{proof}



\begin{proof}[Proof of Theorem \ref{thm: acc lct alg int gfq}]
Suppose that the theorem does not hold. Then there exists a sequence of NQC lc gfq $(X_i,\Ff_i,B_i,\Mm_i)$, $\Rr$-Cartier $\Rr$-divisors $D_i$ on $X_i$, and $\bb$-divisors $\Nn_i$ on $X_i$, such that $\rk\Ff_i=r$, $B_i,D_i\in\Ii$, $\Mm_i,\Nn_i$ are $\Ii$-linear combination of $\bb$-nef$/X$ $\bb$-divisors, and
$$t_i:=\lct(X_i,\Ff_i,B_i,\Mm_i;D_i,\Nn_i)$$ is strictly increasing. By Lemma \ref{lem: find nontrivial divisor on ACSS model}, possibly replacing $\Ii$ with $\Ii\cup\{1\}$, we may assume that
\begin{itemize}
    \item[(i)] $(X_i,\Ff_i,B_i+t_i'D_i,\Mm_i+t_i'\Nn_i)$ is $\Qq$-factorial ACSS for some $0<t_i'<t_i$,
    \item[(ii)] there exists a divisorial contraction $f_i: Y_i\rightarrow X_i$ of a prime divisor $E_i$, such that $-E_i$ is ample$/X_i$, $$a(E_i,\Ff_i,B_i+t_iD_i,\Mm_i+t_i\Nn_i)=-\epsilon_{\Ff_i}(E_i)$$ 
    and
    $$a(E_i,\Ff_i,B_i+sD_i,\Mm_i+s\Nn_i)\not=-\epsilon_{\Ff_i}(E_i)$$ 
    for any $s\not=t_i$, and
    \item[(iii)] let $B_{Y_i},D_{Y_i}$ be the strict transforms of $B_i,D_i$ on $Y_i$ respectively, $\Ff_{Y_i}:=f_i^{-1}\Ff_i$, and $F_i:=(\Supp\Exc(f_i))^{\Ff_i}$. Then
$$(Y_i,\Ff_{Y_i},B_{Y_i}+t_i'D_{Y_i}+F_i,\Mm_i+t_i'\Nn_i)$$
    is $\Qq$-factorial ACSS.
\end{itemize}
We let $E_i^\nu$ be the normalization of $E_i$, $\Ff_{E_i}$ the restricted foliation of $\Ff_{Y_i}$ on $E_i$, $\Mm^{E}_i:=\Mm_i|_{E_i}$, and $\Nn^E_{i}:=\Nn_i|_{E_i}$
For any real number $t$, we let $\Mm^E(t)_i:=\Mm^E_i+t\Nn^E_i$, and
$$K_{\Ff_{E_i}}+B_{E_i}(t)+\Mm^E(t)_{i,E_i^\nu}:=(K_{\Ff_{Y_i}}+B_{Y_i}+tD_{Y_i}+F_i+\Mm_{i,Y_i}+t\Nn_{i,Y_i})|_{E_i^\nu}.$$
Let $V_i$ be the center of $E_i$ on $X_i$. Then there exists an induced birational morphism $\phi_i: E_i^\nu\rightarrow V_i$ such that
$$K_{\Ff_{E_i}}+B_{E_i}(t_i)+\Mm^E(t_i)_{i,E_i^\nu}\sim_{\mathbb R,V_i}0.$$
Since $-E_i$ is ample$/X_i$,
$$K_{\Ff_{E_i}}+B_{E_i}(t_i')+\Mm^E(t_i')_{i,E_i^\nu}$$
is anti-ample$/V_i$.

By Proposition \ref{prop: a.i preserved adjunction}, $\Ff_{E_i}$ is algebraically integrable. By Theorem \ref{thm: precise adj gfq}, 
$$(E_i^\nu,\Ff_{E_i},B_{E_i}(t_i),\Mm^E(t_i)_i)/V_i$$
is lc, and
$$(E_i^\nu,\Ff_{E_i},B_{E_i}(t),\Mm^E(t)_i)/V_i$$
is lc for any $0\leq t\leq t_i$. By Theorem \ref{thm:  ACSS model}, we may let 
$(W_i,\Ff_{W_i},B_{W_i}(t_i),\Mm^E(t_i)_i;G_i)/Z_i$
be an ACSS model of 
$$(E_i^\nu,\Ff_{E_i},B_{E_i}(t_i),\Mm^E(t_i)_i)$$ with induced birational morphism $g_i: W_i\rightarrow E_i^\nu$, and let 
$$B_{W_i}(t):=(g_i^{-1})_*B_{E_i}(t)+(\Supp\Exc(g_i))^{\Ff_{W_i}}$$
for each $i$. Since $$K_{\Ff_{E_i}}+B_{E_i}(t_i')+\Mm^E(t_i')_{i,E_i^\nu}$$
is anti-ample$/V_i$,
$K_{\Ff_{W_i}}+B_{W_i}(t_i')+\Mm^E(t_i')_{i,W_i}$ is not pseudo-effective$/V_i$. Moreover, since $t_i>t_i'$, $B_{E_i}(t_i)\geq B_E(t_i')$, so $(W_i,\Ff_{W_i},B_{W_i}(t_i'),\Mm^E(t_i')_i)$ is lc. Thus we may run a 
$$(K_{\Ff_{W_i}}+B_{W_i}(t_i')+\Mm^E(t_i')_{i,W_i})\text{-MMP}/V_i$$
with scaling of an ample$/V_i$ divisor. By Theorem \ref{thm: existence mfs}, this MMP terminates with a Mori fiber space $\psi_i: (\bar W_i,\Ff_{\bar W_i},B_{\bar W_i}(t_i'),\Mm^E(t_i')_{i})\rightarrow T_i$ of $(W_i,\Ff_{W_i},B_{W_i}(t_i'),\Mm^E(t_i)_i)/V_i$. 
\begin{center}$\xymatrix{
W_i\ar@{-->}[rr]\ar@{->}[d]_{g_i} &  & \bar W_i\ar@{->}[d]^{\psi_i}\\
E_i^\nu\ar@{->}[rd]^{\phi_i} & & T_i\ar@{->}[ld]\\
& V_i &}$
\end{center}
By Lemma \ref{lem: ACSS mmp can run}, $\psi_i$ is also a Mori fiber space$/Z_i$. Since 
$$K_{\Ff_{W_i}}+B_{W_i}(t_i)+\Mm^E(t_i)_{i,W_i}\sim_{\mathbb R,V_i}0,$$
$(\bar W_i,\Ff_{\bar W_i},B_{\bar W_i}(t_i),\Mm^E(t_i)_{i})$ and $(W_i,\Ff_{W_i},B_{W_i}(t_i),\Mm^E(t_i)_i)$ are crepant, where $B_{\bar W_i}(t)$ is the image of $B_{W_i}(t)$ on $\bar W_i$ for any $t$. Then
$$K_{\Ff_{\bar W_i}}+B_{\bar W_i}(t_i)+\Mm^E(t_i)_{i,\bar W_i}\sim_{\mathbb R,V_i}0,$$
so
$$K_{\Ff_{\bar W_i}}+B_{\bar W_i}(t_i)+\Mm^E(t_i)_{i,\bar W_i}\sim_{\mathbb R,T_i}0.$$
Let $L_i$ be a general fiber of $\psi_i$, $B_{L_i}(t):=B_{\bar W_i}(t)|_{L_i}$ for any $t$, and $\Mm^L(t)_i:=\Mm^E(t)_i|_{L_i}$ for any $t$. Then $K_{\Ff_{\bar W_i}}|_{L_i}=K_{L_i}$, $(L_i,B_{L_i}(t_i),\Mm^L(t_i)_i)$ is lc,
$$K_{L_i}+B_{L_i}(t_i)+\Mm^L(t_i)_{i,L_i}\equiv 0,$$
and
$$K_{L_i}+B_{L_i}(t_i')+\Mm^L(t_i')_{i,L_i}$$
is anti-ample. Moreover, since $\psi_i$ is a Mori fiber space$/Z_i$, by Proposition \ref{prop: a.i preserved adjunction}, 
$$\dim L_i\leq\rk\Ff_{\bar W_i}=\rk\Ff_{E_i}\leq\rk\Ff_i=r.$$




We get a contradiction to \cite[Theorem 1.6]{BZ16} by considering the coefficients of $B_{L_i}(t_i)$ and $\Mm^L(t_i)_{i,L_i}$, which can be precisely computed by Theorem \ref{thm: precise adj gfq}. Theorem \ref{thm: acc lct alg int gfq} is proven.
\end{proof}

\subsection{Uniform rational polytopes}\label{subsec: urp}

\begin{defn}\label{defn: r affine functional divisor}
Let $X$ be a normal variety, $D_i$ $\Rr$-divisors on $X$, $\Mm_i$ $\bb$-divisors  on $X$, and $d_i(t):\mathbb R\rightarrow\mathbb R$ $\Rr$-affine functions. Then we call the formal finite sum $\sum d_i(t)D_i$ an \emph{$\Rr$-affine functional divisor}, and call the formal finite sum $\sum d_i(t)\Mm_i$ an \emph{$\Rr$-affine $\bb$-divisor}.
\end{defn}

\begin{defn}
Let $c$ be a non-negative real number, and $\Ii\subset[0,+\infty)$ a set of real numbers. Let $X$ be a normal variety.

For any $\Rr$-affine functional divisor $\Delta(t)$ on $X$, we write $\Delta(t)\in\mathcal{D}_c(\Ii)$ if we may write $\Delta(t)=\sum_id_i(t)D_i$, where $D_i$ are distinct prime divisors, and the following condition is satisfied: For any $i$, either $d_i(t)=1$, or 
$$d_i(t)=\frac{m-1+\gamma+kt}{m},$$
where $m\in\mathbb{N}^{+}$, $\gamma\in\Ii_+$, $k\in\mathbb{Z}$, and $f+kt=\sum_{j}(f_j+k_jt)$, where $f_j\in\Ii\cup\{0\}$, $k_j\in\mathbb{Z}$, and $f_j+k_jc\ge0$  for any $j$.


For any $\Rr$-affine functional $\bb$-divisor $\Mm(t)$ on $X$ and any projective morphism $X\rightarrow Z$, we write $\Mm(t)\in\mathcal{D}_c(\Ii/Z)$ if we can write $\Mm(t)=\sum_i\mu_i(t)\Mm_i$, where $\Mm_i$ are nef$/Z$ $\bb$-Cartier $\bb$-divisors, and the following condition is satisfied: For any $i$, either $\mu_i(t)=1$, or
$$\mu_i(t)=v+nt=\sum_j(v_j+n_jt),$$
where $v_j\in\Ii$, $n_j\in\mathbb Z$, and $v_j+n_jc\geq 0$ for any $j$. Moreover, if $Z=\{pt\}$, then we may omit $Z$ and write $\Mm(t)\in\mathcal{D}_c(\Ii)$.
\end{defn}

\begin{defn}
Let $d$ be a positive integer and  $\Ii\subset[0,+\infty)$ a set of real numbers. We define $\mathcal{B}_{d}(\Ii),\mathcal{B}'_{d}(\Ii)\subset [0,+\infty)$ as follows: $c\in\mathcal{B}_{d}(\Ii)$ (resp. $\mathcal{B}'_{d}(\Ii)$) if and only if there exist a normal projective variety $X$ (resp. a $\Qq$-factorial normal projective variety $X$), an $\Rr$-affine functional divisor $\Delta(t)$ on $X$, and an $\Rr$-affine functional $\bb$-divisor $\Mm(t)$ satisfying the following.
\begin{enumerate}
	\item $\dim X\le d$, 
	\item $\Delta(t)\in\mathcal{D}_{c}(\Ii)$, $\Mm(t)\in\mathcal{D}_c(\Ii)$,
	\item $(X,\Delta(c),\Mm(c))$ is lc,
	\item $K_X+\Delta(c)+\Mm(c)_X\equiv0$, and
	\item $K_X+\Delta(c')+\Mm(c')_X\not\equiv 0$ for any $c'\neq c$.
\end{enumerate}
\end{defn}

\begin{defn}
Let $r$ be a positive integer and  $\Ii\subset[0,+\infty)$ a set of real numbers. We define $\mathcal{C}_{r}(\Ii),\mathcal{C}'_{r}(\Ii)\subset [0,+\infty)$ as follows: $c\in\mathcal{C}_{r}(\Ii)$ (resp. $c\in\mathcal{C}_{r}'(\Ii)$) if and only if there exist a normal projective variety $X$ (resp. a $\Qq$-factorial normal projective variety $X$), an algebraically integrable foliation $\Ff$ on $X$, an $\Rr$-affine functional divisor $\Delta(t)$ on $X$, and an $\Rr$-affine functional $\bb$-divisor $\Mm(t)$ on $X$ satisfying the following.
\begin{enumerate}
	\item $\rk\Ff\le r$, 
	\item $\Delta(t)\in\mathcal{D}_{c}(\Ii)$, $\Mm(t)\in\mathcal{D}_c(\Ii)$,
	\item $(X,\Ff,\Delta(c),\Mm(c))$ is lc,
	\item $K_\Ff+\Delta(c)+\Mm(c)_X\equiv0$, and
	\item $K_\Ff+\Delta(c')+\Mm(c')_X\not\equiv 0$ for any $c'\not=c$.
\end{enumerate}
\end{defn}

\begin{prop}\label{prop: nak special set equal for foliation}
Let $r$ be a positive integer and $\Ii\subset[0,+\infty)$ a set of real numbers. Then $\mathcal{B}_r(\Ii)=\mathcal{C}_r(\Ii)=\mathcal{B}'_r(\Ii)=\mathcal{C}'_r(\Ii)$.
\end{prop}

\begin{proof}
By considering the foliation $\Ff=T_X$, we have $\mathcal{B}_r(\Ii)\subset\mathcal{C}_r(\Ii)$. By the existence of dlt modifications, $\mathcal{B}_r(\Ii)=\mathcal{B}_r'(\Ii)$. By Theorem \ref{thm:  ACSS model}, $\mathcal{C}_r(\Ii)=\mathcal{C}_r'(\Ii)$. We only need show that $\mathcal{C}_r(\Ii)\subset\mathcal{B}_r(\Ii)$.

Pick $c\in\mathcal{C}_r(\Ii)$. The there exists a $\Qq$-factorial normal projective variety $X$, an algebraically integrable foliation $\Ff$ on $X$, an $\Rr$-affine functional divisor $\Delta(t)$ on $X$, and an $\Rr$-affine functional $\bb$-divisor $\Mm(t)$ on $X$, such that
\begin{enumerate}
	\item $\rk\Ff\le r$, 
	\item $\Delta(t)\in\mathcal{D}_{c}(\Ii),\Mm(t)\in\mathcal{D}_c(\Ii)$,
	\item $(X,\Ff,\Delta(c),\Mm(c))$ is lc,
	\item $K_{\Ff}+\Delta(c)+\Mm(c)_X\equiv0$, and
	\item $K_{\Ff}+\Delta(t)+\Mm(t)_X\not\equiv 0$ for any $t\not=c$.
\end{enumerate}
By Theorem \ref{thm:  ACSS model}, we may let $f: X'\rightarrow X$ be an ACSS modification of $(X,\Ff,\Delta(c),\Mm(c))$, $\Ff':=f^{-1}\Ff$, $E:=(\Supp\Exc(f))^{\Ff'}$, and $\Delta'(t):=f^{-1}_*\Delta(t)+E$ for any real number $t$. Then $\rk\Ff'\leq r$, $\Delta'(t)\in\mathcal{D}_c(\Ii)$, $(X',\Ff',\Delta'(c),\Mm(c))$ is lc, and $K_{\Ff'}+\Delta'(c)+\Mm(c)_{X'}\equiv 0$. Moreover, for any $t\not=c$, since $$0\not\equiv K_{\Ff}+\Delta(t)+\Mm(t)_X=f_*(K_{\Ff'}+\Delta'(t)+\Mm(t)_{X'}),$$ $K_{\Ff'}+\Delta'(t)+\Mm(t)_{X'}\not\equiv 0$. Therefore, we may replace $(X,\Ff,\Delta(t),\Mm(t))$ with $(X',\Ff',\Delta'(t),\Mm(t))$, and assume that $(X,\Ff,\Delta(c),\Mm(c))$ is $\Qq$-factorial ACSS. Thus there exists a contraction $f: X\rightarrow Z$ and a reduced divisor $G$ such that $(X,\Ff,\Delta(c),\Mm(c);G)/Z$ is ACSS.

Suppose that for any $0<\delta\ll 1$, $(X,\Ff,\Delta(c+\delta),\Mm(c+\delta);G)/Z$ or $(X,\Ff,\Delta(c-\delta),\Mm(c+\delta);G)/Z$ is not ACSS. By Lemmas \ref{lem: acss smaller coefficient} and \ref{lem: acss f-triple perturb coefficient},
\begin{itemize}
    \item either there exists a component $D$ of $\Delta(c)$, such that $\mult_D\Delta(c)=1$ and $\mult_D\Delta(t)\not=1$ for any $t\not=c$, or
    \item $\Mm(t)=\sum \mu_i(t)\Mm_i$, where each $\Mm_i$ is $\bb$-nef, and $\mu_i(t)=v_{i}+n_{i}t=\sum_i(v_{i,j}+n_{i,j}t)$ for any $v_{i,j}\in\Ii$, $n_{i,j}\in\mathbb Z$, $v_{i}+n_{i}c\geq 0$ for any $i$, and $v_i+n_ic=0$ for some $i$.
\end{itemize}
 By \cite[Lemma 3.7]{Nak16}, $c\in\mathcal{B}_1(\Ii)\subset\mathcal{B}_r(\Ii)$. Therefore, we may assume that $(X,\Ff,\Delta(c+\delta),\Mm(c+\delta);G)/Z$ and $(X,\Ff,\Delta(c-\delta),\Mm(c-\delta);G)/Z$ are ACSS for any $0<\delta\ll 1$. 

Fix $0<\delta\ll 1$. Since $K_{\Ff}+\Delta(t)+\Mm(t)_X\not\equiv 0$ for any $t\not=c$ and $K_{\Ff}+\Delta(c)+\Mm(c)_X\equiv 0$, either $K_{\Ff}+\Delta(c+\delta)+\Mm(c+\delta)_X$ or $K_{\Ff}+\Delta(c-\delta)+\Mm(c-\delta)_X$ is not pseudo-effective. By Theorem \ref{thm: existence mfs}, we may run a $(K_{\Ff}+\Delta(c+\delta)+\Mm(c+\delta)_X)$-MMP (resp. $(K_{\Ff}+\Delta(c-\delta)+\Mm(c-\delta)_X)$-MMP) with scaling of an ample divisor if $K_{\Ff}+\Delta(c+\delta)+\Mm(c+\delta)_X$ (resp. $K_{\Ff}+\Delta(c-\delta))+\Mm(c-\delta)_X$) is not pseudo-effective, which terminates with a Mori fiber space $\phi: (X'',\Ff'',\Delta''(c+\delta),\Mm(c+\delta))\rightarrow T$ (reps. $\phi: (X'',\Ff'',\Delta''(c-\delta),\Mm(c-\delta))\rightarrow T$) of $(X,\Ff,\Delta(c+\delta),\Mm(c+\delta))$ (resp. $(X,\Ff,\Delta(c-\delta),\Mm(c-\delta))$), where $\Delta''(t)$ is the image of $\Delta(t)$ on $X''$ for any $t$. By Lemma \ref{lem: ACSS mmp can run}(4), this MMP is also an MMP$/Z$ and $\phi$ is a contraction$/Z$.

Since $K_{\Ff}+\Delta(c)+\Mm(c)_X\equiv 0$, $(X'',\Ff'',\Delta''(c),\Mm(c))$ and $(X,\Ff,\Delta(c),\Mm(c))$ are crepant, so $K_{\Ff''}+\Delta''(c)+\Mm(c)_{X''}\equiv 0$ and $(X'',\Ff'',\Delta''(c),\Mm(c))$ is lc.

Let $F$ be a general fiber of $\phi$. By Theorem \ref{thm: cone theorem gfq}, $F$ is tangent to $\Ff''$, so $K_{\Ff''}|_F=K_{X''}|_F=K_F$. Let $\Delta_{F}(t):=\Delta''(t)|_F$ and $\Mm^F(t):=\Mm(t)|_F$. Then
\begin{itemize}
   \item $\dim F\leq \dim X-\dim Z=\rk\Ff\leq r$,
    \item $\Delta_F(t)\in\mathcal{D}_c(\Ii)$ and $\Mm^F(t)\in\mathcal{D}_c(\Ii)$,
    \item $(F,\Delta_F(c),\Mm^F(c))$ is lc,
    \item $K_F+\Delta_F(c)+\Mm^F(c)_F\equiv 0$, and
    \item $K_F+\Delta_F(c+\delta)+\Mm^F(c+\delta)_F$ or $K_F+\Delta_F(c-\delta)+\Mm^F(c-\delta)_F$ is anti-ample.
\end{itemize}
Thus $c\in\mathcal{B}_r(\Ii)$.
\end{proof}



\begin{thm}\label{thm: uniform rational polytope foliation one variable}
Let $d,c,m,n$ be positive integers, $r_1,\dots,r_c$ real numbers such that $1,r_1,\dots,r_c$ are linearly independent over $\mathbb Q$, $\bm{r}:=(r_1,\dots,r_c)$, and $s_1,\dots,s_m,\mu_1,\dots,\mu_n: \mathbb R^{c+1}\rightarrow\mathbb R$ $\mathbb Q$-linear functions. Then there exists a positive real number $\delta$ depending only on $d,\bm{r}$ and $s_1,\dots,s_m,\mu_1,\dots,\mu_n$ satisfying the following. Assume that
\begin{enumerate}
    \item 
    $$\left(X,\Ff,B=\sum_{i=1}^ms_i(1,r_1,\dots,r_{c-1},t)B_i,\Mm=\sum_{i=1}^n\mu_i(1,r_1,\dots,r_{c-1},t)\Mm_i\right)\Bigg/X$$ is an lc gfq such that $\Ff$ is algebraically integrable and $\rk\Ff\leq d$,
    \item $B_i\geq 0$ are distinct Weil divisors (possibly $0$) and $s_i(1,\bm{r})\geq 0$ for each $i$,
    \item $\Mm_i$ are nef$/X$ $\bb$-Cartier $\bb$-divisors and $\mu_i(1,\bm{r})\geq 0$ for each $i$, and
    \item $B(t):=\sum_{i=1}^ms_i(1,r_1,\dots,r_{c-1},t)B_i$ and  $\Mm(t):=\sum_{i=1}^n\mu_i(1,r_1,\dots,r_{c-1},t)\Mm_i$ for any $t\in\mathbb R$.
\end{enumerate}
Then $(X,\Ff,B(t),\Mm(t))$ is lc for any $t\in (r_c-\delta,r_c+\delta)$.
\end{thm}
\begin{proof}
We let $s_i(t):=s_i(1,r_1,\dots,r_{c-1},t)$ and $\mu_i(t):=\mu_i(1,r_1,\dots,r_{c-1},t)$ for any $t\in\mathbb R$. If $s_i(r_c)=0$, then $s_i(t)=0$ for any $i$, so we may assume that $s_i(r_c)\not=0$ for any $i$. Let $(X',\Ff',B'(r_c),\Mm(r_c))$ be an ACSS model of $(X,\Ff,B(r_c),\Mm(r_c))$, $f: X'\rightarrow X$ the induced birational morphism, $E:=(\Supp(\Exc(f)))^{\Ff'}$, and $B'(t):=f^{-1}_*B(t)+E$ for any $t$. Then
$$K_{\Ff'}+B'(r_c)+\Mm(r_c)_{X'}=f^*(K_{\Ff}+B(r_c)+\Mm(r_c)_{X}).$$
Since $1,r_1,\dots,r_c$ are linearly independent over $\mathbb Q$, 
$B'(t):=f^{-1}_*B(t)+E$ for any $t$, and
$$K_{\Ff'}+B'(t)+\Mm(t)_{X'}=f^*(K_{\Ff}+B(t)+\Mm(t)_{X})$$
for any $t\in\mathbb R$. Thus possibly replacing $(X,\Ff,B(t),\Mm(t))$ with $(X',\Ff',B'(t),\Mm(t))$, we may assume that $(X,\Ff,B(r_c),\Mm(r_c))$ is $\Qq$-factorial ACSS.

Let $$t_1:=\inf\{t\geq r_c\mid (X,\Ff,B(r_c),\Mm(r_c))\text{ is lc}\}$$
and 
$$t_2:=\sup\{t\leq r_c\mid (X,\Ff,B(r_c),\Mm(r_c))\text{ is lc}\}.$$
If $|t_1-t_0|\leq |t_2-t_0|$ then we let $t_0:=t_1$. Otherwise, we let $t_0:=t_2$. We only need to show that there exists a positive real number $\epsilon$ depending only on $d,\bm{r}$, and $s_1,\dots,s_m,\mu_1,\dots,\mu_n$, such that $|t_0-r_c|\geq\epsilon$. 

Since $1,r_1,\dots,r_c$ are linearly independent over $\mathbb Q$, there exists a positive real number $\delta_1$ depending only on $\bm{r}$ and $s_1,\dots,s_m,\mu_1,\dots,\mu_n$, such that $s_i(t)>0$ and $\mu_i(t)>0$ for any $t\in (r_c-\delta_1,r_c+\delta_1)$. We may assume that $|t_0-r_c|<\delta_1$. In particular, for any $0<\delta\ll 1$, $B(t_0+\delta(t_0-r_c))\geq 0$, and $\mu_i(t_0+\delta(t_0-r_c))>0$. Thus $(X,\Ff,B(t_0),\Mm(t_0))$ has an lc center $V_0$ such that $\dim V_0\leq \dim X-2$, and $V_0$ is not an lc center of $(X,\Ff,B(r_c),\Mm(r_c))$. 

By Lemma \ref{lem: find nontrivial divisor on ACSS model}, possibly replacing $(X,\Ff,B(t),\Mm(t))$, we may assume that there exist a divisorial contraction $g: Y\rightarrow X$ of a prime divisor $\tilde E$ and a real number $s$ satisfying the following: let $B_Y(t)$ be the strict transform of $B(t)$ on $Y$ for any $t$ and $\Ff_Y:=g^{-1}\Ff$, then
\begin{itemize}
\item[(i)] $s\in (r_c,t_0)$ if $r_c>t_0$, and $s\in (t_0,r_c)$ if $t_0<r_c$,
\item[(ii)] $(X,\Ff,B(s),\Mm(s))$ is $\Qq$-factorial ACSS, $(X,\Ff,B(r_c),\Mm(r_c))$ is lc, and  $(X,\Ff,B(t_0),\Mm(t_0))$ is lc,
\item[(iii)] $-\tilde E$ is ample over $X$,
\item[(iv)] $(Y,\Ff_Y,B_Y(s)+\epsilon_{\Ff}(\tilde E),\Mm(s))$ is $\Qq$-factorial ACSS, and
\item[(v)] $a(E,\Ff,B(t_0),\Mm(t_0))=-\epsilon_{\Ff}(\tilde E)$ and $a(E,\Ff,B(r_c),\Mm(r_c))>-\epsilon_{\Ff}(\tilde E)$. In particular, $(Y,\Ff_Y,B_Y(t_0)+\epsilon_{\Ff}(\tilde E),\Mm(t_0))$ is lc and $a(E,\Ff,B(s),\Mm(s))>-\epsilon_{\Ff}(\tilde E)$.
\end{itemize}
We let $E$ be the normalization of $\tilde E$, $\Ff_E$ the restricted foliation of $\Ff_Y$ on $E$, $V:=\Center_X\tilde E$,
$\Mm^E(t):=\Mm(t)|_E$, and
$$K_{\Ff_E}+B_E(t)+\Mm^E(t)_E:=(K_{\Ff_Y}+B_Y(t)+\epsilon_{\Ff}(\tilde E)+\Mm(t)_Y)|_E$$
for any real number $t$. By Theorem \ref{thm: precise adj gfq}, $B_E(t)$ is an $\Rr$-affine functional divisor, $\Mm^E(t)$ is an $\Rr$-affine functional $\bb$-divisor, and
$$(E,\Ff_E,B_E(t_0),\Mm^E(t_0)),(E,\Ff_E,B_E(s),\Mm^E(s))$$
are lc gfqs. By Proposition \ref{prop: a.i preserved adjunction}, $\Ff_E$ is algebraically integrable and $\rk\Ff\leq d$. 

Let $\phi: E\rightarrow V$ be the induced projective surjective morphism. Since $-\tilde E$ is ample$/X$,
$$K_{\Ff_E}+B_E(t_0)+\Mm^E(t_0)_E\sim_{\mathbb R,V}0$$
and
$$K_{\Ff_E}+B_E(s)+\Mm^E(s)_E$$
is anti-ample$/V$. Thus $$K_{\Ff_E}+B_E(t)+\Mm^E(t)_E$$ is anti-ample$/V$ for any $t\in (t_0,s)$ if $t_0<r_c$, and for any $t\in (s,t_0)$ if $t_0>r_c$.

By Theorem \ref{thm:  ACSS model}, we may let $$(W,\Ff_W,B_W(t_0),\Mm^E(t_0);G)/Z$$ be an ACSS model of $(E,\Ff_E,B_E(t_0),\Mm^E(t_0))$ with induced birational morphism $g: W\rightarrow E$. Let $F_W:=(\Supp\Exc(g))^{\Ff_W}$ and let $B_E(t):=g_*^{-1}B_E(t_0)+F_W$ for any $t\in\mathbb R$. Since $s\in (r_c-\delta_1,r_c+\delta_1)$, $s_i(s)>0$ and $\mu_i(s)>0$. By Theorem \ref{thm: precise adj gfq} and Lemma \ref{lem: acss f-triple perturb coefficient}, there exists a real number $u$, such that $u\in (t_0,s)$ if $t_0<r_c$, $u\in (s,t_0)$ if $t_0>r_c$, and 
$(W,\Ff_W,B_W(u),\Mm^E(u);G)/Z$
is ACSS. Since
$$K_{\Ff_E}+B_E(u)+\Mm^E(u)_E$$ is anti-ample$/V$,
$K_{\Ff_W}+B_W(u)+\Mm^E(u)_W$ is not pseudo-effective$/V$. Thus we may run a 
$$(K_{\Ff_W}+B_W(u)+\Mm^E(u)_W)\text{-MMP}/V$$
with scaling of an ample$/V$ divisor. By Theorem \ref{thm: existence mfs}, this MMP terminates with a Mori fiber space$/V$ $\psi: (\bar W,\Ff_{\bar W},B_{\bar W}(u),\Mm^E(u))\rightarrow T$ of $(W,\Ff_{W},B_{W}(u),\Mm^E(u))/V$. By Lemma \ref{lem: ACSS mmp can run}, $\psi$ is also a Mori fiber space$/Z$. 
\begin{center}$\xymatrix{
W\ar@{-->}[rr]\ar@{->}[d]_{g} &  & \bar W\ar@{->}[d]^{\psi}\\
E\ar@{->}[rd]^{\phi} & & T\ar@{->}[ld]\\
& V &}$
\end{center}

Let $B_{\bar W}(t)$ be the image of $B_W(t)$ on $\bar W$ for any $t$. Since 
$$K_{\Ff_E}+B_E(t_0)+\Mm^E(t_0)_E\sim_{\mathbb R,V}0,$$
we have
$$K_{\Ff_W}+B_W(t_0)+\Mm^E(t_0)_W\sim_{\mathbb R,V}0,$$
so $(\bar W,\Ff_{\bar W},B_{\bar W}(u),\Mm^E(u))$ and $(W,\Ff_{W},B_{W}(u),\Mm^E(u))$ are crepant, and
$$K_{\Ff_{\bar W}}+B_{\bar W}(t_0)+\Mm^E(t_0)_{\bar W}\sim_{\mathbb R,V}0.$$
Let $L$ be a general fiber of $\psi$, $B_L(t):=B_{\bar W}(t)|_L$ for any $t$, and $\Mm^L(t):=\Mm^E(t)|_L$ for any $t$. Since $\psi$ is  Mori fiber space$/Z$, the general fibers of $\phi$ are tangent to $\Ff_{\bar W}$. Thus $K_{\Ff_{\bar W}}|_L=K_L$, $(L,B_L(t_0),\Mm^L(t_0))$ is lc,
$$K_L+B_L(t_0)+\Mm^L(t_0)_L\equiv 0,$$
and
$$K_L+B_L(u)+\Mm^L(u)_L$$
is anti-ample. Moreover, since $\psi$ is a Mori fiber space$/Z$, 
$$\dim L\leq\rk\Ff_{\bar W}=\rk\Ff_{E}\leq d.$$
By \cite[Theorem 3.6]{Che20} and considering the coefficients of $B_L(t_0)$ and $\Mm^L(u)$, which can be precisely computed by Theorem \ref{thm: precise adj gfq}, there exists a positive real number $\epsilon$ depending only on $d,\bm{r},s_1,\dots,s_m,\mu_1,\dots,\mu_n$, such that $|t_0-r_c|\geq\epsilon$. This concludes the proof of the theorem.
\end{proof}


\begin{thm}\label{thm: uniform rational polytope}
Let $d,c,m,n$ be positive integers, $r_1,\dots,r_c$ real numbers such that $1,r_1,\dots,r_c$ are linearly independent over $\mathbb Q$, $\bm{r}:=(r_1,\dots,r_c)$, and $s_1,\dots,s_m,\mu_1,\dots,\mu_n: \mathbb R^{c+1}\rightarrow\mathbb R$ $\mathbb Q$-linear functions. Then there exists an open subset $U\ni\bm{r}$ depending only on $d,\bm{r}$ and $s_1,\dots,s_m,\mu_1,\dots\mu_n$ satisfying the following. Assume that
\begin{enumerate}
    \item $$\left(X,\Ff,B(\bm{r}):=\sum_{i=1}^ms_i(1,\bm{r})B_i),\Mm(\bm{r}):=\sum_{i=1}^n\mu_i(1,\bm{r})\Mm_i\right)/X$$ is an lc gfq such that $\Ff$ is algebraically integrable and $\rk\Ff\leq d$,
    \item $B_i\geq 0$ are distinct Weil divisors (possibly $0$) and $s_i(1,\bm{r})\geq 0$,
    \item $\Mm_i$ are nef$/X$ $\bb$-Cartier $\bb$-divisors and $\mu_i(1,\bm{r})\geq 0$, and
    \item $B(\bm{v}):=\sum_{i=1}^ms_i(1,\bm{v})B_i$ and   $\Mm(\bm{v}):=\sum_{i=1}^n\mu_i(1,\bm{v})\Mm_i$ for any $t\in\mathbb R$.
\end{enumerate}
Then $(X,\Ff,B(\bm{v}),\Mm(\bm{v}))$ is lc for any $\bm{v}\in U$.
\end{thm}
\begin{proof}
We apply induction on $c$. When $c=1$, Theorem \ref{thm: uniform rational polytope} directly follows from Theorem \ref{thm: uniform rational polytope foliation one variable}. When $c\geq 2$, by Theorem \ref{thm: uniform rational polytope foliation one variable}, there exists a positive integer $\delta$ depending only on $r_1,\dots,r_c,s_1,\dots,s_m$, such that for any $t\in (r_c-\delta,r_c+\delta)$, $$\left(X,\Ff,\sum_{i=1}^ms_i(1,r_1,\dots,r_{c-1},t)B_i,\sum_{i=1}^n\mu_i(1,r_1,\dots,r_{c-1},t)\Mm_i\right)$$ is lc. We pick rational numbers $r_{c,1}\in (r_c-\delta,r_c)$ and $r_{c,2}\in (r_c,r_c+\delta)$ depending only on $r_1,\dots,r_c,s_1,\dots,s_m$. By induction on $c$, there exists an open subset $U_0\in (r_1,\dots,r_{c-1})$ of $\mathbb R^{c-1}$, such that for any $\bm{v}\in U_0$, $$\left(X,\Ff,\sum_{i=1}^ms_i(1,\bm{v},r_{c,1})B_i,\sum_{i=1}^n\mu_i(1,\bm{v},r_{c,1})\Mm_i\right)$$ and $$\left(X,\Ff,\sum_{i=1}^ms_i(1,\bm{v},r_{c,2})B_i,\sum_{i=1}^n\mu_i(1,\bm{v},r_{c,2})\Mm_i\right)$$ are lc. We may pick $U:=U_0\times (r_{c,1},r_{c,2})$.
\end{proof}



\part{Canonical bundle formula and MMP for generalized pairs}\label{part:cbf}

\section{Canonical bundle formula for lc-trivial fibrations}\label{sec: cbf}

\subsection{Stability of generalized foliated quadruples}\label{subsec: stability gfq}

\begin{prop}[{cf. \cite[Proposition 3.7]{ACSS21}}]\label{prop: bp semistable foliation lc}
    Let $(X,\Ff,B,\Mm)/U$ be a sub-gfq satisfying Property $(*)$ associated with $f: X\rightarrow Z$ and $G$. Assume that $f$ is equi-dimensional and $B$ is horizontal$/Z$. Then $f: (X,B+G,\Mm)\rightarrow Z$ is BP semi-stabl if and only if $(X,\Ff,B,\Mm)$ is sub-lc.
\end{prop}
\begin{proof}
Since $(X,\Ff,B+\Mm)/U$ satisfies Property $(*)$, $f: (X,B+G,\Mm)\rightarrow Z$ satisfies Property $(*)$. By Lemma \ref{lem: basic property (*) gpair}(1), $(X,B+G,\Mm)$ is sub-lc. Let $f': (X',\Sigma_{X'},\Mm)\rightarrow (Z',\Sigma_{Z'})$ be any equi-dimensional model of $(X,B+G,\Mm)$ associated with $h: X'\rightarrow X$ and $h_Z: Z'\rightarrow Z$. By Proposition \ref{prop: weak ss imply *}, there exists an $\Rr$-divisor $\bar B$ on $X'$ satisfying the followings.
\begin{itemize}
    \item $\Supp \bar B\subset\Supp\Sigma_{X'}$.
    \item $K_{X'}+\bar B+\Mm_{X'}=h^*(K_X+B+G+\Mm_X)+F$ for some $F\geq 0$ that is vertical$/Z'$.
\item $(X',\bar B,\Mm)$ and $(X,B+G,\Mm)$ are crepant over the generic point of $Z$. In particular, $(X',\bar B,\Mm)$ and $(X,B,\Mm)$ are crepant over the generic point of $Z$.
\item $f': (X',\bar B,\Mm)\rightarrow Z$ satisfies Property $(*)$. By Lemma \ref{lem: basic property (*) gpair}(1), $(X',\bar B,\Mm)$ is sub-lc.
\end{itemize}
Let $\Ff':=h^{-1}\Ff$, $G'$ the vertical$/Z'$ part of $\bar B$, and $B'$ the horizontal$/Z'$ part of $\bar B$. Then $(X',\Ff',B',\Mm;G')/Z'$ satisfies Property $(*)$. Since $F$ is vertical$/Z'$ and $\Ff'$ is induced by $f'$, $F$ is $\Ff'$-invariant.

We let $B_Z$ and $\Nn$ be the discriminant part and moduli part of $f: (X,B+G,\Mm)\rightarrow Z$ respectively, and let $\bar B_{Z'}$ and $\Nn'$ be the discriminant part and moduli part of $f: (X',\bar B,\Mm)\rightarrow Z$ respectively. Since $(X,\Ff,B,\Mm)/Z$ satisfies Property $(*)$, $Z$ is smooth, so $K_Z+B_Z$ is $\Rr$-Cartier, and we may define
$$K_{Z'}+B_{Z'}:=h_Z^*(K_Z+B_Z).$$
Since $f': (X',\bar B,\Mm)\rightarrow Z'$ satisfies Property $(*)$, $\bar B_{Z'}=f'(G')$. Since $f: (X,B+G,\Mm)\rightarrow Z$ satisfies Property $(*)$, $B_Z=f(G)$.

By Proposition \ref{prop: weak cbf gfq}, $K_{\Ff}+B+\Mm_X\sim\Nn_X$ and $K_{\Ff'}+B'+\Mm_{X'}\sim\Nn'_{X'}$. In particular,  $\Nn_X$ is $\Rr$-Cartier. Let $A:=\Nn'_{X'}-h^*\Nn_{X'}$. Then 
\begin{align*}
    A&=K_{X'}+\bar B+\Mm_{X'}-f'^*(K_{Z'}+\bar B_{Z'})-h^*(K_X+B+G+\Mm_X-f^*(K_Z+B_Z))\\
    &=F-f'^*(K_{Z'}+\bar B_{Z'})+f'(K_{Z'}+B_{Z'})=F-f'^*(\bar B_{Z'}-B_{Z'}).
\end{align*}
In particular, $A$ vertical$/Z'$.
\begin{claim}\label{claim: a' for semistable}
    $(X,\Ff,B,\Mm)$ is sub-lc if and only if $A\geq 0$.
\end{claim}
\begin{proof}
    Let 
    $$A':=K_{\Ff'}+h^{-1}_*B+\Mm_{X'}+(\Supp\Exc(h))^{\Ff'}-h^*(K_{\Ff}+B+\Mm_X).$$
    By Lemma \ref{lem: existence foliated log resolution}, $h$ is a foliated log resolution of $(X,\Ff,B,\Mm)$, so  $(X,\Ff,B,\Mm)$ is sub-lc if and only if $A'\geq 0$. Since
    $$A\sim K_{\Ff'}+B'+\Mm_{X'}-h^*(K_{\Ff}+B+\Mm_X),$$
    for suitable choices of $K_{\Ff}$ and $K_{\Ff'}$, we have
    $$A'-A=h^{-1}_*B-B'+(\Supp\Exc(h))^{\Ff'}.$$
    For any horizontal$/Z$ prime divisor $P$ on $X'$, if $P$ is not exceptional$/X$, then
    $$\mult_PA'=\mult_P(h^{-1}_*B-\bar B)=0$$
    as $G$ and $F$ are vertical$/Z$. If $P$ is exceptional$/X$, then
    $$\mult_PA'=1+\mult_P(h^{-1}_*B-\bar B)\geq 1-\mult_P\bar B.$$
    Since $f': (X',\bar B,\Mm)\rightarrow Z'$ satisfies Property $(*)$, $\mult_P\bar B\leq 1$, so $\mult_PA'\geq 0$.

    For any vertical$/Z'$ prime divisor $P$ on $X'$, since $B$ is horizontal$/Z$ and $B'$ is horizontal$/Z'$, $\mult_PA'=\mult_PA$. The claim follows.
\end{proof}
 Let $B'_{Z'}$ be the discriminant part of $f': (X',\bar B-F,\Mm)\rightarrow Z'$. Then for any prime divisor $D$ on $Z'$,
     $$\mult_D\bar B_{Z'}=1-\sup\{t\mid (X',\bar B+tf'^*D,\Mm)\text{ is sub-lc over the generic point of }D\}$$
     and
    $$\mult_D B'_{Z'}=1-\sup\{t\mid (X',\bar B-F+tf'^*D,\Mm)\text{ is sub-lc over the generic point of }D\}.$$
    Since $\Supp\bar B\subset\Supp\Sigma_{X'}$, by the definition of equi-dimensional model, $\Supp(\bar B-F)\subset\Supp\Sigma_{X'}$. Therefore, if $D\subset\Supp\Sigma_{Z'}$, then
    $$\mult_D B'_{Z'}=\mult_D\bar B_{Z'}=\mult_{f^*D}F=0.$$
    Otherwise,
    $$\mult_D(\bar B_{Z'}-B'_{Z'})=\sup\{t\geq 0\mid F-tf'^*D\geq 0\}.$$
    Therefore, 
    \begin{itemize}
        \item  $F-f'^*(\bar B_{Z'}-B'_{Z'})\geq 0$, and
        \item  $F-f'^*(\bar B_{Z'}-B'_{Z'})-\delta f'^*D\geq 0$ for any prime divisor $D$ on $Z'$ and any $\delta>0$.
    \end{itemize}
Since
$$A=f'^*(B_{Z'}-B'_{Z'})+(F-f'^*(\bar B_{Z'}-B'_{Z'})),$$
we have that $A\geq 0$ if and only if $B_{Z'}-B'_{Z'}\geq 0$. The proposition follows from Claim \ref{claim: a' for semistable}.
\end{proof}

\begin{prop}\label{prop: bp stable nef}
    Let $(X,B,\Mm)/U$ be an lc g-pair and $f: X\rightarrow Z$ a contraction such that
    \begin{itemize}
        \item $f: (X,B,\Mm)\rightarrow Z$ satisfies Property $(*)$,
        \item $(X,B,\Mm)$ is BP semi-stable$/Z$,
        \item $f$ is equi-dimensional, and
        \item $K_X+B+\Mm_X$ is nef$/Z$.
    \end{itemize}
    Let $\Nn$ be the moduli part of $f: X\rightarrow Z$. Then:
    \begin{enumerate}
        \item $\Nn_X$ is nef$/U$.
        \item If  $(X,B,\Mm)$ is BP stable$/Z$, then $\Nn$ descends to $X$. In particular $\Nn$ is nef$/U$.
    \end{enumerate}
\end{prop}
\begin{proof}
Let $\Ff$ be the foliation induced by $\Ff$ and $B^h$ the horizontal$/Z$ part of $B$. By Proposition \ref{prop: bp semistable foliation lc}, $(X,\Ff,B^h,\Mm;G)/Z$ is weak ACSS. Since $K_X+B+\Mm_X$ is nef$/U$, by Lemma \ref{lem: equivalence over bases}, $K_{\Ff}+B+\Mm_X$ is nef$/U$. 
By Proposition \ref{prop: weak cbf gfq},
$$\Nn_X\sim K_{\Ff}+B^h+\Mm_X,$$
so $\Nn_X$ is nef$/U$. This implies (1). If $(X,B,\Mm)$ is BP stable$/Z$, then $\Nn$ descends to $X$, and (2) follows from (1).
\end{proof}

\begin{lem}\label{lem: special acss model}
Let $(X,\Ff,B,\Mm)/U$ be an lc gfq such that $\Ff$ is induced by a contraction $f: X\rightarrow Z$. Let $D_Z$ be a divisor over $Z$. Then there exists an ACSS model $(X',\Ff',B',\Mm)/Z'$ of $(X,\Ff,B,\Mm)$ with induced morphisms $f': X'\rightarrow Z'$ and $g: X'\rightarrow X$, and a birational morphism $h_Z: Z'\rightarrow Z$, such that $h_Z\circ f'=f\circ g$ and $D_Z$ is on $Z'$.
\end{lem}
\begin{proof}
By Definition-Theorem \ref{defthm: weak ss reduction}, there exists an equi-dimensional model $(Y,\Sigma_Y,\Mm)\rightarrow Z$ of $f: (X,B,\Mm)\rightarrow Z$ associated with $h: Y\rightarrow X$ and $h_Z: Z'\rightarrow Z$, such that $D_Z$ is on $Z$. Let $\Ff_Y:=h^{-1}\Ff$ and $B_Y:=h^{-1}_*B+(\Supp\Exc(h))^{\Ff_Y}$, then $(Y,\Ff_Y,B_Y,\Mm)$ is foliated log smooth, and
$$K_{\Ff_Y}+B_Y+\Mm_Y\sim_{\mathbb R,X}\sum_{E\subset\Exc(h)}(\epsilon_{\Ff}(E)-a(E,\Ff,B,\Mm))E\geq 0.$$
By Theorem \ref{thm: mmp very exceptional alg int fol}, we may run a $(K_{\Ff_Y}+B_Y+\Mm_Y)$-MMP$/X$ which terminates with a good minimal model $(X',\Ff',B',\Mm)/X$ of $(Y,\Ff_Y,B_Y,\Mm)/X$. By Lemma \ref{lem: acss model is gmm}, $(X',\Ff',B',\Mm)$ is an ACSS model of $(X,\Ff,B,\Mm)$. By Lemma \ref{lem: ACSS mmp can run}, this MMP is also a $(K_{\Ff_Y}+B_Y+\Mm_Y)$-MMP$/Z'$. Then $(X',\Ff',B',\Mm)/Z'$ satisfies our requirements.
\end{proof}

\begin{thm}\label{thm: lc+weak acc=bpstable}
    Let $(X,\Ff,B,\Mm)/U$ be an lc gfq, $f: X\rightarrow Z$ a contraction, and $G$ a reduced divisor on $X$ such that $(X,\Ff,B,\Mm;G)/Z$ is weak ACSS. Then $(X,B+G,\Mm)$ is BP stable$/Z$.
\end{thm}
\begin{proof}
For any prime divisor $D_Z$ over $Z$, by Lemma \ref{lem: special acss model}, there exist two birational morphisms $h_Z: Z'\rightarrow Z$ and $h: X'\rightarrow X$, and an ACSS model $(X',\Ff',B',\Mm)/Z'$ of $(X,\Ff,B,\Mm)$ with induced morphism $f': X'\rightarrow Z'$, such that $f\circ h=h_Z\circ f'$ and $D_Z$ is on $Z'$. 

We let $G'$ be a divisor on $X'$ such that $(X',\Ff',B',\Mm;G')/Z'$ is ACSS. Let $B_Z$ and $\Nn$ be the discriminant and moduli part of $f: (X,B+G,\Mm)\rightarrow Z$ respectively, $K_{Z'}+B_{Z'}:=h_Z^*(K_Z+B_Z)$, and let $B'_{Z'}$ and $\Nn'$ be the discriminant part and moduli part of $f': (X',B'+G',\Mm)\rightarrow Z'$ respectively. 

By Proposition \ref{prop: weak cbf gfq}, we have
$$\Nn'_{X'}\sim K_{\Ff'}+B'+\Mm_{X'}=h^*(K_{\Ff}+B+\Mm_X)\sim h^*\Nn_X.$$
Thus for suitable choices of $\Nn'$ and $\Nn$, we may assume that $\Nn'_{X'}=h^*\Nn_X$. Let
$$K_{X'}+\tilde B'+\Mm_{X'}:=h^*(K_X+B+G+\Mm_X).$$
Let $\tilde B_{Z'}$ and $\tilde\Nn$ be the discriminant and moduli parts of $f': (X',\tilde B,\Mm)\rightarrow Z'$ respectively. Since
\begin{align*}
    &\Nn'_{X'}-h^*\Nn_X\\
    =&\left(\left(K_{X'}+B'+G'+\Mm_{X'}\right)-f'^*\left(K_{Z'}+B'_{Z'}\right)\right)-h^*\left(K_X+B+G+\Mm_X-f^*\left(K_Z+B_Z\right)\right)\\
    =&B'+G'-\tilde B'-f'^*(B'_{Z'}-B_{Z'}),
\end{align*}
we have that 
$$B'+G'-\tilde B'-f'^*(B'_{Z'}-B_{Z'})=0.$$ Moreover, by Proposition \ref{prop: bp semistable foliation lc}, $(X,B+G,\Mm)$ is BP semi-stable$/Z$. Thus $$\mult_{D_Z}\tilde B_{Z'}\leq\mult_{D_Z}B_{Z'}.$$ 
For any prime divisor $D$ on $X$ with $f'(D)=D_Z$, since $B'$ is horizontal$/Z'$, $\mult_DB'=0$. Thus
$$\mult_DG'=\mult_D\left(\tilde B'+f'^*\left(B'_{Z'}-B_{Z'}\right)\right).$$
There are two cases.

\medskip

\noindent\textbf{Case 1}. $D_Z$ is  not a component of $B'_{Z'}$. In this case, $\mult_DG'=0$, and
$$\mult_D\tilde B'=\mult_Df'^*B_{Z'}=\mult_{D_Z}B_{Z'}\cdot\mult_Df'^*D_Z.$$
Thus $\mult_D\left(\tilde B'-\mult_{D_Z}B_{Z'}f'^*D_Z\right)=0,$ and
$$\mult_D\left(\tilde B'+(1-\mult_{D_Z}B_{Z'})f'^*D_Z\right)=\mult_Df'^*D_Z\geq 1.$$
Therefore,
\begin{align*}
1-\mult_{D_Z}\tilde B_{Z'}&=\sup\left\{t\biggm| \left(X',\tilde B'+tf'^*D_Z,\Mm\right)\text{ is lc over the generic point of }D_Z\right\}\\
&\leq 1-\mult_{D_Z}B_{Z'}.
\end{align*}
Thus
$$\mult_{D_Z}B_{Z'}=\mult_{D_Z}\tilde B_{Z'}.$$

\medskip

\noindent\textbf{Case 2}. $D_Z$ is a component of $B'_{Z'}$. In this case, $\mult_DG'=1$ and $\mult_DB'_{Z'}=1$. Therefore,
$$1=\mult_D\left(\tilde B'+\left(\mult_DB'_{Z'}-\mult_DB_{Z'}\right)f'^*D_Z\right)=\mult_D\left(\tilde B'+\left(1-\mult_DB_{Z'}\right)f'^*D_Z\right).$$
Thus
\begin{align*}
1-\mult_{D_Z}\tilde B_{Z'}&=\sup\{t\mid (X',\tilde B'+tf'^*D_Z,\Mm)\text{ is lc over the generic point of }D_Z\}\\
&\leq 1-\mult_{D_Z}B_{Z'},
\end{align*}
and hence
$$\mult_{D_Z}B_{Z'}=\mult_{D_Z}\tilde B_{Z'}.$$

\medskip

In either case, we have $\mult_{D_Z}B_{Z'}=\mult_{D_Z}\tilde B_{Z'}$. Since $D_Z$ can be any prime divisor over $Z$, $f: (X,B+G,\Mm)\rightarrow Z$ is BP stable.
\end{proof}

\begin{rem}
When $\Mm=\bm{0}$ and $X$ is projective, Theorem \ref{thm: lc+weak acc=bpstable} becomes \cite[Theorem 4.3]{ACSS21} without the condition that $K_X+B$ is $f$-nef. This seems to be an interesting discovery and may be useful for future applications.
\end{rem}

\subsection{Numerical dimension zero generalized foliated quadruples}\label{sub: num 0 mmp}

\begin{prop}\label{prop: weak ss num 0 mmp}
    Let $(X,\Ff,B,\Mm)/U$ be an lc gfq such that
    \begin{itemize}
        \item $(X,\Ff,B,\Mm)$ is weak ACSS,
        \item $\kappa_{\sigma}(X/U,K_{\Ff}+B+\Mm_X)=0$, and
        \item either $X$ is $\Qq$-factorial klt or $\Mm$ is NQC$/U$.
    \end{itemize}
 Then for any ample$/U$ $\Rr$-divisor $A$, there exists a $(K_{\Ff}+B+\Mm_X)$-MMP$/U$ with scaling of $A$, say $\mathcal{P}_0$, satisfying the following. Let $\mathcal{P}=\mathcal{P}_0$ if $X$ is not $\Qq$-factorial, and let $\mathcal{P}$ be any $(K_{\Ff}+B+\Mm_X)$-MMP$/U$ with scaling of $A$ if $X$ is $\Qq$-factorial. After a sequence of steps in $\mathcal{P}$, we get a log birational model $(X',\Ff',B',\Mm)/U$ of $(X,\Ff,B,\Mm)/U$ satisfying the following.
    \begin{enumerate}
        \item For any very general fiber $F'$ of $X'\rightarrow U$, $(K_{\Ff'}+B'+\Mm_{X'})|_{F'}\equiv 0$, and if $\kappa_{\iota}(X/U,K_{\Ff}+B+\Mm_X)=0$, then $(K_{\Ff'}+B'+\Mm_{X'})|_{F'}\sim_{\mathbb R}0.$
        \item Suppose that the associated morphism $\pi:X\to U$ is an equi-dimensional contraction and $U$ is $\Qq$-factorial. 
        \begin{enumerate}
            \item Assume that $K_{\Ff}+B+\Mm_X\equiv_U\text{(resp. }\sim_{\mathbb R,U}\text{) }E^h+E^v$ for some $\Rr$-divisors $E^h$ and $E^v$ such that $E^h\geq 0$ and $E^v$ is vertical$/U$. Then
            $$K_{\Ff'}+B'+\Mm_{X'}\equiv_U\text{(resp. }\sim_{\mathbb R,U}\text{) }0.$$
            In particular, $(X',\Ff',B',\Mm)/U$ is a weak lc model of $(X,\Ff,B,\Mm)/U$.
            \item If $\kappa_{\iota}(X/U,K_{\Ff}+B+\Mm_X)=0$, then:
            \begin{enumerate}
                \item $K_{\Ff'}+B'+\Mm_{X'}\sim_{\mathbb R,U}0.$
                \item $(X',\Ff',B',\Mm)/U$ is a weak lc model of $(X,\Ff,B,\Mm)/U$.
                \item If $(X,\Ff,B,\Mm)/U$ is ACSS, then $(X',\Ff',B',\Mm)/U$ is a good minimal model of $(X,\Ff,B,\Mm)/U$.
            \end{enumerate}
        \end{enumerate}
    \end{enumerate}
\end{prop}
\begin{proof}
Let $(X_0,\Ff_0,B_0,\Mm):=(X,\Ff,B,\Mm)$.
We denote $\mathcal{P}$ by
\begin{center}$\xymatrix{
(X_0,\Ff_0,B_0,\Mm)\ar@{-->}[r] & (X_1,\Ff_1,B_1,\Mm)\ar@{-->}[r] & \dots\ar@{-->}[r] & (X_n,\Ff_n,B_n,\Mm)\ar@{-->}[r] & \dots 
}.$
\end{center}
Let $A_i$ be the strict transform of $A$ on $X_i$, $\pi_i: X_i\rightarrow U$ the induced contraction for each $i$, and
$$\lambda_i:=\inf\{t\geq 0\mid K_{\Ff_i}+B_i+\Mm_{X_i}\text {is nef}/U\}$$
the scaling numbers. By Proposition \ref{prop: run mmp with scaling gfq}, we may choose $\mathcal{P}_0$ so that either $\mathcal{P}$ terminates, or $\lim_{i\rightarrow+\infty}\lambda_i=0$. 

If $\mathcal{P}$ terminates, then we let $m$ be the index so that $(X_m,\Ff_m,B_m,\Mm)/U$ is output of $\mathcal{P}$. If $\mathcal{P}$ does not terminate, then we let $m$ be a positive integer such that $f_i$ is a flip for any $i\geq m$. We let $\phi_i: X_m\dashrightarrow X_i$ be the induced birational map for any $i\geq m$. Since $\mathcal{P}$ contains countable many steps, there are at most countably many closed point $z\in Z$, such that for some $i\geq m$, $\pi_i^{-1}(z)$ is contained in either the flipping locus of $f_i$ or the flipped locus of $f_{i-1}$. Therefore, for a very general point $z\in Z$ and any $i\geq m$, $\pi_i^{-1}(z)$ is neither contained in the flipping locus of $f_i$ nor the flipped locus of $f_{i-1}$ for any $i\geq m$. We let $F_m$ be a very general fiber of $\pi_m$, $z_0:=\pi_m(F_m)$, and let $F_{i}$ be the fiber of $\pi_i$ over $z_0$ for each $i$. Then the induced birational map 
$$\phi_{F,i}:=\phi_i|_{F_m}: F_m\dashrightarrow F_i$$
is small for any $i\geq m$. Let $\Mm^F:=\Mm|_{F_m}$, $B_{F_i}:=B_i|_{F_i}$, and $A_{F_i}:=A_i|_{F_i}$ for each $i\geq m$. Since $F_i$ is a very general fiber of $\pi_i$, $K_{F_i}=K_{\Ff_i}|_{F_i}$ for any $i\geq m$.

We will show that $(X',\Ff',B',\Mm)/U:=(X_m,\Ff_m,B_m,\Mm)/U$ satisfies our requirements.

\begin{claim}\label{claim: movable alon very general fiber}
$K_{\Ff_m}+B_m+\Mm_{X_m}$ is movable$/U$ and $(K_{\Ff_m}+B_m+\Mm_{X_m})|_{F_m}$ is movable.
\end{claim}
\begin{proof}
  If $K_{\Ff_m}+B_m+\Mm_{X_m}$ is nef$/U$ then the claim is obvious, so we may assume that $K_{\Ff_m}+B_m+\Mm_{X_m}$ is not nef$/U$. In particular, $\mathcal{P}$ does not terminate. Since $K_{\Ff_i}+B_i+\Mm_{X_i}+tA_i$ is nef$/U$ for any $i\geq m$,
  $$K_{X_m}+B_{m}+\Mm_{X_m}=\lim_{i\rightarrow+\infty}(\phi_{i}^{-1})_*(K_{X_i}+B_{i}+\Mm_{X_i}+tA_{i})$$
  is movable$/U$, and
  $$K_{F_i}+B_{F_i}+\Mm^F_{F_i}+tA_{F_i}=(K_{\Ff_i}+B_i+\Mm_{X_i}+tA_i)|_{F_i}$$
is nef for each $i\geq m$. Thus
$$K_{F_m}+B_{F_m}+\Mm^F_{F_m}=\lim_{i\rightarrow+\infty}(\phi_{F,i}^{-1})_*(K_{F_i}+B_{F_i}+\Mm^F_{F_i}+tA_{F_i})$$
is movable, and the claim follows.
\end{proof}
\noindent\textit{Proof of Proposition \ref{prop: weak ss num 0 mmp} continued}. Since $\kappa_{\sigma}(X/U,K_{\Ff}+B+\Mm_X)=0$,  $\kappa_{\sigma}(X_m/U,K_{\Ff_m}+B_m+\Mm_{X_m})=0$ and $\kappa_{\sigma}(K_{F_m}+B_{F_m}+\Mm^F_{F_m})=0$. By Claim \ref{claim: movable alon very general fiber} and Lemma \ref{lem: movable num 0 is 0}, $K_{F_m}+B_{F_m}+\Mm^F_{F_m}\equiv 0$. Moreover, if $\kappa_{\iota}(X/U,K_{\Ff}+B+\Mm_X)=0$, then $\kappa_{\iota}(X_m/U,K_{\Ff_m}+B_m+\Mm_{X_m})=0$, so $\kappa_{\iota}(K_{F_m}+B_{F_m}+\Mm^F_{F_m})=0$, and hence $K_{F_m}+B_{F_m}+\Mm^F_{F_m}\sim_{\mathbb R}0$. This implies (1).

We prove (2). From now on, we may assume that $\pi$ is equi-dimensional and $U$ is $\Qq$-factorial. Suppose that $K_{\Ff}+B+\Mm_X\equiv_U\text{(resp. }\sim_{\mathbb R,U}\text{) }E^h+E^v$ where $E^h\geq 0$ and $E^v$ is vertical$/U$. Since $U$ is $\Qq$-factorial, for any prime divisor $D$ on $X$, $D$ is $\Qq$-Cartier, and we may define
$$t_D:=\sup\{s\mid E^v-s\pi^*D\geq 0\text{ over the generic point of }D\}.$$
Let
$$\tilde E^v:=E^v-\sum_{D\mid D\text{ is a prime divisor on U}}t_DD.$$
 Since $\pi$ is equi-dimensional, $\tilde E^v\geq 0$ and $\tilde E^v$ is very exceptional$/U$. Possibly replacing $E^v$ with $\tilde E^v$, we may assume that $0\leq E^v$ is very exceptional$/U$. Let $E^h_{m}$ and $E^v_{m}$ be the strict transforms of $E^h$ and $E^v$ on $X'=X_m$ respectively. Then $E^h_{m}|_{F_m}\equiv 0$, so $E^h_{m}|_{F_m}=0$ and
$$E^v_{m}\equiv_U\text{(resp. }\sim_{\mathbb R,U}\text{) }K_{\Ff_m}+B_m+\Mm_{X_m}$$
is movable$/U$. Therefore, for any prime divisor $S$ on $X_m$ and very general curves $C$ on $S$ over $U$, $E^v_{m}\cdot C\geq 0$. By \cite[Lemma 3.3]{Bir12}, $E^v_{m}=0$. This implies (2.a).

If $\kappa_{\iota}(X/U,K_{\Ff}+B+\Mm_X)=0$, then $K_{\Ff}+B+\Mm_X\sim_{\mathbb R,U}E\geq 0$ for some $\Rr$-divisor $E$ on $X$ (cf. \cite[Definition 2.6]{HH20}). Then (2.b) immediately follows from (2.a) and Lemma \ref{lem: ACSS mmp can run}.
\end{proof}

The following proposition is a direct consequence of Proposition \ref{prop: weak ss num 0 mmp}.

\begin{prop}\label{prop: projective num 0 mmp}
Let $(X,\Ff,B,\Mm)$ be a projective lc gfq such that%and $\pi: X\rightarrow Z$ a contraction such that 
\begin{itemize}
   \item $(X,\Ff,B,\Mm)$ is weak ACSS,
   \item $\kappa_{\sigma}(K_{\Ff}+B+\Mm_X)=0$, and
   \item either $X$ is $\Qq$-factorial klt or $\Mm$ is NQC.
\end{itemize} Then for any ample $\Rr$-divisor $A$, there exists a $(K_{\Ff}+B+\Mm_X)$-MMP with scaling of $A$, say $\mathcal{P}_0$, satisfying the following. Let $\mathcal{P}=\mathcal{P}_0$ if $X$ is not $\Qq$-factorial, and let $\mathcal{P}$ be any $(K_{\Ff}+B+\Mm_X)$-MMP with scaling of an ample $\Rr$-divisor if $X$ is $\Qq$-factorial. Then:
\begin{enumerate}
   \item $\mathcal{P}$ terminates with a weak lc model $(X',\Ff',B',\Mm)$ of $(X,\Ff,B,\Mm)$ such that
   $$K_{\Ff'}+B'+\Mm_{X'}\equiv 0.$$
   \item Suppose that $\kappa_{\iota}(K_{\Ff}+B+\Mm_X)=0$. Then:
   \begin{enumerate}
       \item  $K_{\Ff'}+B'+\Mm_{X'}\sim_{\mathbb R,U}0.$
       \item  If $(X,\Ff,B,\Mm)$ is $\Qq$-factorial ACSS, then $(X',\Ff',B',\Mm)$ is a good minimal model of $(X,\Ff,B,\Mm)$.
   \end{enumerate}
\end{enumerate}
\end{prop}
\begin{proof}
It immediately follows from  Proposition \ref{prop: weak ss num 0 mmp} by taking $U=\{pt\}$.
\end{proof}

\subsection{Refined definition of lc-trivial fibrations}

\begin{defn}[Lc-trivial fibration]\label{defn: lc trivial fibration gfq}
Let $(X,\Ff,B,\Mm)/U$ be a sub-gfq and $f: X\rightarrow Z$ a contraction$/U$, such that the general fibers of $f$ are tangent to $\Ff$. We say that $f: (X,\Ff,B,\Mm)\rightarrow Z$ is an \emph{lc-trivial fibration} if
\begin{enumerate}
\item $(X,\Ff,B,\Mm)$ is sub-lc over the generic point of $Z$,  
\item $K_{\Ff}+B+\Mm_X\sim_{\mathbb R,Z}0$, and
\item there exists a birational morphism $h: Y\rightarrow X$ with $\Ff_Y:=h^{-1}_*\Ff$ and $K_{\Ff_Y}+B_Y+\Mm_Y=h^*(K_{\Ff}+B+\Mm_X)$, such that $-B_Y^{\leq 0}$ is $\Rr$-Cartier and 
$$\kappa_{\sigma}(Y/Z,-B_Y^{\leq 0})=0.$$
\end{enumerate}
It is clear the lc-trivial fibration does not depend on the choice of $U$.
\end{defn}


\begin{rem}\label{rem: lc trivial fibration definition}
It is essential to note that our definition of lc-trivial fibration differs from the classical one, even when $\Mm=\bm{0}$ and $\Ff=T_X$. We have valid reasons for this deviation. For the sake of simplicity, in the rest part of this remark, we will assume that $\Ff=T_X$.
    
In the classical definition, condition (3) is substituted by
\begin{enumerate}
        \item[(3')] $\rk f_*\mathcal{O}_X(\lceil\Aa^*(X,B,\Mm)\rceil)=1.$
\end{enumerate} 
This condition (3') appears in the initial version of the canonical bundle formula \cite[Condition (3) of Theorem 2]{Kaw98}. It has also been adopted in subsequent versions, for instance, \cite[Theorem 0.2]{Amb05} for sub-klt sub-pairs and \cite[Theorem 8.3.7]{Kol07} (also see \cite[Theorem 3.6]{FG14}) for lc-trivial fibrations of sub-lc sub-pairs.

However, for generalized sub-pairs, does not we cannot prove a complete version of the canonical bundle formula under condition (3'). Specifically, for lc-trivial fibrations of NQC generalized pairs defined using condition (3') instead of (3), one must incorporate one of the subsequent conditions to derive the canonical bundle formula:
\begin{itemize}
        \item[(4.1)] $B\geq 0$ over the generic point of $Z$ (rational coefficient case \cite[Theorem 2.20]{FS23}; real coefficient case \cite[Theorem 2.23]{JLX22}).
        \item[(4.2)] $\Mm$ is $\bb$-semi-ample$/Z$ (rational coefficient case \cite[Chapter 6, Theorem 7]{Fil19}; real coefficient case \cite[Theorem 2.23]{JLX22}).
\end{itemize}
While the canonical bundle formula for NQC generalized pairs under conditions (4.1) or (4.2) usually suffices for studying generalized pairs, troubles arise when examining the canonical bundle formula for generalized foliated quadruples. This is because we need to construct equi-dimensional models during the construction of the canonical bundle formula for generalized foliated quadruples, as outlined in \cite[Definition-Theorem 6.12]{LLM23}. This approach could yield a sub-lc g-sub-pair with negative coefficients, typically not complying with (4.1) or (4.2). Consequently, defining the canonical bundle formula for generalized foliated quadruples becomes challenging. Condition (3) was introduced to address this issue.

Indeed, the most important cases of the canonical bundle formula arise when $B\geq 0$ over the generic point of $Z$. But as we often need to consider the coefficients of the discriminant part across any high model of the base $Z$ in order to study the corresponding singularities, base changes are inevitable. Therefore, we need to take consideration of crepant pullbacks of $(X,B,\Mm)$. Furthermore, running minimal model programs over the base to introduce new structures means that crepant transformations over the generic point of $Z$ also become inevitable. This will inevitably introduce more sub-pairs or g-sub-pairs, necessitating a broader category of structures for which the canonical bundle formula needs to be applied. Specifically, we aim to identify a category $\mathcal{D}$ of structures
 $$f: (X,B,\Mm)\rightarrow Z,$$
which satisfies the following two conditions.
\begin{enumerate}
\item[(i)] For any g-sub-pair $(X,B,\Mm)/U$ and contraction$/U$ $f: X\rightarrow Z$ such that $K_X+B+\Mm_X\sim_{\mathbb R,Z}0$ and $(X,B,\Mm)$ is lc over the generic point of $Z$, $f: (X,B,\Mm)\rightarrow Z$ belongs to $\mathcal{D}$.
\item[(ii)] For any g-sub-pairs $(X,B,\Mm)/U$ and $(X',B',\Mm')/U$ and birationally equivalent contractions$/U$ $f: X\rightarrow Z$, $f': X'\rightarrow Z'$ such that $K_X+B+\Mm_X\sim_{\mathbb R,Z}0$, $K_{X'}+B'+\Mm_{X'}\sim_{\mathbb R,Z'}0$, and $(X,B,\Mm)$ and $(X',B',\Mm')$ are crepant over the generic point of $Z$,  $f: (X,B,\Mm)\rightarrow Z$ belongs to $\mathcal{D}$ if and only if  $f': (X',B',\Mm')\rightarrow Z$ belongs to $\mathcal{D}$.
\end{enumerate}
Condition (3') is one natural condition to add in order to form the category $\mathcal{D}$. For generalized pairs, however, the $\mathcal{D}$ category shaped by incorporating condition (3') becomes overly expansive to consistently prove the canonical bundle formula. By comparing our condition (3) with (3'), it becomes evident that (3') can be loosely interpreted as
 $$\kappa(Y/Z,-B_Y^{\leq 0})=0$$
 (cf. \cite[Definitions 8.4.1, 8.4.2]{Kol07}). This essentially indicates some kind of existence of good minimal models should hold for for generalized pairs with Kodaira dimension $0$. But such an assertion is absurd by numerous examples (e.g., \cite[1.1 Example]{Sho00}). In fact, even for usual pairs, since we don't know the existence of good minimal models for pairs with Kodaira dimension of $0$, the theory of mixed Hodge structures is inevitably used to prove the canonical bundle formula in almost all literature, with the exception of \cite{ACSS21}. We also note that \cite{ACSS21} does not extensively address lc-trivial fibrations.

 Given these considerations, we redirect our focus to a new category $\mathcal{D}$ of g-sub-pairs which adhere to (1) and (2) but do not depend on condition (3'). It turns out that condition (3) is a natural alternative choice for us to form the category $\mathcal{D}$. This enables us to bypass the abundance conjecture or the mixed Hodge structure by replacing (3) with (3').  This will eventually lead us to prove the canonical bundle formula for generalized pairs and generalized foliated quadruples in full generality.
\end{rem}


The following lemmas are analogues of Lemma \ref{lem: lc trivial (3) invariant under pullback}, \ref{lem: lc trivial preserved crepant}, and \ref{lem: lc trivial holds for lc gpair} for foliations, and their proofs are similar.

\begin{lem}\label{lem: lc trivial (3) invariant under pullback}
    Let $(X,\Ff,B,\Mm)/U$ be a sub-gfq. Assume that $-B^{\leq 0}$ is $\Rr$-Cartier and $\kappa_{\sigma}(X/U,-B^{\leq 0})=0$. Then for any birational morphism $g: W\rightarrow X$, such that
    \begin{enumerate}
        \item $K_{g^{-1}\Ff}+B_W+\Mm_W:=g^*(K_{\Ff}+B+\Mm_X)$ satisfies that $-B_W^{\leq 0}$ is $\Rr$-Cartier, and
        \item there exists an $\Rr$-Cartier $\Rr$-divisor $0\leq F\subset\Supp\Exc(g)$,
    \end{enumerate}
we have that
$$\kappa_{\sigma}(W/U,-B_W^{\leq 0})=0.$$
\end{lem}
\begin{proof}
Let $D:=-B^{\leq 0}$, $D_W:=-B_W^{\leq 0}$, and $m\gg 0$ an integer. Then we have
$$D_W=g^{-1}_*D+E$$
for some $E\geq 0$ that is exceptional$/X$. Thus
$$0=\kappa_{\sigma}(X/U,D)=\kappa_{\sigma}(W/U,g^*D+mF)\geq\kappa_{\sigma}(W/U,g^{-1}_*D+E)=\kappa_{\sigma}(W/U,D_W)\geq 0.$$
So $\kappa_{\sigma}(W/U,D_W)=0$.
\end{proof}


\begin{lem}\label{lem: lc trivial preserved crepant}
    Let $(X,\Ff,B,\Mm)/U$ and $(X',\Ff',B,\Mm')/U$ be two sub-gfqs. Let $f: X\rightarrow Z$ and $f': X'\rightarrow Z'$ be two birationally equivalent contractions$/U$, such that $(X,\Ff,B,\Mm)$ and $(X',\Ff',B',\Mm')$ are crepant over the generic point of $Z$. Assume that $K_{\Ff}+B+\Mm_X\sim_{\mathbb R,Z}0$ and $K_{\Ff'}+B'+\Mm_{X'}\sim_{\mathbb R,Z'}0$.

    Then $f: (X,\Ff,B,\Mm)\rightarrow Z$ is an lc-trivial fibration if and only if $f': (X',\Ff',B',\Mm')\rightarrow Z'$ is an lc-trivial fibration.
\end{lem}
\begin{proof}
By symmetry, we only need to prove the only if part, and we may assume that $f: (X,\Ff,B,\Mm)\rightarrow Z$ is an lc-trivial fibration.

Let $p: W\rightarrow X$ and $q: W\rightarrow X'$ be a resolution of indeterminacy of the induced birational map $\phi: X\dashrightarrow X'$ such that $\Mm$ descends to $W$, $\Ff_W:=p^{-1}\Ff=q^{-1}\Ff'$,
$K_{\Ff_W}+B_W+\Mm_{W}:=p^*(K_{\Ff}+B+\Mm_X)$, and $K_{\Ff_W}+B'_W+\Mm'_{W}:=q^*(K_{\Ff'}+B'+\Mm'_{X'})$. Moreover, by Lemma \ref{lem: lc trivial (3) invariant under pullback}, possibly replacing $W$ with a higher resolution, we may assume that $W$ is smooth and $\kappa_{\sigma}(W/Z,-B_W^{\leq 0})=0$. 

Since $(X,\Ff,B,\Mm)$ and $(X',\Ff',B',\Mm')$ are crepant over the generic point of $Z$,  over the generic point of $Z$, we have that $B_W=B'_W$, $\Ff=\Ff'$, and $\Mm_W=\Mm'_W$. Thus $\kappa_{\sigma}(W/Z,-B_W'^{\leq 0})=0$. Moreover, since $(X,\Ff,B,\Mm)$ is sub-lc over the generic point of $Z$, $(W,\Ff,B_W,\Mm)$ is sub-lc over the generic point of $Z$, so $(W,\Ff',B_W',\Mm')$ is sub-lc over the generic point of $Z$, and so $(W,\Ff',B_W',\Mm')$ is sub-lc over the generic point of $Z'$, so $(X',\Ff',B',\Mm')$ is sub-lc over the generic point of $Z'$. The lemma follows.
\end{proof}

\begin{lem}\label{lem: lc trivial holds for lc gpair}
    Let $(X,\Ff,B,\Mm)/U$ be a sub-gfq and $f: X\rightarrow Z$ a contraction$/U$, such that $(X,\Ff,B,\Mm)$ is lc over the generic point of $Z$ and $K_{\Ff}+B+\Mm_X\sim_{\mathbb R,Z}0$. Then $f: (X,\Ff,B,\Mm)\rightarrow Z$ is an lc-trivial fibration.
\end{lem}
\begin{proof}
    Over the generic point of $Z$, $B^{\leq 0}=0$, so $\kappa_{\sigma}(X/Z,B^{\leq 0})=0$. The lemma follows from the definition.
\end{proof}


\subsection{Canonical bundle formula for generalized pairs}\label{subsec: cbf gpair}

\begin{defn}\label{defn: cbf gpair}
Let $(X,B,\Mm)/U$ be a g-sub-pair and $f: X\rightarrow Z$ a contraction$/U$ such that $f: (X,B,\Mm)\rightarrow Z$ is an lc-trivial fibration. Then there exists an $\Rr$-divisor $L$ on $Z$ such that $K_X+B+\Mm_X\sim_{\mathbb R}f^*L$. There exists a unique $\bb$-divisor $\Mm^Z$ on $Z$ satisfying the following.
    
Let $f': X'\rightarrow Z'$ be any contraction that is birationally equivalent to $f$ such that the induced birational maps $h: X'\dashrightarrow X$ and $h_Z: Z'\dashrightarrow Z$ are morphisms. We let $$K_{X'}+B'+\Mm_{X'}:=h^*(K_X+B+\Mm_X)$$
and let $B_{Z'}$ be the  discriminant part of $f': (X',B',\Mm)\rightarrow Z'$. Then
$$\Mm^Z_{Z'}=h_Z^*L-K_{Z'}-B_{Z'}.$$
We call $\Mm^Z$ the \emph{base moduli part} of $f: (X,B,\Mm)\rightarrow Z$. If there is no confusion, we may also call $\Mm^Z$ as the \emph{moduli part} of $f: (X,B,\Mm)\rightarrow Z$. It is clear that for any fixed choice of $L$, $\Mm^Z$ is unique. In particular, $\Mm^Z$ is unique up to $\Rr$-linear equivalence.
\end{defn}

\begin{lem}\label{lem: order along generic fiber cbf}
Let $(X,B,\Mm)/U$ be a g-sub-pair and $f: X\rightarrow Z$ a contraction$/U$ such that $f: (X,B,\Mm)\rightarrow Z$ is an lc-trivial fibration. Suppose that $n(K_X+B+\Mm_X)\sim 0$ over the generic point of $Z$ for some positive integer $n$. Then there exists a choice $\Mm^Z$ of the base moduli part of $f: (X,B,\Mm)\rightarrow Z$ such that 
$$n(K_X+B+\Mm_X)\sim nf^*\left(K_Z+B_Z+\Mm^Z_Z\right),$$
where $B_Z$ be the discriminant part of $f: (X,B,\Mm)\rightarrow Z$.
\end{lem}
\begin{proof}
By assumption, there exists a rational function $\psi\in K(X)$ such that $n(K_X+B+\Mm_X)+(\psi)$ is vertical$/Z$. By \cite[Lemma 2.5]{CHL23}, there exists an $\Rr$-Cartier $\Rr$-divisor $L$ on $Z$ such that
$$n(K_X+B+\Mm_X)+(\psi)=nf^*L.$$
The lemma follows from our construction of $\Mm^Z$ as in Definition \ref{defn: cbf gpair}.
\end{proof}

\begin{lem}\label{lem: m preserved under crepant}
    Let $(X,B,\Mm)/U$ and $(X',B,\Mm)/U$ be two g-sub-pairs. Let $f: (X,B,\Mm)\rightarrow Z$ and $f': (X',B',\Mm)\rightarrow Z'$ be two lc-trivial fibrations$/U$ such that $f$ and $f'$ are birationally equivalent, and $(X,B,\Mm)$ and $(X',B',\Mm)$ are crepant over the generic point of $Z$. Let $\Mm^Z$ be the base moduli part of $f: (X,B,\Mm)\rightarrow Z$  and let $\Mm^{Z'}$ be the base moduli part of $Z'$. Then $\Mm^Z\sim_{\mathbb R}\Mm^{Z'}$.
\end{lem}
\begin{proof}
Possibly passing to a common base and resolve indeterminacy of the induced birational map $X\dashrightarrow X'$, we may assume that $f=f'$, $X=X'$, and $Z=Z'$. Now $K_X+B+\Mm_X=K_{X}+B'+\Mm_X$ over the generic point of $Z$, so $B-B'$ is vertical$/Z$. Since $K_X+B+\Mm_X\sim_{\mathbb R,Z}0$ and $K_{X}+B'+\Mm_X\sim_{\mathbb R,Z}0$, $B-B'\sim_{\mathbb R,Z}0$, so $B-B'=f^*P$ for some $\Rr$-divisor $P$ on $Z$.

Let $B_Z$ and $B_Z'$ be the discriminant parts of $f: (X,B,\Mm)\rightarrow Z$ and $f: (X,B',\Mm)\rightarrow Z$ respectively. By the definition of the discriminant part, $B_Z=B_Z'+P$. Since
$$K_Z+B_Z'+P+\Mm^{Z'}_{Z}\sim_{\mathbb R}K_Z+B_Z+\Mm^Z_Z,$$
$\Mm^{Z'}_{Z}\sim_{\mathbb R}\Mm^Z_Z$. Since we may pass to an arbitrarily high base change, we have $\Mm^Z\sim_{\mathbb R}\Mm^{Z'}$.
\end{proof}

\begin{thm}\label{thm: cbf gpair nonnqc}
Let $(X,B,\Mm)/U$ be a g-sub-pair and $f: X\rightarrow Z$ a contraction$/U$ such that $f: (X,B,\Mm)\rightarrow Z$ is an lc-trivial fibration. Let $B_Z$ and $\Mm^Z$ be the discriminant part and a base moduli part of $f: (X,B,\Mm)\rightarrow Z$ respectively. Then $\Mm^Z$ is nef$/U$. Moreover:
\begin{enumerate}
    \item $(Z,B_Z,\Mm^Z)/U$ is a g-sub-pair.
    \item If the vertical$/Z$ part of $B$ is $\geq 0$, then $(Z,B_Z,\Mm^Z)/U$ is a g-pair.
    \item If $(X,B,\Mm)$ is sub-lc (resp. lc, sub-klt, klt), then $(Z,B_Z,\Mm^Z)$ is sub-lc (resp. lc, sub-klt, klt).
    \item Any lc center of $(Z,B_Z,\Mm^Z)$ is the image of an lc center of $(X,B,\Mm)$.
    \item The image of any lc center of $(X,B,\Mm)$ on $Z$ is an lc center of $(Z,B_Z,\Mm^Z)$.
    \item If $\Mm$ is NQC$/U$, then $\Mm^Z$ is NQC$/U$.
\end{enumerate}
\end{thm}
\begin{proof}
By Lemmas \ref{lem: lc trivial (3) invariant under pullback} and \ref{lem: lc trivial preserved crepant}, possibly replacing $f$, we may assume that $-B^{\leq 0}$ is $\Rr$-Cartier and $\kappa_{\sigma}(X/Z,-B^{\leq 0})=0$. By Definition-Theorem \ref{defthm: weak ss reduction}, $f: (X,B,\Mm)\rightarrow Z$ has an equi-dimensional model $f': (X',\Sigma_{X'},\Mm)\rightarrow Z'$ with associated morphisms $h: X'\to X$ and $h_Z: Z'\rightarrow Z$. Let
$$K_{X'}+\tilde B'+\Mm_{X'}:=h^*(K_X+B+\Mm_X),$$
$\tilde B'^h$ the horizontal$/Z'$ part of $\tilde B'$, and $B':=(\tilde B'^h)^{\geq 0}$. Let $G'$ be the vertical$/Z'$ part of $\Sigma_{X'}$, $\tilde B'^v$ the vertical$/Z'$ part of $\tilde B'$, $E^h:=-(\tilde B'^h)^{\leq 0}$, and $E^v:=G'-\tilde B'^v$. Then $E^h\geq 0$ and $E^v$ is vertical$/Z'$. By Lemma \ref{lem: lc trivial (3) invariant under pullback}, $\kappa_{\sigma}(X'/Z,E^h)=0$. We have
\begin{align*}
    K_{X'}+B'+G'+\Mm_{X'}&=h^*(K_X+B+\Mm_X)+B'+G'-\tilde B'\\
    &\sim_{\mathbb R,Z'}\left(B'-\tilde B'^h\right)+G'-\tilde B'^v=-\left(\tilde B'^h\right)^{\leq 0}+\left(G'-\tilde B'^v\right)=E^h+E^v.
\end{align*}
Since $(X,B,\Mm)$ is sub-lc over the generic point of $Z$, $\Sigma_{X'}\geq B'\geq 0$. Let $\Ff'$ be the foliation induced by $f': X'\rightarrow Z'$. By Lemma \ref{lem: existence foliated log resolution}, $(X',\Ff',B',\Mm;G')/Z'$ is ACSS. By Proposition \ref{prop: weak cbf gfq}, 
$$K_{\Ff'}+B'+\Mm_{X'}\sim_{\mathbb R,Z'}K_{X'}+B'+G'+\Mm_{X'}\sim_{\mathbb R,Z'}E^h+E^v.$$
Thus
$$\kappa_{\sigma}(X'/Z',K_{\Ff'}+B'+\Mm_{X'})=\kappa_{\sigma}(X'/Z',E^h)=0.$$
By Proposition \ref{prop: weak ss num 0 mmp}, we may run a $(K_{\Ff'}+B'+\Mm_{X'})$-MMP$/Z'$ which terminates with a good minimal model $(X'',\Ff'',B'',\Mm)/Z'$ of $(X',\Ff',B',\Mm)/Z'$. Let  $G''$ be the image of $G'$ on $X''$. By Lemma \ref{lem: ACSS mmp can run}, $(X'',\Ff'',B'',\Mm;G'')/Z'$ is ACSS.

Since $X'\rightarrow U$ factors through $Z'$, $X'\dashrightarrow X''$ is a sequences of steps of a $(K_{\Ff'}+B'+\Mm_{X'})$-MMP$/U$. By Lemma \ref{lem: equivalence over bases}, $K_{\Ff''}+B''+\Mm_{X''}$ is nef$/U$. By Theorem \ref{thm: lc+weak acc=bpstable}, $(X'',B''+G'',\Mm)$ is BP stable$/Z'$. Let $f'': X''\rightarrow Z'$ be the induced contraction and let $\Nn$ be the moduli part of $f'': (X'',B''+G'',\Mm)\rightarrow Z'$. By Proposition \ref{prop: bp stable nef}, $\Nn$ is nef$/U$ and $\Nn$ descends to $X$. By Proposition \ref{prop: weak cbf gfq}, $$K_{X''}+B''+G''+\Mm_{X''}\sim_{\mathbb R,Z'}0.$$
Let $\Mm'$ be the base moduli part of  $f'': (X'',B''+G'',\Mm)\rightarrow Z'$, then by the definition of base moduli part, $\Mm'$ descends to $Z'$ and $f''^*\Mm'_{X'}=\Nn_{X''}$ is nef, so $\Mm'_{X'}$ is nef, hence $\Mm'$ is nef.

Let $\tilde B''^h$ be the image of $\tilde B'^h$ on $X''$. Since $K_{X'}+\tilde B'+\Mm_{X'}\sim_{\mathbb R,Z'}0$,  $K_{X'}+\tilde B'^h+\Mm_{X'}\sim_{\mathbb R}0$ over the generic point of $Z'$. Thus $K_{X''}+\tilde B''^h+\Mm_{X''}\sim_{\mathbb R}0$ over the generic point of $Z'$. Since $B''\geq\tilde B''^h$ and $K_{X''}+B''+\Mm_{X''}\sim_{\mathbb R,Z'}0$, $B''=\tilde B''^h$ over the generic point of $Z'$. Since $(X',\tilde B'^h,\Mm)$ and $(X'',\tilde B''^h,\Mm)$ are crepant over the generic point of $Z'$, $(X',\tilde B',\Mm)$ and $(X'',B''+G'',\Mm)$ are crepant over the generic point of $Z'$. Thus $(X,B,\Mm)$ and $(X'',B''+G'',\Mm)$ are crepant over the generic point of $Z$. By Lemma \ref{lem: m preserved under crepant}, $\Mm^Z=\Mm'$. The main part of the theorem follows. (1) immediately follows.

(2-4) immediately from the definition of the discriminant part. (5) follows from the definition of the discriminant part and Lemma \ref{lem: alg int foliation lct achieved}. By \cite[Theorem 2.23]{JLX22}, if $\Mm$ is NQC$/U$, then $\Mm'$ is NQC$/U$, hence $\Mm^Z$ is NQC$/U$. (6) follows.
\end{proof}

\subsection{Canonical bundle formula for generalized foliated quadruples}\label{subsec: cbf gfq}

\begin{deflem}\label{deflem: cbf gfq}
    Let $(X,\Ff,B,\Mm)/U$ be a sub-gfq and $f: X\rightarrow Z$ a contraction$/U$, such that the general fibers of $f$ are tangent to $\Ff$ and $f: (X,\Ff,B,\Mm)\rightarrow Z$ is an lc-trivial fibration. We define two $\bb$-divisors $\Bb$ and $\Mm^Z$ on $Z$ in the following way.

 By Lemma \ref{lem: gen fiber tangent mean induce}, there exists a foliation $\Ff_Z$ on $Z$ such that $\Ff=f^{-1}\Ff_Z$. Let $f': (X',\Sigma_{X'},\Mm)\rightarrow (Z',\Sigma_{Z'})$ be any equi-dimensional model of $f: (X,B,\Mm)\rightarrow Z$ with associated morphisms $h: X'\rightarrow X$ and $h_Z: Z'\rightarrow Z$. Let $\Ff_{Z'}:=h_Z^{-1}\Ff_Z$ and $\Ff':=h^*\Ff$, then $\Ff'=f'^{-1}\Ff_{Z'}$. We define
 $$R':=\sum\left(f'^*D-f^{-1}\left(D\right)\right),$$
where $D$ runs over all $\Ff_{Z'}$-invariant prime divisors on $Z$. By \cite[2.9]{Dru17}, we have 
 $$K_{\Ff'/\Ff_{Z'}}=K_{X'/Z'}-R'.$$
Let $K_{\Ff'}+B'+\Mm_{X'}:=h^*(K_{\Ff}+B+\Mm_X)$. Then $K_{\Ff'}'+B'+\Mm_{X'}\sim_{\mathbb R,Z'}0$, so 
$$K_{X'}+B'-R'+\Mm_{X'}\sim_{\mathbb R,Z'}0.$$
Since $R'=0$ and $K_{X'}=K_{\Ff'}$ over the generic point of $Z'$, $f': (X',B'-R',\Mm)\rightarrow Z'$ is an lc-trivial fibration. By Theorem \ref{thm: cbf gpair nonnqc}, there exist two $\bb$-divisors $\Bb$ and $\Mm^Z$ on $Z$, such that $\Bb$ is uniquely determined and $\Mm^Z$ is uniquely determined up to $\Rr$-linear equivalence, and the following conditions are satisfied:
    \begin{enumerate}
        \item[(i)] $K_{X'}+B'-R'+\Mm_{X'}\sim_{\mathbb R}f'^*(K_{Z'}+\Bb_{Z'}+\Mm^Z_{Z'})$.
        \item[(ii)] $\Mm^Z$ is nef$/U$.
        \item[(iii)] For any birational morphism $g_Z: Z''\rightarrow Z'$ and $g: X''\rightarrow X'$ such that the induced map $f'': X''\dashrightarrow Z''$ is a morphism, we let 
        $$K_{X''}+\tilde B''+\Mm_{X''}:=g^*(K_{X'}+B'-R'+\Mm_{X'}),$$ 
        then $\Bb_{Z''}$ is the discriminant part of $f'': (X'',\tilde B'',\Mm)\rightarrow Z''$.
    \end{enumerate}
 We call $\Bb$ as the \emph{discriminant $\bb$-divisor} of $f: (X,\Ff,B,\Mm)\rightarrow Z$ and call $\Mm^Z$ as the \emph{base moduli part} of $f: (X,\Ff,B,\Mm)\rightarrow Z$. We also call $\Bb_Z$ the \emph{discriminant part} of $f: (X,\Ff,B,\Mm)\rightarrow Z$. Then:
    \begin{enumerate}
      \item  $\Bb$ and $\Mm^Z$ are well-defined, i.e. $\Bb$ 
 and the $\Rr$-linear equivalence class of $\Mm^Z$ are independent of the choices of the equi-dimensional model of $f: (X,B,\Mm)\rightarrow Z$.  
 \item $(Z,\Ff_Z,B_Z:=\Bb_Z,\Mm^Z)/U$ is a sub-gfq.  We say that $(Z,\Ff_Z,B_Z,\Mm^Z)/U$ is a sub-gfq \emph{induced by a canonical bundle formula$/U$} of $f: (X,\Ff,B,\Mm)\rightarrow Z$.
 \item If $\Mm$ is NQC$/U$, then $\Mm^Z$ is NQC$/U$.
    \end{enumerate}
\end{deflem}

\begin{proof}
By \cite[Definition-Lemma 6.11]{LLM23}, $\Bb$ is independent of the choices of the equi-dimensional model of $f: (X,B,\Mm)\rightarrow Z$. 

Since $K_{\Ff}+B+\Mm_X\sim_{\mathbb R,Z}0$, there exists an $\Rr$-divisor $L$ on $Z$ which is uniquely determined up to $\Rr$-linear equivalence, such that
$$K_{\Ff}+B+\Mm_X\sim_{\mathbb R}f^*L.$$
By condition (i), we have
$$K_{\Ff'}+B'+\Mm_{X'}\sim_{\mathbb R}f'^*\left(K_{\Ff_{Z'}}+\Bb_{Z'}+\Mm^Z_{Z'}\right).$$
Therefore, for any birational morphism $g_Z: Z''\rightarrow Z'$ with $\Ff_{Z''}:=g_Z^{-1}\Ff_{Z'}$, we have
$$\Mm^Z_{Z''}\sim_{\mathbb R}(h_Z\circ g_Z)^*L-K_{\Ff_{Z''}}-\Bb_{Z''}.$$
Thus $\Mm^Z_{Z''}$ is uniquely determined up to the choices of $L$ in its $\Rr$-linear equivalence class, so $\Mm^Z$ is uniquely determined up to $\Rr$-linear equivalence. This implies (1).

We have
$$L=(h_Z)_*h_Z^*L\sim_{\mathbb R}(h_Z)_*\left(K_{\Ff_{Z'}}+\Bb_{Z'}+\Mm^Z_{Z'}\right)=K_{\Ff_Z}+B_Z+\Mm^Z_Z,$$
so $K_{\Ff_Z}+B_Z+\Mm^Z_Z$ is $\Rr$-Cartier. By condition (ii), $(Z,\Ff_Z,B_Z:=\Bb_Z,\Mm^Z)/U$ is a sub-gfq. This implies (2).

(3) follows from Theorem \ref{thm: cbf gpair nonnqc}(6).
\end{proof}


\begin{lem}\label{lem: order along generic fiber cbf gfq}
Let $(X,\Ff,B,\Mm)/U$ be a sub-gfq and $f: X\rightarrow Z$ a contraction$/U$ such that $f: (X,\Ff,B,\Mm)\rightarrow Z$ is an lc-trivial fibration. Let $B_Z$ be the discriminant part of $f: (X,\Ff,B,\Mm)\rightarrow Z$ and $\Ff_Z$ a foliation on $Z$ such that $\Ff=f^{-1}\Ff_Z$. Let $n$ be a positive integer such that $n(K_{\Ff}+B+\Mm_X)\sim 0$ over the generic point of $Z$. Then there is a choice $\Mm^Z$ of the base moduli part of  $f: (X,\Ff,B,\Mm)\rightarrow Z$, such that 
$$n(K_{\Ff}+B+\Mm_X)\sim nf^*\left(K_{\Ff_Z}+B_Z+\Mm^Z_Z\right).$$
\end{lem}
\begin{proof}
Let $f': (X',\Sigma_{X'},\Mm)\rightarrow (Z',\Sigma_{Z'})$ be a sufficiently high equi-dimensional model of $f: (X,B,\Mm)\rightarrow Z$ with associated morphisms $h: X'\rightarrow X$ and $h_Z: Z'\rightarrow Z$. Let $\Ff_{Z'}:=h_Z^{-1}\Ff_Z$ and let  
$$R':=\sum_{D\mid D\text{ is an }\Ff_{Z'}\text{-invariant prime divisor}}(f'^*D-f'^{-1}(D)).$$
Then $f': (X',B'-R',\Mm)\rightarrow Z'$ is an lc-trivial fibration.
%and $\Mm^Z$ is the base moduli part of $f': (X',B'-R',\Mm)\rightarrow Z'$. 
Since $R'$ is vertical$/Z'$, $n(K_{X'}+B'-R'+\Mm_{X'})\sim 0$ over the generic point of $Z$. The lemma follows from Lemma \ref{lem: order along generic fiber cbf}.
\end{proof}



\begin{lem}\label{lem: m preserved under crepant gfq}
Let $(X,\Ff,B,\Mm)/U$ and $(X',\Ff',B',\Mm)/U$ be two sub-gfqs. Let $f: (X,\Ff,B,\Mm)\rightarrow Z$ and $f': (X',\Ff',B',\Mm)\rightarrow Z'$ be two lc-trivial fibrations$/U$ such that $f$ and $f'$ are birationally equivalent, and $(X,\Ff,B,\Mm)$ and $(X',\Ff',B',\Mm)$ are crepant over the generic point of $Z$. Let $\Mm^Z$ be the base moduli part of $f: (X,\Ff,B,\Mm)\rightarrow Z$  and let $\Mm^{Z'}$ be the base moduli part of $f': (X',\Ff',B',\Mm)\rightarrow Z$. Then $\Mm^Z\sim_{\mathbb R}\Mm^{Z'}$.
\end{lem}
\begin{proof}
Possibly passing to a common base and resolve indeterminacy of the induced birational map $X\dashrightarrow X'$, we may assume that $f=f'$, $X=X'$, $Z=Z'$, and $\Ff=\Ff'$ over the generic point of $Z$, $f: (X,\Sigma)\rightarrow (Z,\Sigma_Z)$ is equi-dimensional toroidal for some $\Sigma\supset\Supp B\cup\Supp B'$, and $(Z,\Sigma_Z)$ is log smooth. Let $\Ff_Z$ and $\Ff_Z'$ be two foliations on $Z$ such that $\Ff=f^{-1}\Ff_Z$ and $\Ff'=f'^{-1}\Ff_Z'$,
 $$R:=\sum_{D\mid D\text{ is an }\Ff_{Z}\text{-invariant prime divisor}}(f^*D-f^{-1}(D)),$$
 and
$$R':=\sum_{D\mid D\text{ is an }\Ff'_{Z}\text{-invariant prime divisor}}(f^*D-f^{-1}(D)).$$
Then $\Mm^Z$ and $\Mm^{Z'}$ are the moduli parts of $f: (X,B-R,\Mm)\rightarrow Z$ and $f': (X,B'-R',\Mm)\rightarrow Z$ respectively. Since $(X,B-R,\Mm)$ and $(X',B'-R',\Mm)$ are crepant over the generic point of $Z$, by Lemma \ref{lem: m preserved under crepant}, $\Mm^Z\sim_{\mathbb R}\Mm^{Z'}$.
\end{proof}

%$B_Z:=\Bb_Z$ the discriminant part of $f: (X,\Ff,B,\Mm)\rightarrow Z$,
\begin{lem}\label{lem: td=bd}
 Let $(X,\Ff,B,\Mm)/U$ be a sub-gfq and $f: X\rightarrow Z$ a contraction$/U$ such that $f: (X,\Ff,B,\Mm)\rightarrow Z$ is an lc-trivial fibration with discriminant $\bb$-divisor $\Bb$. Let $\Ff_Z$ be a foliation on $Z$ such that $\Ff=f^{-1}\Ff_Z$. Then for any prime divisor $D$ on $Z$, 
$$\mult_D\Bb_Z=\epsilon_{\Ff_Z}(D)-\sup\{t\geq 0\mid (X,\Ff,B+tf^*D,\Mm)\text{ is sub-lc over the generic point of } D\}.$$
Moreover, there exists an lc center of $$(X,\Ff,B+(\epsilon_{\Ff_Z}(D)-\mult_D\Bb_Z)f^*D,\Mm)$$ over the generic point of $D$.
\end{lem}
\begin{proof}
Let $B_Z:=\Bb_Z$. By Definition-Lemma \ref{deflem: cbf gfq}, possibly replacing $f: X\rightarrow Z$ with an equi-dimensional model of $f: (X,B,\Mm)\rightarrow Z$, we may assume that $X$ is $\Qq$-factorial klt with at most toric quotient singularities, $f$ is equi-dimensional, $\Mm$ descends to $X$, and there exists a toroidal morphism $f: (X,\Sigma_X,\Mm)\rightarrow (Z,\Sigma_Z)$ such that $\Supp B\subset\Sigma_X$. We define
$$R:=\sum_{D\mid D\text{ is an }\Ff_Z\text{-invariant prime divisor}}(f^*D-f^{-1}(D)).$$ 

For any prime divisor $D$ on $Z$, we define 
$$b_D:=1-\sup\{t\geq 0\mid (X,B-R+tf^*D,\Mm)\text{ is lc over the generic point of } D\}$$
and
$$t_D:=\epsilon_{\Ff_Z}(D)-\sup\{t\geq 0\mid (X,\Ff,B+tf^*D,\Mm)\text{ is lc over the generic point of } D\}.$$
By definition, $\mult_DB_Z=b_D$ for any prime divisor $D$ on $Z$. There are three cases.

\medskip

\noindent\textbf{Case 1}. $D$ is not $\Ff_Z$-invariant. In this case, $R=0$ and $K_{\Ff}=K_X$ over the generic point of $D$, so
\begin{align*}
    &\sup\{t\mid (X,B-R+tf^*D,\Mm)\text{ is sub-lc over the generic point of } D\}\\
    =&\sup\{t\mid (X,\Ff,B+tf^*D,\Mm)\text{ is sub-lc over the generic point of } D\}.
\end{align*}
Thus $b_D=t_D$. Moreover, any lc center of $(X,B-R+(1-b_D)f^*D,\Mm)$ over the generic point of $D$ is an lc center of $(X,\Ff,B+(1-b_D)f^*D,\Mm)$ over the generic point of $D$. Since $(X,B-R+(1-b_D)f^*D,\Mm)$ is a g-sub-pair over the generic point of $D$, by Lemma \ref{lem: alg int foliation lct achieved}, there exists an lc center of  $(X,B-R+(1-b_D)f^*D,\Mm)$ over the generic point of $D$. Thus there exists an lc center of $(X,\Ff,B+(1-b_D)f^*D,\Mm)$ over the generic point of $D$.

\medskip

\noindent\textbf{Case 2}. $D$ is $\Ff_Z$-invariant and $D\not\subset\Sigma_Z$. Let $B^h$ be the horizontal$/Z$ part of $B$, then $B=B^h$ over the generic point of $D$. Since $(X,\Ff,B,\Mm)$ is sub-lc over the generic point of $Z$, $\Sigma_X\geq B^h$. By \cite[Lemma 6.6]{LLM23}, $(X,B^h+f^{-1}(D),\Mm)$ is sub-lc over the generic point of $D$. Since
$$(X,B-R+f^*D,\Mm)=(X,B^h+f^{-1}(D),\Mm)$$
over the generic point of $D$, $b_D=0$. Thus $\mult_DB_Z=0$. Since $D$ is $\Ff_Z$-invariant, any component of $f^{-1}(D)$ is $\Ff$-invariant. Since $B=B^h$ over the generic point of $D$, any component of $f^{-1}(D)$ is an lc center of $(X,\Ff,B,\Mm)$. In particular,
$b_D=0=t_D.$

\medskip

\noindent\textbf{Case 3}. $D$ is $\Ff_Z$-invariant and $D\subset\Sigma_Z$. Then
$$-b_D=\sup\{t\mid (X,B+f^{-1}(D)+tf^*D,\Mm)\text{ is sub-lc over the generic point of } D\}.$$
Since $f: (X,\Sigma_X,\Mm)\rightarrow (Z,\Sigma_Z)$ is toroidal, there exists a component $S$ of $f^*D$, such that
\begin{itemize}
    \item $\mult_S(B+f^{-1}(D)-b_Df^*D)=1$, and
    \item $0\geq \lfloor B+f^{-1}(D)-tf^*D\rfloor$ over the generic point of $D$ for any $t<-b_D$.
\end{itemize}
Therefore, $\mult_S(B-b_Df^*D)=0$, and $0\geq B-b_Df^*D$ over the generic point of $D$. Thus
$$-b_D\geq \sup\{t\geq 0\mid (X,\Ff,B+tf^*D,\Mm)\text{ is sub-lc over the generic point of } D\}=-t_D.$$
Suppose that $-b_D>-t_D$. Let $s\in (-t_D,-b_D)$ be a real number, then $$(X,B+f^{-1}(D)+sf^*D,\Mm)$$ is sub-lc over the generic point of $D$, and $(X,\Ff,B+sf^*D,\Mm)$ is not sub-lc over the generic point of $D$. Then there exists a prime divisor $D_X$ over $X$, such that the image of $D_X$ on $Z$ is $D$, and $a(D_X,\Ff,B+sf^*D,\Mm)<-\epsilon_{\Ff}(D_X)$. By Definition-Theorem \ref{defthm: weak ss reduction}, there exists an equi-dimensional model $f': (X',\Sigma_{X'},\Mm)\rightarrow (Z',\Sigma_{Z'})$ of $f: (X,\Supp B+\Supp f^*D,\Mm)\rightarrow Z$ associated with $h: X'\rightarrow X$ and $h_Z: Z'\rightarrow Z$, such that $D_X$ is on $X'$. Let $\Ff':=h^{-1}\Ff$, $\Ff_{Z'}:=h_Z^{-1}\Ff_Z$, $K_{\Ff'}+B'+\Mm_{X'}:=h^*(K_{\Ff}+B+\Mm_X)$, $D':=(h_Z^{-1})_*D$, and 
$$R':=\sum_{L\mid L\text{ is an }\Ff_{Z'}\text{-invariant prime divisor}}(f'^*L-f'^{-1}(L)).$$
Then $D_X$ is a component of $f'^{-1}(D')$. Since $D'$ is $\Ff_{Z'}$-invariant and $\Ff'=f'^{-1}\Ff_{Z'}$, $D_{X}$ is $\Ff'$-invariant. Since $a(D_X,\Ff,B+sf^*D,\Mm)<-\epsilon_{\Ff}(D_X)$, $\mult_{D_X}(B'+sf'^*D')>0$. By Definition-Lemma \ref{deflem: cbf gfq}(1),
\begin{align*}
   -b_D&=\sup\{t\geq 0\mid (X',B'-R'+tf'^*D',\Mm)\text{ is lc over the generic point of } D'\}-1\\
   &=\sup\{t\geq 0\mid (X',B'+f'^{-1}(D)+tf'^*D',\Mm)\text{ is lc over the generic point of } D'\}\\
   &<s<-b_D,
\end{align*}
a contradiction. Thus $b_D=t_D$. Since $\mult_S(B-t_Df^*D)=0$, $S$ is an lc center of $(X,B-b_Df^*D,\Mm)$ over the generic point of $D$. The lemma follows in this case.
\end{proof}


\begin{prop}\label{prop: gfq cbf preserve sing}
Let $(X,\Ff,B,\Mm)/U$ be a sub-gfq and $f: X\rightarrow Z$ a contraction$/U$ such that $f: (X,\Ff,B,\Mm)\rightarrow Z$ is an lc-trivial fibration. Let $\Bb$ be the discriminant $\bb$-divisor of $f: (X,\Ff,B,\Mm)\rightarrow Z$, $B_Z:=\Bb_Z$, and $\Mm^Z$ the base moduli part of  $f: (X,\Ff,B,\Mm)\rightarrow Z$. Let $\Ff_Z$ be a foliation on $Z$ such that $\Ff=f^{-1}\Ff_Z$. Then:
\begin{enumerate}
    \item If the vertical$/Z$ part of $B$ is $\geq 0$, then $B_Z\geq 0$.
    \item If $(X,\Ff,B,\Mm)$ is sub-lc (resp. lc), then $(Z,\Ff_Z,B_Z,\Mm^Z)$ is sub-lc (resp. lc).
    \item Any lc center of $(Z,\Ff_Z,B_Z,\Mm^Z)$ is the image of an lc center of $(X,\Ff,B,\Mm)$.
    \item The image of any lc center of $(X,\Ff,B,\Mm)$ on $Z$ is an lc center of $(Z,\Ff_Z,B_Z,\Mm^Z)$.
\end{enumerate}
\end{prop}
\begin{proof}
The proposition immediately follows from Lemma \ref{lem: td=bd}.
\end{proof}

Finally, we state the following proposition that can be useful for inductive purposes.

\begin{prop}\label{prop: composition lc trivial fibration}
    Let $(X,\Ff,B,\Mm)$ be a sub-gfq and $X\xrightarrow{f}Y\xrightarrow{g}Z$ two contractions$/U$. Let $h:=g\circ f$. Suppose that $h: (X,\Ff,B,\Mm)\rightarrow Z$ is an lc-trivial fibration. Let $(Z,\Ff_Z,B_Z,\Mm^Z)$ be the sub-gfq induced by $h: (X,\Ff,B,\Mm)\rightarrow Z$. Then:
    \begin{enumerate}
        \item $f: (X,\Ff,B,\Mm)\rightarrow Y$ is an lc-trivial fibration.
        \item Let $(Y,\Ff_Y,B_Y,\Mm^Y)$ be a sub-gfq induced by $f: (X,\Ff,B,\Mm)\rightarrow Y$. Then:
        \begin{enumerate}
            \item $g: (Y,\Ff_Y,B_Y,\Mm^Y)\rightarrow Z$ is an lc-trivial fibration.
            \item The discriminant part of $g: (Y,\Ff_Y,B_Y,\Mm^Y)\rightarrow Z$ is $B_Z$.
            \item $(Z,\Ff_Z,B_Z,\Mm^Z)$ is a sub-gfq induced by  $g: (Y,\Ff_Y,B_Y,\Mm^Y)\rightarrow Z$.
        \end{enumerate}
    \end{enumerate}
\end{prop}
\begin{proof}
Possibly replacing $X$ and $Y$ with high models, we may assume that $X$ and $Y$ are smooth, and $\kappa_{\sigma}(X/Z,-B^{\leq 0})=0.$

(1) Since $(X,\Ff,B,\Mm)$ is sub-lc over the generic point of $Z$,  $(X,\Ff,B,\Mm)$ is sub-lc over the generic point of $Y$. Since $K_{\Ff}+B+\Mm_X\sim_{\mathbb R,Z}0,$ $K_{\Ff}+B+\Mm_X\sim_{\mathbb R,Y}0$. Since $\kappa_{\sigma}(X/Z,-B^{\leq 0})=0$, $\kappa_{\sigma}(X/Y,-B^{\leq 0})=0$. This implies (1).

(2.a) Since $(X,\Ff,B,\Mm)$ is sub-lc over the generic point of $Z$, by Theorem \ref{thm: cbf gpair nonnqc}, $(Y,\Ff_Y,B_Y,\Mm^Y)$ is sub-lc over the generic point of $Z$. Since 
$$f^*(K_{\Ff_Y}+B_Y+\Mm_Y)\sim_{\mathbb R}K_{\Ff}+B+\Mm_X\sim_{\mathbb R,Z}0,$$
    $K_{\Ff_Y}+B_Y+\Mm_Y\sim_{\mathbb R,Z}0$. By Lemma \ref{lem: td=bd}, for any component $D$ of $B_Y^{\leq 0}$ and any irreducible component $D_X$ of $f^{-1}(D)$ over the generic point of $D$, $D_X$ is a component of $B^{\leq 0}$. Therefore, over the generic point of $Z$, there exists a positive real number $\delta$ such that
    $$-B^{\leq 0}\geq \epsilon f^*(-B_Y^{\leq 0}).$$
    Thus
    $$0\leq \kappa_{\sigma}(X/Z,f^*(-B_Y^{\leq 0}))=\kappa_{\sigma}(X/Z,\epsilon f^*(-B_Y^{\leq 0}))\leq \kappa_{\sigma}(X/Z,-B^{\leq 0})=0,$$
    so
    $$\kappa_{\sigma}(Y/Z,-B_Y^{\leq 0})=\kappa_{\sigma}(X/Z,f^*(-B_Y^{\leq 0}))=0.$$
    Therefore, $g: (Y,\Ff_Y,B_Y,\Mm^Y)\rightarrow Z$ is an lc-trivial fibration.

(2.b) Let $B_{Z}'$ be the discriminant part of $g: (Y,\Ff_Y,B_Y,\Mm^Y)\rightarrow Z$. For any prime divisor $D$ over $Z$, let $s_D:=\epsilon_{\Ff_Z}(D)-\mult_DB_Z$ and  $s'_D:=\epsilon_{\Ff_Z}(D)-\mult_DB_Z'$. 

By Lemma \ref{lem: td=bd}, for any positive real number $t$ and any prime divisor $D$ on $Z$, 
$$(Y,\Ff_Y,B_Y+tg^*D,\Mm)$$ 
is the sub-gfq induced by $f: (X,\Ff,B+th^*D,\Mm)\rightarrow Y$ over the generic point of $D$. By Proposition \ref{prop: gfq cbf preserve sing}(3)(4),
\begin{align*}
s'_D=&\sup\{t\geq 0\mid (Y,\Ff_Y,B_Y+tg^*D,\Mm^Y)\text{ is sub-lc over the generic point of }D\}\\
=&\sup\{t\geq 0\mid (X,\Ff,B+th^*D,\Mm)\text{ is sub-lc over the generic point of }D\}=s_D.
\end{align*}
Thus $B_Z=B_Z'$. 

(2.c) By applying (2.b) to all high models of $Z$, we get (2.c).
\end{proof}



\section{Canonical bundle formula for lc-trivial morphisms and subadjunction formula}\label{sec: subadj}

\subsection{Canonical bundle formula for lc-trivial morphisms}

\begin{deflem}[{\cite[Proposition 3.4]{Dru21}; cf. \cite[Proposition 3.7]{Spi20}}]\label{deflem: hurwitz foliation}
    Let $f: X'\rightarrow X$ be a surjective finite morphism between normal varieties and $\Ff$ a foliation on $X$. Assume that $K_{\Ff}$ is $\Qq$-Cartier and $\Ff':=f^{-1}\Ff$. For any prime divisor on $X$, we let $r_D$ be the ramification index of $f$ along $D$. We call
    $$R:=\sum_{D\mid D\text{ is a non-}\Ff\text{-invariant prime divisor}}(r_D-1)D$$
    the \emph{ramification divisor} of $f$ with respect to $\Ff$. Then we have
    $$K_{\Ff'}=f^*K_{\Ff}+R.$$
\end{deflem}

\begin{deflem}\label{deflem: cbf finite}
Let $(X,\Ff,B,\Mm)/U$ be a sub-gfq and $f: X\rightarrow Z$ a finite morphism$/U$. Suppose that there exists a foliation $\Ff_Z$ on $Z$ such that $\Ff=f^{-1}\Ff_Z$, and suppose that $K_{\Ff}+B+\Mm_X\sim_{\mathbb R,Z}0$.

We define two $\bb$-divisors, $\Bb$ on $\Mm^Z$ on $Z$, in the following way. Let $h_Z: Z'\rightarrow Z$ be any birational morphism, $X'$ the main component of $Z'\times_{Z}X$, $f': X'\rightarrow Z'$ and $h: X'\rightarrow X$ the induced morphisms, $\Ff':=h^{-1}\Ff$, and $\Ff_{Z'}:=h_Z^{-1}\Ff_Z$. We let $$K_{\Ff'}+B'+\Mm_{X'}:=h^*(K_{\Ff}+B+\Mm_X).$$
Let $Z'^0$ be the largest open subset of $Z'$ which does not contain $\Sing(\Ff_{Z'})\cup\Sing(Z')$ and let $X'^0:=f'^{-1}(Z'^0)$. By Definition-Lemma \ref{deflem: hurwitz foliation}, 
$$K_{\Ff'|_{X'^0}}=(f'|_{X'^0})^*K_{\Ff_{Z'}|_{Z'^0}}+R'^0$$
where $R'^0$ is the ramification divisor of $f'|_{X'^0}$ with respect to $\Ff_{Z'}|_{Z'^0}$. We let $R'$ be the closure of $R'^o$ in $X'^o$. We let $\Bb$ and $\Mm^Z$ be the $\bb$-divisors such that $\Bb_{Z'}=\frac{1}{\deg f}f'_*(R'+B')$ and $\Mm^Z_{Z'}=\frac{1}{\deg f}f'_*\Mm_{X'}$ for any choices of $Z'$. Then:
\begin{enumerate}
   \item $\Bb$ and $\Mm$ are well-defined and uniquely determined.
    \item For any choice of $Z'$,
    $$K_{\Ff'}+B'+\Mm_{X'}\sim_{\mathbb R}f'^*(K_{\Ff_{Z'}}+B_{Z'}+\Mm^Z_{Z'}).$$
    \item $\Mm^Z$ is nef$/U$.
    \item If $B\geq 0$, then $\Bb_Z\geq 0$.
    \item If $(X,\Ff,B,\Mm)$ is (sub-)lc, then $(Z,\Ff_Z,\Bb_Z,\Mm^Z)$ is (sub-)lc, and for any lc center $T$ of $(Z,\Ff_Z,B_Z,\Mm^Z)$, any component of $f^{-1}(T)$ is an lc center of $(X,\Ff,B,\Mm)$.
    \item If $\Mm$ is NQC$/U$, then $\Mm^Z$ is NQC$/U$.
\end{enumerate}
We call $\Bb$ the \emph{discriminant $\bb$-divisor} of $f: (X,B,\Mm)\rightarrow Z$, and call $B_Z:=\Bb_Z$ the \emph{discrminant part} of $f: (X,B,\Mm)\rightarrow Z$. We call $\Mm^Z$ the \emph{base moduli part} of $f: (X,B,\Mm)\rightarrow Z$. We say that $(Z,\Ff_Z,B_Z,\Mm^Z)/U$ is the sub-gfq \emph{induced} by $f: (X,\Ff,B,\Mm)\rightarrow Z$.
\end{deflem}
\begin{proof}
(1) We only need to show that for any birational morphism $g_Z: Z''\rightarrow Z'$, $(g_Z)_*\Bb_{Z''}=\Bb_{Z'}$ and $(g_Z)_*\Mm^Z_{Z''}=\Mm^Z_{Z'}$. We let $X''$ be the main component of $X'\times_{Z'}Z''$ and $g: X''\rightarrow X'$, $f'': X''\rightarrow Z''$ the induced morphisms. Let $\Ff'':=g^{-1}\Ff',\Ff_{Z''}:=g^{-1}_Z\Ff_{Z'}$, $Z''^0$ be the largest open subset of $Z''$ which does not contain $\Sing(\Ff_{Z''})\cup\Sing(Z'')$, $X''^0:=f'^{-1}(Z''^0)$, $R''^0$ the ramification divisor of $f''|_{X''^0}$ with respect to $\Ff_{Z''}|_{Z''^0}$, and $R''$ the closure of $R''^0$ in $X''$. Then
$$\Bb_{Z'}=\frac{1}{\deg f}f'_*(B'+R')=\frac{1}{\deg f}f'_*g_*(B''+R'')=\frac{1}{\deg f}(g_Z)_*f''_*(B''+R'')=(g_Z)_*\Bb_{Z''}$$
and
$$\Mm^Z_{Z'}=\frac{1}{\deg f}f'_*\Mm_{X'}=\frac{1}{\deg f}f'_*g_*\Mm_{X''}=\frac{1}{\deg f}(g_Z)_*f''_*\Mm_{X''}=(g_Z)_*\Mm^Z_{Z''}.$$

(2) By (1), we only need to prove (2) for any sufficiently high model $Z'$ of $Z$. In particular, we may assume that $Z'$ is $\Qq$-factorial. Then $f'^*(\frac{1}{\deg f}f'_*R')=R'$, $f'^*(\frac{1}{\deg f}f'_*B')=B'$, and $f'^*(\frac{1}{\deg f}f'_*\Mm_{X'})=\Mm_{X'}$, so (2) immediately follows.

(3)(6) By \cite[Lemma 4.2]{HL21b}, there exists a birational morphism $h_Z: Z''\rightarrow Z$ satisfying the following. Let $X''$ be the main component of $Z''\times_ZX$, then $\Mm$ descends to $X''$. By definition, $\Mm^Z$ descends to $Z''$. Since $\Mm_{X''}$ is nef, $\Mm^Z_{Z''}$ is nef. Thus $\Mm^Z$ is nef. This implies (3). Moreover, if $\Mm$ is NQC$/U$, then $\Mm_{X''}$ is NQC$/U$, so  $\Mm^Z_{Z''}$ is NQC$/U$, hence $\Mm^Z$ is NQC$/U$. This implies (6).

(4) It is obvious from the definition of $\Bb$.

(5) By (4), we only need to prove the sub-lc case. Suppose that $(X,\Ff,B,\Mm)$ is sub-lc, then $(X',\Ff',B',\Mm)$ is sub-lc. Let $D$ be a prime divisor on $Z'$. Let $E_1,\dots,E_m$ be all components of $f'^{-1}(D)$ and let $r_i$ be the ramification index of $r_i$ along $E_i$. 

If $D$ is $\Ff_{Z'}$-invariant, then each $E_i$ is $\Ff$-invariant $E_i\not\subset\Supp R'$. Since $(X',\Ff',B',\Mm)$ is sub-lc, $\mult_{E_i}B'\leq 0$ for any $i$. Thus
$$\mult_D\Bb_{Z'}=\mult_D\frac{1}{\deg f}f'_*(B'+R')=\sum_{i=1}^m\frac{1}{\deg f}(\mult_{E_i}B')\leq 0=\epsilon_{\Ff_{Z'}}(D).$$
Moreover, if $D$ is an lc place of $(Z,\Ff_Z,B_Z,\Mm^Z)$, then $\mult_D\Bb_{Z'}=0$, so $\mult_{E_i}B'=0$ for each $i$. Therefore, each $E_i$ is an lc place of $(X',\Ff',B',\Mm)$, hence an lc place of $(X,\Ff,B,\Mm)$.

If $D$ is $\Ff_Z$-invariant, then each $E_i$ is not $\Ff$-invariant, and $\sum_{i=1}^mr_i\leq\deg f$. Since $(X',\Ff',B',\Mm)$ is sub-lc, $\mult_{E_i}B'\leq 1$ for any $i$. Thus
$$\mult_D\Bb_{Z'}=\mult_D\frac{1}{\deg f}f'_*(B'+R')=\sum_{i=1}^m\frac{1}{\deg f}(r_i-1+\mult_{E_i}B')\leq\frac{\sum_{i=1}^mr_i}{\deg f}\leq 1=\epsilon_{\Ff_{Z'}}(D).$$
Moreover, if $D$ is an lc place of $(Z,\Ff_Z,B_Z,\Mm^Z)$, then $\mult_D\Bb_{Z'}=1$, so $\mult_{E_i}B'=1$ for each $i$. Therefore, each $E_i$ is an lc place of $(X',\Ff',B',\Mm)$, hence an lc place of $(X,\Ff,B,\Mm)$.

Since $h_Z: Z'\rightarrow Z$ can be any birational morphism, we get (5).
\end{proof}

\begin{defn}[lc-trivial morphism]\label{defn: lc trivial morphism}
Let $(X,\Ff,B,\Mm)/U$ be a sub-gfq and $f: X\rightarrow Z$ a projective surjective morphism over $U$. Let $X\xrightarrow{\tau}\tilde Z\xrightarrow{\gamma}Z$ be the Stein factorization of $f$. We say that $f: (X,\Ff,B,\Mm)\rightarrow Z$ is an \emph{lc-trivial morphism}, if
\begin{enumerate}
\item $K_{\Ff}+B+\Mm_X\sim_{\Rr,Z}0$,
\item $\tau: (X,\Ff,B,\Mm)\rightarrow\tilde Z$ is an lc-trivial fibration, and
\item there exists a foliation $\Ff_{Z}$ on $Z$ such that $\Ff=f^{-1}\Ff_Z$. 
\end{enumerate}
\end{defn}

\begin{defthm}[Canonical bundle formula for lc-trivial morphisms]\label{defthm: cbf lctrivial morphism}
    Let $$(X,\Ff,B,\Mm)/U$$ be a sub-gfq and $f: X\rightarrow Z$ an lc-trivial morphism$/U$, and let $\Ff_Z$ be a foliation on $Z$ such that $\Ff=f^{-1}\Ff_Z$. Then there is a sub-gfq $(Z,\Ff_Z,B_Z,\Mm^Z)/U$, such that $B_Z$ is uniquely determined and $\Mm^Z$ is determined up to $\Rr$-linear equivalence, defined in the following way.

    Let $X\xrightarrow{\tau}\tilde Z\xrightarrow{\gamma}Z$ be the Stein factorization of $f$. By Definition-Lemma \ref{deflem: cbf gfq}, there exists a sub-gfq $$(\tilde Z,\Ff_{\tilde Z},B_{\tilde Z},\tilde\Mm^Z)/U$$
    induced by $\tau: (X,\Ff,B,\Mm)\rightarrow\tilde Z$, such that $B_{\tilde Z}$ is uniquely determined, and  $\tilde\Mm^Z$ is uniquely determined up to $\Rr$-linear equivalence. Moreover, we have $\Ff_{\tilde Z}=\gamma^{-1}\Ff_Z$ and
    $$K_{\Ff_{\tilde Z}}+B_{\tilde Z}+\tilde\Mm^Z_{\tilde Z}\sim_{\mathbb R,Z}0.$$
    By Definition-Lemma \ref{deflem: cbf finite}, there exists a sub-gfq  
  $$(Z,\Ff_Z,B_Z,\Mm^Z)/U$$
  induced by $\gamma: (\tilde Z,\Ff_{\tilde Z},B_{\tilde Z},\tilde\Mm^Z)\rightarrow Z$, such that $B_Z$  is uniquely determined, and  $\Mm^Z$ is uniquely determined up to $\Rr$-linear equivalence. We say that $B_Z$ is the \emph{discriminant part} of $f: (X,\Ff,B,\Mm)\rightarrow Z$, $\Mm^Z$ the \emph{base moduli part} of  $f: (X,\Ff,B,\Mm)\rightarrow Z$, and say that $(Z,\Ff_Z,B_Z,\Mm^Z)$ is a sub-gfq induced by $f: (X,\Ff,B,\Mm)\rightarrow Z$.

 Moreover, we have the following:
\begin{enumerate}
\item If the vertical$/Z$ part of $B$ is $\geq 0$, then $B_Z\geq 0$.
\item  If $(X,\Ff,B,\Mm)$ is (sub-)lc, then $(Z,\Ff_Z,B_Z,\Mm^Z)$ is (sub-)lc.
\item $B_Z$ is uniquely determined, and $\Mm^Z$ is uniquely determined up to $\Rr$-linear equivalence.
\item Suppose that $(X,\Ff,B,\Mm)$ is sub-lc. Then for any lc center $T$ of $(Z,\Ff_Z,B_Z,\Mm^Z)$, $T$ is the image of an lc center of $(X,\Ff,B,\Mm)$ on $Z$.
\end{enumerate}
\end{defthm}
\begin{proof}
(1) It follows from Definition-Lemma \ref{deflem: cbf finite}(4) and Proposition \ref{prop: gfq cbf preserve sing}(1).

(2) It follows from Definition-Lemma \ref{deflem: cbf finite}(5) and Proposition \ref{prop: gfq cbf preserve sing}(2).

(3) It follows from Definition-Lemma \ref{deflem: cbf gfq}(1) and Definition-Lemma \ref{deflem: cbf finite}(1).

(4) It follows from Proposition \ref{prop: gfq cbf preserve sing}(3) and Definition-Lemma \ref{deflem: cbf finite}(5).
\end{proof}

\subsection{Subadjunction formula for g-pairs}

In this section, we shall introduce and discussion the subadjunction formula for lc g-pairs. Since the canonical bundle formula for lc-trivial fibrations for gfqs requires that the general fibers are tangent to the foliation, the subadjunction formula for foliations is more subtle and we will omit it in this paper.

\begin{defthm}[Subadjunction formula via log resolutions]\label{defthm: subadjun}
    Let $(X,B,\Mm)/U$ be a g-sub-pair and $V$ an lc center of $(X,B,\Mm)$ with normalization $\nu: W\rightarrow V$, such that $B\geq 0$ near the generic point of $V$. Then there exists a naturally defined g-sub-pair $(W,B_W,\Mm^W)/U$ defined in the following way. 
    
    Let $S$ be an lc place of $(X,B,\Mm)$ so that $\Center_XS=V$. Let $h: Y\rightarrow X$ be a log resolution of $(X,\Supp B)$ such that $\Mm$ descends to $Y$ and $S$ is on $Y$. We let 
    $$K_Y+B_Y+\Mm_Y:=h^*(K_X+B+\Mm_X)$$
    and let $(S,B_S,\Mm^S)/U$ be the g-sub-pair induced by the adjunction
    $$K_S+B_S+\Mm^S_S:=(K_Y+B_Y+\Mm_Y)|_S.$$
    Then there exists an induced projective surjective morphism $h_S: S\rightarrow W$ such that $\nu\circ f_S=h|_S$. By construction, we have
    $$K_S+B_S+\Mm^S_S\sim_{\mathbb R,W}0.$$
    Since  $B\geq 0$ near the generic point of $V$, $B_W\geq 0$ near the generic point of $S$. Therefore, $h_S: (S,B_S,\Mm^S)\rightarrow W$ is an lc-trivial morphism. By Definition-Theorem \ref{defthm: cbf lctrivial morphism}, there exists a g-sub-pair $(W,B_W,\Mm^W)/U$ induced by $h_S: (S,B_S,\Mm^S)\rightarrow W$. Moreover, we have the following:
    \begin{enumerate}
        \item For any fixed choice and $S$, $B_W$ is uniquely determined, and $\Mm^W$ is uniquely determined up to $\Rr$-linear equivalence. In particular, $B_W$ and the $\Rr$-linear equivalence class of $\Mm^W$ are independent of the choice of $h$.
        \item $K_W+B_W+\Mm_W\sim_{\mathbb R}(K_X+B+\Mm_X)|_{W}$.
        \item If $(X,B,\Mm)$ is sub-lc near $V$, then $(W,B_W,\Mm^W)$ is sub-lc.
         \item Suppose that $(X,B,\Mm)$ is sub-lc near $V$.  Then for any lc center $T$ of $(W,B_W,\Mm^W)$, $\nu(T)$ is an lc center of $(X,B,\Mm)$.
    \end{enumerate}
    We say that $(W,B_W,\Mm^W)/U$ is a g-sub-pair induced by subadjunction
    $$K_W+B_W+\Mm^W_W:=(K_X+B+\Mm_X)|_W$$
    and say that $(W,B_W,\Mm^W)$ is associated with $S$.
\end{defthm}
\begin{proof}
    The construction is clear so we only need to prove (1-4). 

    (1) We let $h': Y'\rightarrow X$ be a log resolution of $(X,\Supp B)$ such that $\Mm$ descends to $Y'$ and $S$ on $Y$, so that the induced birational map $g: Y'\rightarrow Y$ is a morphism. Let $S':=g^{-1}_*S$, 
    $$K_{Y'}+B_{Y'}+\Mm_{Y'}:=h'^*(K_X+B+\Mm_X),$$
    and let $(S',B_{S'},\Mm^S)/U$ be the g-sub-pair induced by the adjunction
    $$K_{S'}+B_{S'}+\Mm^S_{S'}:=(K_{Y'}+B_{Y'}+\Mm_{Y'})|_S.$$
    Then $g|_{S'}: S'\rightarrow S$ is a morphism, and we have
    \begin{align*}
       K_{S'}+B_{S'}+\Mm^S_{S'}&=(K_{Y'}+B_{Y'}+\Mm_{Y'})|_{S'}=g^*(K_Y+B_Y+\Mm_Y)|_{S'}\\
       &=g|_{S'}^*((K_Y+B_Y+\Mm_Y)|_S)=g|_{S'}^*(K_S+B+S+\Mm^S_S).
    \end{align*}
    By our construction, the g-sub-pair induced by $h_S\circ g|_{S'}: (S',B_{S'},\Mm^S)\rightarrow W$ is equal to the g-sub-pair induced by $h_S: (S,B_{S},\Mm^S)\rightarrow W$ modulo $\Rr$-linear equivalence of the base moduli part. Since $h'$ can be any high log resolution of $(X,\Supp B)$, (1) follows.

    (2) It immediately follows from the definition. 

    (3) Since $(X,B,\Mm)$ is sub-lc near $V$, $(W,B_W,\Mm)$ is sub-lc near $S$. Thus $(S,B_S,\Mm^S)$ is sub-lc. By Definition-Lemma \ref{defthm: cbf lctrivial morphism}(1), we get (3).

    (4) By Definition-Theorem \ref{defthm: cbf lctrivial morphism}, $T$ is the image of an lc center $T_S$ of $(S,B_S,\Mm^S)$ on $W$. Since $(S,B_S,\Mm^S)$ is log smooth, $T_S$ is also an lc center of $(Y,B_Y,\Mm)$. Thus $h(T_S)$ is an lc center of $(X,B,\Mm)$. By construction, $\nu(T)=h(T_S)$.
\end{proof}



\begin{prop}[Subadjunction formula via dlt models]\label{prop: lc subadj is lc}
       Let $(X,B,\Mm)/U$ be an g-sub-pair and $V$ an lc center of $(X,B,\Mm)$ with normalization $\nu: W\rightarrow V$, such that $(X,B,\Mm)$ is lc near $W$. Let $S$ be an lc place of $(X,B,\Mm)$ such that $\Center_XS=V$. Let $(W,B_W,\Mm^W)/U$ be a g-sub-pair induced by subadjunction
    $$K_W+B_W+\Mm^W_W:=(K_X+B+\Mm_X)|_W$$
    and is associated with $S$.

    Suppose that $f: Y\rightarrow X$ is a dlt modification of $(X,B,\Mm)$ near $W$ such that $S$ is on $Y$. Let 
    $$K_Y+B_Y+\Mm_Y:=f^*(K_X+B+\Mm_X),$$
     $(S,B_{S},\Mm^S)/U$ the g-sub-pair induced by the adjunction
    $$K_{S}+B_{S}+\Mm^S_{S}:=(K_Y+B_Y+\Mm_Y)|_{S},$$
    and $f_{S}: S\rightarrow W$ the induced projective surjective morphism such that $\nu\circ f_S=f|_{S}$. Then:
    \begin{enumerate}
        \item $(W,B_W,\Mm^W)$ is the g-pair induced by $f_S: (S,B_S,\Mm^S)\rightarrow W$.
        \item $(W,B_W,\Mm^W)$ is lc.
    \end{enumerate}
\end{prop}
\begin{proof}
Let $g: Y'\rightarrow Y$ be a log resolution of $(Y,\Supp B_Y)$ such that $\Mm$ descends to $Y'$, 
    $$K_{Y'}+B_{Y'}+\Mm_{Y'}:=g^*(K_Y+B_Y+\Mm_Y),$$
    $S':=g^{-1}_*S$, and let $(S',B_{S'},\Mm^S)/U$ be the g-sub-pair induced by the adjunction
    $$K_{S'}+B_{S'}+\Mm^S_{S'}:=(K_{Y'}+B_{Y'}+\Mm_{Y'})|_{S'}.$$
    Then $g|_{S'}: S'\rightarrow S$ is a morphism, and we have
    \begin{align*}
       K_{S'}+B_{S'}+\Mm^S_{S'}&=(K_{Y'}+B_{Y'}+\Mm_{Y'})|_{S'}=g^*(K_Y+B_Y+\Mm_Y)|_{S}\\
       &=g|_{S'}^*((K_Y+B_Y+\Mm_Y)|_{S})=g|_{S'}^*(K_{S}+B_{S}+\Mm^S_{S}).
    \end{align*}
    By our construction, $(W,B_W,\Mm^W)/U$ is the g-sub-pair induced by $f_S\circ g|_{S'}: (S',B_{S'},\Mm^S)\rightarrow W$, which is equal to the g-sub-pair induced by $f_S: (S,B_{S},\Mm^S)\rightarrow W$ modulo $\Rr$-linear equivalence of the base moduli part. By Definition-Theorem \ref{defthm: cbf lctrivial morphism}(2), $(W,B_W,\Mm^W)$ is lc.
\end{proof}

\begin{defthm}\label{thm: spring and source for glc crepant log structure}
Let $(X,B,\Mm)/U$ be a dlt g-pair and $f: (X,B,\Mm)\rightarrow Y$ a dlt crepant log structure$/U$ (Definition \ref{defn: lc cls}). Let $Z\subset Y$ be an lc center of $f: (X,B,\Mm)\rightarrow Y$ with normalization $\nu: Z^n\rightarrow Z$. Let $\mathcal{S}$ be the set of all lc centers of $(X,B,\Mm)$ which dominate $Z$ and let $S\in\mathcal{S}$ be an element that is minimal under inclusion. Let $(S,B_S,\Mm^S)$ be the g-pair induced by adjunction
$$K_S+B_S+\Mm^S_S:=(K_X+B+\Mm_X)|_S,$$
$f_S: S\rightarrow Z^n$ the induced morphism such that $\nu\circ f_S=f|_S$, and let $f^n_S: S\xrightarrow{\tau} V\xrightarrow{\gamma} Z^n$ be the Stein factorization of $f|_S: S\rightarrow Z$.
Then:
\begin{enumerate}
\item[(1)] (Crepant log structure) $(S,B_S,\Mm^S)$ is dlt, $K_S+B_S+\Mm^S_S\sim_{\Rr,Z}0$, and $(S,B_S,\Mm^S)$ is klt over the generic point of $Z$. In particular, $f|_S: (S,B_S,\Mm^S)\rightarrow Z$ is a dlt crepant log structure and an lc-trivial morphism.
\end{enumerate}
We let
$$(V,B_{V},\Mm^V)/U$$
be the g-pair induced by the lc-trivial fibration $\tau: (S,B_S,\Mm^S)\rightarrow V$. Then: 
\begin{enumerate}
    \item[(2)] (Uniqueness of sources) The crepant birational equivalence class of $(S,B_S,\Mm^S)$ does not depend on the choice of $S$. We call the crepant birational equivalence class of $(S,B_S,\Mm^S)$ as the \emph{source} of $Z$ with respect to $f: (X,B,\Mm)\rightarrow Y$, and is denoted by $\Src(Z,X,B,\Mm)$.
    \item[(3)] (Uniqueness of springs) $(V,B_V,\Mm^V)$ modulo the $\Rr$-linear equivalence class of $\Mm^V$ is unique up to isomorphism. We call $(V,B_V,\Mm^V)$ as the \emph{spring} of $Z$ with respect to $f: (X,B,\Mm)\rightarrow Y$, and is denoted by $\Spr(Z,X,B,\Mm)$.
    \item[(4)] (Adjunction) Let $W\subset X$ be an lc center such that $Z\subset Y_W:=f(W)$, and let $(W,B_W,\Mm^W)/U$ be the lc g-pair induced by repeatedly applying adjunction
    $$K_W+B_W+\Mm^W_W:=(K_X+B+\Mm_X)|_W.$$  
    Let $\nu_Y: Y_W^n\rightarrow Y_W$ be the normalization of $Y_W$, $f_W: W\rightarrow Y_W^n$ the induced morphism such that $\nu_Y\circ f_W=f|_W$, and let
    $$W\xrightarrow{\tau_W} V_W\xrightarrow{\gamma_W}Y_W$$
    be the Stein factorization of $f_W$. Let $Z_W\subset V_W$ be an irreducible subvariety such that $(\nu_Y\circ\gamma_W)(Z_W)=Z$, and $(V_W,B_{V_W},\Mm^{V_W})/U$ a g-pair induced by the lc-trivial fibration $\tau_W: (W,B_W,\Mm^W)\rightarrow V_W$. Then:
    \begin{enumerate}
    \item $Z_W$ is an lc center of $(V_W,B_{V_W},\Mm^{V_W})$.
    \item $\Src(Z,X,B,\Mm)=\Src(Z_W,W,B_W,\Mm^W)$.
    \item $\Spr(Z,X,B,\Mm)=\Spr(Z_W,W,B_W,\Mm^W)$.
    \end{enumerate}
\end{enumerate}
\end{defthm}
\begin{proof}
(1) By \cite[Lemma 2.9]{HL22}, $(S,B_S,\Mm^S)$ is dlt. Since $K_X+B+\Mm_X\sim_{\mathbb R,Z}0$, $K_S+B_S+\Mm^S_S\sim_{\mathbb R,Z}0$. By Lemma \ref{lem: inversion of adjunction gdlt} and since $S$ is minimal in $\mathcal{S}$, $(S,B_S,\Mm^S)$ is klt over the generic point of $Z$. (1) follows.

(2) By Theorem \ref{thm: P1 link for gdlt crepant log structure}, different choices of $S$ are $\mathbb P^1$-linked to each other, hence they are crepant equivalent to each other by Definition \ref{defn: p1 link}(3). 

(3) It follows from (2) and Definition \ref{defn: cbf gpair}.

(4) By Lemma \ref{lem: gdlt crepant log structure is compatible under subadjunction}(3) and Theorem \ref{thm: cbf gpair nonnqc}, $Z_W$ is an lc center of $(V_W,B_{V_W},\Mm^{V_W})$ and an lc center of $\tau_W: (W,B_W,\Mm^W)\rightarrow V_W$. This implies (4.a).

Let $S'$ be a minimal lc center of $(W,B_W,\Mm^W)$ which dominates $Z_W$, then $S'$ is also an lc center of $(X,B,\Mm)$ which dominates $Z_W$. In particular, $S'$ dominates $Z$. If $S'$ is not minimal in $\mathcal{S}$, then there exists $S''\subsetneq S'$ such that $S''$ dominate $Z$, so $\tau_W(S'')\subset Z_W$ and $\tau_W(S'')$ dominates $Z$. This is not possible as $Z_W$ is irreducible and $\gamma_W$ is finite. Therefore, $S'$ is minimal in $\mathcal{S}$. This implies (4.b). (4.c) follows from (4.b) and (3).
\end{proof}


\begin{deflem}[Subadjunction formula via minimal lc centers]\label{deflem: subadj minimal lc center}
    Let $(X,B,\Mm)/U$ be an g-sub-pair and $V$ an lc center of $(X,B,\Mm)$ with normalization $\nu: W\rightarrow V$, such that $(X,B,\Mm)$ is lc near $W$. 

    Suppose that $f: Y\rightarrow X$ is a dlt modification of $(X,B,\Mm)$ near $W$ and let
    $$K_Y+B_Y+\Mm_Y:=f^*(K_X+B+\Mm_X).$$
    Let $\mathcal{S}$ be the set of all lc center of $(Y,B_Y,\Mm)$ whose image on $X$ is $V$, and let $S$ be a minimal element of $\mathcal{S}$ up to inclusion. Let $(S,B_S,\Mm^S)/U$ be the g-pair induced by repeating applying adjunction
    $$K_S+B_S+\Mm^S_S:=(K_Y+B_Y+\Mm_Y)|_S,$$
    and let $f_S: S\rightarrow W$ be the induced projective surjective morphism such that $\nu\circ f_S:=f|_S$. 

    We let $(W,B_W,\Mm^W)/U$ be a g-pair induced by a canonical bundle formula of $f_S: (S,B_S,\Mm^S)\rightarrow W$. Then:
    \begin{enumerate}
        \item There exists an lc place $S'$ of $(X,B,\Mm)$ such that $\Center_XS'=V$, $(W,B_W,\Mm^W)$ is a g-pair induced by subadjunction
    $$K_W+B_W+\Mm^W_W:=(K_X+B+\Mm_X)|_W,$$
    and $(W,B_W,\Mm^W)$ is associated with $S'$. 
        \item $K_W+B_W+\Mm_W\sim_{\mathbb R}(K_X+B+\Mm_X)|_W$.
            \item $(W,B_W,\Mm^W)$ is lc.
            \item For any lc center $T$ of $(W,B_W,\Mm^W)$, $\nu(T)$ is an lc center of $(X,B,\Mm)$.
            \item $W$ does not depend on the choice of $S$ (but may depend on the choice of $f$).
    \end{enumerate}
    We say that $(W,B_W,\Mm^W)/U$ is \emph{associated to} $f$.
\end{deflem}
\begin{proof}
(1) We let $g: Y'\rightarrow Y$ be the blow-up of the generic point of $S$ and let $S'$ be the reduced exceptional divisor. Let 
$$K_{Y'}+B_{Y'}+\Mm_{Y'}=g^*(K_Y+B_Y+\Mm_Y).$$
Then $(Y',B_{Y'},\Mm)$ is dlt over a neighborhood of $W$.  Let $(S',B_{S'},\Mm^{S'})/U$ be the g-pair induced by adjunction
$$K_{S'}+B_{S'}+\Mm^{S'}_{S'}:=(K_{Y'}+B_{Y'}+\Mm_{Y'})|_{S'}.$$
Since $(Y,B_Y)$ is log smooth near the generic point of $S$ and $\Mm$ descends to $Y$ near the generic point of $S$, $g|_{S'}: S'\rightarrow S$ is a contraction, and $(S,B_S,\Mm^S)$ is induced by $g|_{S'} (S',B_{S'},\Mm^{S'})\rightarrow S$.

Thus the Stein factorization of the induced morphism $S'\rightarrow W$ factors through $S$. By Proposition \ref{prop: composition lc trivial fibration}, we get (1).

(2) It follows from (1) and Definition-Theorem \ref{defthm: subadjun}(2).

(3)  It follows from (1) and Proposition \ref{prop: lc subadj is lc}(2).

(4) It follows from (1) and Definition-Theorem \ref{defthm: subadjun}(4).

(5) It follows from Definition-Theorem \ref{thm: spring and source for glc crepant log structure}.
\end{proof}



\section{Stratification of generalized pairs and Du Bois property}\label{sec: du bois}

The goal of this section is to study the stratification properties of lc generalized pairs and prove Theorem \ref{thm: glc sings are Du Bois}.

\subsection{Stratification}

In this subsection we recall some basic definitions of stratifications.

\begin{defn}[{\cite[Definition 9.15]{Kol13}}] 
Let $X$ be a scheme. A {\it stratification} of $X$ is a decomposition of $X$ into a finite disjoint union of reduced locally closed subschemes. We will consider stratifications where the strata are of pure dimensions
and are indexed by their dimensions. We write $X=\cup_{i}S_iX$ where $S_iX\subset X$ is the $i$-th
dimensional stratum. Such a stratified scheme is denoted by $(X,S_*)$. We also
assume that $\cup_{i\le j}S_iX$ is closed for every $j$. The {\it boundary} of $(X,S_*)$ is the closed subscheme
$$
B(X,S_*):=\cup_{i<\dim X}S_iX=X\backslash S_{\dim X}X,
$$
and is denoted by $B(X)$ if the stratification $S_*$ is clear. 

Let $(X, S_*)$ and $(Y, S_*)$ be stratified schemes. We say that $f: X\to Y$ is a {\it stratified morphism} if $f(S_iX)\subset S_iY$ for every $i$. Since $S_iX$ are disjoint with each other, $f: X\to Y$ is a stratified morphism if and only if $S_iX=f^{-1}(S_iY)$.

Let $(Y, S_*)$ be a stratified scheme and $f:X\to Y$ a quasi-finite morphism such that $f^{-1} (S_iY)$ has pure dimension $i$ for every $i$ . Then $S_iX:=f^{-1}(S_iY)$ defines a stratification of $X$. We denote it by $(X,f^{-1}S_*)$, and we say that $f:X\to(Y,S_*)$ is \emph{stratifiable}.
\end{defn}

\begin{defn}[{\cite[Definition 9.16]{Kol13}}]
Let $(X, S_*)$ be stratified variety. A relation $(\sigma_1,\sigma_2): R\rightrightarrows (X,S_*)$ is {\it stratified} if each $\sigma_i$ is stratifiable and $\sigma_1^{-1}S_*=\sigma_2^{-1}S_*$. Equivalently,
there exists a stratification $(R,\sigma^{-1}S_i)$, such that $r\in\sigma^{-1}S_iR$ if and only if $\sigma_1(r)\in S_iX$ and if and only if $\sigma_2(r)\in S_iX$.
\end{defn}

\begin{defn}[{\cite[Definition 9.18]{Kol13}}]
Let $(X,S_*)$ be a stratified scheme such that $X$ is an excellent scheme. The normality conditions (N), (SN), (HN), and (HSN)  are defined in the following ways.
\begin{enumerate}
    \item[(N)] We say that $(X,S_*)$ has {\it normal strata}, or that it satisfies (N), if each $S_iX$ is normal.
    \item[(SN)] We say that $(X,S_*)$ has {\it semi-normal boundary}, or that it satisfies (SN), if $X$ and $B(X,S_*)$ are both semi-normal.
    \item[(HN)] We say that $(X,S_*)$ has {\it hereditarily normal strata}, or that it satisfies (HN), if \begin{enumerate}
            \item the normalization $\pi: (X^n,\pi^{-1}S_*)\to (X,S_*)$ is stratifiable,
            \item $(X^n,\pi^{-1}S_*)$ satisfies (N), and
            \item $B(X^n,\pi^{-1}S_*)$ satisfies (HN).
    \end{enumerate}       
    \item[(HSN)] We say that $(X,S_*)$ has {\it hereditarily semi-normal boundary}, or that it
    satisfies (HSN), if \begin{enumerate}
            \item the normalization $\pi: (X^n,\pi^{-1}S_*)\to (X,S_*)$ is stratifiable,
            \item $(X,\pi^{-1}S_*)$ satisfies (SN), and
            \item $B(X^n,\pi^{-1}S_*)$ satisfies (HSN).
    \end{enumerate}
\end{enumerate}
\end{defn}

Next we give a special stratification that is induced by the lc crepant log structure. 

\begin{defn}[Lc stratification for generalized pairs]
Let $f:(X,\Delta,\Mm)\to Z$ be an lc crepant log structure. Let $S^*_i(Z,X,\Delta,\Mm)\subset Z$ be the union of all $\le i$-dimensional lc centers of $f:(X,\Delta,\Mm)\to Z$, and
$$
S_i(Z,X,\Delta,\Mm):=S^*_i(Z,X,\Delta,\Mm)~\backslash ~S^*_{i-1}(Z,X,\Delta,\Mm).
$$
If the lc crepant log structure $f:(X,\Delta,\Mm)\to Z$ is clear from the context, we will use $S_i(Z)$ for abbreviation. It is clear that each $S_i(Z)$ is a locally closed subspace of $Z$ of pure dimension $i$, and $Z$ is the disjoint union of all $S_i(Z)$. 

The stratification of $Z$ induced by $S_i(Z)$ is called the \emph{lc stratification} of $Z$ induced by $f:(X,\Delta,\Mm)\to Z$.  Since this is the only stratification we are going to use in the rest of this paper, we usually will not emphasize the lc crepant structure $f:(X,\Delta,\Mm)\to Z$, and we will denote the corresponding stratified scheme by $(Z,S_*)$. The \emph{boundary} of $(Z,S_*)$ is the closed subspace
$$B(Z,S_*):=Z\backslash S_{\dim Z}(Z)=\cup_{i<\dim Z}S_i(Z).$$
\end{defn}

\begin{defn}\label{defn: of glc origin}
We say that a semi-normal stratified space $(Y,S_*)$ is \textit{of lc origin} if $S_i(Y)$ is unibranch for any $i$, and there are lc crepant log structures $f_j:(X_j,\Delta_j,\Mm^j)\to Z_j$ with lc stratifications $(Z_j,S_{*}^j)$ and a finite surjective stratified morphism $\pi: \amalg_j(Z_j,S_{*}^j)\to (Y,S_*)$.
\end{defn}

\subsection{Semi-normality of lc centers and lc origin}
In this subsection we show that lc centers of lc generalized pairs are semi-normal.

\begin{thm}\label{thm: glc locus is semi-normal}
Let $f:(X,\Delta,\Mm)\to Z$ be an lc crepant log structure. Let $W\subset Z$ be the union of all lc centers of $f:(X,\Delta,\Mm)\to Z$ except $Z$, and $B(W)\subset W$ the union of all non-maximal (with respect to inclusion) lc centers that are contained in $W$. Then
\begin{enumerate}
    \item $W$ is semi-normal, and
    \item $W\backslash B(W)$ is normal.
\end{enumerate}
\end{thm}
\begin{proof}
Let $(Z,\Delta_Z,\NN)/U$ be an lc g-pair induced by the canonical bundle formula$/U$ of $f: (X,\Delta,\Mm)\rightarrow Z$. By Theorem \ref{thm: cbf gpair nonnqc}, the lc centers of $(Z,\Delta_Z,\NN)$ are exactly the lc centers of $f: (X,\Delta,\Mm)\rightarrow Z$. Possibly replacing $(X,\Delta,\Mm)$ with a dlt model of $(Z,\Delta_Z,\NN)$, we may assume that $f$ is birational and $(X,\Delta,\Mm)$ is $\mathbb Q$-factorial dlt. We have $W=f(\lf\Delta\rf)$. Let $\Delta':=\{\Delta\}$. We consider the exact sequence 
$$
0\to\Oo_X(-\lf\Delta\rf)\to\Oo_X\to\Oo_{\lf\Delta\rf}
$$
and its push-forward
$$
\Oo_Z=f_*\Oo_X\to f_*\Oo_{\lf\Delta\rf}\stackrel{\delta}{\longrightarrow}R^1f_*\Oo_X(-\lf\Delta\rf).
$$
By \cite[Lemma 3.4]{HL22}, we can find an $\Rr$-divisor $\Delta''\ge 0$ such that $$-\lf\Delta\rf\sim_{\Rr,Z}K_X+\Delta'+\Mm_X\sim_{\Rr,Z}K_X+\Delta''$$ and $(X,\Delta'')$ is klt. Since $-\lfloor\Delta\rfloor$ is a Weil divisor, by \cite[Lemma 5.3, Theorem 5.6]{HLS19}, possibly perturbing $\Delta''$, we may assume that $\Delta''$ is a $\Qq$-divisor and 
$$-\lf\Delta\rf\sim_{\mathbb Q,Z}K_X+\Delta''.$$
By \cite[Corollary 10.40]{Kol13}, $R^if_*\Oo_{X}(-\lf\Delta\rf)$ is torsion free for every $i$. On the other hand, $f_*\Oo_{\lf\Delta\rf}$ is supported on $W$, hence it is a torsion sheaf. Thus the connecting map $\delta$ is zero, hence $\Oo_Z\twoheadrightarrow f_*\Oo_{\lf\Delta\rf}$ is surjective. Since this map factors through $\Oo_W$, we conclude that $\Oo_W\twoheadrightarrow f_*\Oo_{\lf\Delta\rf}$ is also surjective, hence an isomorphism.



Note that $\lf\Delta\rf$ has only nodes at codimension 1 points and it is $S_2$ by \cite[Corollary 2.88]{Kol13}. By \cite[Lemma 10.14]{Kol13}, $\lf\Delta\rf$ is semi-normal. By \cite[Lemma 10.15]{Kol13}, $W$ is semi-normal. This is (1).

To prove (2), let $V\subset\lf\Delta\rf$ be an irreducible component of its non-normal locus. Then $V$ is an lc center of $(X,\Delta)$, hence an lc center of $(X,\Delta,\Mm)$. Thus $f(V)$ is an lc center of $f: (X,\Delta,\Mm)\rightarrow Z$. Hence either $f(V)$ is an irreducible component of $W$, or $f(V)\subset B(W)$. Thus \cite[Complement 10.15.1]{Kol13} implies that $W \backslash B(W)$ is normal.
\end{proof}

\begin{cor}
Let $(X,\Delta,\Mm)$ be an lc g-pair. Then $\Nklt(X,\Delta,\Mm)$ is semi-normal.
\end{cor}
\begin{proof}
It follows from Theorem \ref{thm: glc locus is semi-normal} when $f$ is the identity morphism.
\end{proof}



\begin{lem}\label{lem: (Z,S) is U and SN}(cf. \cite[Lemma 5.26]{Kol13})
Let $f:(X,\Delta,\Mm)\to Z$ be a lc crepant log structure and $(Z,S_*)$ the induced lc stratification. Then
\begin{enumerate}
    \item  $S_i(Z)$ is unibranch for every $i$, and 
    \item  $B(Z,S_*)$ is semi-normal.
\end{enumerate}
\end{lem}

\begin{proof}
(1) follows from Lemma \ref{lem: intersection of lc center gpair}(2) and (2) follows from Theorem \ref{thm: glc locus is semi-normal}.
\end{proof}


\begin{lem}\label{lem: stratification is compatible under adjunction} (cf. \cite[Proposition 4.42]{Kol13})
Let $f: (X,\Delta,\Mm)\to Z$ be a dlt crepant log structure, $(Z,S_*)$ its induced lc stratification, and $Y\subset X$ an lc center of $(X,\Delta,\Mm)$. Let $(Y,\Delta,\Mm^Y)/Z$ be the dlt g-pair induced by adjunction to higher-codimensional lc center $Y$, i.e. 
$$K_Y+\Delta_Y+\Mm^Y_Y:=(K_X+\Delta+\Mm_X)|_Y.$$
We consider the Stein factorization of $f|_Y$
$$(Y,\Delta_Y,\Mm^Y)\stackrel{f_Y}{\longrightarrow}W\stackrel{\pi}{\longrightarrow}Z.$$
Then:
\begin{enumerate}
    \item $f_Y:(Y,\Delta_Y,\Mm^Y)\to W$ is a dlt crepant log structure which induces an lc stratification $(W,S_*)$.
    \item $S_i(W)=\pi^{-1}(S_i(Z))$ for every $i$.
\end{enumerate}
\end{lem}
\begin{proof}
It follows from Lemma \ref{lem: intersection of lc center gpair}.
\end{proof}

\begin{thm}\label{thm: (Z,S) is HN and HSN}
Let $f:(X,\Delta,\Mm)\to Z$ be an lc crepant log structure and $(Z,S_*)$ the induced lc stratification. Then $(Z,S_*)$ satisfies (HN) and (HSN).
\end{thm}
\begin{proof}
By Lemma \ref{lem: (Z,S) is U and SN} and \cite[Definitions 9.18,~9.19]{Kol13}, $(Z,S_*)$ satisfies (HU) and (HSN). By \cite[Theorem 9.21]{Kol13}, $(Z,S_*)$  satisfies (HN).
\end{proof}

\begin{lem}(cf. \cite[5.29]{Kol13})\label{lem: glc stratification is of glc origin}
Every lc stratification is of lc origin. More precisely, let $f:(X,\Delta,\Mm)\to W $ be an lc crepant log structure and $Y\subset W$ any union of lc centers. Then $(Y, S_*)$ is of lc origin, where $S_i(Y)=Y\cap S_i(W)$ for each $i$.
\end{lem}

\begin{proof}
By Theorem \ref{thm: (Z,S) is HN and HSN} and \cite[Theorem 9.26]{Kol13}, we know that $Y$ is semi-normal and $S_i(Y)$ is unibranch for each $i$. Then we can apply Lemma \ref{lem: stratification is compatible under adjunction} to each lc center of $f: (X,\Delta,\Mm)$ contained in $Y$ to conclude that $(Y,S_*)$ is of lc origin.
\end{proof}

\subsection{Du Bois property}\label{subsec: du bois}

In this subsection, we show that lc generalized pairs have Du Bois singularities. This subsection is parallel to \cite[Section 6]{LX23b}. 

We recall the following definition in \cite{Kov11} (cf. \cite[Definition 6.10]{Kol13}). 

\begin{defn}
A \emph{DB pair} $(X,\Sigma)$ consists of a reduced scheme $X$ of finite type and a closed reduced subscheme $\Sigma$ in $X$ such that the natural morphism
$$
\mathcal{I}_{\Sigma\subset X}\to \underline{\Omega}_{X,\Sigma}^0
$$
is a quasi-isomorphism. We will also say $(X,\Sigma)$ is \emph{DB} in this case.
\end{defn}

The definition of DB pairs is subtle but what really matters here is the following lemma:

\begin{lem}[{\cite[Proposition 6.15]{Kol13}}]\label{lem: property of DB pairs}
Let $(X,\Sigma)$ be a DB pair. Then $X$ has Du Bois singularities if and only if $\Sigma$ has Du Bois singularities.
\end{lem}

The following theorems are analogues of \cite[Theorems 6.31, 6.33]{Kol13} for g-pairs and the proofs are similar. For the reader's convenience, we provide full proofs here.

\begin{thm}\label{thm: (Z,W) is DB for glc crepant log structure}
Let $(X,B,\Mm)/U$ be an lc g-pair and $f: (X,B,\Mm)\rightarrow Z$ an lc-trivial fibration.
Let $W\subset Z$ be the union of lc centers of $f: (X,B,\Mm)\rightarrow Z$ except $Z$. Then $(Z,W)$ is a DB pair.
\end{thm}

\begin{proof}
Let $(Z,B_Z,\Mm^Z)/U$ be a g-pair induced by $f: (X,B,\Mm)\rightarrow Z$. By Theorem \ref{thm: cbf gpair nonnqc}, the lc centers of $(Z,B_Z,\Mm^Z)$ are exactly the lc centers of $f: (X,B,\Mm)\rightarrow Z$. Thus we can assume that $f$ is the identity morphism, $(X,B,\Mm)=(Z,B_Z,\Mm^Z)$, and $W=\Nklt(X,B,\Mm)$.

Let $g: Y\to X$ be a log resolution of $(X,\Supp B)$ such that $\Mm$ descends to $Y$ and $F:=g^{-1}(W)$ is an snc divisor. Let
$$K_Y+B_Y+\Mm_Y:=g^*(K_X+B+\Mm_X)$$
and $D:=B_Y^{=1}$. Since $\Mm_Y$ is nef$/X$ and big$/X$, there exists $0\le B'_Y\sim_{\Rr,X}\Mm_Y$ such that $(Y,B_Y-D+B'_Y)$ is sub-klt. Possibly replacing $Y$ with a higher resolution, we may assume that $(Y,\Supp B_Y\cup\Supp D\cup\Supp B_Y')$ is log smooth. Let 
$$\bar B_Y:=(B_Y-D+B'_Y)^{\ge0}+\{(B_Y-D+B'_Y)^{\le 0}\}$$ and 
$$E:=\lfloor (B_Y-D+B'_Y)^{\le 0}\rfloor,$$ then $\lf\bar B_Y\rf=0$ and $E$ is a g-exceptional Weil divisor. In particualr, $(Y,\bar B_Y)$ is klt.

Since $E-D\ge-F$, we have natural maps:
$$
g_*\Oo_Y(-F)\to Rg_*\Oo_Y(-F)\to Rg_*\Oo_Y(E-D).
$$
Since $E-D\sim_{\Rr,X}K_Y+\bar B_Y$ and $E-D$ is a Weil divisor, by \cite[Lemma 5.3, Theorem 5.6]{HLS19}, $E-D\sim_{\Qq,X}K_Y+\bar B_Y'$ for some klt $\Qq$-pair $(Y,\bar B_Y')$. by \cite[Theorem 10.41]{Kol13},
$$
Rg_*\Oo_Y(E-D)\simeq_{qis}\sum_{i}R^ig_*\Oo_Y(E-D)[i].
$$
Thus we get a morphism 
$$
g_*\Oo_Y(-F)\to Rg_*\Oo_Y(-F)\to Rg_*\Oo_Y(E-D)\to g_*\Oo_Y(E-D).
$$
Note that 
$$
g_*\Oo_Y(E-D)=g_*\Oo_Y(E-D)\cap g_*\Oo_Y(E)=g_*\Oo_Y(E-D)\cap g_*\Oo_Y=g_*\Oo_Y(-D).
$$
Since $D$ is reduced and $g(D)=W$, we have $g_*\Oo_Y(-D)=\mathcal{I}_W$, the ideal sheaf of $W$ in $Z=X$. Moreover, $g_*\Oo_Y(-F)=\mathcal{I}_W$ since $F$ is also reduced. Therefore, we get an isomorphism $\mathcal{I}_W=g_*\Oo_Y(-F)\to g_*\Oo_Y(E-D)$, which implies that 
$$
\rho: \mathcal{I}_W\simeq g_*\mathcal{I}_F\to Rg_*\mathcal{I}_F
$$
has a left inverse. Since $Y$ is smooth and $F$ is an snc divisor, we see that $(Y,F)$ is a DB pair, thus by \cite[Theorem 3.3]{Kov12} (cf. \cite[Theorem 6.27]{Kol13}), $(Z,W)$ is also a DB pair. \end{proof}

\begin{defn}
We say a commutative diagram of schemes
\begin{align*}
\xymatrix{
\mathcal{C}\ar@{->}[r]^j\ar@{->}[d]_q & Y\ar@{->}[d]^{p}\\
 \mathcal{D}\ar@{->}[r]^i & X\\
}
\end{align*}
is a \emph{universal push-out diagram} if for any scheme $T$, the induced diagram
$$\xymatrix{
\Hom{(X,T)}\ar@{->}[r]^{\circ i}\ar@{->}[d]_{\circ p} & \Hom{(\mathcal{D},T)}\ar@{->}[d]^{{\circ q}}\\
 \Hom{(Y,T)}\ar@{->}[r]^{\circ j} & \Hom{(\mathcal{C},T)}\\
}$$
is a universal pull-back diagram of sets. 
\end{defn}


\begin{thm}\label{thm: of glc origin implies DB}
Let $(X,S_*)$ be a stratified scheme of lc origin (Definition \ref{defn: of glc origin}). Then $X$ is Du Bois.
\end{thm}

\begin{proof}
We use induction on the dimension. When $\dim X=1$ the theorem is trivial.

Let $\pi: (X^n,S^n_*)\to (X,S_*)$ be the normalization. Let $B(X)\subset X$ and
$B(X^n)\subset X^n$ denote the corresponding boundaries. By \cite[9.15.1]{Kol13}, we have a universal push-out diagram
\begin{center}
$\xymatrix{
B(X^n)\ar@{^(->}[r]\ar@{->}[d] & X^n\ar@{->}[d]^{\pi}\\
 B(X)\ar@{^(->}[r]& X\\
}$
\end{center}
Notice that $B(X)$ and $B(X^n)$ are of lc origin by Lemma \ref{lem: glc stratification is of glc origin}, hence Du Bois by induction.

Since $\pi$ is finite, it follows that $R\pi_*\mathcal{I}_{B(X^n)\subset X^n}=\pi_*\mathcal{I}_{B(X^n)\subseteq X^n}$. Furthermore, $\pi_*\mathcal{I}_{B(X^n)\subseteq X^n}=\mathcal{I}_{B(X)\subseteq X}$ by \cite[Theorem 9.30]{Kol13}. By \cite[Theorem 3.3]{Kov12} and Lemma \ref{lem: property of DB pairs}, we only need to show that $X^n$ is Du Bois. By assumption, for each irreducible component $X_i^n\subset X^n$, there exists an lc crepant log structure $f_i:(Y_i,\Delta_i,\Mm)\to Z_i$ and a finite surjection $Z_i\to X_i^n$. By \cite[Corollary 2.5]{Kov99}, we only need to show that $Z_i$ is Du Bois for each $i$. Let $B(Z_i)\subset Z_i$ be the boundary of the lc stratification of $Z_i$. Then  $B(Z_i)$ is of lc origin by Lemma \ref{lem: glc stratification is of glc origin}, hence Du Bois by induction. By Theorem \ref{thm: (Z,W) is DB for glc crepant log structure}, $(Z_i,B(Z_i))$ is a DB pair, hence $Z_i$ is Du Bois and we are done.
\end{proof}

\begin{proof}[Proof of Theorem \ref{thm: glc sings are Du Bois}]
   Let $W$ be any union of the glc centers, then by Lemma \ref{lem: glc stratification is of glc origin} the induced stratified space $(W,S_*)$ is of lc origin. Theorem \ref{thm: glc sings are Du Bois} follows from Theorem \ref{thm: of glc origin implies DB}.
\end{proof}


\section{Vanishing and contraction theorems for lc generalized pairs}\label{sec: vanishing gpair}

The goal of this section is to prove the vanishing theorems and contraction theorems for lc generalized pairs. This section is parallel to \cite{CLX23}, except that the canonical bundle formula and the subadjunction formulas are replaced by the ones established in Sections \ref{sec: cbf} and \ref{sec: subadj}.

\subsection{Adjacent lc centers and universal push-out diagram}

\begin{defn}[Union of lc centers] Let $(X,B,\Mm)$ be an lc g-pair. A \emph{union of lc centers} of $(X,B,\Mm)$ is a reduced scheme $Y=\cup Y_i$, where each $Y_i$ is an lc center of $(X,B,\Mm)$. We denote by $S(X,B,\Mm)$ the set of all unions of lc centers of $(X,B,\Mm)$. We remark that 
\begin{enumerate}
    \item $\emptyset$ is also considered as a union of lc centers, and
    \item a union of lc center may be represented in different ways. For example, if $Y_1$ and $Y_2$ are two lc centers such that $Y_1\subsetneq Y_2$, then $Y_1\cup Y_2$ and $Y_2$ are the same union of lc centers.
\end{enumerate}
\end{defn}


\begin{defn}[Adjacent unions of lc centers]
Let $(X,B,\Mm)$ be an lc g-pair. For any two unions of lc centers $Y,Y'\in S(X,B,\Mm)$, we say that $Y$ and $Y'$ are \emph{adjacent} in $S(X,B,\Mm)$ if
\begin{enumerate}
    \item $Y\subsetneq Y'$ or $Y'\subsetneq Y$, and
    \item there does not exist any $Y''\in S(X,B,\Mm)$ such that $Y\subsetneq Y'' \subsetneq Y'$ or $Y'\subsetneq Y''\subsetneq Y$. 
\end{enumerate}
An lc center $V$ is called \emph{minimal} in $S(X,B,\Mm)$ if $V$ and $\emptyset$ are adjacent in $S(X,B,\Mm)$.
\end{defn}


\begin{lem}\label{lem:pushout2}
Let $(X,B,\Mm)$ be an lc g-pair. Let $Y$ and $Y'$ be two unions of lc centers, such that $Y'\subsetneq Y$, and $Y$ and $Y'$ are adjacent in $S(X,B,\Mm)$. Let $\pi: Y^n\rightarrow Y$ be the normalization of $Y$ and let $Y'':=\pi^{-1}(Y')$ with the reduced scheme structure. Denote the induced morphism $Y''\rightarrow Y'$ by $\pi''$. Then there exist a universal push-out diagram
\begin{center}
$\xymatrix{
Y'' \ar@{^(->}[r]^j\ar@{->}[d]_{\pi''} & Y^n\ar@{->}[d]^{\pi}\\
Y' \ar@{^(->}[r]^i& Y\\
}$
\end{center}
and a short exact sequence
\begin{align*}
0\to \Oo_{Y}\xrightarrow{\pi^*\oplus i^*} \pi_*\Oo_{Y^n}\oplus\Oo_{Y'}\xrightarrow{j^*-\pi''^*} \pi''_*\Oo_{Y''}\to 0,
\end{align*}
where $i,j$ are the natural closed immersions.
\end{lem}
\begin{proof}
By Theorem \ref{thm: glc locus is semi-normal} and \cite[Theorem 9.26]{Kol13}, $Y$ is semi-normal. Let $L$ be an lc center contained in $Y$ but not contained in $Y'$. Since $Y'$ and $Y$ are adjacent in $S(X,B,\Mm)$, we have
$$Y\backslash Y'=L\backslash (L\cap Y'),$$
and $L\cap Y'$ is the union of all lc centers of $(X,B,\Mm)$ that are contained in $L$ but not equal to $L$. By Theorem \ref{thm: glc locus is semi-normal}, $Y\setminus Y'$ is normal. The lemma follows from \cite[Lemma 2.6]{CLX23}.
\end{proof}

\subsection{Vanishing theorems}

The following lemma is very similar to \cite[Lemma 2.4]{Xie22}.

\begin{lem}\label{lem: perturb glc pair to nlc pair}
Let $(X,B,\Mm)/U$ be an lc g-pair, and $L$ a nef $\Rr$-divisor such that $L-(K_X+B+\Mm_X)$ is nef$/U$ and big$/U$. Then there exists an $\Rr$-divisor $\Delta\geq 0$ such that $L-(K_X+\Delta)$ is ample over $U$ and $\Nlc(X,\Delta)=\Nklt(X,B,\Mm)$. 
\end{lem}
\begin{proof}
Let $f: Y\to X$ be a log resolution of $(X,\Supp B)$ such that $\Mm$ descends on $Y$, and let
$$K_Y+B_Y+\Mm_Y:=f^*(K_X+B+\Mm_X).$$
Since $L-(K_X+B+\Mm_X)$ is nef$/U$ and big$/U$, $f^*L-(K_Y+B_Y+\Mm_Y)$ is nef$/U$ and big$/U$. Then there exists an $\Rr$-divisor $E\geq 0$ on $Y$, such that for any positive integer $n$, there exists an ample$/U$ $\Rr$-divisor $A_n$ on $Y$ such that
$$f^*L-(K_Y+B_Y+\Mm_Y)\sim_{\Rr,U} A_n+\frac{1}{n}E.$$
We let $m$ be a positive integer such that
$$\Nklt(Y,B_Y,\Mm)=\lfloor B_Y\rfloor=\Nklt(Y,B_Y+\frac{1}{m}E,\Mm).$$
Let $0<\delta\ll 1$ a real number such that $A_m-\delta\lfloor B_Y\rfloor$ is ample$/U$. Pick a general ample $\Rr$-divisor $A_Y\in|(A_{m}+\Mm_Y-\delta\lfloor B_Y\rfloor)/U|_{\mathbb R}$, set
$$B_Y':=B_Y+A_Y+\delta\lfloor B_Y\rfloor+\frac{1}{m}E,$$
and $B':=f_*B_Y'$. Then $0\leq B'\sim_{\mathbb R,U}B+\Mm_X$ and
$$\Nlc(X,B')=\Nklt(X,B')=\Nklt(X,B,\Mm).$$
Since $L-(K_X+B+\Mm_X)$ is big$/U$ and nef$/U$, there exist an $\Rr$-divisor $F\geq 0$ on $X$ and an ample$/U$ $\Rr$-divisor $H$ on $X$ such that
$$L-(K_X+B+\Mm_X)\sim_{\mathbb R,U}H+F.$$
Let $l\gg 0$ be an integer, then
$$L-\left(K_X+B'+\frac{1}{l}F\right)\sim_{\mathbb R,U}L-\left(K_X+B+\Mm_X+\frac{1}{l}F\right)\sim_{\mathbb R,U}\frac{1}{l}H+\frac{l-1}{l}(L-(K_X+B+\Mm_X))$$
is ample$/U$, and
$$\Nlc\left(X,B'+\frac{1}{l}F\right)\subset\Nklt(X,B')=\Nlc(X,B')\subset\Nlc\left(X,B'+\frac{1}{l}F\right).$$
Hence
$$\Nlc\left(X,B'+\frac{1}{n}F\right)=\Nlc(X,B')=\Nklt(X,B,\Mm).$$
Thus $\Delta:=B'+\frac{1}{l}F$ has the required property.
\end{proof}

\begin{lem}\label{lem: vanishing keep under finite mor}
    Let  $f: X\rightarrow U$ be a projective morphism, $h: Y\rightarrow X$ a finite morphism between normal schemes, and $g:=f\circ h$. Let $W\subset X$ and $V\subset Y$ be two reduced subschemes such that $h^{-1}(W)=V$ with defining ideal sheaves $\mathcal{I}_W$ and $\mathcal{I}_V$. Let $L$ be a line bundle on $X$ such that $R^ig_*(h^*L\otimes\mathcal{I}_V)=0$ for some positive integer $i$. Then $R^if_*(L\otimes\mathcal{I}_W)=0$.
\end{lem}
\begin{proof}
    Notice that $I_W$ is a direct summand of $h_*I_V$ (via the splitting $\Oo_X\to h_*\Oo_Y\to \Oo_X$), so it suffices to prove that $R^if_*(L\otimes h_*\mathcal{I}_V)=0$.
    Since $R^ih_*(G)=0$ for any coherent sheaf $G$ and $i>0$, we have
    $$R^if_*(L\otimes h_*\mathcal{I}_V)=R^if_*(h_*(h^*L\otimes\mathcal{I}_V))=R^ig_*(h^*L\otimes\mathcal{I}_V)=0.$$
\end{proof}



\begin{lem}\label{lem:3.1}
Let $f: X\rightarrow U$ be a projective morphism, $L$ an $\Rr$-Cartier $\Rr$-divisor on $X$, and $D$ a Cartier divisor on $X$. Let $h: Y\rightarrow X$ be a finite morphism, $(Y,B_Y,\Mm^Y)/U$ an lc g-pair such that 
$$K_Y+B_Y+\Mm^Y_Y\sim_{\mathbb R,U}h^*L,$$
and $V:=\Nklt(Y,B_Y,\Mm^Y)$ with the reduced scheme structure. Set $W:=h(V)$ with the reduced scheme structure. Let $\mathcal{I}_W,\mathcal{I}_V$ be the defining ideal sheaves of $W$ and $V$ respectively, $g:=f\circ h$, and $D_Y:=h^*D$. Suppose that $D_Y-(K_Y+B_Y+\Mm^Y_Y)$ is nef$/U$ and log big$/U$ with respect to $(Y,B_Y,\Mm^Y)$. Then:
\begin{enumerate}
	\item $R^ig_*(\mathcal{I}_V\otimes\mathcal{O}_Y(D_Y))=0$ for any $i>0$.
    \item $g_*\mathcal{O}_Y(D_Y)\rightarrow g_*\mathcal{O}_V(D_Y)$ is surjective.
    \item Suppose that $V=h^{-1}(W)$. Then $R^if_*(\mathcal{I}_W\otimes\mathcal{O}_X(D))=0$ for any $i>0$.
    \item Suppose that $V=h^{-1}(W)$. Then $f_*\mathcal{O}_X(D)\rightarrow f_*\mathcal{O}_W(D)$ is surjective.
\end{enumerate}
\end{lem}
\begin{proof}
By Lemma \ref{lem: perturb glc pair to nlc pair}, there exists a pair $(Y,\Delta_Y)$ such that $D_Y-(K_Y+\Delta_Y)$ is ample/$U$ and $V=\Nlc(X,\Delta_Y)$. (1) follows from  \cite[Theorem 8.1]{Fuj11}. 
 (2) follows from (1) and the long exact sequence
$$0\rightarrow g_*(\mathcal{I}_V\otimes\mathcal{O}_Y(D_Y))\rightarrow g_*\mathcal{O}_Y(D_Y)\rightarrow g_*\mathcal{O}_V(D_Y)\rightarrow R^1g_*(\mathcal{I}_V\otimes\mathcal{O}_Y(D_Y))\rightarrow\dots.$$
(3) follows from (1) and Lemma \ref{lem: vanishing keep under finite mor}. (4) follows from (3) and the long exact sequence
$$0\rightarrow f_*(\mathcal{I}_W\otimes\mathcal{O}_X(D))\rightarrow f_*\mathcal{O}_X(D)\rightarrow f_*\mathcal{O}_W(D)\rightarrow R^1f_*(\mathcal{I}_W\otimes\mathcal{O}_X(D))\rightarrow\dots.$$
\end{proof}

\begin{lem}\label{lem:3.2}
Let $(X,B,\Mm)/U$ be an lc g-pair associated with morphism $f: X\rightarrow U$, and $D$ a Cartier divisor on $X$ such that $D-(K_X+B+\Mm_X)$ is nef$/U$ and log big$/U$ with respect to $(X,B,\Mm)$. Let $Y$ and $Y'$ be two unions of lc centers, such that $Y'\subsetneq Y$, and $Y$ and $Y'$ are adjacent in $S(X,B,\Mm)$. Let $\pi: Y^n\rightarrow Y$ be the normalization of $Y$, $Y'':=\pi^{-1}(Y')$ with the reduced scheme structure, and $\pi'':=\pi|_{Y''}$.
\begin{center}
$\xymatrix{
Y'' \ar@{^(->}[r]^j\ar@{->}[d]_{\pi''} & Y^n\ar@{->}[d]^{\pi}\\
Y' \ar@{^(->}[r]^i& Y\\
}$
\end{center}
Then the induced map
$$f_*\pi_*\mathcal{O}_{Y^n}(D|_{Y^n}) \rightarrow f_*\pi''_*\mathcal{O}_{Y''}(D|_{Y''})$$
is surjective.
\end{lem}

\begin{proof}
We only need to show that
$$f_*\pi_*\mathcal{O}_{Y_0}(D|_{Y^n_0}) \rightarrow f_*\pi''_*\mathcal{O}_{Y_0''}(D|_{Y^n_0\cap Y''})$$
is surjective for any connected component $Y^n_0$ of $Y^n$. Let $Y^n_0$ be a connected component of $Y^n$, $Y_0'':=Y''\cap Y^n_0$, $Y_0:=\pi(Y^n_0)$, and $Y'_0:=\pi''(Y_0'')$. Since $Y$ and $Y'$ are adjacent in $S(X,B,\Mm)$, either $Y_0=Y'_0$, or $Y_0'$ and $Y_0$ are adjacent in $S(X,B,\Mm)$ and $Y_0'\subsetneq Y_0$. Possibly replacing $Y$ with $Y_0$ and $Y'$ with $Y_0'$, we may assume that $Y$ is an lc center of $(X,B,\Mm)$. Since  $Y$ and $Y'$ are adjacent in $S(X,B,\Mm)$, $Y'$ is the union of all lc centers of $(X,B,\Mm)$ that are contained in $Y$.

We let $(W,B_W,\Mm)$ be a dlt model of $(X,B,\Mm)$ with induced birational morphism $h: W\rightarrow X$. Let $S$ be an lc center of $(W,B_W,\Mm)$ which is minimal in all lc centers which dominate $Y$, $(S,B_S,\Mm^S)/U$ the dlt g-pair induced by adjunction 
$$K_S+B_S+\Mm^S_S:=(K_W+B_W+\Mm_W)|_W,$$
and $h_S: S\rightarrow Y^n$ the induced morphism such that $\pi\circ h_S=h|_S$. Let
$$S\xrightarrow{\tau} Z\xrightarrow{\gamma} Y^n$$
be the Stein factorization of $h_S$, and let $(Z,B_Z,\Mm^Z)/U$ be the lc g-pair induced by $\tau: (S,B_S,\Mm^S)\rightarrow Z$. Let $(Y^n,B_{Y^n},\Mm^{Y^n})/U$ be the lc g-pair induced by $\gamma: (Z,B_Z,\Mm^Z)\rightarrow Y^n$, and let $Y'_Z:=\gamma^{-1}(Y'')$.

 By Lemma \ref{lem: gdlt crepant log structure is compatible under subadjunction}(3) and Theorem \ref{thm: cbf gpair nonnqc}(5), for any lc center $V$ of $(X,B,\Mm)$ such that $V\subset Y^n$, any irreducible component of $\gamma^{-1}(V)$ is an lc center of $(Z,B_Z,\Mm^Z)$. In particular, any irreducible component of $Y'_Z$ is an lc center of $(Z,B_Z,\Mm^Z)$, so $\Nklt(Z,B_Z,\Mm^Z)\subset Y'_Z$. By Theorem \ref{thm: cbf gpair nonnqc}(4), $(\pi\circ\gamma)(\Nklt(Z,B_Z,\Mm^Z))$ is a union of lc centers of $(X,B,\Mm)$ that are contained in $Y$, so $\Nklt(Z,B_Z,\Mm^Z)\subset Y'_Z$. Thus  $$\Nklt(Z,B_Z,\Mm^Z)=Y'_Z=\gamma^{-1}(Y'').$$

Let $D_{Y^n}:=D|_{Y^n}$ and $D_Z:=\gamma^*D_{Y^n}$. Since $D-(K_X+B+\Mm_X)$ is nef$/U$, $D_{Y^n}-(K_{Y^n}+B_{Y^n}+\Mm^{Y^n}_{Y^n})$ is nef$/U$, so $D_Z-(K_Z+B_Z+\Mm^Z_Z)$ is nef$/U$. For any lc center $V_Z$ of $(Z,B_Z,\Mm^Z)$ with normalization $V_Z^n$, $(\pi\circ\gamma)(V_Z)$ is an lc center of $(X,B,\Mm)$, so  $(D-(K_X+B+\Mm_X))|_{\pi(V_Z)^n}$ is big$/U$, where $\pi(V_Z)^n$ is the normalization of $\pi(W)$. Since $\pi$ is finite, $(D_Z-(K_Z+B_Z+\Mm^Z_Z))|_{V_Z^n}$ is big$/U$. Therefore, $D_Z-(K_Z+B_Z+\Mm^Z_Z)$ is log big$/U$.

The lemma follows from Lemma \ref{lem:3.1}(4).
\end{proof}




\begin{thm}\label{thm: kod vanishing with lc strata}
Let $(X,B,\Mm)/U$ be an lc g-pair associated with projective morphism $f: X\rightarrow U$, $D$ a Cartier divisor on $X$ such that $D-(K_X+B+\Mm_X)$ is nef$/U$ and log big$/U$ with respect to $(X,B,\Mm)$, and $Y$ a union of lc centers of $(X,B,\Mm)$ such that $Y\not=X$. Then:
\begin{enumerate}
	\item $R^if_*\mathcal{O}_Y(D)=0$ for any positive integer $i$.
	\item $R^if_*\mathcal{O}_X(D)=0$ for any positive integer $i$.
    \item The map $f_*\mathcal{O}_X(D)\rightarrow f_*\mathcal{O}_Y(D)$ is surjective.
    \item $R^if_*(\mathcal{I}_Y\otimes\mathcal{O}_X(D))=0$ for any positive integer $i$, where $\mathcal{I}_Y$ is the defining ideal sheaf of $Y$ on $X$.
\end{enumerate}
\end{thm}
\begin{proof}
We apply induction on $\dim X$. When $\dim X=1$ the theorem is obvious. 

For any union of lc centers $Z$ of $(X,B,\Mm)$, we define $m(Z)$ to be the number of lc centers of $(X,B,\Mm)$ that are contained in $Z$. We let $W:=\Nklt(X,B,\Mm)$, associated with the reduced scheme structure.

\medskip

\noindent\textbf{Step 1}. In this step we prove (1) when $Y$ is minimal in $S(X,B,\Mm)$ the set of all unions of lc centers of $(X,B+\Mm)/U$.

By Theorem \ref{thm: (Z,S) is HN and HSN}, $Y$ is normal. If $\dim Y=0$ then we are done. Otherwise, by Definition-Theorem \ref{defthm: subadjun} and Proposition \ref{prop: lc subadj is lc}, there exists a klt g-pair $(Y,B_Y,\Mm^{Y})/U$ such that $K_{Y}+B_{Y}+\Mm^{Y}_{Y}\sim_{\Rr,U}(K_X+B+\Mm_X)|_{Y}$. Hence $D|_Y-(K_{Y}+B_{Y}+\Mm^{Y}_{Y})$ is nef$/U$ and big$/U$. By Lemma \ref{lem: perturb glc pair to nlc pair}, there exists a klt pair $(Y,\Delta_Y)$ such that $D|_Y-(K_{Y}+\Delta_Y)$ is ample$/U$. (1) follows from the usual Kawamata-Viehweg vanishing theorem (cf. \cite[Theorem 1-2-7]{KMM87}).

\medskip

\noindent\textbf{Step 2}. In this step we prove (1). 

We apply induction on $m(Y)$. When $m(Y)=1$, $Y$ is minimal in $S(X,B,\Mm)$ and we are done by \textbf{Step 1}. Thus we may assume that $m(Y)>1$. In particular, $\dim Y\geq 1$. We let $Y'\in S(X,B,\Mm)$ be a union of lc centers such that $Y'\subsetneq Y$ and $Y',Y$ are adjacent in $S(X,B,\Mm)$.  Let $\pi: Y^n\rightarrow Y$ be the normalization of $Y$, $Y'':=\pi^{-1}(Y')$ with the reduced scheme structure, and $\pi'':=\pi|_{Y''}$. By Lemma \ref{lem:pushout2}, there exists a universal push-out diagram
\begin{center}
$\xymatrix{
Y'' \ar@{^(->}[r]^j\ar@{->}[d]_{\pi''} & Y^n\ar@{->}[d]^{\pi}\\
Y' \ar@{^(->}[r]^i& Y\\
}$
\end{center}
and a short exact sequence
\begin{align}\label{eq: short exact sequence in main theorem}
0\to \Oo_{Y}\xrightarrow{\pi^*\oplus i^*} \pi_*\Oo_{Y^n}\oplus\Oo_{Y'}\xrightarrow{j^*-\pi''^*} \pi''_*\Oo_{Y''}\to 0.
\end{align}
where $i,j$ are the natural closed immersions. Since $m(Y')<m(Y)$, by induction on $m(Y)$, we have
\begin{align}\label{eq3.1}
R^if_*\mathcal{O}_{Y'}(D)=0
\end{align}
for any positive integer $i$. 

By Definition-Theorem \ref{defthm: subadjun} and Proposition \ref{prop: lc subadj is lc}, there exists an lc g-pair $(Y^n,B_{Y^n},\Mm^{Y_n})/U$ such that $K_{Y^n}+B_{Y^n}+\Mm^{Y_n}_{Y^n}\sim_{\mathbb R}(K_X+B+\Mm_X)|_{Y^n}$, and the image of any lc center of $(Y^n,B_{Y^n},\Mm^{Y_n})$ in $X$ is an lc center of $(X,B,\Mm)$. Since $\dim Y^n<\dim X$ and $\pi$ is a finite morphism, by induction on $\dim X$, we have
\begin{align}\label{eq3.2}
&R^i(f\circ \pi)_*\mathcal{O}_{Y^n}(D|_{Y^n})=R^if_*\big(\pi_*(\mathcal{O}_{Y^n}(D|_{Y^n})\big)
=0.
\end{align}
\begin{claim}
    For any positive integer $i$,
    \begin{align}\label{eq3.3}
&R^i(f\circ \pi'')_*\mathcal{O}_{Y''}(D|_{Y''})= R^if_*\big(\pi''_*\mathcal{O}_{Y''}(D|_{Y''})\big)=0.
\end{align}
\end{claim}
\begin{proof}
    We only need to show that,
    $$R^i(f\circ \pi'')_*\mathcal{O}_{Y''\cap Y^n_0}(D|_{Y''\cap Y^n_0})= R^if_*\big(\pi''_*\mathcal{O}_{Y''\cap Y^n_0}(D|_{Y''\cap Y^n_0})\big)=0$$
for any irreducible component $Y^n_0$ of $Y^n$. Let $Y^n_0$ be a connected component of $Y^n$, $Y_0'':=Y''\cap Y^n_0$, $Y_0:=\pi(Y^n_0)$, and $Y'_0:=\pi''(Y_0'')$. Since $Y$ and $Y'$ are adjacent in $S(X,B,\Mm)$, either $Y_0=Y'_0$, or $Y_0'$ and $Y_0$ are adjacent in $S(X,B,\Mm)$ and $Y_0'\subsetneq Y_0$. Possibly replacing $Y$ with $Y_0$ and $Y'$ with $Y_0'$, we may assume that $Y$ is an lc center of $(X,B,\Mm)$. Since  $Y$ and $Y'$ are adjacent in $S(X,B,\Mm)$, $Y'$ is the union of all lc centers of $(X,B,\Mm)$ that are contained in $Y$.
    
We let $(X',B',\Mm)$ be a dlt model of $(X,B,\Mm)$ with induced birational morphism $h: X'\rightarrow X$. Let $S$ be an lc center of $(X',B',\Mm)$ which is minimal in all lc centers which dominate $Y$, $(S,B_S,\Mm^S)/U$ the dlt g-pair induced by adjunction 
$$K_S+B_S+\Mm^S_S:=(K_{X'}+B'+\Mm_{X'})|_S,$$
and $h_S: S\rightarrow Y^n$ the induced morphism such that $\pi\circ h_S=h|_S$. Let
$$S\xrightarrow{\tau} Z\xrightarrow{\gamma} Y^n$$
be the Stein factorization of $h_S$, and let $(Z,B_Z,\Mm^Z)/U$ be the lc g-pair induced by $\tau: (S,B_S,\Mm^S)\rightarrow Z$. Let $(Y^n,B_{Y^n},\Mm^{Y^n})/U$ be the lc g-pair induced by $\gamma: (Z,B_Z,\Mm^Z)\rightarrow Y^n$, and let $Y'_Z:=\gamma^{-1}(Y'')$.

 By Lemma \ref{lem: gdlt crepant log structure is compatible under subadjunction}(3) and Theorem \ref{thm: cbf gpair nonnqc}(5), for any lc center $V$ of $(X,B,\Mm)$ such that $V\subset Y^n$, any irreducible component of $\gamma^{-1}(V)$ is an lc center of $(Z,B_Z,\Mm^Z)$. In particular, any irreducible component of $Y'_Z$ is an lc center of $(Z,B_Z,\Mm^Z)$, so $\Nklt(Z,B_Z,\Mm^Z)\subset Y'_Z$. By Theorem \ref{thm: cbf gpair nonnqc}(4), $(\pi\circ\gamma)(\Nklt(Z,B_Z,\Mm^Z))$ is a union of lc centers of $(X,B,\Mm)$ that are contained in $Y$, so $\Nklt(Z,B_Z,\Mm^Z)\subset Y'_Z$. Thus  $$\Nklt(Z,B_Z,\Mm^Z)=Y'_Z=\gamma^{-1}(Y'').$$
 Since $\dim Z<\dim X$, by induction on $\dim X$,
     $$R^i(f\circ \pi''\circ\gamma)_*\mathcal{O}_{Y'_Z}(D|_{Y'_Z})=0.$$
     By Lemma \ref{lem: vanishing keep under finite mor}, the claim follows.
\end{proof}
\noindent\textit{Proof of Theorem \ref{thm: kod vanishing with lc strata} continued}. By the short exact sequence \eqref{eq: short exact sequence in main theorem}, we have a short exact sequence
$$0\rightarrow \mathcal{O}_Y(D)\xrightarrow{\pi^*\oplus i^*}\pi_*\mathcal{O}_{Y^n}(D|_{Y^n})\oplus \mathcal{O}_{Y'}(D) \xrightarrow{j^*-\pi''^*} \pi''_*\mathcal{O}_{Y''}(D|_{Y''})\rightarrow 0,$$
which induces a long exact sequence
\begin{align*}
0&\rightarrow f_*\mathcal{O}_Y(D)\rightarrow f_*\pi_*\mathcal{O}_{Y^n}(D|_{Y^n})\oplus f_*\mathcal{O}_{Y'}(D)  \xrightarrow{j^*-\pi''^*} f_*\pi''_*\mathcal{O}_{Y''}(D|_{Y''})\rightarrow \cdots\\
\cdots &\rightarrow R^if_*\mathcal{O}_Y(D) \rightarrow R^if_*\big(\pi_*(\mathcal{O}_{Y^n}(D|_{Y^n})\big)\oplus R^if_*\mathcal{O}_{Y'}(D)\rightarrow  R^if_*\big(\pi''_*\mathcal{O}_{Y''}(D|_{Y''})\big)\rightarrow\cdots.
\end{align*}
Hence, it follows from \eqref{eq3.1}, \eqref{eq3.2}, \eqref{eq3.3} and Lemma \ref{lem:3.2} that $R^if_*\mathcal{O}_Y(D)=0$ for any positive integer $i$.

\medskip

\noindent\textbf{Step 3}. In this step we prove (2) and prove (3)(4) when $Y=W=\Nklt(X,B,\Mm)$. 

We have the long exact sequence
\begin{align*}
    0&\rightarrow f_*(\mathcal{I}_W\otimes \mathcal{O}_X(D))\rightarrow f_*\mathcal{O}_X(D)\rightarrow f_*\mathcal{O}_W(D)\rightarrow\dots\\
    \dots&\rightarrow R^if_*(\mathcal{I}_W\otimes \mathcal{O}_X(D))\rightarrow R^if_*\mathcal{O}_X(D)\rightarrow R^if_*\mathcal{O}_W(D)\rightarrow\dots
\end{align*}
By (1), $R^if_*\mathcal{O}_W(D)=0$ for any positive integer $i$. By Lemma \ref{lem:3.1}(1), $R^i(\mathcal{I}_W\otimes f_*\mathcal{O}_X(D))=0$ for any positive integer $i$. This implies (2), and also implies (3)(4) when $Y=W$.

\medskip

\noindent\textbf{Step 4}. We prove (3)(4) in this step, hence conclude the proof of the theorem.

We apply induction on $m(W)-m(Y)$. When $m(W)-m(Y)=0$, $Y=W$ and we are done by \textbf{Step 3}. Thus we may assume that $m(W)-m(Y)>0$. Then there exists a union of lc centers $\tilde Y$ such that $Y\subsetneq\tilde Y\subset W$, and $Y$ and $\tilde Y$ are adjacent in $S(X,B,\Mm)$.


Let $\tilde\pi:\tilde{Y}^n\rightarrow \tilde{Y}$ be the normalization of $\tilde Y$, and let $\widehat Y:=\tilde\pi^{-1}(Y)$ with the reduced scheme structure. Let $\tilde i: Y\hookrightarrow\tilde Y$ and  $\tilde j: \widehat{Y}\hookrightarrow\tilde Y^n$  be the natural inclusions, and let $\widehat{\pi}:=\tilde\pi|_{\widehat{Y}}$. By Lemma \ref{lem:pushout2}, there exists a universal push-out diagram
\begin{center}
$\xymatrix{
\widehat{Y}\ar@{^(->}[r]^{\tilde j}\ar@{->}[d]_{\widehat\pi} & \tilde Y^n\ar@{->}[d]^{\tilde \pi}\\
Y\ar@{^(->}[r]^{\tilde i}&\tilde Y\\
}$
\end{center}
and a short exact sequence
\begin{align*}
0\to \Oo_{\tilde Y}\xrightarrow{\tilde\pi^*\oplus\tilde i^*} \tilde\pi_*\Oo_{\tilde Y^n}\oplus\Oo_{Y}\xrightarrow{\tilde j^*-\widehat{\pi}^*} \widehat{\pi}_*\Oo_{\widehat{Y}}\to 0.
\end{align*}
which induces a short exact sequence
$$0\rightarrow \mathcal{O}_{\tilde{Y}}(D)\xrightarrow{\tilde\pi^*\oplus\tilde i^*} \tilde\pi_*\mathcal{O}_{\tilde{Y}^n}(D|_{\tilde{Y}^n})\oplus \mathcal{O}_{Y}(D)  \xrightarrow{\tilde j^*-\widehat{\pi}^*} \widehat{\pi}_*\mathcal{O}_{\widehat{Y}}(D|_{\widehat{Y}})\rightarrow 0.$$
So we have the left exact sequence
\begin{align}\label{eq:long}
0\rightarrow f_*\mathcal{O}_{\tilde{Y}}(D)\xrightarrow{\tilde\pi^*\oplus\tilde i^*} f_*\tilde\pi_*\mathcal{O}_{\tilde{Y}^n}(D|_{\tilde{Y}^n})\oplus f_*\mathcal{O}_{Y}(D)  \xrightarrow{\tilde j^*-\widehat{\pi}^*} f_*\widehat{\pi}_*\mathcal{O}_{\widehat{Y}}(D|_{\widehat{Y}}).
\end{align}
By Lemma \ref{lem:3.2},
$$\tilde j^*: f_*\tilde\pi_*\mathcal{O}_{\tilde{Y}^n}(D|_{\tilde{Y}^n})\rightarrow f_*\widehat{\pi}_*\mathcal{O}_{\widehat{Y}}(D|_{\widehat{Y}})$$
is surjective. Thus by an easy map tracing of \eqref{eq:long} we have that
$$\tilde i^*: f_*\mathcal{O}_{\tilde{Y}}(D)\rightarrow f_*\mathcal{O}_{Y}(D)$$
is also surjective. Since  $m(W)-m(\tilde Y)<m(W)-m(Y)$, by induction on $m(W)-m(Y)$, $$f_*\mathcal{O}_X(D)\rightarrow f_*\mathcal{O}_{ \tilde{Y}}(D)$$ is surjective. This implies (3).

We have the long exact sequence
\begin{align*}
    0&\rightarrow f_*(\mathcal{I}_Y\otimes \mathcal{O}_X(D))\rightarrow f_*\mathcal{O}_X(D)\rightarrow f_*\mathcal{O}_Y(D)\rightarrow\dots\\
    \dots&\rightarrow R^if_*(\mathcal{I}_Y\otimes \mathcal{O}_X(D))\rightarrow R^if_*\mathcal{O}_X(D)\rightarrow R^if_*\mathcal{O}_Y(D)\rightarrow\dots,
\end{align*}
so (4) follows immediately from (1)(2)(3).
\end{proof}



\subsection{Base-point-freeness theorem and contraction theorem}\label{subsec: bpf nonnqc}

\begin{lem}\label{lem: non-vanishing of lc gpair}
 Let $(X,B,\Mm)/U$ be an lc g-pair and $D$ a nef$/U$ Cartier divisor on $X$ such that $aD-(K_X+B+\Mm_X)$ is ample$/U$ for some positive real number $a$. Let $Y$ be a minimal lc center of $(X,B,\Mm)$ if $(X,B,\Mm)$ is not klt, and let $Y:=X$ if $(X,B,\Mm)$ is klt. Let $D_Y:=D|_Y$. Then for any integer $m\gg 0$,
\begin{enumerate}
    \item  $\mathcal{O}_{Y}(mD_Y)$ is globally generated over $U$,
    \item  $|mD/U|\not=\emptyset$, and
    \item $Y$ is not contained in $\Bs|mD/U|$.
\end{enumerate}
\end{lem}
\begin{proof}
When $(X,B,\Mm)$ is klt, by \cite[Lemma 3.4]{HL22}, there exists a klt pair $(X,\Delta)$ such that $D-(K_X+\Delta)$ is ample$/U$. By the usual base-point-freeness theorem (cf. \cite[Theorem 3-1-1]{KMM87}), the lemma follows.

When $(X,B,\Mm)$ is not klt, by Theorem \ref{thm: (Z,S) is HN and HSN}, $Y$ is normal. By Theorem \ref{thm: kod vanishing with lc strata}(3), the map $f_*\mathcal{O}_X(mD)\rightarrow f_*\mathcal{O}_Y(mD_Y)$ is surjective for any positive integer $m\geq a$. Thus (2)(3) follow from (1) and we only need to prove (1). If $\dim Y=0$ then there is nothing left to prove. If $\dim Y>0$, then by  Definition-Lemma \ref{deflem: subadj minimal lc center}, there exists a klt g-pair $(Y,B_Y,\Mm^{Y})/U$ such that $K_{Y}+B_{Y}+\Mm^{Y}_{Y}\sim_{\Rr,U}(K_X+B+\Mm_X)|_{Y}$. Thus $D_Y-(K_{Y}+B_{Y}+\Mm^{Y}_{Y})$ is nef$/U$ and log big$/U$ with respect to $(Y,B_Y,\Mm^Y)$. By \cite[Lemma 3.4]{HL22}, there exists a klt pair $(Y,\Delta_Y)$ such that $D_Y-(K_Y+\Delta_Y)$ is ample$/U$. By the usual base-point-freeness theorem (cf. \cite[Theorem 3-1-1]{KMM87}), the lemma follows.
\end{proof}



\begin{proof}[Proof of Theorem \ref{thm:base-point-freeness intro}]
By Lemma \ref{lem: non-vanishing of lc gpair}, we may let $m_0$ be the minimal positive integer such that $|mD|\not=\emptyset$ for any integer $m\geq m_0$.

\begin{claim}\label{claim: induction bs}
Let $\{p_i\}_{i=1}^{+\infty}$ be a strictly increasing sequence of positive integers. There exist a non-negative integer $M$ and integers $i_1<i_2<\dots<i_{M+1}$ satisfying the following. Let $s_k:=\prod_{l=1}^kp_{i_l}$ for any $1\leq k\leq M+1$, then
\begin{enumerate}
    \item $|s_1D/U|\not=\emptyset$,
    \item $\Bs|s_kD/U|\supsetneq\Bs|s_{k+1}D/U|$ for any $1\leq k\leq M$, and
    \item $\Bs|s_{M+1}D/U|=\emptyset$.
\end{enumerate}
\end{claim}
\begin{proof}
We may take $i_1$ to be any integer such that $p_{i_1}\geq m_0$, then (1) holds. 

Suppose that we have already found $i_1,\dots,i_k$ for some positive integer $k$. Let $d:=\dim X$, let $H_{1},\cdots,H_{d+1}$ be $d+1$ be general elements in $|s_kD/U|$, and let $H:=H_{1}+\cdots+H_{d+1}$. Then $(X,B+H,\Mm)$ is lc outside $\Bs|s_kD/U|$. If $\Bs|s_kD/U|=\emptyset$, then we may let $M:=k-1$ and we are done. Thus we may assume that $\Bs|s_kD/U|\not=\emptyset$.

Since every $H_{j}$ contains $\Bs|s_kD/U|$, by \cite[Theorem 18.22]{Kol+92}, $(X,B+H,\Mm)$ is not lc near $\Bs|s_kD/U|$. Let 
$$c:=\sup\{t\mid t\geq 0, (X,B+tH,\Mm)\text{ is lc}\},$$
then $c\in [0,1)$, and there exists at least one lc center of $(X,B+cH,\Mm)$ which is contained in $\Bs|s_kD/U|$. Let
$\mathcal{S}$ be the set of all lc centers of $(X,B+cH,\Mm)$ that are contained in $\Bs|s_kD/U|$, and let $Y$ be a minimal lc center in $\mathcal{S}$. Since
$$(a+s_k(d+1))D-(K_X+B+cH+\Mm_X)\sim_{\mathbb R}s_k(d+1)(1-c)D+(aD-(K_X+B+\Mm_X))$$
is ample$/U$, by Lemma \ref{lem: non-vanishing of lc gpair}, there exists a positive integer $n$, such that for any integer $m\geq n$, $|ms_kD/U|\not=\emptyset$ and $\Bs|ms_kD/U|$ does not contain $Y$. In particular, $\Bs|ms_kD/U|\subsetneq\Bs|s_kD/U|$. We may let $i_{k+1}$ be any integer such that $i_{k+1}>i_{k}$ and $p_{i_{k+1}}\geq n$. This construction implies (2). (3) follows from (2) and the Noetherian property.
\end{proof}

\noindent\textit{Proof of Theorem \ref{thm:base-point-freeness intro} continued}. We let $p$ and $q$ be two different prime numbers. By Claim \ref{claim: induction bs}, there exist two non-negative integers $M,N$ such that $\mathcal{O}_X(p^MD)$ and $\mathcal{O}_X(q^ND)$ are globally generated$/U$. Since $p^M$ and $q^N$ are coprime, for any integer $m\gg 0$, we may write $m=bp^M+cq^N$ for some non-negative integers $b,c$, hence
$$\Bs|mD/U|\subset\Bs|p^MD/U|\cup\Bs|q^ND/U|=\emptyset.$$
Therefore, $\mathcal{O}_X(mD)$ is globally generated over $U$ for any integer $m\gg 0$.
\end{proof}





\begin{thm}[Contraction theorem for lc generalized pairs, cf. {\cite[Theorem 1.5]{Xie22}}]\label{thm: cont thm gpair}
Let $(X,B,\Mm)/U$ be an lc generalized pair and $F$ a $(K_X+B+\Mm_X)$-negative extremal face$/U$. Then there exists a contraction$/U$ $\cont_F: X\rightarrow Z$ of $F$ satisfying the following.
\begin{enumerate}
    \item For any integral curve $C$ on $X$ such that the image of $C$ in $U$ is a closed point, $\cont_F(C)$ is a point if and only if $[C]\in F$.
    \item $\mathcal{O}_Y=(\cont_F)_*\mathcal{O}_X$. In other words, $\cont_F$ is a contraction.
    \item For any Cartier divisor $D$ on $Y$ such that $D\cdot C=0$ for any curve $C$ contracted by $\cont_F$, there exists a Cartier divisor $D_Y$ on $Y$ such that $D=\cont_F^*D_Y$.
\end{enumerate}
\end{thm}
\begin{proof}[Proof of Theorem \ref{thm: cont thm gpair}]
(1)(2) By Theorem \ref{thm: cone theorem gfq},  $F$ is a finitely dimensional rational  $(K_X+B+\Mm_X)$-negative extremal face$/U$. Thus there exists a nef Cartier divisor $L$ on $X$ that is the supporting function of $F$. Then $L-(K_X+B+\Mm_X)$ is ample. By Theorem \ref{thm:base-point-freeness intro}, $mL$ is base-point-free$/U$, hence defines a contraction$/U$. Denote this contraction by $\cont_F$. Then $\cont_F$ satisfies (1) and (2).

(3) Since $D-(K_X+B+\Mm_X)$ is ample$/Z$, by Theorem \ref{thm:base-point-freeness intro}, $\mathcal{O}_X(mD)$ is globally generated over $Z$ for any integer $m\gg 0$. Since $D\cdot C$ for any curve $C$ contracted by $\cont_F$, $\cont_F$ is defined by $|mD|$ for any integer $m\gg 0$. Thus $mD=f^*D_{Y,m}$ and $(m+1)D=f^*D_{Y,m+1}$ for any integer $m\gg 0$. We may let $D_Y:=D_{Y,m+1}-D_{Y,m}$.
\end{proof}


\begin{proof}[Proof of Theorem \ref{thm: semi-ampleness intro}]
We write $D=\sum_{i=1}^c r_iD_i$ where $r_1,\dots,r_c$ are linearly independent over $\mathbb Q$ and each $D_i$ is a $\Qq$-divisor. We define $D(\bm{v}):=\sum_{i=1}^cv_iD_i$ for any $\bm{v}=(v_1,\dots,v_c)\in\mathbb R^c$, and let $\bm{r}:=(r_1,\dots,r_c)$. By \cite[Lemma 5.3]{HLS19}, each $D_i$ is $\Qq$-Cartier, so $\mathcal{D}(\bm{v})$ is $\Qq$-Cartier for any $\bm{v}\in\mathbb R^c$. 

Let $L:=D-(K_X+B+\Mm_X)$. Since ample$/U$ is an open condition, there exists an open set $V\ni\bm{r}$ in $\mathbb R^c$, such that 
$\frac{1}{2}L+D(\bm{v})-D$ is ample$/U$ for any $\bm{v}\in V$.

By Theorem \ref{thm: cone theorem gfq}, there exist finitely many $(K_X+B+\Mm_X+\frac{1}{2}L)$-negative extremal rays$/U$ $R_1,\dots,R_l$, and each $R_j=\mathbb R_+[C_j]$ for some rational curve $C_j$ such that
$$-2\dim X\leq (K_X+B+\Mm_X+\frac{1}{2}L)\cdot C_j<0.$$
Since $D$ is nef, $D\cdot C_j\geq 0$ for each $j$. Thus possibly shrinking $V$, we may assume that for any $\bm{v}\in V$, we have that $D(\bm{v})\cdot C_j>0$ for any $j$ such that $D\cdot C_j>0$. Since $r_1,\dots,r_c$ are linearly independent over $\mathbb Q$, for any $j$ such that $D\cdot C_j=0$, we have $D(\bm{v})\cdot C_j=0$ for any $\bm{v}\in\mathbb R^c$. Therefore,  $D(\bm{v})\cdot C_j\geq 0$ for any $j$ and any $\bm{v}\in V$.

For any extremal ray $R$ in $\overline{NE}(X/U)$ and any $\bm{v}\in V$, if $R=R_j$ for some $j$, then $D(\bm{v})\cdot R_j\geq 0$. If $R\not=R_j$ for any $j$, then
$$D(\bm{v})\cdot R_j=(K_X+B+\Mm_X+\frac{1}{2}L)\cdot R+(\frac{1}{2}L+D(\bm{v})-D)\cdot R>0.$$
Therefore, $D(\bm{v})$ is nef$/U$ for any $\bm{v}\in V$. Moreover,
$$D(\bm{v})-(K_X+B+\Mm_X)=\frac{1}{2}L+\frac{1}{2}L+D(\bm{v})-D$$
is ample$/U$.

We let $\bm{v}_1,\dots,\bm{v}_{c+1}\in V\cap\mathbb Q^c$ be rational points such that $\bm{r}$ is in the interior of the convex hull of $\bm{v}_1,\dots,\bm{v}_{c+1}$. Then there exists positive real numbers $a_1,\dots,a_{c+1}$ such that $\sum_{i=1}^{c+1}a_i=1$ and $\sum_{i=1}^{c+1}a_i\bm{v}_i=\bm{r}$. Since $D(\bm{v}_i)$ is a nef$/U$ $\Qq$-divisor and $D(\bm{v}_i)-(K_X+B+\Mm_X)$ is ample$/U$, by Theorem \ref{thm:base-point-freeness intro}, $D(\bm{v}_i)$ is semi-ample$/U$ for any $i$. Therefore, $D=\sum a_iD(\bm{v}_i)$ is semi-ample$/U$.
\end{proof}



\section{Existence of flips for generalized pairs}\label{sec: eof gpair}
%contractions and 

The goal of this section is to show the existence of flips for $\Qq$-factorial lc generalized pairs. We remark that  \cite[Theorem 1.2]{HL21a} proves the case when the generalized pairs are NQC. Our proof does not rely on the results in \cite{HL21a}.

The following lemma is crucial for the proof of the existence of flips.

\begin{lem}\label{lem: flip reduce special gpair to pair}
Let $(X,B,\Mm)/U$ be an lc g-pair such that the induced morphism $\pi: X\rightarrow U$ is birational. Assume that there exists a non-empty open subset $U^0\subset U$, such that
\begin{enumerate}
    \item all lc centers of $(X,B,\Mm)$ intersect $X^0:=X\times_U U^0$, and
    \item $\Mm^0:=\Mm\times_U U^0$ descends to $X^0$, and $\Mm^0_{X^0}\sim_{\Rr,U^0}0$.
\end{enumerate}
Then there exists an $\Rr$-divisor $0\leq G\sim _{\Rr,U}\Mm _X$ such that $(X,B+G)$ is lc and $\Nklt(X,B+G)=\Nklt(X,B,\Mm)$.
\end{lem}

\begin{proof}
Let $f:\tilde{X}\to X$ be a log resolution of $(X,\Supp B)$ such that $\Mm$ descends to $\tilde{X}$, we may write 
$$K_{\tilde{X}}+\tilde B+\Mm_{\tilde{X}}=f^*(K_X+B+\Mm_X)$$
for some $\Rr$-divisor $\tilde{B}$. Since 
$\pi$ is birational, $\Mm_{\tilde{X}}$ is big$/U$ and nef$/U$. Thus there exists a $\Qq$-divisor $E\ge 0$, such that for any positive integer $k$, there exists an ample$/U$ $\Rr$-divisor $A_k$ on $\tilde X$ such that
$$\Mm_{\tilde{X}}=A_k+\frac{1}{k}E.$$ 
Moreover, for any $k\gg 1$, $\Nklt(X,\tilde B+\frac{1}{k}E)$ is contained in the strata of $\lfloor \tilde B\rfloor$. By \cite[Lemma 5.3]{HLS19}, there exist $\Qq$-divisors $\tilde{M}_i$ and positive real numbers $a_i$, such that 
\begin{itemize}
\item $\sum a_i=1$,
    \item $\Mm_{\tilde{X}}=\sum a_i\tilde{M}_i$,
    \item $\tilde{M}_i-\Mm_{\tilde{X}}+A_k$ is ample$/U$ for each $i$, and
    \item $\tilde{M}_i|_{\tilde{X}^0}\sim_{\Qq,U^0}0$, where $\tilde{X}^0:=\tilde{X}\times_{U}U_0$.
\end{itemize}
Let $m$ be a positive integer such that $m\tilde{M}_i$ is a Weil divisor and $m\tilde{M}_i|_{\tilde{X}^0}\sim_{U^0}0$ for each $i$. Then there exists a very ample divisor $H\ge 0$ on $U$, such that $\phi_{*}\mathcal{O}_{\tilde{X}}(m\tilde{M}_i)\otimes H$ is globally generated for each $i$, where $\phi:=\pi\circ f$. In particular, $\mathcal{O}_{\tilde{X}}(m\tilde{M}_i)\otimes \phi^{*}H$ is globally generated over $U^0$. Thus for any general element $D_i\in |m\tilde{M}_i+\phi^{*}H|$, any non-klt center of $(X,\tilde B+\frac{1}{m}\sum a_iD_i)$ is contained in $\tilde{X}\backslash \tilde{X}^0$. 

Since 
$$m\tilde{M}_i+\phi^{*}H=m(\tilde{M}_i-\Mm_{\tilde{X}}+A_k)+\frac{m}{k}E+\phi^{*}H,$$ possibly replacing $m$ by a multiple, we may
may assume that $\frac{m}{k}E$ is Cartier. Thus $m(\tilde{M}_i-\Mm_{\tilde{X}}+A_k)$ is ample$/U$ and Cartier, and for any general element $\tilde{G}_i\in |m\tilde{M}_i+\phi^{*}H|$, any non-klt center of  $(X,\tilde B+\frac{1}{m}\sum a_i\tilde{G}_i)$ is a stratum of $\lfloor\tilde B\rfloor$ which intersects $\tilde{X}^0$. 


Therefore, for each $i$, we may choose $\tilde{G}_i\in |m\tilde{M}_i+\phi^{*}H|$, such that 
$$\left(X,\tilde B+\frac{1}{m}\sum a_i\tilde{G}_i\right)$$ is sub-lc. Let $\tilde{G}=\frac{1}{m}\sum a_i\tilde{G}_i$, and $G:=f_*\tilde{G}$. Then $0\leq G\sim _{\Rr,U}\Mm _X$, $(X,B+G)$ is lc, and $\Nklt(X,B+G)=\Nklt(X,B,\Mm)$.
\end{proof}



\begin{defn}[Flipping contraction]\label{defn: flipping contraction}
Let $X\rightarrow U$ be a projective morphism such that $X$ is normal quasi-projective and $D$ an $\Rr$-Cartier $\Rr$-divisor on $X$. A \emph{$D$-flipping contraction} over $U$ is a contraction $f: X\rightarrow Z$ over $U$ satisfying the following:
\begin{enumerate}
\item $X$ is $\Qq$-factorial,
 \item $f$ is a small birational morphism, and
    \item $f$ is the contraction of a $D$-negative extremal ray $R$ in $\overline{NE}(X/U)$. In particular, $\rho(X/Z)=1$.
\end{enumerate}
\end{defn}

\begin{defn}[Flip]\label{defn: flip}
Let $X$ be a normal quasi-projective variety, $D$ an $\Rr$-Cartier $\Rr$-divisor on $X$, and $f: X\rightarrow Z$ a $D$-flipping contraction. A \emph{$D$-flip} is a birational contraction $f^+: X^+\rightarrow Z$ satisfying the following.
\begin{enumerate}
\item $X^+$ is a normal quasi-projective variety,
    \item $f^+$ is small, and
    \item $D^+$ is $\Rr$-Cartier and ample$/Z$, where $D^+$ is the strict transform of $D$ on $X^+$.
\end{enumerate}
We call $f^{+}$ the \emph{flip} of $f$.
\end{defn}

\begin{lem}\label{lem: flip keep under rlinearequivalence}
Let $X$ be a normal quasi-projective variety and $D$ and $D'$ two $\Rr$-Cartier $\Rr$-divisors on $X$.  Let $f: X\rightarrow Z$ a $D$-flipping contraction such that $D\equiv_{Z}rD'$ for some real number $r>0$. Then:
\begin{enumerate}
    \item $f$ is a $D'$-flipping contraction. 
    \item Suppose that $f^+: X^+\rightarrow Z$ is a $D'$-flip. Assume that either $D\sim_{\mathbb R,Z}rD'$, or $D=K_X+B+\Mm_X$ for some lc g-pair $(X,B,\Mm)/Z$. Then $f^+$ is also a $D$-flip.
\end{enumerate}
\end{lem}
\begin{proof}
    Since $D\sim_{\mathbb R,U}rD'$, the unique $D$-negative extremal ray in $\overline{NE}(X/Z)$ is also a $D'$-negative extremal ray, and we get (1). 

By Theorem \ref{thm: cont thm gpair}(3) we may assume that  $D\sim_{\mathbb R,Z}rD'$. Let $D^+$ and $D'^+$ be the strict transform of $D$ and $D'$ on $X^+$ respectively. We have $D-rD'\sim_{\mathbb R}f^*L$ for some $\Rr$-Cartier $\Rr$-divisor $L$ on $Z$. Since $X\dashrightarrow X^+$ is small, $D^+-rD'^+\sim_{\mathbb R}(f^+)^*L$. Since $D^+$ is $\Rr$-Cartier and ample$/Z$, $D'^+$ is $\Rr$-Cartier and ample$/Z$. This implies (2).
\end{proof}

\begin{lem}\label{lem: uniqueness flip}
    Let $(X,B,\Mm)/U$ be an lc g-pair and $f: X\rightarrow Z$ a $(K_X+B+\Mm_X)$-flipping contraction$/U$. Suppose that the flip $f^+: X^+\rightarrow Z$ of $f$ exists. Then:
 \begin{enumerate}
     \item $f^+$ is unique.
     \item For any $\Rr$-Cartier $\Rr$-divisor $D$ on $X$, the strict transform of $D$ on $X^+$ is $\Rr$-Cartier.
     \item If $X$ is $\Qq$-factorial, then $X^+$ is $\Qq$-factorial and $\rho(X)=\rho(X^+)$.
 \end{enumerate}
\end{lem}
\begin{proof}
   Let $H$ be an anti-ample$/Z$ divisor on $X$. Since $\rho(X/Z)=1$ and $K_X+B+\Mm_X$ is anti-ample$/Z$, there exist a positive real number $r$ and a real number $s$ such that $H-r(K_X+B+\Mm_X)\equiv_Z0$ and $D-s(K_X+B+\Mm_X)\equiv_Z0$. By Theorem \ref{thm: cont thm gpair}(3),  $H-r(K_X+B+\Mm_X)\sim_{\mathbb R,Z}0$ and $D-s(K_X+B+\Mm_X)\sim_{\mathbb R,Z}0$.  

   (1) By Lemma \ref{lem: flip keep under rlinearequivalence}, $f^+$ is an $H$-flip. Thus
   $$X^+=\proj\left(\bigoplus_{m=0}^{+\infty}f_*\mathcal{O}_X(mH)\right)$$
   is unique.

   (2) We have $D-s(K_X+B+\Mm_X)\sim_{\mathbb R}f^*L$ for some $\Rr$-Cartier $\Rr$-divisor $L$ on $Z$. Let $D^+$ and $B^+$ be the strict transform of $D$ and $B$ on $X^+$ respectively. Since $X\dashrightarrow X^+$ is small, $D^+-s(K_{X^+}+B^++\Mm_{X^+})\sim_{\mathbb R}(f^+)^*L$. Since $K_{X^+}+B^++\Mm_{X^+}$ is $\Rr$-Cartier, $D^+$ is $\Rr$-Cartier.

   (3) Since $X\dashrightarrow X^+$ is an isomorphism in codimension 1, there is a natural isomorphism between the groups of Weil divisors on $X$ and $X^+$. By (2), if $X$ is $\Qq$-factorial, then $X^+$ is $\Qq$-factorial. Since $X$ and $X^+$ are both $\mathbb Q$-factorial, $\rho(X)=\rho(X^+)$. 
\end{proof}

\begin{thm}[Existence of flips]\label{thm: existence of q-factorial glc flips}
Let $(X,B,\Mm)/U$ be an lc g-pair and $f: X\rightarrow Z$ a $(K_X+B+\Mm_X)$-flipping contraction$/U$. Assume that $\Mm_X$ is $\Rr$-Cartier. Then the $(K_X+B+\Mm_X)$-flip $f^+: X^+\rightarrow Z$ of $f$ exists. Moreover,
\begin{enumerate}
    \item $\rho(X^+/Z)=1$,
    \item $\Mm_{X^+}$ is $\mathbb R$-Cartier, and
    \item If $X$ is $\Qq$-factorial, then $X^+$ is $\mathbb Q$-factorial and $\rho(X)=\rho(X^+)$.
\end{enumerate}
\end{thm}
\begin{proof}[Proof of Theorem \ref{thm: existence of q-factorial glc flips}]
Let $h: \tilde X\rightarrow X$ be a birational morphism such that $\Mm$ descends to $\tilde X$. Since $\Mm_X$ is $\Rr$-Cartier and $\Mm_{\tilde X}$ is nef$/X$, we have
$$\Mm_{\tilde X}+E=h^*\Mm_X$$
for some $E\geq 0$ that is exceptional over $X$. Let $C$ be any flipping curve contracted by $f$. 

There are two cases:

\medskip

\noindent\textbf{Case 1}. $\Mm_X\cdot C\geq 0$. Then $(K_X+B)\cdot C<0$, and $f$ is also a $(K_X+B)$-flipping contraction. By \cite[Corollary 1.2]{Bir12}, \cite[Corollary 1.8]{HX13}, there exists a $(K_X+B)$-flip $f^+: X^+\rightarrow Z$, such that $\rho(X^+/Z)=1$. (1) follows. By Lemma \ref{lem: flip keep under rlinearequivalence}, $f^+: X^+\rightarrow Z$ is a $(K_X+B+\Mm_X)$-flip. (2-3) follow from Lemma \ref{lem: uniqueness flip}.


\medskip

\noindent\textbf{Case 2}. $\Mm_X\cdot C<0$. In this case, $C\subset h(E)$. Let $Z^0:= Z\backslash\{f(h(\Supp E))\}$, $X^0:=X\times_Z Z^0$, $B^0:=B\times_Z Z^0$, and $\Mm^0:=\Mm\times_Z Z^0$. Since $h(\Supp E)$ does not contain any lc center of $(X,B,(1-\epsilon)\Mm)$, for any $\epsilon\in (0,1)$,
\begin{itemize}
\item all lc centers of $(X,B,(1-\epsilon)\Mm)$ intersect $X^0$,
\item $\Mm^0$ descends to $X^0$ and $\Mm^0_{X^0}\sim_{\Rr,Z^0}0$. 
\end{itemize}
Let $\epsilon_0\in (0,1)$ be a real number such that $f$ is also a $(K_X+B+(1-\epsilon_0)\Mm_X)$-flipping contraction. By Lemma \ref{lem: flip reduce special gpair to pair}, there exists an $\Rr$-divisor $G\sim _{\Rr,Z}(1-\epsilon_0)\Mm _X$ such that $(X,B+G)$ is lc. By Lemma \ref{lem: flip keep under rlinearequivalence}, $f$ is a $(K_X+B+G)$-flip. By \cite[Corollary 1.2]{Bir12}, \cite[Corollary 1.8]{HX13}, the flip $f^+: X^+\rightarrow Z$ of $f$ exists and $\rho(X^+/Z)=1$. (1) follows. By Lemma \ref{lem: flip keep under rlinearequivalence}, $f^+$ is a $(K_X+B+\Mm_X)$-flip.  (2-3) follow from Lemma \ref{lem: uniqueness flip}.
\end{proof}

\part{Good minimal model and the proofs of the main theorems}\label{part:gmm}

\section{Existence of good minimal models and \texorpdfstring{$\bb$}{}-semi-ampleness}\label{sec: gmm fdlt}

\subsection{Good minimal models for polarized foliations}

We note that the subsequent lemma does not necessarily require $\Ff$ to be algebraically integrable, allowing its application to other scenarios.
\begin{lem}\label{lem: +a keep under mmp}
Let $(X,\Ff,B,\Mm)/U$ be an lc gfq and $A$ an ample$/U$ $\Rr$-divisor on $X$. Let 
$$\phi: (X,\Ff,B+A,\Mm)\dashrightarrow (X',\Ff',B'+A',\Mm)$$
be a sequence of steps of a $(K_{\Ff}+B+A+\Mm_X)$-MMP$/U$, where $B'$ and $A'$ are the images of $B$ and $A$ on $X'$ respectively. Then there exist a nef$/U$ $\bb$-divisor $\Nn$ and an ample$/U$ $\Rr$-divisor $\tilde A'$ on $X'$, such that 
\begin{enumerate}
    %\item $\tilde A'$ is ample$/U$,
    \item $(X',\Ff',B',\Nn)/U$ is lc,
    \item $\Nn_{X'}+\tilde A'\sim_{\mathbb R,U}\Mm_{X'}+A'$,
    \item $\Nn-\Mm$ is nef$/U$, and
    \item for any contraction $f: X\rightarrow Z$ and divisor $G$ on $X$ such that $(X,\Ff,B,\Mm;G)/Z$ satisfies Property $(*)$ (resp. is weak ACSS, is ACSS), $(X',\Ff',B',\Nn;G':=\phi_*G)/Z$ satisfies Property $(*)$ (resp. is weak ACSS, is ACSS).
\end{enumerate}
\end{lem}
\begin{proof}
    We may assume that $\phi$ is a single step of a MMP$/U$, and we have the following diagram$/U$
    \begin{center}$\xymatrix{
   X\ar@{->}[rd]^{g}\ar@{-->}[rr]^{\phi} & & X'\ar@{->}[dl]^{h}\\
    & T &
}$
\end{center}
such that either $\phi=g$ is a divisorial contraction, or $\phi$ is a flip, $g$ is the flipping contraction, and $h$ is the flipped contraction. Then there exists an ample$/T$ divisor $H$ on $X$ such that $K_{\Ff}+B+A+H+\Mm_X\sim_{\mathbb R,T}0$. Let $H':=\phi_*H$, then $-H'$ is ample$/T$. Since $A$ is ample$/U$, there exists an ample$/U$ $\Rr$-divisor $C$ on $T$ such that $A-g^*C$ is ample$/U$. 

Let $0<\epsilon\ll 1$ be a real number. Then $\tilde A':=h^*C-\epsilon H'$ and $L:=A-g^*C+\epsilon H$ are ample$/U$, and $\phi$ is a step of a $(K_{\Ff}+B+A+\Mm_X+\epsilon H)$-MMP$/T$. Since
$$K_{\Ff}+B+A+\Mm_X\sim_{\mathbb R,T}K_{\Ff}+B+L+\Mm_X-\epsilon H,$$
$\phi$ is a step of a $(K_{\Ff}+B+L+\Mm_X)$-MMP$/T$, hence a step of a $(K_{\Ff}+B+L+\Mm_X)$-MMP$/U$. Let $L':=\phi_*L$, then
$$\phi_*L+\tilde A'=A'.$$
We let $\Nn:=\Mm+\overline{L}$. By our construction, $\Nn$ and $\tilde A'$ satisfy (2) and (3). Since $\Nn-\Mm$ descends to $X$ and $(X,\Ff,B,\Mm)$ is lc,  $(X,\Ff,B,\Nn)$ is lc. Since $\phi$ is a step of a $(K_{\Ff}+B+\Nn_X)$-MMP$/U$, $(X',\Ff',B',\Nn)$ is lc, which implies (1). If $(X,\Ff,B,\Mm;G)/Z$ satisfies Property $(*)$ (resp. is weak ACSS, is ACSS), then $(X,\Ff,B,\Nn;G)/Z$ satisfies Property $(*)$ (resp. is weak ACSS, is ACSS). (4) follows from Lemma \ref{lem: ACSS mmp can run}.
\end{proof}

\begin{lem}\label{lem:superundercbf}
Let $(X,B,\Mm)/U$ be an lc g-pair and $f:X\to Z$ a contraction such that $B$ is super$/Z$. Assume that $\phi:X\to T$ is a contraction$/U$ such that $K_X+B+\Mm_X\sim_{\Rr,T}0$ and $\phi$ is also a contraction$/Z$. Let $B_T$ be the discriminant part of $f: (X,B,\Mm)\rightarrow T$. Then $B_T$ is super$/Z$.
\end{lem}
\begin{proof}
Let $d:=\dim X$. Since  $B$ is super$/Z$, there exist ample Cartier divisors $H_1,\dots,H_{2d+1}$ on $Z$ such that $B\geq\sum_{i=1}^{2d+1} f^*H_i$. In particular, $B-\sum_{i=1}^{2d+1} f^*H_i\ge0$, and 
$$K_X+B-\sum_{i=1}^{2d+1} f^*H_i+\Mm_X\sim_{\Rr,T}0.$$
Let $B_{T}'$ be the discriminant part of $\phi: (X,B-\sum_{i=1}^{2d+1} f^*H_i,\Mm)\rightarrow T$ and let $\psi: T\rightarrow Z$ be the induced contraction. Then $B_T=B_T'+\sum_{i=1}^{2d+1}\psi^*H_i$, so $B_T$ is super$/Z$.
\end{proof}

\begin{thm}\label{thm: bpf induced gfq}
Let $d$ and $m$ be two positive integers, $(X,\Ff,B,\Mm)/U$ an lc gfq of dimension $d$, and $A\geq 0$ an ample$/U$ $\Rr$-divisor on $X$, such that
\begin{itemize}
  \item $\Ff$ is induced by a contraction $f: X\rightarrow Z$, 
  \item $K_{\Ff}+B+A+\Mm_X$ is nef$/U$, and
  \item $K_{\Ff}+B+\Mm_X\sim_{\mathbb R,Z}K_X+\Delta+\Nn_X$ for some lc g-pair $(X,\Delta,\Nn)/U$.
\end{itemize}
Then the followings hold.
\begin{enumerate}
  \item  $K_{\Ff}+B+A+\Mm_X$ is semi-ample$/U$.
  \item  The contraction$/U$ defined by $K_{\Ff}+B+A+\Mm_X$ is a contraction$/Z$.
  \item Suppose that 
  $$m(K_{\Ff}+B+\Mm_X)\sim_{Z}m(K_X+\Delta+\Nn_X)$$ 
  and $m(K_{\Ff}+B+A+\Mm_X)$ is Cartier. Then $$\mathcal{O}_X(nm(K_{\Ff}+B+A+\Mm_X))$$ is globally generated$/U$ for any integer $n\gg 0$.
\end{enumerate}
\end{thm}
\begin{proof}
Let $\pi: X\rightarrow U$ be the induced morphism and let $H'$ be a sufficiently ample Cartier divisor on $U$. Possibly replacing $A$ with $A+\pi^*H'$, we may assume that $A$ is ample.  Let $H_1,\dots,H_{2\dim X+1}$ be ample Cartier divisor on $Z$. Possibly replacing $\Delta$ with $\Delta+\sum_{i=1}^{2\dim X+1}f^*H_i$, we may assume that $\Delta$ is super$/Z$. By Lemma \ref{lem: equivalence over bases}, $K_{X}+\Delta+A+\Nn_X$ is nef$/U$.
%Possibly replacing $\Delta$ with $\Delta+f^*H_Z'$, we may assume that any $(K_{X}+\Delta+A+\Nn_X)$-negative extremal ray$/U$ is also over $Z$.
%Then there is no $(K_{\Ff}+B+A+\Mm_X)$-negative extremal ray$/U$, so there is no $(K_{\Ff}+B+A+\Mm_X)$-negative extremal ray$/U$ that is also over $Z$, and so there is no $(K_{X}+\Delta+A+\Nn_X)$-negative extremal ray$/U$ that is also over $Z$. 
By Theorem \ref{thm: semi-ampleness intro}, $K_{X}+\Delta+A+\Nn_X$ is semi-ample$/U$, so $K_{X}+\Delta+A+\Nn_X$ defines a contraction$/U$ $\phi: X\rightarrow T$. %Moreover, by Theorem \ref{thm:base-point-freeness intro}, if $K_X+\Delta+A+\Nn_X$ is Cartier, then $\phi$ is defined by $n(K_X+\Delta+A+\Nn_X)$ for any integer $n\gg 0$.
Since $\Delta$ is super$/Z$, by the the length of extremal rays (Theorem \ref{thm: cone theorem gfq}(2)), any $(K_{X}+\Delta+A+\Nn_X)$-trivial extremal ray in $\overline{NE}(X/U)$ is an extremal ray$/Z$, so $\phi$ is a contraction$/Z$. Therefore, $K_{\Ff}+B+A+\Mm_X\sim_{\mathbb R,T}0$. Let $\Ff_T$ be the foliation induced by the induced contraction $\psi: T\rightarrow Z$, then $\Ff=\phi^{-1}\Ff_T$. 

We let $H_T$ be a general ample$/U$ $\Rr$-divisor on $T$ such that $H:=A-\phi^*H_T$ is ample$/U$. By Theorem \ref{thm: cbf gfq}, there exist an lc gfq $(T,\Ff_T,B_T,\Mm^T)/U$ induced by a canonical bundle formula of $\phi: (X,\Ff,B,\overline{H}+\Mm)\rightarrow T$, and an lc g-pair $(T,\Delta_T,\Nn^T)/U$ induced by a canonical bundle formula of $\phi: (X,\Delta,\overline{H}+\Nn)\rightarrow T$. By Lemma \ref{lem:superundercbf}, $\Delta_T$ is super$/Z$.

We have
$$K_{\Ff_T}+B_T+\Mm^T_T\sim_{\mathbb R,Z}K_T+\Delta_T+\Nn^T_T.$$ 
Since $\phi$ is the morphism$/U$ defined by $K_{X}+\Delta+A+\Nn_X$, $K_T+\Delta_T+H_T+\Nn^T_T$ is ample$/U$. Thus $K_T+\Delta_T+(1-\delta)H_T+\Nn^T_T$ is ample$/U$ for any $0<\delta\ll 1$. By Theorem \ref{thm: cone theorem gfq}, $(K_{\Ff_T}+B_T+(1-\delta)H_T+\Mm^T_T)$ is nef$/U$. Thus $K_{\Ff_T}+B_T+H_T+\Mm^T_T$ is ample$/U$, so $K_{\Ff}+B+A+\Mm_X$ is semi-ample$/U$, and $\phi$ is the contraction$/U$ defined by $K_{\Ff}+B+A+\Mm_X$. This implies (1)(2). 

We prove (3). Since $\phi$ is a contraction defined by $K_X+\Delta+A+\Nn_X$, $m(K_{\Ff}+B+A+\Mm_X)$ is Cartier, and $K_{\Ff}+B+A+\Mm_X\sim_{\mathbb Q,T}0$, by Theorem \ref{thm: cont thm gpair}(3), there exists a Cartier divisor $L$ on $T$ such that
$$m(K_{\Ff}+B+A+\Mm_X)=\phi^*L.$$
Since $L\sim_{\mathbb R}K_{\Ff_T}+B_T+H_T+\Mm^T_T$, $L$ is ample$/U$. Thus $nL$ is very ample$/U$ for any integer $n\gg 0$, so 
$$\mathcal{O}_X(nm(K_{\Ff}+B+A+\Mm_X))=\mathcal{O}_X(\phi^*(nL))$$
is globally generated$/U$ for any integer $n\gg 0$.
\end{proof}



\begin{thm}\label{thm: gmm polarized gfq}
    Let $(X,\Ff,B,\Mm)/U$ be an lc gfq, and $A,H$ two ample$/U$ $\Rr$-divisors on $X$. Assume that
    \begin{itemize}
        \item $\Ff$ is induced by a contraction $X\rightarrow Z$,
        \item $K_{\Ff}+B+A+\Mm_X$ is pseudo-effective$/U$, 
        \item $K_{\Ff}+B+\Mm_X\sim_{\mathbb R,Z}K_X+\Delta+\Nn_X$ for some lc g-pair $(X,\Delta,\Nn)/U$, and
        \item either $X$ is $\Qq$-factorial klt or $\Mm$ is NQC$/U$.
    \end{itemize}
   Then there exists a $(K_{\Ff}+B+A+\Mm_X)$-MMP$/U$ with scaling of $H$, say $\mathcal{P}_0$, satisfying the following. Let $\mathcal{P}=\mathcal{P}_0$ if $X$ is not $\Qq$-factorial, and let $\mathcal{P}$ be any $(K_{\Ff}+B+A+\Mm_X)$-MMP$/U$  with scaling of an ample$/U$ $\Rr$-divisor if $X$ is $\Qq$-factorial.  Then 
   \begin{enumerate}
       \item  $\mathcal{P}$ terminates at a model $X'$ such that $K_{\Ff'}+B'+A'+\Mm_X$ is semi-ample$/U$, where $B',A'$ are the images of $B,A$ on $X'$ respectively, and $\Ff'$ is the pushforward of $\Ff$ on $X'$, and
       \item the contraction$/U$ defined by $K_{\Ff'}+B'+A'+\Mm_{X'}$ is a contraction$/Z$.
   \end{enumerate}
\end{thm}
\begin{proof}
Let $\Mm':=\Mm+\bar A$. Then $\mathcal{P}$ is a $(K_{\Ff}+B+\Mm'_X)$-MMP$/U$ and $(X,\Ff,B,\Mm')$ is lc. By Proposition \ref{prop: run mmp with scaling gfq}, $\mathcal{P}$ terminates with a weak lc model $(X',\Ff',B',\Mm')/U$ of $(X,\Ff,B,\Mm')/U$. By Lemma \ref{lem: +a keep under mmp}, there exists a nef$/U$ $\bb$-divisor $\Mm''$ and an ample$/U$ $\Rr$-divisor $A''$ such that $\Mm''_{X'}+A''\sim_{\mathbb R,U}\Mm_{X'}+A'$ and $(X',\Ff',B',\Mm'')/Z$ is lc. By Lemma \ref{lem: ACSS mmp can run}, $\mathcal{P}$ is also a $(K_X+\Delta+A+\Nn_X)$-MMP$/U$. Since $(X,\Delta,\Nn+\bar A)$ is lc, $(X',\Delta',\Nn+\bar A)$ is lc, where $\Delta'$ is the image of $\Delta$ on $X'$. Moreover,
$$K_{\Ff'}+B'+A''+\Mm''_{X'}\sim_{\mathbb R,Z}K_{X'}+\Delta'+\Nn_{X'}+\bar A_{X'}.$$
The theorem follows from Theorem \ref{thm: bpf induced gfq}.
\end{proof}

\subsection{A special case of Prokhorov-Shokurov's effective \texorpdfstring{$\bb$}{}-semi-ampleness conjecture}


\begin{thm}\label{thm: a special b-semiampleness}
Let $(X,B,\Mm)/U$ be an lc g-pair and $f:(X,B,\Mm)\rightarrow Z$ a contraction satisfying Property $(*)$. Let $\Nn$ be the moduli part of $f: (X,B,\Mm)\rightarrow Z$. Assume that
\begin{enumerate}
  \item $f$ is equi-dimensional,
  \item $K_X+B+\Mm_X$ is nef$/Z$, 
  \item $(X,B,\Mm)$ is BP semi-stable$/Z$, and
  \item there exists an ample$/U$ $\Rr$-divisor $H$ such that either $B^h\geq H$ or $\Mm-\bar H$ is nef$/U$, where $B^h$ the horizontal$/Z$ part of $B$.
\end{enumerate}
Then $\Nn$ descends to $X$ and $\Nn_X$ is semi-ample$/U$.
\end{thm}
\begin{proof}
    Let $\Ff$ be the foliation induced by $f$. By Proposition \ref{prop: bp semistable foliation lc}, $(X,\Ff,B^h,\Mm)$ is lc and $(X,\Ff,B^h,\Mm;B-B^h)/Z$ is weak ACSS. By Theorem \ref{thm: lc+weak acc=bpstable}, $(X,B,\Mm)$ is BP stable$/Z$. By Proposition \ref{prop: bp stable nef}, $\Nn$ descends to $X$ and is nef$/U$. By Proposition \ref{prop: weak cbf gfq}, $K_{\Ff}+B^h+\Mm_X=\Nn_X$ is nef$/U$. By Theorem \ref{thm: bpf induced gfq}, $\Nn_X=K_{\Ff}+B^h+\Mm_X$ is semi-ample$/U$.
\end{proof}


When we have an lc-trivial fibration, we can prove stronger $\bb$-semi-ampleness.

\begin{thm}\label{thm: a special effective b-semiampleness}
Let $d$ and $m$ be two positive integers. Then there exists a positive integer $I$ depending only on $d$ and $m$ satisfying the following.

Assume that $(X,B,\Mm)/U$ is an lc g-pair and $f: (X,B,\Mm)\rightarrow Z$ is a contraction$/U$ satisfying Property $(*)$. Let $\Nn$ be the moduli part of $f: (X,B,\Mm)\rightarrow Z$. Assume that
 \begin{enumerate}
        \item $f$ is equi-dimensional,
        \item $X$ is of Fano type over $Z$,
        \item $K_X+B+\Mm_X\sim_{\mathbb Q,Z}0$,
        \item $(X,B,\Mm)$ is BP semi-stable$/Z$,
        \item $mB$ is a Weil divisor and $m\Mm$ is $\bb$-base-point-free$/U$, and
        \item there exists an ample$/U$ $\Rr$-divisor $H$ such that either $B^h\geq H$ or $\Mm-\bar H$ is nef$/U$, where $B^h$ the horizontal$/Z$ part of $B$.
    \end{enumerate}
Then $\Nn$ descends to $X$, $I\Nn_X$ is Cartier, and $\mathcal{O}_X(nI\Nn_X)$ is globally generated$/U$ for any integer $n\gg 0$.
\end{thm}
\begin{proof}
Let $B_Z$ be the discriminant part of $f: (X,B,\Mm)\rightarrow Z$. By \cite[Lemma 4.2]{Has22} (cf. \cite[Proposition 6.3]{Bir19}), there exist a positive integer $q$ depending only on $d$ and $m$ and a choice $\Mm^Z$  of the moduli part of $f: (X,B,\Mm)\rightarrow Z$, such that 
$$q(K_X+B+\Mm_X)\sim qf^*(K_Z+B_Z+\Mm^Z_Z)$$ 
and $q\Mm^Z$ is nef$/U$. Since $f: (X,B,\Mm)\rightarrow Z$ satisfies Property $(*)$, $Z$ is smooth and $B_Z$ is reduced. In particular, $q(K_X+B+\Mm_X)$ is Cartier.

By Theorem \ref{thm: a special b-semiampleness}, $\Nn$ descends to $X$ and $\Nn_X$ is semi-ample$/U$. By Proposition \ref{prop: weak cbf gfq}, 
$$\Nn_X\sim K_{\Ff}+B^h+\Mm_X\sim_ZK_X+B+\Mm_X$$ 
is semi-ample$/U$, where $\Ff$ is the foliation induced by $f$. Since $Z$ is smooth and $q(K_X+B+\Mm_X)$ is Cartier, $q\Nn_X$ and $q(K_{\Ff}+B^h+\Mm_X)$ are Cartier, and we may let $I:=q$. By Theorem \ref{thm: bpf induced gfq}(3), $\mathcal{O}_X(nI(K_{\Ff}+B^h+\Mm_X))=\mathcal{O}_X(nI\Nn_X)$ is globally generated$/U$ for any integer $n\gg 0$.
\end{proof}



\begin{thm}\label{thm: special b-semiampleness lc trivial}
Let $(X,B,\Mm)/U$ be an lc g-pair, $G\geq 0$ an $\Rr$-Cartier $\Rr$-divisor on $X$, and $f: X\rightarrow Z$ an equi-dimensional contraction$/U$. Assume that
\begin{itemize}
    \item $G$ is vertical$/Z$,
    \item $f: (X,B+G,\Mm)\rightarrow Z$ satisfies Property $(*)$,
    \item $(X,B+G,\Mm)$ is BP semi-stable$/Z$,
    \item $K_X+B+\Mm_X\sim_{\mathbb R,Z}0$,
    \item there exists an ample$/U$ $\Rr$-divisor $H$ such that either $B^h\geq H$ or $\Mm-\bar H$ is nef$/U$, where $B^h$ the horizontal$/Z$ part of $B$, and
    \item either $X$ is $\Qq$-factorial klt or $\Mm$ is NQC$/U$.
\end{itemize}
Let $\Nn$ be the moduli part of  $f: (X,B,\Mm)\rightarrow Z$. Then:
\begin{enumerate}
    \item $\Nn$ descends to $X$ and $\Nn_X$ is semi-ample$/U$.
    \item Suppose that there exists a positive integer $m$ such that $mB^h$ is a Weil divisor, $m\Mm$ is $\bb$-base-point-free$/U$, and $X$ is of Fano type over $Z$. Then there exists a positive integer $I$ depending only on $\dim X$ and $m$, such that $I\Nn_X$ is Cartier and $\mathcal{O}_X(nI\Nn_X)$ is globally generated$/U$ for any integer $n\gg 0$.
\end{enumerate}
\end{thm}
\begin{proof}
For any prime divisor $D$ on $Z$, we let 
$$t_D:=\sup\{t\geq 0\mid G-tf^*D\geq 0\}$$
and let 
$$G_0:=G-\sum_{D\mid D\text{ is a prime divisor on }Z}t_D\pi^*D.$$ 
Then $G_0\geq 0$ and $G_0$ is very exceptional$/Z$.

Let $\Ff$ be the foliation induced by $f$. By Proposition \ref{prop: bp semistable foliation lc}, $(X,\Ff,B^h,\Mm)$ is lc, so $(X,\Ff,B^h,\Mm;G+B-B^h)/Z$ is weak ACSS. By Proposition \ref{prop: weak cbf gfq}, $$K_{\Ff}+B^h+\Mm_X\sim_{\mathbb R,Z}K_X+B+G+\Mm_X\sim_{\mathbb R,Z}G\sim_{\mathbb R,Z}G_0.$$
By Theorem \ref{thm: mmp very exceptional alg int fol}, we may run a $(K_{\Ff}+B^h+\Mm_X)$-MMP$/Z$ which terminates with a weak lc model $(X',\Ff',(B^h)',\Mm)/Z$ of $(X,\Ff,B^h,\Mm)/Z$, such that $K_{\Ff'}+(B^h)'+\Mm_{X'}\sim_{\mathbb R,Z}0$. 

Let $B'$ and $G'$ be the images of $B$ an $G$ on $X'$ respectively, $f': X'\rightarrow Z$ the induced contraction, and let $\Nn'$ be the moduli part of $f': (X',B'+G',\Mm)\rightarrow Z$. By Lemma \ref{lem: ACSS mmp can run}, $(X',B'+G',\Mm)/Z$ satisfies Property $(*)$. By Proposition \ref{prop: bp semistable foliation lc}, $(X',B'+G',\Mm)/Z$ is BP semi-stable. By Proposition \ref{prop: weak cbf gfq}, $$K_{X'}+B'+G'+\Mm_{X'}\sim_{\mathbb R,Z}K_{\Ff'}+B'+\Mm_{X'}\sim_{\mathbb R,Z}0.$$
By Lemma \ref{lem: +a keep under mmp}, there exists an ample$/U$ $\Rr$-divisor $H'\geq 0$ on $X'$ such that either $(B^h)'\geq H'\geq 0$ or $\Mm-\bar H'$ is nef$/U$. By Theorem \ref{thm: a special b-semiampleness}, $\Nn'$ descends to $X'$ and $\Nn'_{X'}$ is semi-ample$/U$. Moreover, under the condition of (2), there exists a positive integer $I$ depending only on $d$ and $m$ such that $I\Nn'_{X'}$ is Cartier and $\mathcal{O}_X(nI\Nn'_{X'})$ is globally generated$/U$ for any integer $n\gg 0$.

Let $\Mm^Z$ and $\Mm'^Z$ be the base moduli part of $f: (X,B,\Mm)\rightarrow Z$ and $f': (X',B'+G',\Mm)\rightarrow Z$ respectively. Since $\Nn'$ descends to $X'$ and $\Nn'_{X'}$ is semi-ample$/U$, $\Mm'^Z$ descends to $Z$ and $\Mm'^Z_Z$ is semi-ample$/U$. Since the induced birational map $\phi$ is a $G'$-MMP and $G'$ is vertical$/Z$, $\phi$ is an isomorphism over the generic point of $Z$. Thus $f: (X,B,\Mm)\rightarrow Z$ and $f': (X',B'+G',\Mm)\rightarrow Z$ are crepant over the generic point of $Z$. By Lemma \ref{lem: m preserved under crepant}, $\Mm^Z=\Mm'^Z$. Thus $\Nn=\Nn'$, and the theorem follows.
\end{proof}


\section{Proofs of the main theorems}\label{sec: proof of the main theorems}

In this section, we prove all theorems that are listed in Sections \ref{sec:Introduction} and \ref{sec: statement of main results}. We recall the theorems that are already proven in the previous parts of the paper.
\begin{enumerate}
    \item Theorems \ref{thm:base-point-freeness intro} and \ref{thm: semi-ampleness intro} were proven in Subsection \ref{subsec: bpf nonnqc}.
    \item Theorem \ref{thm: glc sings are Du Bois} was proven in Subsection \ref{subsec: du bois}.
    \item Theorem \ref{thm: cone theorem gfq} was proven in Subsection \ref{subsec: proof of cone}.
    \item Theorems \ref{thm: lc adjunction foliation nonnqc}, \ref{thm: dcc adjunction is dcc}, \ref{thm:  ACSS model} were proven in Subsection \ref{subsec: proof of adj}.
    \item Theorem \ref{thm: global acc alg int gfq} was proven in Subsection \ref{subsec: global acc}.
    \item Theorem \ref{thm: acc lct alg int gfq} was proven in Subsection \ref{subsec: acc}.
\end{enumerate}

\begin{thm}[{cf. \cite[Conjecture 4.2(1)]{CS23a}}]\label{thm: fdlt is acss}
    Let $(X,\Ff,B,\Mm)/U$ be a $\Qq$-factorial F-dlt gfq. Then $(X,\Ff,B,\Mm)$ is ACSS.
\end{thm}
\begin{proof}
    Let $h: Y\rightarrow X$ be a foliated log resolution of  $(X,\Ff,B,\Mm)$ such that $a(D,\Ff,B,\Mm)>-\epsilon_{\Ff}(D)$ for any prime $h$-exceptional divisor $D$. Let $\Ff_Y:=h^{-1}\Ff$ and $B_Y:=h^{-1}_*B+(\Supp\Exc(h))^{\Ff_Y}$, then $(Y,\Ff_Y,B_Y,\Mm)$ is $\Qq$-factorial ACSS and $K_{\Ff_Y}+B_Y+\Mm_Y\sim_{\mathbb R,X}E\geq 0$
    for some $h$-exceptional prime divisor $E$ such that $\Supp E=\Supp\Exc(h)$. By Theorem \ref{thm: mmp very exceptional alg int fol}, we may run a $(K_{\Ff}+B+\Mm_X)$-MMP$/X$ with scaling of an ample$/U$ divisor $A$ which terminates with a good minimal model $(X',\Ff',B',\Mm)/X$ of $(X,\Ff,B,\Mm)/X$, such that $E$ is contracted by this MMP. Thus the induced birational morphism $X'\rightarrow X$ is small. Since $X$ is $\Qq$-factorial, $X'\rightarrow X$ is the identity morphism. The theorem follows.
\end{proof}

\begin{proof}[Proof of Theorem \ref{thm: mmp fdlt}]
    It follows from Theorem \ref{thm: fdlt is acss} and Proposition \ref{prop: run mmp with scaling gfq}.
\end{proof}

\begin{proof}[Proof of Theorem \ref{thm: +a gmm fdlt}]
    It follows from Theorems \ref{thm: fdlt is acss}, \ref{thm: gmm polarized gfq}, and Proposition \ref{prop: run mmp get mfs}.
\end{proof}


\begin{proof}[Proof of Theorem \ref{thm: +a abundance fdlt}]
    We may assume that $K_{\Ff}+B+A$ is pseudo-effective$/U$. The theorem follows from  Theorems \ref{thm: fdlt is acss} and \ref{thm: gmm polarized gfq}.
\end{proof}

\begin{proof}[Proof of Theorem \ref{thm: bpf fdlt}]
   It follows from Theorems \ref{thm: fdlt is acss} and \ref{thm: bpf induced gfq}.
\end{proof}

\begin{proof}[Proof of Theorem \ref{thm: eomfs}]
    It is a special case of Theorem \ref{thm: existence mfs}.
\end{proof}

\begin{proof}[Proof of Theorem \ref{thm: gmm ai num0}] 
First we prove (3). By Theorem \ref{thm: fdlt is acss} and Proposition \ref{prop: projective num 0 mmp}, we may run a $(K_{\Ff}+B)$-MMP with scaling of an
ample $\Rr$-divisor, and any such MMP terminates with a log minimal model $(X',\Ff',B')$ of $(X,\Ff,B)$ such that $K_{\Ff'}+B'\equiv 0$. By \cite[Theorem 1.4]{DLM23}, $K_{\Ff'}+B'\sim_\Rr 0$. 

(2) follows from (3) and Theorem \ref{thm:  ACSS model}. (1) follows from (2).
\end{proof}


\begin{proof}[Proof of Theorem \ref{thm: abundance num 0 no restriction to f}]
Let $(Y,\Ff_Y,\bar B_Y;G)/Z$ be a proper ACSS model of $(X,\Ff,B)$ with induced birational morphism $g: Y\rightarrow X$, whose existence is guaranteed by Theorem \ref{thm: property * induction}. Let $K_{\Ff_Y}+B_Y:=g^*(K_{\Ff}+B)$ and $K_Y+B'_Y:=g^*(K_X+B)$. Since $(X,B)$ is lc, the coefficient of any component of $B'_Y$ is $\leq 1$. In particular, any coefficient of $B$ is $\leq 1$. We let
$$\bar B_Y:=g^{-1}_*B+(\Supp\Exc(g))^{\Ff_Y}$$
and $E:=B_Y-\bar B_Y$. Then $E\geq 0$ and $E$ is exceptional$/X$.

Suppose that $K_{\Ff_Y}+\bar B_Y$ is not pseudo-effective. We let
$$F:=\sum_{D\mid D\text{ is a }g\text{-exceptional prime divisor}}D.$$
Then $G\geq F$. Since $(Y,\bar B_Y+G)$ is lc and $G\geq F$, $(Y,\bar B_Y+F)$ is lc. By \cite[Theorem 5.3]{ACSS21}, $K_Y+\bar B_Y+F$ is not pseudo-effective.  For any prime $f$-exceptional divisor $D$ such that $D$ is not $\Ff_Y$-invariant, we have $\mult_D\bar B_Y=1$. Therefore, 
$$\bar B_Y+F=g^{-1}_*B+\Supp\Exc(g).$$
Since the coefficient of any component of $B'_Y$ is $\leq 1$, we have
$$E':=g^{-1}_*B+\Supp\Exc(g)-B'_Y\geq 0$$
and $E'$ is exceptional$/X$. Therefore, 
\begin{align*}
    -\infty&=\kappa_{\sigma}(K_Y+\bar B_Y+F)=\kappa_{\sigma}(K_Y+g^{-1}_*B+\Supp\Exc(g))\\
    &=\kappa_{\sigma}(g^*(K_X+B)+E')=\kappa_{\sigma}(K_X+B)\geq 0,
\end{align*}
a contradiction. Thus $K_{\Ff_Y}+\bar B_Y$ is not pseudo-effective. Since
$$0\leq\kappa_{\sigma}(K_{\Ff_Y}+\bar B_Y)\leq\kappa_{\sigma}(K_{\Ff_Y}+B_Y)=\kappa_{\sigma}(K_{\Ff}+B)=0,$$
we have $\kappa_{\sigma}(K_{\Ff_Y}+\bar B_Y)=0$. By Theorem \ref{thm: gmm ai num0}(1), $\kappa_{\iota}(K_{\Ff_Y}+\bar B_Y)=0$, so 
$$0=\kappa_{\iota}(K_{\Ff_Y}+\bar B_Y)\leq \kappa_{\iota}(K_{\Ff_Y}+B_Y)= \kappa_{\iota}(K_{\Ff}+B)\leq \kappa_{\sigma}(K_{\Ff}+B)=0.$$
Thus $\kappa_{\iota}(K_{\Ff}+B)=0$ and we are done.
\end{proof}

\begin{proof}[Proof of Theorem \ref{thm: cs23 4.2(1)}]
    It immediately follows from Theorem \ref{thm: fdlt is acss}.
\end{proof}

\begin{thm}\label{thm: almost holomorphic strong}
    Let $(X,\Ff,B,\Mm)$ be a $\Qq$-factorial lc generalized foliated quadruple such that $\Ff$ is algebraically integrable. Assume that there exists a foliated log resolution $h: Y\rightarrow X$ such that $a(D,\Ff,B,\Mm)>-1$ for any $h$-exceptional prime divisor $D$. Then $\Ff$ is induced by an almost holomorphic map.
\end{thm}
\begin{proof}
Let $\Ff_Y:=h^{-1}\Ff$ and let $B_Y:=h^{-1}_*B+(\Supp\Exc(h))^{\Ff_Y}$, then $(Y,\Ff_Y,B_Y,\Mm)$ is $\Qq$-factorial ACSS and $K_{\Ff_Y}+B_Y+\Mm_Y\sim_{\mathbb R,X}E\geq 0$ for some $h$-exceptional prime divisor $E$. Let $F$ be the sum of all non-$\Ff_Y$-invariant prime $h$-exceptional divisors. By assumption, $F\subset\Supp E$.

By Theorem \ref{thm: mmp very exceptional alg int fol}, we may run a $(K_{\Ff}+B+\Mm_X)$-MMP$/X$ with scaling of an ample$/U$ divisor $A$ which terminates with a good minimal model $(X',\Ff',B',\Mm)/X$ of $(X,\Ff,B,\Mm)/X$, such that $E$ is contracted by this MMP. Then $F$ is contracted by this MMP, and $(X',\Ff',B',\Mm)/Z$ is ACSS for some contraction $f': X'\rightarrow Z$. Since $X$ is $\Qq$-factorial, The induced morphism $g: X'\rightarrow X$ only extracts $\Ff'$-invariant divisors, so $g$ is an isomorphism over the generic point of $Z$. In particular, 
$$f:=f'\circ g^{-1}: X\dashrightarrow Z$$
is an almost holomorphic map which induces $\Ff$.
\end{proof}

\begin{proof}[Proof of Theorem \ref{thm: canonical almost holomorphic}]
    It is a special case of Theorem \ref{thm: almost holomorphic strong}.
\end{proof}


\begin{proof}[Proof of Theorem \ref{thm: cone theorem nonnqc gpair}]
It follows from Theorems \ref{thm: cone theorem gfq} and \ref{thm: cont thm gpair}.
\end{proof}


\begin{proof}[Proof of Theorem \ref{thm: eof nonnqc}]
It is a special case of Theorem \ref{thm: existence of q-factorial glc flips}.
\end{proof}


\begin{proof}[Proof of Theorem \ref{thm: qfact nonnqc mmp can run intro}]
It follows from Theorems \ref{thm: cone theorem nonnqc gpair} and \ref{thm: eof nonnqc}.
\end{proof}


\begin{proof}[Proof of Theorem \ref{thm: kod vanishing gpair intro}]
It immediately follows from Theorem \ref{thm: kod vanishing with lc strata}(2) by letting $U=\{pt\}$.
\end{proof}

\begin{proof}[Proof of Theorem \ref{thm: kv vanishing gpair intro}]
It immediately follows from Theorem \ref{thm: kod vanishing with lc strata}(2).
\end{proof}


\begin{proof}[Proof of Theorem \ref{thm: subadj intro}] It is a special case of Definition-Lemma \ref{deflem: subadj minimal lc center}.
\end{proof}




\begin{proof}[Proof of Theorem \ref{thm: cbf gfq}]
(1-4) follow from Definition-Theorem \ref{defthm: cbf lctrivial morphism}. (5) follows from Definition-Theorem \ref{defthm: cbf lctrivial morphism} and Proposition \ref{prop: gfq cbf preserve sing}. (6) follows from Proposition \ref{prop: gfq cbf preserve sing}. (7) follows from Lemma \ref{lem: td=bd}. (8) follows from Lemma \ref{lem: m preserved under crepant gfq} and Definition-Lemma \ref{deflem: cbf finite}. (9) follows from Definition-Lemma \ref{deflem: cbf gfq}(3) and Definition-Lemma \ref{deflem: cbf finite}(6).
\end{proof}

\begin{proof}[Proof of Theorem \ref{thm: fol adj intro}]
It is a special case of Theorem \ref{thm: lc adjunction foliation nonnqc}.
\end{proof}

\begin{proof}[Proof of Corollary \ref{cor: global acc rank 1 gfq}]  
If $K_{\Ff}$ is pseudo-effective, then $K_{\Ff}\equiv B\equiv\Mm_X\equiv 0$. By Lemma \ref{lem: trivial trace nef imply trivial}, $\Mm\equiv\bm{0}$ so there is nothing left to prove. So we may assume that $K_{\Ff}$ is not pseudo-effective. Since $\rk\Ff=1$, by Theorem \ref{thm: subfoliation algebraic integrable}, $\Ff$ is algebraically integrable. The corollary follows from Theorem \ref{thm: global acc alg int gfq}.
\end{proof}


\begin{proof}[Proof of Theorem \ref{thm: uniform rational polytope gfq}]
It immediately follows from Theorem \ref{thm: uniform rational polytope}.
\end{proof}

\begin{proof}[Proof of Theorem \ref{thm: gfq mmp very exceptional intro}]
It is a special case of Theorem \ref{thm: mmp very exceptional alg int fol}.
\end{proof}


\begin{proof}[Proof of Theorem \ref{thm: ps intro}]
It follows from Theorem \ref{thm: a special b-semiampleness}.
\end{proof}

\begin{proof}[Proof of Theorem \ref{thm: main mmp foliation}]
(1) follows from Theorem \ref{thm: fdlt is acss}, Proposition \ref{prop: weak cbf gfq}, Theorem \ref{thm: cone theorem gfq}, and the contraction theorem and the existence of flips for lc pairs. (2) is a special case of Theorem \ref{thm: bpf fdlt}. (3) is a special case of Theorem \ref{thm: +a gmm fdlt}. (4) follows from Theorems \ref{thm: fdlt is acss} and \ref{thm: bpf induced gfq}.
\end{proof}

\begin{proof}[Proof of Theorem \ref{thm: main mmp gpair}]
    (1) follows from Theorem \ref{thm: cone theorem nonnqc gpair} and \ref{thm: eof nonnqc} and (2) follows from Theorem \ref{thm: semi-ampleness intro}.
\end{proof}





\begin{thebibliography}{99}



\bibitem[AK00]{AK00} D. Abramovich and K. Karu, \textit{Weak semistable reduction in characteristic 0}, Invent. Math. \textbf{139} (2000), no. 2, 241--273.

\bibitem[Amb03]{Amb03} F. Ambro, \textit{Quasi-log varieties}, Tr. Mat. Inst. Steklova \textbf{240} (2003), Biratsion. Geom. Linein. Sist. Konechno Porozhdennye Algebry, 220--239; translation in Proc. Steklov Inst. Math. \textbf{240} (2003), no. 1, 214--233.

\bibitem[Amb05]{Amb05} F. Ambro, \textit{The moduli b-divisor of an lc-trivial fibration}, Compos. Math. \textbf{141} (2005), no. 2, 385--403.

\bibitem[ACSS21]{ACSS21} F. Ambro, P. Cascini, V. V. Shokurov, and C. Spicer, \textit{Positivity of the moduli part}, arXiv:2111.00423.

\bibitem[AD13]{AD13} C. Araujo and S. Druel, \textit{On Fano foliations}, Adv. Math., \textbf{238} (2013), 70--118.

\bibitem[ABBDILW23]{ABBDILW23} K. Ascher, D. Bejleri, H. Blum, K. DeVleming, G. Inchiostro, Y. Liu, and X. Wang, \textit{Moduli of boundary polarized Calabi-Yau pairs}, arXiv:2307.06522.

\bibitem[Ber23]{Ber23} F. Bernasconi, \textit{Counterexamples to the MMP for 1-foliations in positive characteristic}, arXiv:2309.13978.

\bibitem[Bir12]{Bir12} C. Birkar, \textit{Existence of log canonical flips and a special LMMP}, Pub. Math. IHES., \textbf{115} (2012), 325--368.

\bibitem[Bir19]{Bir19} C. Birkar, \textit{Anti-pluricanonical systems on Fano varieties}, Ann. of Math. (2), \textbf{190} (2019), 345--463.


\bibitem[Bir20]{Bir20} C. Birkar, \textit{On connectedness of non-klt loci of singularities of pairs}, arXiv:2010.08226v2, to appear in J. Differential Geom.

\bibitem[Bir21]{Bir21} C. Birkar, \textit{Generalised pairs in birational geometry}, EMS Surv. Math. Sci. \textbf{8} (2021), no. 1--2, 5--24.

\bibitem[BZ16]{BZ16} C. Birkar and D.-Q. Zhang, \textit{Effectivity of Iitaka fibrations and pluricanonical systems of polarized pairs}, Pub. Math. IHES., \textbf{123} (2016), 283--331.

\bibitem[BM16]{BM16} F. Bogomolov and F. McQuillan, \textit{Rational curves on foliated varieties}, In: \textit{Foliation theory in algebraic
geometry}, Simons Symp. Springer, Cham (2016), 21--51.

\bibitem[Bru02]{Bru02} M. Brunella, \textit{Foliations on complex projective surfaces}, arXiv:math/0212082.


\bibitem[Bru15]{Bru15} M. Brunella, \textit{Birational geometry of foliations}, IMPA Monographs \textbf{1} (2015), Springer, Cham.

\bibitem[BCHM10]{BCHM10}
C. Birkar, P. Cascini, C. D. Hacon and J. M\textsuperscript{c}Kernan, \textit{Existence of minimal models for varieties of log general type}, J. Amer. Math. Soc. \textbf{23} (2010), no. 2, 405--468.

\bibitem[Che20]{Che20} G. Chen, \textit{Boundedness of $n$-complements for generalized pairs}, arXiv:2003.04237.


\bibitem[CHL23]{CHL23} G. Chen, J. Han, and J. Liu, \textit{On effective Iitaka fibrations and existence of complements}, arXiv:2301.04813.

\bibitem[Che22]{Che22} Y.-A. Chen, \textit{ACC for foliated log canonical thresholds}, arXiv:2202.11346.

\bibitem[Che23]{Che23} Y.-A. Chen, \textit{Log canonical foliation singularities on surfaces}, Math. Nachr. \textbf{00} (2023), 1--35.

\bibitem[CP19]{CP19} F.~Campana and M. P\u{a}un, \textit{Foliations with positive slopes and birational stability of orbifold cotangent bundles}, Pub. Math. IHES., \textbf{129} (2019), 1--49.

\bibitem[Can04]{Can04} F. Cano, \textit{Reduction of the singularities of codimension one singular foliations in dimension three}, Ann. Math. (2) \textbf{160} (2004), no. 3, 907--1011.

\bibitem[CS20]{CS20} P. Cascini and C. Spicer, \textit{On the MMP for rank one foliations on threefolds}, arXiv:2012.11433.

\bibitem[CS21]{CS21} P.~Cascini and C. Spicer, \textit{MMP for co-rank one foliations on threefolds}, Invent. math. \textbf{225} (2021), 603--690.

\bibitem[CS23a]{CS23a} P. Cascini and C. Spicer, \textit{On the MMP for algebraically integrable foliations}, to appear in Shokurov's 70th birthday's special volume, arXiv:2303.07528.

\bibitem[CS23b]{CS23b} P. Cascini and C. Spicer, \textit{Foliation adjunction}, arXiv:2309.10697.


\bibitem[CD23]{CD23} P. Chaudhuri and O. Das, \textit{A basepoint free theorem for algebraically integrable foliations}, arXiv:2307.03530v1.

\bibitem[CLX23]{CLX23} B. Chen, J. Liu, and L. Xie, \textit{Vanishing theorems for generalized pairs}, arXiv:2305.12337.

\bibitem[DH23]{DH23} O. Das and C. D. Hacon, \textit{On the Minimal Model Program for K\"ahler 3-folds}, arXiv:2306.11708.

\bibitem[DHY23]{DHY23} O. Das, C. D. Hacon, and J. Y\'a\~nez, \textit{MMP for generalized pairs on K\"ahler 3-folds}, arXiv:2305.00524.

\bibitem[DLM23]{DLM23} O. Das, J. Liu, and R. Mascharak, \textit{ACC for lc thresholds for algebraically integrable foliations}, arXiv:2307.07157.

\bibitem[DO23a]{DO23a} O. Das and W. Ou, \textit{On the Log Abundance for Compact K\"ahler 3-folds}, Manuscripta Math. (2023)

\bibitem[DO23b]{DO23b} O. Das and W. Ou, \textit{On the Log Abundance for Compact K\"ahler threefolds II}, arXiv:2306.00671.

\bibitem[dFKX17]{dFKX17} T. de Fernex, J. Koll\'ar, and C. Xu, \textit{The dual complex of singularities}, in \textit{Higher dimensional algebraic geometry: in honor of Professor Yujiro Kawamata’s sixtieth birthday}, Adv. Stud. Pure Math., \textbf{74} (2017), Math. Soc. Japan, Tokyo, 103--129. 

\bibitem[Dru17]{Dru17} S. Druel, \textit{On foliations with nef anti-canonical bundle}, Trans. Amer. Math. Soc., \textbf{369} (2017), no. 11, 7765--7787.

\bibitem[Dru21]{Dru21} S. Druel, \textit{Codimension 1 foliations with numerically trivial canonical class on singular spaces}, Duke Math. J., \textbf{170} (2021), no. 1, 95--203.

\bibitem[Eck04]{Eck04} T. Eckl, \textit{Numerically trivial foliations}, Ann. Inst. Fourier (Grenoble) \textbf{54} (2004), 887--938.

\bibitem[Fil19]{Fil19} S. Filipazzi, \textit{Generalized pairs in birational geometry}, 2019. PhD thesis, University of Utah.

\bibitem[Fil20]{Fil20} S. Filipazzi, \textit{On a generalized canonical bundle formula and generalized adjunction}, Ann. Sc. Norm. Super. Pisa Cl. Sci. (5) Vol. XXI (2020), 1187--1221.

\bibitem[FS23]{FS23} S. Filipazzi and R. Svaldi, \textit{On the connectedness principle and dual complexes for generalized pair}, Forum Math. Sigma \textbf{11} (2023), E33.

\bibitem[Flo14]{Flo14} E. Floris, \textit{Inductive approach to effective b-semiampleness}, Int. Math. Res. Not. \textbf{6} (2014), 1465--1492.

\bibitem[Fuj11]{Fuj11} O. Fujino, \textit{Fundamental theorems for the log minimal model program}, Publ. Res. Inst. Math. Sci. \textbf{47} (2011), no. 3, 727--789.


\bibitem[Fuj17]{Fuj17} O. Fujino, \textit{Foundations of the minimal model program}, MSJ Memoirs, \textbf{35}, Mathematical Society of Japan, Tokyo (2017).

\bibitem[FM00]{FM00} O. Fujino and S. Mori, \textit{A canonical bundle formula}, J. Differential Geom. \textbf{56} (2000), no. 1, 167--188.

\bibitem[FG12]{FG12} O. Fujino and Y. Gongyo, \textit{On canonical bundle formulas and subadjunctions}, Michigan Math. J. \textbf{61} (2012), 255--264.

\bibitem[FG14]{FG14} O. Fujino and Y. Gongyo, \textit{On the moduli b-divisors of lc-trivial fibrations}, Ann. Inst. Fourier (Grenoble), \textbf{64} (2014), no. 4, 1721--1735.

\bibitem[Gon11]{Gon11} Y. Gongyo, \textit{On the minimal model theory for dlt pairs of numerical Kodaira dimension zero}, Math. Rest. Lett. \textbf{18} (2011), no. 5, 991--1000.

\bibitem[HL21a]{HL21a} C. D. Hacon and J. Liu, \textit{Existence of flips for generalized lc pairs}, arXiv:2105.13590, to appear in Camb. J. Math. 

\bibitem[HMX14]{HMX14} C. D. Hacon, J. M\textsuperscript{c}Kernan, and C. Xu, \textit{ACC for log canonical thresholds}, Ann. of Math. \textbf{180} (2014), no. 2, 523--571.

\bibitem[HX13]{HX13} C. D. Hacon and C. Xu, \textit{Existence of log canonical closures}, Invent. Math. \textbf{192} (2013), no. 1, 161--195.

\bibitem[HL22]{HL22} J. Han and Z. Li, \textit{Weak Zariski decompositions and log terminal models for generalized polarized pairs}, Math. Z. \textbf{302} (2022), 707--741.

\bibitem[HLS19]{HLS19} J. Han, J. Liu, and V. V. Shokurov, \textit{ACC for minimal log discrepancies of exceptional singularities}, arXiv:1903.04338.

\bibitem[HL21b]{HL21b} J. Han and W. Liu, \textit{On a generalized canonical bundle formula for generically finite morphisms},  Ann. Inst. Fourier (Grenoble), \textbf{71} (2021), no. 5, 2047--2077.


\bibitem[Har77]{Har77} R. Hartshorne, \textit{Algebraic geometry}, Springer-Verlag, New York-Heidelberg (1977), Graduate Texts in Mathematics, no. 52.


\bibitem[Has22]{Has22} K. Hashizume, \textit{Iitaka fibrations for dlt pairs polarized by a nef and log big divisor}, Forum Math. Sigma. \textbf{10} (2022), Article No. 85.

\bibitem[HH20]{HH20}  K. Hashizume and Z. Hu, \textit{On minimal model theory for log abundant lc pairs}, J. Reine Angew. Math., \textbf{767} (2020), 109--159. 


\bibitem[Hu20]{Hu20} Z. Hu, \textit{Log abundance of the moduli b-divisors for lc-trivial fibrations}, arXiv:2003.14379.

\bibitem[KMM87]{KMM87} Y. Kawamata, K. Matsuda, and K. Matsuki, \textit{Introduction to the minimal model problem}, Algebraic geometry, Sendai (1985), 283--360, Adv. Stud. Pure Math., \textbf{10}, North-Holland, Amsterdam (1987).

\bibitem[JLX22]{JLX22} J. Jiao, J. Liu, and L. Xie, \textit{On generalized lc pairs with b-log abundant nef part}, arXiv:2202.11256.

\bibitem[Kaw98]{Kaw98} Y. Kawamata, \textit{Subadjunction of log canonical divisors, II}, Amer. J. Math. \textbf{120} (1998), no. 5, 893--899.

\bibitem[Kod64]{Kod64} K. Kodaira, \textit{On the structure of compact complex analytic surfaces}, I, Amer. J. Math. \textbf{86} (1964), 751--798.


\bibitem[Kol07]{Kol07} J. Koll\'ar, \textit{Kodaira’s canonical bundle formula and adjunction}, In: \textit{Flips for 3-folds and 4-folds}, Ed. by A. Corti. \textbf{35}. Oxford Lecture Series in Mathematics and its Applications. Oxford: Oxford University Press (2007), Chap. 8, 134--162.

\bibitem[Kol13]{Kol13} J. Koll\'ar, \textit{Singularities of the minimal model program}, Cambridge Tracts in Math. \textbf{200} (2013), Cambridge Univ. Press. With a collaboration of S\'andor Kov\'acs.

\bibitem[Kol23]{Kol23} J. Koll\'ar, \textit{Families of varieties of general type}, Cambridge Tracts in Math. \textbf{231} (2023), Cambridge Univ. Press. With the collaboration of Klaus Altmann and S\'andor Kov\'acs.


\bibitem[Kol$^+$92]{Kol+92} J. Koll\'{a}r et al., \textit{Flip and abundance for algebraic threefolds}. Ast\'{e}risque no. \textbf{211}, (1992).

\bibitem[KM98]{KM98} J. Koll\'{a}r and S. Mori, \textit{Birational geometry of algebraic varieties}, Cambridge Tracts in Math. \textbf{134} (1998), Cambridge Univ. Press.

\bibitem[Kov99]{Kov99} S. J. Kov\'acs, \textit{Rational, log canonical, Du Bois singularities: on the conjectures of Koll\'ar and Steenbrink}, Compos. Math. \textbf{118} (1999), no. 2, 123--133.

\bibitem[Kov11]{Kov11} S. J. Kov\'acs, \textit{DB pairs and vanishing theorems}, Kyoto Journal of Mathematics, Nagata Memorial Issue \textbf{51} (2011), no. 1, 47--69.

\bibitem[Kov12]{Kov12} S. J. Kov\'acs, \textit{The splitting principle and singularities}, Compact moduli spaces and vector bundles, Contemp. Math. \textbf{564} (2012), Amer. Math. Soc. Providence, RI, 195--204.



\bibitem[LT22]{LT22} V. Lazi\'c and N. Tsakanikas, \textit{Special MMP for log canonical generalised pairs (with an appendix joint with Xiaowei Jiang)}, Sel. Math. New Ser. \textbf{28} (2022), Article No. 89.

\bibitem[LLM23]{LLM23} J. Liu, Y. Luo, and F. Meng, \textit{On global ACC for foliated threefolds}, arXiv:2303.13083, to appear in Trans. of Amer. Math. Soc.

\bibitem[LMX23a]{LMX23a} J. Liu, F. Meng, and L. Xie, \textit{Complements, index theorem, and minimal log discrepancies of foliated surface singularities}, arXiv:2305.06493.

\bibitem[LMX23b]{LMX23b} J. Liu, F. Meng, and L. Xie, \textit{Uniform rational polytope of foliated threefolds and the global ACC}, arXiv:2306.00330.

\bibitem[LX23a]{LX23a} J. Liu and L. Xie, \textit{Relative Nakayama-Zariski decomposition and minimal models of generalized pairs}, Peking Math. J. (2023).

\bibitem[LX23b]{LX23b} J. Liu and L. Xie, \textit{Semi-ampleness of generalized pairs}, Adv. Math. \textbf{427} (2023), 109126.

\bibitem[McQ98]{McQ98} M. McQuillan, \textit{Diophantine approximation and foliations}, Pub. Math. IHES. \textbf{87} (1998), 121--174.

\bibitem[McQ08]{McQ08} M. McQuillan, \textit{Canonical models of foliations}, Pure Appl. Math. Q. \textbf{4} (2008), no. 3, Special Issue: In honor of Fedor Bogomolov, Part 2, 877--1012.

\bibitem[Miy87]{Miy87} Y. Miyaoka, \textit{Deformations of a morphism along a foliation and applications}, Algebraic geometry, Bowdoin, Proc. Sympos. Pure Math. \textbf{46} (1985) (Brunswick, Maine, 1985), Amer. Math. Soc., Providence, RI (1987), 245--268.


\bibitem[Nak16]{Nak16} Y. Nakamura, \textit{On minimal log discrepancies on varieties with fixed Gorenstein index}, Michigan Math. J. \textbf{65} (2016), no. 1, 165--187.

\bibitem[Nak04]{Nak04} N. Nakayama, \textit{Zariski-decomposition and abundance}, MSJ Memoirs, \textbf{14} (2004), Mathematical Society of Japan, Tokyo.

\bibitem[PS09]{PS09} Y.G. Prokhorov and V. V. Shokurov, \textit{Towards the second main theorem on complements}, J. Algebraic Geom., \textbf{18} (2009), no. 1, 151--199.

\bibitem[Roc97]{Roc97}  R. T. Rockafellar, \textit{Convex analysis} (1997), vol. 11, Princeton University Press.

\bibitem[Sei68]{Sei68} A. Seidenberg, \textit{Reduction of singularities of the differential equation A dy = B dx}, Amer. J. Math. \textbf{90} (1968), 248--269.

\bibitem[Sho00]{Sho00} V. V. Shokurov, \textit{Complements on surfaces}, J. Math. Sci. (New York) \textbf{102} (2000), no. 2, 3876--3932.

\bibitem[Siu10]{Siu10} Y.-T. Siu, \textit{Abundance conjecture}, in \textit{ Geometry and analysis}, no. 2, Ed. by L. Ji, 271--317. Advanced Lectures in Mathematics. Boston International Press.

\bibitem[Spi20]{Spi20} C. Spicer, \textit{Higher dimensional foliated Mori theory}, Compos. Math. \textbf{156} (2020), no. 1, 1--38.

\bibitem[SS22]{SS22} C. Spicer and R. Svaldi, \textit{Local and global applications of the Minimal Model Program for co-rank 1 foliations on threefolds}, J. Eur. Math. Soc. \textbf{24} (2022), no. 11, 3969--4025.

\bibitem[Sza94]{Sza94} E. Szab\'o, \textit{Divisorial log terminal singularities}, J. Math. Sci. Univ. Tokyo, \textbf{1} (1994), no. 3, 631--639.

\bibitem[TX23]{TX23} N. Tsakanikas and L. Xie, \textit{Remarks on the existence of minimal models of log canonical generalized pairs}, arXiv:2301.09186.

\bibitem[Xie22]{Xie22} L. Xie, \textit{Contraction theorem for generalized pairs}, arXiv:2211.10800.

\bibitem[Xu23]{Xu23} Z. Xu, \textit{Abundance for threefolds in positive characteristic when $\nu=2$}, arXiv:2307.03938.

\end{thebibliography}
\end{document}

