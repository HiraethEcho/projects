\documentclass{article}

\usepackage{amsfonts}
\usepackage[all]{xy}
\usepackage{amssymb}
\usepackage{amsmath}
\usepackage{mathrsfs}
\usepackage{amsthm}
\usepackage{enumerate}
\usepackage[hidelinks]{hyperref}
\usepackage{ulem}

\usepackage{geometry}
\geometry{a4paper,left=2cm,right=2cm,top=2cm,bottom=2cm}

\newtheorem{defn}{Definition}[section]
\newtheorem{prop}[defn]{Proposition}
\newtheorem{lem}[defn]{Lemma}
\newtheorem{thm}[defn]{Theorem}
\newtheorem{cor}[defn]{Corollary}
\newtheorem{rmk}[defn]{Remark}
\newtheorem{fact}[defn]{Fact}
\newtheorem{problem}{Problem}
\newtheorem*{ques}{Question}

\setcounter{section}{0}

\title{Cremona Group 1}
\author{wyz}
\date{\today}

\begin{document}
  \maketitle
\section{Conventions}
All surface $S$ means smooth projective $2$-dimensional integral variety over $\mathbb{C}$. Let $f:X\dashrightarrow Y$ be a birational map, then there are maximal pairs $U\subset X $ and $V \subset Y$ such that $f:u\to V$ is an isomorphism. And we call
\[
  \operatorname{Exc}f=X\setminus U
.\]
the exceptional set.

\section{Blow up on Surface}
Let $\pi :Y\to X$ be the blowing up of a point $p$ on surface $X$, and let $x,y$ be the local coordinates near $p$, then
\begin{enumerate}
  \item If $C$ is a curve on $X$ defined by $f_m(x,y)+ r(x,y)$ near $p$, where $f_m$ is a homogeneous polynomial of degree $m$, then
    \[
      m_pC=\operatorname{mult}_pC=m
    .\];
  \item Let $C'$ be the strict transform of $C$ on $Y$, then
    \[
      \pi^*C=C'+mE
    .\]
  \item we have intersections
    \[
      \begin{aligned}
        E^2&=-1\\
        C_{1}'.C_{2}'&=C_{1}.C_{2}-m_{1}m_{2}\\
        K_{X'}&=\pi^*K_{X}+E\\
        K_{X'}^2&=K_{X}^2-1
      \end{aligned}
    .\]
\end{enumerate}
Let $Y\to X_n\to X_{n-1}\to \cdots \to X_{0}=X$ be a sequence of blowing up points, then
\begin{enumerate}
  \item If $p$ is a point on $X_{k},k\neq 0$, then define
    \[
      \operatorname{mult}_pC=\operatorname{mult}_pC'
    .\]
    where $C'$ is strict transform of $C$ on $X_i$.
  \item If $C_Y$ is strict transform of $C$ on $X$ which is smooth, then
    \[
      p_a(C)=g(C_Y)+ \sum^{n}_{i=0} \frac{m_{i}(m_{i-1})}{2}  
    .\]
  and $\operatorname{mult}_{p_{i}}\geqslant \operatorname{mult}_{p_{j}}$ for $p_{j}$ over $p_{i}$.
\end{enumerate}
Let $\Gamma$ be a linear system on $X$, then define
\[
  \operatorname{mult}_p\Gamma=\min_{D \in \Gamma}\operatorname{mult}_pD 
.\]
Then we have 
\begin{enumerate}
  \item 
    \[
      \pi^*\Gamma=\Gamma'+mE
    .\]
\end{enumerate}
\section{Resolve}
\begin{lem}
  Let $f:X\dashrightarrow \mathbb{P}^{n}$ be a rational map, then there is a surface $W$ and morphisms $p:W\to X$ $q:W\to \mathbb{P}^{n}$ resolving the raional map, and $p$ is sequence of blowing ups.
\end{lem}
\begin{cor}
  Let $f:X\dashrightarrow Y$ be a birational map, and $p\in \operatorname{Ind}(f^{-1})$, then there is a curve $C \in Y$ such that $(f^{-1})(C)=p$; If there is a point $q\notin \operatorname{Ind}(f^{-1})$, and $(f^{-1})(q)=p$, then there is a curve $C \subset Y$ such that $q\in C$.
\end{cor}
\begin{lem}
  Let $f:X\to Y$ be a birational morphism and $p\in \operatorname{Ind}(f^{-1})$, then
  \[
    X\to \operatorname{Bl}_pY
  .\]
\end{lem}
\begin{defn}[Minimal resolution]
  $W$ is called the \textbf{minimal resolution} of birational map $f:X\dashrightarrow Y$ if any other resolution $W'$ factors through $W$.
  Furthermore, the points blown up by $p:W\to X$ are called \textbf{base points} of $f^{-1}$. 
\end{defn}
\section{Hizebruch Surface}
Define Hizebruch Surface as follows:
\begin{defn}
 \[
   \mathbb{F}_{n}\subset \mathbb{P}^{2}\mathbb{P}^{1}\quad [x:y:z]\times [u:v]
 .\]
 $\mathbb{F}_{n}$ is defined by $yv^n-zu^n$.
\end{defn}
\section{Groups}
\subsection{Birational groups}
First of all, the Cremona group:
\begin{defn}
  $\operatorname{Cr}_n=\operatorname{Bir}(\mathbb{P}^{n})$.
\end{defn}
Jonqui\`{e}re map:
\begin{defn}
  Let $\eta:\mathbb{P}^{2}\dashrightarrow C$ be a rational map, then $\operatorname{Bir}(\mathbb{P}^{2},\eta)$ is the group of birational maps $f:\mathbb{P}^{2}\dashrightarrow \mathbb{P}^{2}$ such that 
  \[
    \xymatrix{
      \mathbb{P}^{2}\ar@{-->}[r]\ar[d]_\eta&\mathbb{P}^{2}\ar[d]^\eta  \\
    C\ar[r]^{\bar{f}}&C \\ }
  .\]
  where $\bar{f}$ is an isomorphism. Furthermore, if $\eta$ is
  \[
    \begin{aligned}
      \mathbb{P}^{2}&\dashrightarrow \mathbb{P}^{1}\\ 
      [x:y:z]&\mapsto [y:z]
    .\end{aligned}
  .\]
  then the corresponding group $\operatorname{Bir}(\mathbb{P}^{2},\eta)$ is called \bf{Jonqui\`{e}re group}.
\end{defn}

\end{document}
