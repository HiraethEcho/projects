\chapter{下降法}
在这一章中,代数簇对$(X,B)$的边界除子 $B$ 是 $\mathbb{Q}$-除子。 

首先回顾Corti \cite{cortiFactoringBirationalMaps} 给出的三维终端奇点的Sarkisov纲领。令 $f: X\to S$ 和 $f':X'\to S'$ 是双有理等价的两个有终端奇点的三维森纤维空间。
取$S'$上的丰沛除子 $A'$ (整系数Weil除子) 使得 对某个 $\mu'>0$有 $X'$ 上的一般的丰沛除子 $H'$满足$H'\sim -\mu'K_{X'}+f'^*A'$,并令 $H$是  $H'$ 在 $X$上的严格双有理变换 (strict birational transform)。 取一个公共解消$p: W\to X$ 和 $q:W \to X'$。
\begin{enumerate}
  \item 令 $\mu= \max \{c \in \mathbb{R} : K_{X}+\frac{1}{c}H \text{在} S \text{上数值有效} \}$;
  \item 令 $\lambda = \min \{c\in \mathbb{R}: (X,\frac{1}{c}H) \text{有典范奇点}  \}$;
  \item 令 $p:(W, \frac{1}{\lambda} H_{W})\to (X,\frac{1}{\lambda}H)$的解消,且$H_{W}$是$H'$ 在$W$ 上的严格双有理变换,则$e$是$p$-无差的例外除子的个数。
\end{enumerate}
如果 $\lambda \leqslant \mu$,在 $X$运行相对于恰当基底的 $(K_X+\frac{1}{\mu}H)$-MMP ;如果 $\lambda > \mu$ 则构造一个除子解压 (divisorial extraction) $ p:Z \to X$,并运行相对于 $ S$ 的 $ (K_Z+\frac{1}{\lambda}H_Z)$-MMP,这样得到第一个 Sarkisov 连接 $\psi_1: X\dashrightarrow  X_{1}$。这两种情况都是双射线MMP。用 $X_{1}$ 和$\Phi_{1}=\Phi\circ\psi_1^{-1}: X_1\dashrightarrow X'$替换 $X$ 和 $\Phi$ ,并重复这个过程,这样递归地构造一系列Sarkisov 连接。在这个过程中不变量$(\mu,\lambda,e)$将按 字典序下降,最终得到 $\Psi_{N}:X_{N-1} \dashrightarrow X_{N}$,且 $X_{N}\cong X'$。 这就是三维终端奇点代数簇的Sarkisov纲领。

对于klt奇点的 $\mathbb{Q}$-分解代数簇对,考虑MMP-相关的森纤维空间$(X,B)$ 和 $(X',B')$ 。自然的想法是按如下定义 $\mu,\lambda,e$:
\begin{enumerate}
  \item 令 $\mu= \max \{c \in \mathbb{R} : K_{X}+B+\frac{1}{c}H \text{在} S \text{上数值有效} \}$;
  \item 令 $\lambda = \min \{c\in \mathbb{R}: (X,B+\frac{1}{c}H) \text{有典范奇点}  \}$;
  \item 令 $e $为$  (X,B+\frac{1}{\lambda}H)$ 的无差的例外除子的个数。
\end{enumerate}

对 $\lambda$ 的定义将导致一些困难。当 $\lambda > \mu$时,为了构造 Sarkisov连接需要构造除子解压 $p:Z \to X$,并在$Z$ 上运行 $(K_Z+B_Z+\frac{1}{\lambda}H_Z)$-MMP。这个除子解压会解压出一个素除子$E$,这个素除子 $E$ 在 $Z$ 的边界  $B_Z$中的系数是 $1$。如果 $E$ 是 $(X',B')$的边界 $B'$的一项,那么 $E$ 在 $B'$中的系数小于 $1$,这两者不匹配。另一方面,这是需要在具有lc奇点的代数簇对上运行MMP,这比在具有klt奇点的代数簇对上的MMP有技术上的困难。除此之外,由于具有klt奇点的Fano代数簇对的有界性失效,在证明 Sarkisov 纲领的终结性也有困难。

Bruno 和 Matsuki给出了$\lambda$的另一种定义 (见定义\ref{sarkisovdegree}) ,且取决于特定的一个包含$(X,B)$和$(X',B')$ 的代数簇对的集合 $\mathcal{C}_{\theta}$ ,这个集合满足:
\begin{itemize}
  \item
   对任意两个  $\mathcal{C}_{\theta}$中的代数簇对 $(X,B),(X',B')$ ,存在$\mathcal{C}_{\theta}$中的代数簇对 $(W,B_W)$和对数公共解消 $p:W\to (X,B)$ 和 $q:W\to (X',B')$,使得 $(W,B_W)$具有klt奇点且 $p_*B_W=B,q_*B_W=B'$。
% (By the construction of $Z$, the condition $q_*B_W = B'$ implies that the coefficients of $B_Z$ are compatible with $B'$.)
  \item  在集合中的任意代数簇对$(X,B)$ 和 $(Z,B_Z)$上可以运行  $(K_X+B+cH)$-MMP 和 $(K_Z+B_Z+cH_Z)$-MMP,并且所有结果都仍然在集合 $\mathcal{C}_{\theta}$中;
  \item 所有$\mathcal{C}_{\theta}$ 中的代数簇对都具有 $\delta$-lc 奇点,其中$\delta$ 是取决于 $\mathcal{C}_{\theta}$的正数。 
\end{itemize}

\section{定义与引理}
令 $ K=K(X) $ 是双有理等价类的有理函数域 (注意到双有理等价的代数簇有相同的有理函数域),$ \Sigma=\{\nu\} $是有理函数域的离散赋值的集合。
\begin{definition}\label{thetacategory}
  \cite[Definition 3.5]{brunoLogSarkisovProgram1995}
  取一个函数$\theta:\Sigma\to [0,1)_{\mathbb{Q}}$, 那么 可以定义关于  $\theta$的集合$ \mathcal{C}_{\theta} $,包含满足下列条件的具有klt奇点的$\mathbb{Q}$-分解代数簇对$ (X,B=\sum a_{i}B_{i}) $:
  \begin{enumerate}
    \item $ a_i=\theta(B_i) $;
    \item 对所有 $ X $上的例外除子 $E $ 有$ a(E;X,B)>-\theta(E) $。
  \end{enumerate}
\end{definition}
\begin{remark}
例如取 $\theta \equiv 0$为常值函数,那么 $\mathcal{C}_{\theta}$ 是所有和$X$双有理等价的具有终端奇点的代数簇 $Y$ (不带有边界)。
\end{remark}
  根据这个集合可以定义 $\theta$-差异数 ($\theta$-discrepancy ):
\begin{definition}[$\theta$-差异数]
  令 $\mathcal{C}_{\theta}$ 为上述代数簇对的集合,且$(X, B)$ 是有理函数域满足 $K(X)=K$的代数簇对。令  $f: Y\to X$是$(X, B)$的一个对数奇点解消,有分歧公式:
  \[ K_{Y}+B_{Y}+C=f^*(K_{X}+B) ,\]
  其中 $B_{Y}=f^{-1}_*B+ \sum_{E_{i}\text{ 例外除子}} \theta(E_{i})E_{i}$。则 $X$ 的例外除子 $E_{i}$的 $\theta$-差异数定义为
  \[ a_{\theta}(E_{i};X,B)=-\operatorname{mult}_{E_{i}}C. \]
  或等价的,可以定义为
  \[ a_{\theta}(E_{i};X,B)=a(E_{i};X,B)+\theta(E_{i}). \]
  如果 $(X,B)$ 上的所有 例外除子 $E$ 满足   $a_{\theta}(E;X,B)\geqslant 0$ (对应的, $a_{\theta}(E;X,B)> 0$),则称代数簇对 $(X,B)$ 具有$\theta$-典范奇点 (对应的 ,$\theta$-终端奇点)  。
\end{definition}
\begin{remark}
 $\theta$-典范 代数簇对并不总在集合 $\mathcal{C}_{\theta}$中。
\end{remark}

 对MMP-相关的森纤维空间内,可以构造运行Sarkisov纲领所需要的集合 $\mathcal{C}_{\theta} $:
\begin{proposition}\cite[Lemma 3.6]{brunoLogSarkisovProgram1995}\label{cat}
  令 $ f:(X,B)\to S$ 和 $f':(X',B')\to S' $ 是两个 MMP-相关的具有klt奇点的 $ \mathbb{Q} $-分解森纤维空间,有双有理映射 $\Phi$:
  \[ \xymatrix{
      {(X,B)}\ar[d]_{f}\ar@{.>}[r]^\Phi&(X',B')\ar[d]^{f'}\\
      S&S'} \]
假设 $ B=\sum_{i}b_{i}B_{i}+\sum_{j}d_{j}D_j $ 和 $ B'=\sum_{j}d_{j}'D_{j}+\sum_{k}b_{k}'B_{k}' $,其中 $ B_{i} $ 是在 $ X $上但不在 $ X' $上的除子, $ B_k' $ 是在 $ X' $上但不在  $ X $上的除子,而 $ D_j $ 是在 $ X $ 和 $ X' $上的除子。由引理\ref{MMPrelatedConditation},有 $ d_{j}=d_{j}' $。取一个有理数 $\epsilon$满足 
\[ -\operatorname{totdiscrep}(X,B),-\operatorname{totdiscrep}(X',B')  <\epsilon <1 ,\]
并按如下定义函数 $ \theta: \{ \nu \} \to [0,1)_{\mathbb{Q}} $ :
  \begin{itemize}
    \item 对于边界  $B,B'$的除子,有$ \theta(B_i)=b_i, \theta(D_{j})=d_{j},\theta(B_{k}')=b_{k}'$;
    \item  如果 $E$ 是 $X$ 和 $X'$ 上的例外除子,则    $ \theta(E)=\epsilon $;
    \item   如果 $ D $ 是 $ X $ 和 $ X' $上的除子,但不是$ B $ 或 $ B' $的部分,则$ \theta(D)=0 $。
  \end{itemize}
  那么定义\ref{thetacategory} 构造的集合$ \mathcal{C}_{\theta} $ 满足:
  \begin{enumerate}
    \item $ (X,B) $ 和 $ (X',B') $在集合 $ \mathcal{C}_{\theta} $中;
    \item   对 $ \mathcal{C}_{\theta} $中任意有限多个具有klt奇点的代数簇对$ \{(X_l,B_l)\} $,有$ (Z,B_Z)\in \mathcal{C}_{\theta} $ 和 射影双有理态射 $ Z\to X_l $使得 $X_{l}$ 是 相对于$X_{l}$的   $ (K_{Z}+B_{Z}) $-MMP的输出,因此也是相对于 $ \mathrm{Spec}\,\mathbb{C} $的$(K_Z+B_Z)$-MMP 结果;
    \item 任何从  $ \mathcal{C}_{\theta} $中一个元素出发的 $ (K+B) $-MMP ,其结果依然落入 $ \mathcal{C}_{\theta} $。对任何$ c\in \mathbb{Q}_{>0} $和无基点 (base point free)的整系数除子 $H$ 给出的$ (K+B+cH) $-MMP也成立。
  \end{enumerate}
\end{proposition}
\begin{remark}\label{delta-lc}
  令 $\delta=1-\epsilon$,那么所有$\mathcal{C}_{\theta}$ 中的代数簇对都是具有$\delta$-lc。
\end{remark}

使用命题 \ref{cat}中的假设和记号,可以定义Sarkisov 次数 ( Sarkisov degree )。取 $S'$ 上的极丰沛的除子 $A'$和足够大和足够可除的整数 $\mu'>1 $使得
\[ \mathcal{H}'=|-\mu' (K_{X'}+B') +f'^*A'| \]
是 $ X' $ 在 $ \mathrm{Spec}\,\mathbb{C}$上的极丰沛的完全线性系 。 令 $ (W,B_W) $ 是 $ X $ 和 $ X' $ 在 $ \mathcal{C}_{\theta} $ 中的公共对数解消,有射影态射 $ \sigma:W\to X$和   $\sigma':W\to X' $ 满足 $\sigma_*B_W=B, \sigma'_*B_W=B' $。令$\mathcal{H}_W:=\sigma'^*\mathcal{H}'$,那么  $\mathcal{H}:=\Phi^{-1}_*\mathcal{H}'=\sigma_*\mathcal{H}_W$。进一步,如果 $ \mathcal{H} $ 不是无基点的,那么
\[ \sigma^*\mathcal{H}=\mathcal{H}_W+F ,\]
其中 $ F=\sum f_{l}F_{l}\geqslant0 $ 是固定部分(fixed part)。线性系 $ \mathcal{H}' $中的一个一般除子 $H'$使得 $ H_W:=\sigma'^*H'=\sigma'^{-1}_*H'\in \mathcal{H}_W $,并记 $ H:=\Phi^{-1}_*H'=\sigma_*H_{W} $。那么 $H$ 是 $f$-丰沛的,且$ \sigma^*H=H_W+F $。 通过取进一步的解消,不妨设 $H_{W}$ 与$\sigma$ 和 $\sigma'$的例外除子各部分光滑且互相横截相交(即$(W,H_{W}+ \operatorname{Exc}\sigma+ \operatorname{Exc}\sigma')$是对数光滑的)。


接下来定义在 $\mathcal{C}_{\theta}$中关于 $H'$ (或 $\mathcal{H}'$)的Sarkisov次数:
\begin{definition}\label{sarkisovdegree}
  \cite[Definition 3.8]{brunoLogSarkisovProgram1995}
 $\mathcal{C}_{\theta}$中关于 $H'$ (或 $\mathcal{H}'$)的Sarkisov次数是一个按字典序排序的三元组$ (\mu,\lambda,e) $,其中:
  \begin{itemize}
    \item \textbf{ 数值有效阈值$ \mu $}:令 $ C\subset X  $ 是被$ f $压缩的曲线,那么
          \[ \mu:=-\frac{H\cdot C}{(K_X+B)\cdot C} .\]
          即 $ K_X+B+\frac{1}{\mu} H \equiv_S0$;
    \item \textbf{$ \theta $-典范 阈值  $ \frac{1}{\lambda} $}:  若$ \mathcal{H} $无基点则定义 $\lambda=0$;否则定义
          \[ \frac{1}{\lambda}:=\max\{t:a_{\theta}(E;X,B+tH)\geqslant 0,  \forall \ X\text{上例外除子}E \};\]
    \item \textbf{ $(K_{X}+B_{X}+\frac{1}{\mu}H)$的$\theta$-无差除子个数}:  $ e=0 $ 若 $ \mathcal{H} $ 无基点 (此时 $ \lambda=0 $)则定义  $e=0$;否则定义
          \[ e=\#\{E; E \text{ 是 }\sigma\text{-例外除子,且 } a_{\theta}(E;X,B+\frac{1}{\lambda} H)=0 \} .\]
  \end{itemize}
\end{definition}

\begin{remark}
  对于Sarkisov次数,有下列性质:
  \begin{enumerate}
    \item  Sarkisov次数取决于  $A', H'$ 和  $\theta$的选取。
    \item   取公共对数解消$ (W,B_W)\in \mathcal{C}_{\theta} $,其中 $ B_W=\sum \theta(E)E $ ,并且有射影双有理态射 $ \sigma:W\to X , \sigma':W\to X' $。 由于 $\sigma^*\mathcal{H}=\mathcal{H}_W+\sum f_{l}F_{l}$,所以有分歧公式:
          \[ K_W+B_W+tH_W=\sigma^*(K_X+B+tH)+\sum(a_l-tf_l)E_l ,\]
          其中 $ \sum a_l E_l $ 是有效除子且支撑在 $ \mathrm{Exc}\,\sigma $上。那么 $\lambda:=\max\{ \frac{f_l}{a_l}\}$。如果$ \mathcal{H} $是无基点的,那么 $ \sum f_lF_l=0 $ 且$\lambda=0  $。
    \item   $ e $ 是公式
          \[ K_W+B_W+\frac{1}{\lambda} H_W=\sigma^*(K_X+B+\frac{1}{\lambda} H)+\sum(a_l-\frac{1}{\lambda} f_l)E_l .\]
      中系数$\sum(a_l-\frac{1}{\lambda}f_l)E_l$ 为 $ 0 $的部分的个数。
          这样的素除子 $E_{1},\ldots, E_{e}$ 称作$\theta$-无差的。
  \end{enumerate}
\end{remark}
需要构造在集合 $\mathcal{C}_{\theta}$中的解压态射:
\begin{lemma}\label{thetaextraction}
  使用定义\ref{sarkisovdegree}中的记号,并假设 $\lambda \neq 0$,那么存在压缩态射  $f: Z\to X$ 满足:
  \begin{itemize}
    \item $(Z,B_{Z})\in \mathcal{C}_{\theta}$ 且 $(Z,B_{Z}+\frac{1}{\lambda}H_{Z})$ 具有$\theta$-终端奇点的 $\mathbb{Q}$-分解代数簇对;
    \item  $\rho(Z)=\rho(X)+1$;
    \item $(K_{X}+B+\frac{1}{\lambda}H)$是 $f$-无差的,即
          \[ K_{Z}+B_{Z}+\frac{1}{\lambda}H_{Z}=f^*(K_{X}+B+\frac{1}{\lambda}H) .\]
  \end{itemize}
\end{lemma}
\begin{proof}
  按照\cite[Proposition 1.6]{brunoLogSarkisovProgram1995}的思路来证明。取 定义\ref{sarkisovdegree}中的$ (W,B_{W})\in \mathcal{C}_{\theta}$和公共对数解消  $\sigma:W\to X,\sigma':W \to X'$ 。将 $(K_{X}+B+\frac{1}{\lambda}H)$的$\theta$-无差除子重新编号 $E_{1},\ldots ,E_{e}$ ,那么有
  \[ K_W+B_W+\frac{1}{\lambda} H_W=\sigma^*(K_X+B+\frac{1}{\lambda} H)+\sum_{l=1}^{e} 0\cdot E_{l}+\sum_{l>e}(a_l-\frac{1}{\lambda} f_l)E_l .\]
  在 $W$ 上运行相对于 $X$的对某丰沛除子标量的  $(K_{W}+B_{W}+\frac{1}{\lambda}H_{W})$-MMP,将终结于 $(W, B_{W}+\frac{1}{\lambda}H_{W})$相对于 $X$ 极小模型 $p:(Y, B_{Y}+\frac{1}{\lambda}H_{Y})\to X$,且 $p$ 的例外除子恰好是 $\cup_{i=1}^{e}E_{i}$,并且$p$是无差的:
  \[ K_{Y}+B_{Y}+\frac{1}{\lambda}H_{Y}=p^*(K_{X}+B+\frac{1}{\lambda}H) .\]
  接下来运行相对于 $X$ 的对某丰沛除子标量的  $(K_{Y}+B_{Y})$-MMP ,将终结于  $(Y,B_{Y})$ 相对于 $X$的极小模型,这个极小模型就是$(X,B)$。令 $f: Z\to X$ 是MMP中的最后一个除子压缩,那么$f$就是满足条件的除子解压态射。 \end{proof}

\section{构造Sarkisov连接}
这一节主要按照 \cite[\S1]{brunoLogSarkisovProgram1995}的内容。

\textbf{终结判定:}如果 $ \lambda\leqslant\mu $ 且 $ K_X+B+\frac{1}{\mu}H $ 是数值有效的,那么两个森纤维空间是同构的 (见定理\ref{nfi}),Sarkisov纲领在此结束。

\textbf{构造连接:}
如果上述条件不成立,则进行Sarkisov纲领的归纳构造。
\begin{lemma}
  \begin{enumerate}
    \item 如果$ \lambda\leqslant\mu $ 且 $ K_X+B+\frac{1}{\mu}H $ 不是数值有效的,那么存在压缩态射 $X \to T$和 第三型或第四型 Sarkisov连接 $\psi_{1}:X\dashrightarrow X_{1}$ ;
    \item  如果 $ \lambda>\mu $那么存在除子压缩 (除子解压) $p:Z\to X$ 和第一型或第二型Sarkisov 连接 $ \psi_{1}:X\dashrightarrow X_{1}$。
  \end{enumerate}
\end{lemma}
\begin{proof}
  分两种情况考虑:
  \begin{enumerate}
    \item 假设 $\lambda\leqslant \mu$ 和  $ K_X+B+\frac{1}{\mu}H $ 不是数值有效的。记 $ f $ 是 $ (K_X+B) $-负性的极端射线$ R= \overline{\operatorname{ NE }}(X/S) $的压缩态射,那么由 $\mu$ 的定义有$ (K_X+B+\frac{1}{\mu}H)\cdot R=0 $ 。存在极端射线 $ P \subset \overline{\operatorname{ NE }}(X) $使得$ (K_X+B+\frac{1}{\mu}H)\cdot P<0 $ 且 $ F:=P+R $是极端面  (细节见 \cite [5.4.2]{cortiFactoringBirationalMaps} )。取  $ 0<\delta\ll 1 $ 使得 $ (K_X+B+(\frac{1}{\mu}-\delta)H)\cdot P<0 $,由于  $H$ 是 $f$-丰沛的,有 $  (K_X+B+(\frac{1}{\mu}-\delta)H)\cdot R<0 $。因此 $ F $ 是 $  (K_X+B+(\frac{1}{\mu}-\delta)H) $-负性的极端面。由于 $ (X,B+(\frac{1}{\mu}-\delta)H) $有 klt奇点,由压缩定理,存在关于 $F$ 的压缩态射 $ g:X\to T $  穿过 $ f:X\to S $。
  \[ \xymatrix{
      X\ar[d]\ar[rdd]& \\
      S\ar[rd]&\\
         &T } \]
      由于  $ (X,B+\frac{1}{\mu}H) $具有klt奇点,且 $ \rho(X/T)=2 $,可以运行 相对于$T$ 的 关于某丰沛除子标量的 $ (K_X+B+\frac{1}{\mu}H) $-MMP。由于 $ B+\frac{1}{\mu}H $是相对于 $T$的大除子,这个 MMP终结。有下列情况:
    \begin{enumerate}
      \item 在有限多步翻转复合 $ X\dashrightarrow Z $后,第一个非翻转的压缩态射是一个除子压缩 $ p:Z\to X_1 $,之后是一个森纤维空间的压缩态射 $f_{1}:(X_{1},B_{1}+\frac{1}{\mu}H_{1})\to S_1$。这个压缩态射$f_1$也是关于$(X_{1},B_{1})$的森纤维空间。
      \[ \xymatrix{
          X\ar@{.>}[r]\ar[d]& Z\ar[rd] \\
          S\ar[rd]& & X_{1}\ar[d]\\
               &T\ar[r]^{\sim}& S_{1} } \]
      这是第三型的Sarkisov连接。
      \item 在有限多步翻转复合$ X\dashrightarrow X_1 $后,第一个非翻转的压缩态射是森纤维空间的压缩态射 $ f_1:(X_1,B_1+\frac{1}{\mu}H_1)\to S_{1} $。这个压缩态射 $f_1$ 也是关于 $(X_{1},B_{1})$的森纤维空间。
      \[ \xymatrix{
          X\ar@{.>}[rr]\ar[d]& &X_{1}\ar[d] \\
          S\ar[rd]& & S_{1}\ar[ld]\\
             &T& } \]
        这是第四型的Sarkisov连接。
      \item 在有限多步翻转复合 $ X\dashrightarrow Z $后,第一个非翻转的压缩态射是一个除子压缩 $ p:Z\to X_1 $,且
            \[ K_Z+B_Z+\frac{1}{\mu}H_Z=p^*(K_{X_1}+B_1+\frac{1}{\mu}H_1)+eE ,\]
            其中 $ e>0 ,E=\operatorname{Exc}\,p$ 且  $f_{1}: (X_1,B_1+\frac{1}{\mu}H_1) \to T$ 是关于 $(X,B+\frac{1}{\mu}H)$ 在 $T$ 上的极小模型。事实上  $ \overline{\operatorname{NE}}(X_1/T) $ 唯一的极端射线是$ (K_{X_1}+B_1+\frac{1}{\mu}H_1) $-平凡的,所以是 $ (K_{X_1}+B_1) $-负性的。所以 $ f_1:(X_1, B_1)\to T $ 是森纤维空间。
      \[ \xymatrix{
          X\ar@{.>}[r]\ar[d]& Z\ar[rd] \\
          S\ar[rd]& & X_{1}\ar[d]\\
               &T\ar[r]^{\sim}& S_{1} } \]
            取 $ S_1=T $,这是第三型Sarkisov连接。
      \item 在有限多步翻转复合 $ X\dashrightarrow Z $后,  $(K_{X}+B+\frac{1}{\mu}H)$-MMP 终结于 $T$ 上的极小模型 $ (X_1,B_1+\frac{1}{\mu}H_1) $。那么存在$ \overline{\operatorname{NE}}(X_1/T) $的极端射线 $R$ ,并且是 $ (K_{X_1}+B_1+\frac{1}{\mu}H_1) $-平凡的和 $ (K_{X_1}+B_1) $-负性的。
            令$ f_1:X_1\to S_1 $ 为 关于$R$的压缩态射,这是第四型的Sarkisov连接。
            \[ \xymatrix{
                X\ar@{.>}[rr]\ar[d]& &X_{1}\ar[d] \\
                S\ar[rd]& & S_{1}\ar[ld]\\
                  &T& } \]
    \end{enumerate}
  \item 假设 $\lambda>\mu$。取引理\ref{thetaextraction}构造的除子解压 $ p:(Z,B_Z+\frac{1}{\lambda}H_Z)\to (X,B+\frac{1}{\lambda}H) $,即   $ (Z,B_Z) $ 具有$ \theta $-终端奇点 且 $ p^*(K_X+B+\frac{1}{\lambda}H)=K_Z+B_Z+\frac{1}{\lambda}H_Z $,其中 $ B_Z=\sum\theta(E_{\nu})E_\nu $,$Z$ 上的除子$E_{\nu}$对应赋值$\nu$。
    运行相对于 $S$ 的对某丰沛除子标量的 $ (K_Z+B_Z+\frac{1}{\lambda}H_Z) $-MMP,由于 $Z$被 $ (K_Z+B_Z+\frac{1}{\lambda}H_Z) $-负性的曲线覆盖,$ (K_Z+B_Z+\frac{1}{\lambda}H_Z) $ 不是相对伪有效的。因此由定理\ref{notpseudoeffmfs},这个 MMP终结于森纤维空间 。有下列两种情况:
    \begin{enumerate}
      \item 在有限多步翻转复合 $ Z\dashrightarrow X_1 $后,第一个非翻转的压缩态射是森纤维空间$f_1:(X_1,B_1+\frac{1}{\lambda}H_1)\to S_1$ 由于 $ (K_{X_1}+B_1+\frac{1}{\lambda}H_1) $ 是在$S_1$上反丰沛(anti-ample)的除子,且 $ H_1 $ 是 $ f_1 $-丰沛的所以 $(K_{X_1}+B_1) $在 $S_1$上反丰沛。
      \[ \xymatrix{
        &Z\ar@{.>}[r]\ar[ld] &X_{1}\ar[d] \\
          X\ar[d]& &S_{1}\ar[lld]\\
          S   & & } \]
        因此$ f_1:(X_1, B_1)\to S_1 $ 是森纤维空间,这是第一型的Sarkisov连接。
      \item 在有限多步翻转复合 $ X\dashrightarrow X' $后,第一个非翻转的压缩态射是一个除子压缩 $ q:Z'\to X_1 $,接着是森纤维空间压缩态射$f_1:(X_1,B_1+\frac{1}{\lambda}H_1)\to S$。
      \[ \xymatrix{
        &Z\ar@{.>}[r]\ar[ld] &Z'\ar[rd] \\
          X\ar[d]& & &X_{1}\ar[d]\\
          S\ar[rrr]^{\sim}   & & & S_{1} } \]
        令$ S_1=S $,那么压缩态射$f_1$ 同时也是关于 $(X_1, B_1)$的森纤维空间。这是第二型的Sarkisov连接。
    \end{enumerate}
  \end{enumerate}
\end{proof}
用  $(X_1,B_1)$ 和$\Phi_{1}=\Phi\circ\psi_1^{-1}$替换 $(X,B)$ 和 $\Phi$,并重复上述引理内容。 
\begin{assertion}\label{R-Sarkisovdeg}
  Sarkisov次数在此过程中按字典序下降:
  \begin{itemize}
    \item 对于(1):
      \begin{itemize}
            \item  (1).1 和 (1).2的情况下,由于 $ K_{X_1}+B_1+\frac{1}{\mu}H_1 $ 在 $S_1$上反丰沛,所以有 $\mu_1<\mu$。
            \item (1).3 和(1).4 的情况下,由于$ (K_{X_1}+B_1+\frac{1}{\mu}H_1) $ 在射线 $ R=\overline{\operatorname{NE}}(X_1/S_1) $,所以有 $\mu_1=\mu$。
              注意到 $ (X_1,B_1+\frac{1}{\mu}H_1) $任然具有$ \theta $-典范奇点,所以 $\lambda_1\leqslant \mu=\mu_1$,所以下一个Sarkisov连接依然是情况(1)。对于(1).3的情况,还有 $\rho(X_1)=\rho(X)-1$。
      \end{itemize}
    \item 对 (2).1,有 $\mu_1\leqslant \mu$和  $\lambda_1\leqslant \lambda$。且如果 $ \lambda_1=\lambda $,那么 $e_1<e$。
  \end{itemize}
\end{assertion}

\section{下降法的终结性}\label{termination1}
首先说明在终结判定满足时,得到的森纤维空间和$X'\to S'$同构:
\begin{theorem}[Noether-Fano-Iskovskikh 判定法]\label{nfi}
  按照定义\ref{sarkisovdegree}中的记号,有
  \begin{enumerate}
    \item $ \mu\geqslant \mu' $;
    \item 如果 $ \mu \geqslant \lambda $ 且 $ (K_X+B+\frac{1}{\mu} H) $ 是数值有效的,那么 $\Phi$ 是森纤维空间的同构,即有交换图表:
          \[ \xymatrix{
              X\ar[r]^\sim_{\Phi}\ar[d]_f&X'\ar[d]^{f'}\\
              S\ar[r]^\sim& S' } \]
  \end{enumerate}
\end{theorem}

\begin{proof}
  按照 \citet[Claim 13.20]{haconMinimalModelProgram2012}, \citet[Theorem 5.1]{liuSarkisovProgramGeneralized2021} 和 \citet[Theorem 4.2]{cortiFactoringBirationalMaps}的思路给出证明:
  \begin{enumerate}
    \item 只需证明 $ (K_X+B+\frac{1}{\mu'}H) $ 是 $ f $-数值有效的。 取公共解消 $\sigma:W\to X$ 和 $\sigma':W\to X'$,有分歧公式
          \[ \begin{aligned}
              K_W+B_W+\frac{1}{\mu'}H_W= & \sigma'^*(K_{X'}+B'+\frac{1}{\mu'}H')+\sum e'_jE_j+ \sum g_k'G_k' \\
              =                          & \sigma^*(K_{X}+B+\frac{1}{\mu'}H)+\sum g_iG_i+\sum e_jE_j,
            \end{aligned} \]
          其中 $ \{G_i\}, \{E_j\} $ 是 $ \sigma $-例外除子,  $ \{E_j\}, \{G'_k\} $ 是 $ \sigma' $-例外除子。即$ G_{i}$是只在 $\sigma$上例外的除子,$ G'_{k}$是只在 $\sigma'$上例外除子,$E_{j}$是在二者上都例外的除子。 由于  $H_W=\sigma'^*H' $,所以 $ g_k'>0 $ 或者没有这样的 $ G'_k $ (这是由$B_{W}$的构造得到的)。取被 $f$ 压缩的的一般曲线 $ C\subset X $ ,且它在 $W$ 的双有理原像$ \tilde{C} $和 $ G_i, E_j $无交,并且不包含在$ G'_k $中。那么有
          \[ \begin{aligned}
              C\cdot\left(K_X+B+\frac{1}{\mu'}H\right)= & \tilde{C}\cdot\left(\sigma^*\left(K_X+B+\frac{1}{\mu'}H\right)+\sum g_iG_i+\sum e_jE_j\right)           \\
              =                                     & \tilde{C}\cdot\left(\sigma'^*\left(K_{X'}+B'+\frac{1}{\mu'}H'\right)+\sum e'_jE_j+ \sum g_k'G_k'\right) \\
              =                                     & \tilde{C}\cdot\sigma'^*f'^*A'+\tilde{C}\cdot\left(\sum g_k'G_k'\right) \geqslant0 .
            \end{aligned} \]
          由此推出 $ (K_X+B+\frac{1}{\mu'}H) $ 是 $ f $-数值有效的,且 $ \mu\geqslant \mu' $;
    \item 首先证明 $ \mu=\mu' $。只需证$\mu'\geqslant \mu $ ,同上,只需证$ (K_{X'}+B'+\frac{1}{\mu}H') $ 是 $ f' $-数值有效的。同上取 被 $f'$ 压缩的 $X'$ 上的一般曲线$ C' \subset X'$ ,并且在 $W$ 上的双有理原像$\tilde{C}'$ 和  $ G'_k, E_j $无交,并且不包含在 $ G_i $。那么同上可得到$C'\cdot\left(K_{X'}+B'+\frac{1}{\mu}H'\right)\geqslant 0$,即 $ (K_{X'}+B'+\frac{1}{\mu}H') $ 是 $ f' $-数值有效的。而$ (K_{X'}+B'+\frac{1}{\mu'}H')\equiv_{f',\mathbb{Q}}0 $,所以$ \frac{1}{\mu}\geqslant \frac{1}{\mu'} $,这就推出$\mu'\geqslant \mu $。

接下来证明它们同构。 取 $X$ 上  极丰沛除子 $ D $,且$D'  $ 是在$ X' $上的严格双有理变换。那么  $ D' $ 是 $ f' $-丰沛的,所以存在$ 0<d\ll1 $使得:
          \begin{itemize}
            \item $ K_X+B+\frac{1}{\mu }H+dD $ 是丰沛除子;
            \item $ K_{X'}+B'+\frac{1}{\mu }H'+dD' $ 是丰沛除子;
            \item $(W,B_{W}+\frac{1}{\mu}H_{W}+dD_{W})$具有klt奇点 (因为 $\mu \geqslant \lambda$,所以可以做到)。
          \end{itemize}
          因此 $X$ 和 $X'$ 都是 $(W,B_{W}+\frac{1}{\mu}H_{W}+dD_{W})$的对数典范模型,由对数典范模型的唯一性, $X\cong X'$。更进一步, $f$ 和  $f'$压缩相同的曲线数值等价类,所以两个森纤维空间同构。
  \end{enumerate}
\end{proof}

Bruno和Matsuki\cite{brunoLogSarkisovProgram1995}给出的下降法的Sarkisov纲领的终结性需要下列条件:
\begin{enumerate}
  \item 数值有效阈值$\mu$的离散性 (或者降链条件,descending chain condition, DCC)。由于本章的代数簇对$(X,B)$的边界除子是$\mathbb{Q}$-除子,由$\delta$-lc Fano 代数簇对的有界性和相关定理 (\cite[Theorem 1.1]{birkarSingularitiesLinearSystems2020}),任意维数下此条件都成立 (离散的正有理数集满足DCC)。
  \item 翻转的终结性 (这对高维情况还没有完全证明); 
  \item lct的ACC (ascending chain condition of log canonical thresholds);
  \item 对具有终端奇点的代数簇的Sarkisov纲领需要局部lct的有限性;对具有klt奇点的代数簇对的Sarkisov纲领需要局部$\theta$-ct ($\theta$-canonical thresholds)的有限性。这些在四维及以上的情况还不清楚。
\end{enumerate}
\textbf{用反证法部分证明终结性: }
简要的说明思路: 如果上一节中构造的Sarkisov连接构成无限长的序列,即有无限多个 $ X_i $ 和构造的映射:
\[ X=X_0\dashrightarrow X_1\dashrightarrow \cdots\dashrightarrow X_i \dashrightarrow\cdots\dashrightarrow X'.\]
\begin{enumerate}
  \item 由 $\mu_{i}$的离散性(DCC),在有限步后 $\mu_{i}$ 是常值,不再下降。不妨设 $\mu=\mu_{0}=\mu_{i}$ 对所有 $i$成立。
  \item 如果序列中有一个第三型或第四型的 Sarkisov连接 $\psi_i$ ,那么后续每一个 Sarkisov 连接 $\psi_j, j>i$都是第三型或第四型 (见断言 \ref{R-Sarkisovdeg})。由于第三型 Sarkisov 连接的 Picard数严格下降,所以只有有限多个。于是在$j\gg 0$时 $\psi_j$ 全部为第四型Sarkisov连接,同时也是无限多个翻转的复合。对于三维代数簇对和四维伪有效的代数簇对,这样的翻转只有有限多步,这是一个矛盾。在高维其他情况还不清楚。
  \item 假设所有Sarkisov连接都是第一型或者第二型的。lct的ACC对所有维数都成立 \cite{HMX14},所以存在正数 $\alpha$使得 对 $i\gg 0$ 有$(X_i,B_i+\alpha H_i)$ 具有klt奇点,并且每个Sarkisov 连接 $\psi_i,i\gg 0$ 都 同时是 相对于$S_{i}$的  $(K_{Z_i}+B_{Z_i}+\alpha H_{Z_i})$-MMP 结果。在具有终端奇点的三维代数簇的情况,这和 $\theta$-典范阈值的有限性矛盾 (具体见\cite[Claim 2.2]{brunoLogSarkisovProgram1995})。
\end{enumerate}

