\documentclass{article}

\usepackage{amsfonts}
\usepackage[all]{xy}
\usepackage{amssymb}
\usepackage{amsmath}
\usepackage{mathrsfs}
\usepackage{amsthm}
\usepackage{enumerate}
\usepackage[hidelinks]{hyperref}
\usepackage{tikz}  

\usepackage{geometry}
\geometry{a4paper,left=2cm,right=2cm,top=2cm,bottom=2cm}

\newtheorem{definition}{Definition}[section]
\newtheorem{proposition}[definition]{Proposition}
\newtheorem{lemma}[definition]{Lemma}
\newtheorem{theorem}[definition]{Theorem}
\newtheorem{corollary}[definition]{Corollary}
\newtheorem{remark}[definition]{Remark}
\newtheorem{fact}[definition]{Fact}
\newtheorem{assertion}[definition]{Assertion}
\newtheorem{example}[definition]{Example}
\newtheorem{problem}{Problem}
\newtheorem*{ques}{Question}
\setcounter{section}{0}

\title{On the Du Bois property of secant varieties}

\author{ Lei Song}
%\date{\today}

\begin{document}
\maketitle
%\tableofcontents
%\newpage
\section{Introduction to D.B}
$X$ smooth projective variety over $\mathbb{C}$, and $ \dim X =n $
complex:
\[
  0\to \mathcal{O}_{X} \xrightarrow{d=\partial} \Omega^{1}\to \cdots \to Omega^{n}\to 0
\]
there is a filtration of the cotangent sheaf:
\[
  F^{0}: 0\to \mathcal{O}_{X} \xrightarrow{d=\partial} \Omega^{1}\to \cdots \to Omega^{n}\to 0
\]
\[
  F^{1}: 0\to  \Omega^{1}\to \cdots \to \Omega^{n}\to 0
\]

let $ Gr^{p}_{F}\underline{\Omega}_{X}^{\cdot}=\Omega_{X}^{p}$

$ E_{1}^{p,q}=\mathbb{H}^{q}(X,\underline{\Omega}_{X}^{p}) \implies H^{p+q}(X,\mathbb{C}) $

\begin{definition}
  If $ \mathcal{O}_{X}\to \underline{\Omega}_{X^{0}} $ is quasi-isomorphism, then $ X $ is Du Bois.
\end{definition}

\begin{proposition}[property of Du Bois]
  \begin{enumerate}
    \item 
      klt $\implies $ rational singularity $\implies $ Du Bois

      klt $\implies $ log canonical $\implies $ Du Bois
    \item A fibration $ \mathcal{X}\to B $ such that $ \mathcal{X}_{b} $ is D.B for all $b$. If $ \mathcal{X}_{B_{0}} $ has $ S_{k} $ condition, then so is any $ \mathcal{X}_{b} $
      \item $ R^{q}f_{*}\mathcal{O}_{X} $ is locally free and compatible with case change.
  \end{enumerate}
\end{proposition}

\textbf{Higher Du Bois}
\begin{definition}[Shen-Veriketesh-Vo]
  Fix $ p\in \mathbb{N} $, $X$ is $p$-Du bois if $X$ is semi-normal, and
  \begin{enumerate}
    \item $ h^{0}(\underline{\Omega}_{X}^{k})\to \underline{\Omega}_{X}^{k} $ is quasi-isomorphism for $ 0\leq k\leq p $ (pre-$ p $-D.B.)
    \item $ Codim (X_{sing},X)\geq 2p+1 $
    \item $  h^{0}(\underline{\Omega}_{X}^{k})\xrightarrow{\sim} \underline{\Omega}_{X}^{[k]} $
  \end{enumerate}
\end{definition}
\begin{definition}
  $X$ is $p$-rational singularity if $X$ is normal and
  \begin{enumerate}
    \item $ Codim (X_{sing},X)> 2p+1 $
    \item For log resolution $ f:\hat{X}\to X $, and exceptional divisor $E$, there is  $ R^{i}f_{*}(\Omega_{X}^{K}(\log E))=0 $ for $ \forall i\geq -, 0\leq k\leq p $ (pre-$ p $-rational)
  \end{enumerate}
\end{definition}
\begin{remark}[some results]
  \begin{enumerate}
    \item And $ p $-rational $ \implies $ $ p $=Du bois.
      \item In lc I $ p-DB \implies (p-q) $-rational
      \item In Non lc I $ p-DB \not\implies (p-q) $-rational
  \end{enumerate}
  
\end{remark}
\section{Singularity of secant varieties}
Secant varieties $ \Sigma_{k} $ of a variety $X \hookrightarrow \mathbb{P}^{N}$ is defined as  union of $k$-planes generated by points in $X$.

\begin{remark}
  For $ X \hookrightarrow \mathbb{P}^{N} $ by $ |L| $, if $L$ hhas insufficient positive, then degenerate: $ \dim \Sigma_{k} < $ expected. (expecte $ \dim \Sigma^{k}=k+(k+1)n $)

  Example: consider $ \mathbb{P}^{2}\hookrightarrow \mathbb{P}^{5} $, then $ \Sigma $ is a hypersurface.
\end{remark}

\begin{remark}
  Assume that 
\end{remark}

\begin{theorem}[Wlley,Vermeire]
  $ \Sigma $ is normal.
\end{theorem}
\begin{theorem}[Chou-S]
  \begin{itemize}
    \item $ \Sigma $ has worst D.B.
    \item $ \Sigma $ is C.M.   if and only if $ h^{1}(X,\mathcal{O})=\cdots = h^{n-1}(X,\mathcal{O})=0. $
      \item $ \omega^{GR}_{X}\cong \omega  $   if and only if $ h^{n}(X,\mathcal{O})=0 $
  \end{itemize}
\end{theorem}
\section{Geometry of secant varieties}
$ X^{[2]} $ Hibert scheme of $ 2 $-points on $X$. Then
\[
  \xymatrix{
    \Phi\ar[r]& X^{[2]} \times X \ar[ld]\ar[rd] &\\
    X^{[2]}\ar[rr] & &X 
  } 
\]
\section{ref}
ref: https://doi.org/10.1112/plms.12635
\end{document}
