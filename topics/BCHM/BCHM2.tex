%!TeX program = xelatex
\documentclass{article}

\usepackage{amsfonts}
\usepackage[all]{xy}
\usepackage{amssymb}
\usepackage{amsmath}
\usepackage{mathrsfs}
\usepackage{amsthm}
\usepackage{enumerate}
\usepackage[hidelinks]{hyperref}
\usepackage{ulem}
\usepackage{tikz}  
\usetikzlibrary{arrows.meta}%画箭头用的包

\usepackage{geometry}
\geometry{a4paper,left=2cm,right=2cm,top=2cm,bottom=2cm}

\newtheorem{defn}{Definition}[section]
\newtheorem{prop}[defn]{Proposition}
\newtheorem{lem}[defn]{Lemma}
\newtheorem{thm}[defn]{Theorem}
\newtheorem{cor}[defn]{Corollary}
\newtheorem{rmk}[defn]{Remark}
\newtheorem{fact}[defn]{Fact}
\newtheorem{problem}{Problem}
\newtheorem*{ques}{Question}

\setcounter{section}{0}

\title{lecture 2: Main results of BCHM}
\author{wyz}
\date{\today}

\begin{document}

  \maketitle
  % \tableofcontents
  % \newpage

\section{MMP with scaling}
also called \textbf{directed MMP}:
Let $f:(X,B)\to S$ dlt $\mathbb{Q}$-factorial pair. Let $H$ effective $\mathbb{R}$-Cartier divisor (e.g.  $H$ is sufficiently ample ) such that
\begin{enumerate}
  \item $(X,B+H)$ dlt
  \item $K_{X}+B+H$ nef.
\end{enumerate}

This imoplies $(X,B+H)$ is minimal model over $S$. 

Let 
\[
v_{1}=\min \{t\geqslant 0: K_{X}+B+tH \text{nef over }S\} 
\]
if $v_{1}=0$, this ends; if $v_{1}>0$, then 
\textbf{Claim}: there is extremal ray $R$ s.t. 
\begin{itemize}
  \item 
    $(K_{X}+B+tH).R=0$
  \item $(K_{X}+B)=0$
\end{itemize}
then  we have contraction $(X_{1},B_{1})=(X,B)\to Y$ and $\phi: X_{1}\dashrightarrow X_{2}$ (if dvisorial contraction then $Y=X_{1}$, if small contraction then the flip) with
\begin{enumerate}
  \item $(X_{2},B_{2}+v_{1}H_{2})$ dlt
  \item $K_{X_{2}}+B_{2}+v_{1}H_{2}$ nef.
\end{enumerate}
Let 
\[
  v_{2}=\min \{t\geqslant 0: K_{X_{2}}+B_{2}+tH_{2} \text{nef over }S\} \leqslant v_{1}
\]
Then there is a sequence
\[
  (X_{1},B_{1})\dashrightarrow   (X_{2},B_{2})\dashrightarrow   (X_{3},B_{3})\dashrightarrow  \cdots\dashrightarrow  (X_{k},B_{k})\dashrightarrow 
\]
\begin{proof}[proof of claim]
  if $(X,B)$ is klt and $B$ ample effective. By cone theorem, there are finite $R$ such that $(K_{X}+B).R<0$. Then there is $R$ such that$(K_{X}+B+tH).R=0$
\end{proof}
Assume all $X_{i} \dashrightarrow X_{i+1}$ flip, then
\[
  N^1(X_{1}/S) \cong N^1(X_{i+1}/S)
\]
Then we can draw the nef cone of $\operatorname{Nef}(X_{i}/S)$ and $\operatorname{Nef}(X_{i+1}/S)$ in the same space $  N^1(X_{1}/S) \cong N^1(X_{i+1}/S)$ 
\section{ model}
\begin{defn}[minimal model]
  
$(X,B)\to S$ $\mathbb{Q}$-factorial dlt pair. Then $(Y,C)\to S$ is minimal model of $(X,B)\to S$ if there is birational map  $\phi:X \dashrightarrow  Y$ such that 
\begin{enumerate}
  \item $(Y,C)$ $\mathbb{Q}$-factorial dlt
  \item $\phi$ surjective in codimention one and $C=\phi_*B$
  \item $(K_{Y}+C)$ is nef over $S$ 
  \item common resolution such that 
    \[
      p^*(K_{X}+B)-q^*(K_{Y}+C)=E
    \]
    where $E$ effective and $\operatorname{Supp}E$ contains all divisors on $X$ contracted by $\phi$. Or, $\operatorname{Supp}p_*E= \cup (\text{divisors on} X \text{contracted by } \phi.)$   
\end{enumerate}
\end{defn}
Fact: if MMP terminates, then such $K_{X_{k}}+B_{k}$ is minimal model.
\begin{defn}[Canonical model]
  With same notation, $h$ is canoical model if  $K_{Y}+C$ is semiample. Equivalently, there is $h:Y\to Z$ and $H$ on $Z$ ample over $S$ and $K_{Y}+C=h^*H$
\end{defn}
remark: canonical model $Z$ is unique. Infact, 
\[
  Z=\operatorname{Proj}_{S}(R(Z,H))\cong \operatorname{Proj}_{S}(R(Y,K_{Y}+C))\cong\operatorname{Proj}_{S}(R(X,K_{X}+B))
\]
In particular, flip $(X^+,B^+)$ is an canonical model and minimal model.

We need existence of a special flip to show MMP, then MMP implies existence of al flip. 

If $K_{X}+B$ is psedueffective over  $S$, and such MMP terminates, then it ends with a minimal model.
\section{BCHM}
Setting \emph{\textbf{BCHM condition}: } 
\begin{enumerate}
  \item $X$ is $n$- dim $\mathbb{Q}$-factorial and $B$ is $\mathbb{R}$-Cartier and $f:X\to S$ projective morphism, and $S$ is quasi-projective
  \item $(X,B)$ dlt
  \item $B=  \left\lfloor B \right\rfloor +A+E$ where $A$ ample effective and $E$ effective. 

\end{enumerate}
we have klt BCHM condation
\begin{enumerate}
  \item $X$ is $n$- dim $\mathbb{Q}$-factorial and $B$ is $\mathbb{R}$-Cartier and $f:X\to S$ projective morphism, and $S$ is quasi-projective
  \item $(X,B)$ klt
  \item $B $ is big
\end{enumerate}
remark: if $X\to S$ is birational, then the condition holds.

\textbf{BCHM theorems}
\begin{enumerate}
  \item 
\begin{thm}[Existence of Flips]
 For a small contraction of  $K_{X}+B$, such flip exits.
\end{thm}
\item
\begin{thm}[Existence of PL flip]
  For a small contraction of  $K_{X}+B$, if  $- \left\lfloor B \right\rfloor $ ample over $S$, then flip exits
\end{thm}
\item 
\begin{thm}[Existence of Minimal models]
  if $K_{X}
  B$ is psedueffective over $S$, then there exits miminal model of $(X,B)/S$ .
\end{thm}
\item 
\begin{thm}[Finiteness of Minimal models]
  Suppose $\mathcal{P} \subset \operatorname{WDiv}_{\mathbb{R}}(X)$ is polytope spaned by effecitive divisors such that the BCHM conditiaion holds for any $B \in \mathcal{P}$, then there are finitely many rational maps $X \dashrightarrow Y_{k}$ such that  any minimal model of $(X,B)$ for any $B \in \mathcal{P}$ is isomorpic to  $Y_{k}$ for some $k$.
\end{thm}
\item
\begin{thm}[Termination of MMP with scaling]
  Assume  $f:(X,B')\to S$ and $f:(X,B)\to S$ with BCHM conditiaion and $B'\geqslant B$ and  $K_{X}+B'$ is nef over $S$, then any  $(K_{X}+B)$-MMP with scaling of $B'-B$ terminates.
\end{thm}
\item 
\begin{thm}[Special termination of MMP with scaling]
  In above theorem setting, any $(K_{X}+B)$-MMP with scaling of  $B'-B$ is isomorphism along  $\left\lfloor B \right\rfloor $ after finitely many steps.
\end{thm}
\item
\begin{thm}[Nonvanshing]
  If $K_{X}+B$ is psedueffective over $S$ and BCHM, there exits $D$ effective $\mathbb{R}$-divisor such that 
  \[
    D\equiv_{S}K_{X}+B
  \]
\end{thm}
\end{enumerate}
Clearly
\begin{itemize}
  \item 1 implies 2
  \item 3 implies 1,7
  \item 5 implies 3,6
\end{itemize}
\section{Sketch}
\begin{enumerate}[a)]
  \item Exist of flip $_n$ + Termination $_{n-1}$ $\implies$ Specila termination $_n$ 
  \item Exit of PL n + SpT n + NV $(X,B)$ n  $\implies$ Exit of minimal model for $(X,B)$ n
  \item Exist PL n +SpT n $\implies$ Exit of flip
  \item exit and finite minimal model n $\implies$ Exit of PL flip n 
  \item Exit of minimal model n $\implies$ Finite of minimal model n 
  \item Finite minimal model n $\implies$ Termination n 
  \item Exit of PL n + SpT n + Exitence of minimal model for $K_{X}+B$ big over $S$ n $\implies$ NV n 
\end{enumerate}
\textbf{Remark}:$d)$ is the begining
\end{document}
