%!TeX program = xelatex
\documentclass{article}

\usepackage{amsfonts}
\usepackage[all]{xy}
\usepackage{amssymb}
\usepackage{amsmath}
\usepackage{mathrsfs}
\usepackage{amsthm}
\usepackage{enumerate}
\usepackage[hidelinks]{hyperref}
\usepackage{ulem}
\usepackage{tikz}  
\usetikzlibrary{arrows.meta}%画箭头用的包

\usepackage{geometry}
\geometry{a4paper,left=2cm,right=2cm,top=2cm,bottom=2cm}

\newtheorem{defn}{Definition}[section]
\newtheorem{prop}[defn]{Proposition}
\newtheorem{lem}[defn]{Lemma}
\newtheorem{thm}[defn]{Theorem}
\newtheorem{cor}[defn]{Corollary}
\newtheorem{rmk}[defn]{Remark}
\newtheorem{fact}[defn]{Fact}
\newtheorem{problem}{Problem}
\newtheorem*{ques}{Question}

\setcounter{section}{0}

\title{Lecture 1: Introduce MMP}
\author{Jiang chen}
\date{\today}

\begin{document}
  \maketitle
  % \tableofcontents
  % \newpage

\begin{lem}
  $X ,Y$  birational smooth varieties, then 
  \[
    h^0(X,\omega ^n )=h^0(Y,\omega ^n)
  .\]
 
\end{lem}
\section{Higher dimension generaliation}
\begin{itemize}
  \item $(-1)$-curve $\to $ extrmal rays
  \item contraction of $(-1)$-curves  $\to $ contraction theorem.
\end{itemize}
\section{Singularity of pairs}
Everything over $\mathbb{C}$.

Let $X$ be a normnal variety. $B=\sum b_i B_{i}$ where $B_{i}$ are primes divisors, and $b_{i}\in \mathbb{R}_{geqslant 0}$. And $K_{X}+B$ is $\mathbb{R}$-Cartier. 

$\mathbb{R}$-Cartier means $K_{X}+B= \sum^{n}_{i=1} c_{i}D_{i}$ where $D_{i}$ cartier and $c_{i}\in \mathbb{R}_{geqslant 0}$. Here $=$ means as weil divisor or $\mathbb{R}$-linear equivalent

$f:Y\to X$ is log resolution of $(X,B)$ if $f$ is projective birational, $Y$ is smooth,$\operatorname{Exc}f$ is a diviosr and  $\operatorname{Exc}f \subset Y \cup  f^{-1}*B$ are SNC.     

\begin{defn}
  \[
    K_{Y}+C=f^*(K_{X}+B)
  .\]
 \begin{itemize}
   \item $(X,B)$ klt  if $C<1$  (here exit such $f$ implies for any $f$  )
   \item $(X,B)$ is dlt if $B\leqslant 1$ and $\exists f$ such that $C-f^{-1}_*<1$ 

 \end{itemize} 
\end{defn}
remark: klt $\implies$ dlt.

\section{cone of curves}
Let $f:X\to S$ be a projective surjective morphsim, and $X$ is normal.
\begin{defn}
  $N^1(X /S)$ is numerical equivalent classes of $\mathbb{R}$-Cartier $\mathbb{R}$-divisor.

  $N_1(X /S)= \operatorname{Span}_{\mathbb{R}}(curves) /\equiv$ 

\end{defn}
\begin{thm}[Klein]
 $D$ is ample over $S$ iff $(D.v)>0$ for all $v \in \overline{\operatorname{NE}} (X /S)-0$.    
\end{thm}
\begin{thm}[cone thm]
  $(X,B)$ is dlt, $f:X\to S$ projective, and $A$ ample over $S$, and $\epsilon>0$, then 
  \[
    \overline{\operatorname{NE}}(X /S) = \overline{\operatorname{NE}}(X /S_{K_{X}+B+\epsilon A\geqslant 0}) + \sum^{s}_{i=1} R_{i}
  .\]
 where $R_{i}$ finitely many extremal ray.

 Furthermore, any $R_{i}$ gives a contraction
 \[
   h_{i}:X\to Y_{i} \to S
 .\]
such that 
\begin{enumerate}
  \item $h_{i*} \mathcal{O}_{X}=\mathcal{O}_{Y_{i}}$
  \item $\rho(X /Y_{i})=1$.
  \item $-(K_{X}+B)$ is ample over $Y_{i}$ 
\end{enumerate}
\end{thm}
\section{MMP}
 Start from: $f:(X,B)\to S$ dlt pair and $\mathbb{Q}$-factorial.
 \begin{enumerate}
   \item if $K_{X}+B$ is nef, then we end.
   \item if $K_{X}+B$ is not nef, then there is a exremal ray $R$ , then $h:X\to Y$
     \begin{enumerate}[a]
       \item $h$ birational and contract a prime divisor, called divisorial contraction.
       \item $h$ birational and isomorphsim in codim $1$.
       \item $\dim X >\dim Y$
     \end{enumerate}
 \end{enumerate}
 For 3 contraction maps:
 \begin{enumerate}[a]
   \item  replace $X$  by $Y$
   \item  $K_{Y}+ h_*B $ is not $\mathbb{R}$-Cartier. Replace by flip.
   \item end.
 \end{enumerate}
 We have 2 questions:
 \begin{enumerate}
   \item What is a flip?
   \item When MMP ends?
 \end{enumerate}

\begin{defn}
  $f:X\to Y$ is small contraction with respect to $(K_{X}+B)$-negative ray, then $h^+ :X^+\to Y$ is a flip if 
  \begin{enumerate}
    \item $h^+$ is small;
    \item $K^{X^+}+B^+$ is ample over $Y$. 
  \end{enumerate}
\end{defn}
\begin{lem}
  If flip exits, then it is unique, and $(X^+,B^+)$ is stasifies start condition. Furthermore, if $B$ is $\mathbb{Q}$-divisor, then flip exits iff
  \[
   \mathcal{R}= \oplus h_*\mathcal{O}E_X(\lceil m(K_{X}+B)\rceil)
  .\]
 is finite generated $\mathcal{O}_Y$-algebra.
 In fact,
 \[
   X^+=\operatorname{Proj} \mathcal{R}
 .\]
 
\end{lem}
Need MMP to prove finitely generated; and inverse. Some of Main theorem of BCHM.
\begin{enumerate}
  \item flips
  \item FG of canonical ring.
\end{enumerate}
\end{document}
