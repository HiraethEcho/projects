\documentclass{article}

\usepackage{amsfonts}
\usepackage[all]{xy}
\usepackage{amssymb}
\usepackage{amsmath}
\usepackage{mathrsfs}
\usepackage{amsthm}
\usepackage{enumerate}
\usepackage[hidelinks]{hyperref}
\usepackage{ulem}

\newtheorem{defn}{Definition}[section]
\newtheorem{prop}[defn]{Proposition}
\newtheorem{lem}[defn]{Lemma}
\newtheorem{thm}[defn]{Theorem}
\newtheorem{cor}[defn]{Corollary}
\newtheorem{rmk}[defn]{Remark}
\newtheorem{fact}[defn]{Fact}
\newtheorem{problem}{Problem}
\newtheorem*{ques}{Question}

\title{NFI }
\author{wyz}
\date{}

\begin{document}

Proof of Theorem 2.10.1 (1): Assume any nbhd $ U $ of $ x $ intersects infinitely many $ H_j $, then there is a rational line in the smallest real linear space containing $ P $ passing through a sufficiently small nbhd $ U  $ of $ x $ with following property: $ L\cap U $  is an open subset of the rational closed interval $ [B,C]=L\cap P $, intersecting infinitely many $ H_j $ at distinct points.

By  shifting the coordinates WMA $ x=0\in \mathbb{R}^t=\left<B_1,\ldots,B_t\right> $. Take the norm as $ ||(a_1,\ldots,a_t)||=\sum |a_i| $. If there are infinitely many $ H_j $ passing through $ x $, then there are $ 1\leqslant i<j\leqslant t $ such that $ \{H_j\cap P_{i,j}\} $ is an infinite set, where $ L_{ij} $ is the plane generated by $ B_i,B_j $. Clearly there is a line $ L $ in $ P_{ij} $ not passing through $ x $, intersecting infinitely many $ H_j $  at distinct points. On the other hand, if there are only finitely many $ H_j $ passing through $ x $, claim that one of the coordinate axises satisfying the condition. Take $ 0<\epsilon_0\ll 1 $ such that $ U_0=U(x=0,\epsilon_0)\subset P $, by assumption there is a hyperplane $ H_j $ intersecting $ U_0 $  but not passing though $ x $. Thus $ H_j $ is defined as
$$ \sum\frac{x_i}{a_i}=1 $$
with $ |a_i|<\epsilon_0 $ for some $ i $. Take $ \epsilon_1=|a_i| $ and repeat the procedure, clearly that one of $ L_i $ satisfying the condition.

	
\end{document}