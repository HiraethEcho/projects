%!TeX program = xelatex
\documentclass{article}

\usepackage{amsfonts}
\usepackage[all]{xy}
\usepackage{amssymb}
\usepackage{amsmath}
\usepackage{mathrsfs}
\usepackage{amsthm}
\usepackage{enumerate}
\usepackage[hidelinks]{hyperref}
\usepackage{ulem}
\usepackage{tikz}
\usetikzlibrary{arrows.meta}%画箭头用的包

\usepackage{geometry}
\geometry{a4paper,left=2cm,right=2cm,top=2cm,bottom=2cm}

\newtheorem{defn}{Definition}[section]
\newtheorem{prop}[defn]{Proposition}
\newtheorem{lem}[defn]{Lemma}
\newtheorem{thm}[defn]{Theorem}
\newtheorem{cor}[defn]{Corollary}
\newtheorem{rmk}[defn]{Remark}
\newtheorem{fact}[defn]{Fact}
\newtheorem{problem}{Problem}
\newtheorem*{ques}{Question}

\setcounter{section}{0}

\title{Introduction to klt MMP}
\author{wyz}
\date{\today}

\begin{document}

\maketitle
%\tableofcontents
%\newpage
\section{Pairs}
\subsection{Divisors}
\subsubsection{Intersection}
\subsubsection{Cohomologies of divisors}
% TODO: weak R-R
\subsubsection{Cone of divisors}
% TODO: definitions of ample, nef, big;
% TODO: criterion for ampleness etc
% TODO: kodaira lemma etc
\subsection{Pairs}
\subsubsection{Resolution}

\begin{lem}[coefficients after blowup]
  Let $ (X,B=\sum_i a_i D_i) $ be a pair and $ X $ is smooth. Let $ Z\subset X $ be a smooth subvariety of codimension $ k $. Blowing up  $ X $ along the smooth center $ Z $ we obtain a birational proper map: $ p:Y=\mathrm{Bl}_ZX\to X $. Then $ E=\mathrm{Exc}\, p $ is an irreducible exceptional divisor and
  \[ a(E;X,B)=k-1-\sum_ia_i\mathrm{mult}_ZD_i \]
\end{lem}
\begin{proof}
  This is local, we may assume $ X=\mathbb{A}^n $, $ D_i $ is defined by $ (f_i),f_i\in \mathbb{C}[x_1,\ldots,x_n] $ and $ Z=V(x_1,\ldots,x_{k}) $. Then $ Y\hookrightarrow X\times \mathbb{P}^{k-1} $ is defined by $ \{ x_iy_j-x_jy_i\} $. Take an open subset
  \[ V=\{ y_1=1 \} =\mathrm{Spec}\frac{\mathbb{C}[x_1,\ldots,x_n,1,y_2,\ldots,y_k]}{\left <x_j-y_jx_1 \right >_{1\leqslant j\leqslant k}}\]
  of $ Y $, then
  \begin{equation*}
    \begin{aligned}
      p^*(dx_1\wedge\cdots\wedge dx_n) & =dx_1\wedge (d(y_2x_1)\wedge\cdots\wedge d(y_kx_1))\wedge dx_{k+1}\wedge\cdots\wedge dx_n \\
                                       & =x_1^{k-1}dx_1\wedge (dy_2\wedge\cdots\wedge y_k)\wedge dx_{k+1}\wedge\cdots\wedge dx_n
    \end{aligned}
  \end{equation*}
  And $ m_i=\mathrm{mult}_ZD_i $,
  \[
    f_i\in I^{m_i}-I^{m_i+1}, I=\left <x_1,\ldots,x_k\right >
  \]
\end{proof}

\begin{cor}[discrepancy by resolution]
  Given  $ X $ normal and boundary $ B=\sum_ia_iD_i $ with $ a_i\leqslant 1 $. Take a resolution $ f:Y\to X $ such that $ Y $ is smooth and $ f^{-1}_*D_i $ smooth, $ \sum f^{-1}_*D_i \cup E$ snc. Suppose $ a(E_j;X,B)\geqslant -1 $ for every irreducible exceptional divisor $ E_j $, then
  \[ \mathrm{discrep}\, (X,B)=\min\{\min\{a(E_j;X,B)\},\min\{1-a_i\},1\} \]
\end{cor}
\begin{proof}
  Denote $ K_Y+B_Y=f^*(K_X+B) $, and $ b_j=-a(E_j;X,B)\leqslant 1 $ is coefficient of $ E_j $ in $ B_Y $. Again we have
  \begin{equation*}
    \begin{aligned}
      \mathrm{discrep}\,(X,B) & =\min\{ \mathrm{discrep}\,(Y,B_Y), \min\{a(E_i,X,B)\}\} \\
                              & =\min\{ \mathrm{discrep}\,(Y,B), \min\{-b_j\}\}
    \end{aligned}
  \end{equation*}
  By proposition,  notice that $ B_Y $ has componets $ f^{-1}_*D_i $ and $ E_j $, then

  \[ \mathrm{discrep}\,(Y,B_Y)=\min \{ 1-a_i-b_j, 1-b_j-b_{j'}, 1-a_i, 1-b_j,1 \} \]
  There is no $ 1-a_i-a_{i'} $ since $ f^{-1}_*D_i $ and $ f^{-1}_*D_{i'} $ are disjoint. Furthermore, $ -b_j\leqslant 1-b_j-b_{j'} $ and $ -b_j\leqslant 1-b_j-a_i $, then
  \begin{equation*}
    \begin{aligned}
        & \mathrm{discrep}\, (X,B)                                                  \\
      = & \min\{\min\{a(E_j;X,B)\},\min\{1-a_i\},1\}                                \\
      = & \min\{ \min \{ 1-a_i-b_j, 1-b_j-b_{j'}, 1-a_i, 1-b_j,1 \}, \min\{-b_j\}\} \\
      = & \min\{  1-a_i,-b_j,1 \}                                                   \\
      = & \min\{\min\{a(E_j;X,B)\},\min\{1-a_i\},1\}
    \end{aligned}
  \end{equation*}

\end{proof}

\begin{lem}[resolution keep $H^0$]
  Let $ f:Y\to X $ be a proper birational morphism of varieties, and $ X $ a normal variety. Let $ D $ be a cartier divisor on $ X $, and $ F $ an exceptional effecitve cartier divisor on $ Y $, and $ E=\mathrm{Exc}(f) $, then 
  \[ H^0(X,D)\cong H^0(Y,f^*D+F) \] 
\end{lem}
\begin{proof}
  Since $ f $ is birational proper, then $  \underline{\mathrm{Spec}}f_*\mathcal{O}_Y\xrightarrow{\sim}X $, and $ f_*\mathcal{O}_Y\cong \mathcal{O}_X $. By projection formula, 
  \[ H^0(Y,f^*\mathcal{O}_X(D))=H^0\left(X,f_*\left(f^*\mathcal{O}_X(D)\otimes \mathcal{O}_Y\right)\right)=H^0(X,\mathcal{O}_X(D))\]
  Since $ F $ is effective,
  \[ H^0(Y,f^*D)\subset H^0(Y,f^*D+F) \]
  Since $ F $ is exceptional, i.e. $ F\subset E $,
  \[ H^0(Y,f^*D+F)\subset H^0(Y-E,f^*D) \]
  Notice that $ Y-E\cong X-f(E)  $,
  \[ H^0(Y-E,f^*D)= H^0(X-f(E),D) \]
  But $ X $ is normal, and $ \mathrm{codim} f(E)\geqslant 2 $,
  \[  H^0(X-f(E),D) =H^0(X,D) \]
  Now we have 
  \[ H^0(X,D) \subset H^0(Y,f^*D+F)\subset H^0(X,D) \]
\end{proof}

\subsubsection{Singularity}

\begin{defn}[singularities]
  ([KM98], Definition 2.34) Let $ (X,B) $ be a pair where $ X $ is a normal variety and $ B=\sum a_iD_i $ is sum of distinct prime divisors with rational coefficients. Assume $ (K_X+B) $ is $ \mathbb{Q} $-cartier, then we call $ (X,B) $ is 
  \begin{itemize}
    \item \textbf{terminal} if $ \mathrm{discrep}(X,B) >0$;
    \item \textbf{canonical} if $ \mathrm{discrep}(X,B) \geqslant 0$;
    \item \textbf{klt} if $ \mathrm{discrep}(X,B) >-1$ and $ a_i<1 $;
    \item \textbf{plt} if $ \mathrm{discrep}(X,B) >-1$;
    \item \textbf{lc} if $ \mathrm{discrep}(X,B) \geqslant -1 $. 
    \item If $ B=0 $, then klt, plt, dlt coincide, and is called \textbf{log terminal};
  \end{itemize}
\end{defn}

\section{X-method}

\section{Non-vanishing Theorem}

\begin{thm}
  Let $ X $ be a projective smooth variety of dimension $ n $ over $ \mathbb{C} $, and let  $ A $ be an ample divisor. Then
  \[ H^i(X,-A)=0,i<n \]
  or by serre duality, equivalent to
  \[ H^i(X,K_X+A)=0,i>0  \]
\end{thm}

Our main method to prove such theorem is to find another better variety $ f:Y\to X $ and line bundle $ A' $ on $ Y $, such that
\begin{itemize}
  \item $ H^i(Y,-A')=0,i<n  $ or $ H^i(Y,K_Y+A')=0,i>0  $;
  \item $ H^i(Y,-A')=0 \Rightarrow H^i(X,-A)=0,i<n  $  or
  
  $ H^i(Y,K_Y+A')=0 \Rightarrow  H^i(X,K_X+A)=0,i>0 $.
\end{itemize}

\section{Rationality Theorem}

\section{Base Point Free Theorem}

\section{Cone Theorem}

\section{MMP}

\subsection{Contraction}

\subsection{Flip}

\end{document}
