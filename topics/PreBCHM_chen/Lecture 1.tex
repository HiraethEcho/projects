 \documentclass{article}

\usepackage{amsfonts}
\usepackage[all]{xy}
\usepackage{amssymb}
\usepackage{amsmath}
\usepackage{mathrsfs}
\usepackage{amsthm}
\usepackage{enumerate}
\usepackage[hidelinks]{hyperref}
\usepackage{ulem}
\usepackage{longtable,booktabs,array}

\newtheorem{defn}{Definition}[section]
\newtheorem{prop}[defn]{Proposition}
\newtheorem{lem}[defn]{Lemma}
\newtheorem{thm}[defn]{Theorem}
\newtheorem{cor}[defn]{Corollary}
\newtheorem{rmk}[defn]{Remark}
\newtheorem{fact}[defn]{Fact}
\newtheorem{problem}{Problem}
\newtheorem*{ques}{Question}

\title{Notes of Seminar of MMP for surfaces}
\author{wyz}
\date{\today}

\begin{document}
	\maketitle
8.25

ref: Classification of Complex Algebraic Surfaces, Ciro Ciliberto
	
Let $ X $ be a irreducible projective smooth surface over $ \mathbb{C} $, and assume its canonical divisor $ K_X $ is not nef.
	
	
\section{Rationality theorem}

First we state a lemma about irrational numbers:
\begin{lem}
	If $ r>0 $ is a irrational number, then there are infinitely many pairs $ (u,v)\in \mathbb{Z} $ such that 
	$$ 0<\frac{v}{u}-r< \frac{1}{3u}$$
\end{lem}

\begin{thm}
	(Rationality theorem). Let $ A $ be an ample (cartier) divisor on X, define 
	$$ r_A:=\sup \{ r\in \mathbb{R}_{>0}:L_r=A+rK_X \text{ is nef  } \} $$
	$ r_A $ is called  the canonical threshold of $ A $. Then $ r_A\in \mathbb{Q} $, and $ L_{r_A} $ is nef but NOT ample.
\end{thm}

\begin{proof}
	For any $ 0<r<r_A $, the $ \mathbb{R} $-divisor $ L_r=A+rK_X $ is ample. Indeed, $ L_r=(r_A-r)A+L_{r_A} $, where $ (r_A-r)A $ is ample, and $ L_{r_A} $ is nef, hence $ L_r $ is ample. Since $ K_X $ is not nef, there is a curve $ C $, such that $ (C.K_X)<0 $. Therefore  $ (C.L_r)<0 $ for $ r\gg 0 $, hence $ L_r $ is not nef. This shows $ r_A<+\infty $.
	
	Consider a function $ P(x,y)=\chi (xA+yK_X), x,y\in \mathbb{Z} $. This is a polynomial in $ x,y $ of degree $ 2 $. Let $ P_{(u,v)}(k)=P(uk,vk) $, which is a polynomial in $ k $ of degree at most $ 2 $. $ P_{(u,v)}(k)\equiv0 $ if and only if the line $ uy-vx=0 $ contained in the zero locus $ P(x,y)=0 $. But $ P(x,y) $ is a polynomial of degree $ 2 $, so there are only finitely many pairs $ (u,v) $ such that $ P_{(u,v)}\equiv 0 $. 
	
	We prove the theorem by oantradiction: If $ r_A $ is irrational, by the lemma above, we can take such a pair $ (u_0,v_0) $ and $ P_{(u_0,v_0)}(k) $ is not identically $ 0 $. Take $ k_0\in \{1,2,3\} $ such that $ P_{(u_0,v_0)}(k_0)\neq 0  $. Set
	$$ M=k_0(u_0A+v_0K_X)=K_X+k_0u_0(A+\frac{k_0v_0-1}{k_0u_0}K_X) $$
	By the choice of $ (u_0,v_0) $ and $ k_0 $, $ 0<\frac{k_0v_0-1}{k_0u_0}<r_A $, therefore $ A+\frac{k_0v_0-1}{k_0u_0}K_X $ is an ample $ \mathbb{Q} $-cartier divisor, and $ k_0u_0(A+\frac{k_0v_0-1}{k_0u_0}K_X)  $ is an ample Cartier divisor. By Kodaira vanishing theorem, 
	$$\chi(M)=h^0(M)=P_{(u_0,v_0)}(k_0)\neq 0 $$
	By following lemma, this implies $ r_A\in \mathbb{Q} $, which is a contradicion.
	
	Assume $ L=L_{r_A} $ is ample, since it is a $ \mathbb{Q} $-Cartier divisor, there is  an integer $ m\gg0 $ such that $ mL-A $ is ample. Then 
	$$ (m-1)A+mr_AK_X=(m-1)(A+\frac{m}{m-1}r_AK_X) $$
	is ample. But $ \frac{m}{m-1}r_A>r_A $, this is contradict to the definition of $ r_A $, and implies that $ L $ is not ample.
\end{proof}


\begin{lem}
	If there is a rational number $ r_0>r_A $ such that $ L_{r_0}=A+r_0K_X $ is effective ( $ h^0(kL_{r_0})\neq 0 $ for some large integer $ k $), then $ r_A $ is rational.
\end{lem}
\begin{proof}
	If $ kL_{r_0} $ is an effective Cartier divisor, then we may assume $ kL_{r_0}\sim \sum_{i=1}^{n}d_iD_i  $, where $ D_i $ are irreducible curves on $ X $, and $ d_i>0 $. Then for $ 0<r\leqslant r_A< r_0 $,
	$$ L_r=A+rK_X=\frac{r_0-r}{r_0}A+\frac{r}{kr_0} \sum_{i=1}^{n}d_iD_i$$
	Thus $ L_r $ is nef if and only if  
	$$ (C.L_r)=\frac{r_0-r}{r_0}(C.A)+\frac{r}{kr_0} \sum_{i=1}^{n}d_i(C.D_i) \geqslant 0$$
	for all curves $ C $. Notice that $ A $ is ample, and $ D_i $ are irreducible curves, this fails only when  $ C=D_j $ for some $ j $. Set $ f_j(r):=(D_j.L_r)=ar+b $, by definition of $ r_A $ and $ r_0 $, $ a,b\in\mathbb{Q} $ and $  f_j(r)\geqslant0 $ for $ r\leqslant r_A $. If  $ f_j(r)<0 $ for some $ r>r_A $, then $ a<0 $. Let $ r_j $ be the solution of $ f_j(r)=0 $, then 
	$$ r_A=\min\{r_j>0\} $$
	Notice that $ r_0 $ is rational and there are only finitely many $ D_i $, $ r_A $ is rational.
\end{proof}

\section{Base point free thm}
Let $ X $ be a smooth projective surface. A \emph{pencil}  or a \emph{fibration}  on $ X $ is a surjiective morphism $ f:X\to C $ to a smooth irreducible curve $ C $ with connected fibres. Let $ F=\sum_{i=1}^{m}n_iF_i , n_i>0$ be a fibre of $ f $, and $ F_i $ are irreducible components of $ F $. Then $ F^2=0 $, and 
$$ n_j^2F_j^2=(n_jF_j.F-\sum_{i\neq j}n_iF_i)=-\sum_{i\neq j}n_in_j(F_i.F_j)<0 $$
$ \sum_{i\neq j}(F_i.F_j) >0$ since $ F $ is connected. 
\begin{lem}
	(Zarisiki lemma). Let $ f:X\to C $ be a pencil  and $ F $ is a fibre. Suppose $ D $ is a non-zero curve  in $ Div(X)\otimes \mathbb{Q} $ such that $ \mathrm{Supp}\,D \subset \mathrm{Supp}\,F  $. Then $ D^2\leqslant 0 $, and equality holds if and only if $ D=kF $ for some $ k\in \mathbb{Q} $.
\end{lem}
\begin{proof}
	Suppose $ F=\sum_{i=1} ^{m}n_iF_i, n_i\in \mathbb{Z}_{>0}$ and $ D=\sum_{i=1} ^{m}d_in_iF_i, d_i\in \mathbb{Q}_{\geqslant0} $. Let $ G_i=n_iF_i $, then
	\begin{equation*}
	\begin{aligned}
	D^2&=\sum_{i=1}^{m}d_i^2G_i^2+2\sum_{i<j}d_id_jG_iG_j=\sum_{i=1}^{m}d_i^2(G_i.F-\sum_{j\neq i}G_j)+ 2\sum_{i<j}d_id_jG_iG_j\\
	&=-\sum_{i<j}(d_i^2+d_j^2-2d_id_j)G_iG_j=-\sum_{i<j}(s_i-s_j)^2G_iG_j\leqslant0
	\end{aligned}
	\end{equation*}
	The equlity holds if and only if $ s_i=s_j $ for all $ G_i.G_j>0 $. But $ F $ is connected, hence $ s_i=s_j $ for all $ i,j $.
\end{proof}

\begin{thm}
	(Base point free theorem). Let $ L=A+r_AK_X=A+rK_X $ as in the rationality theorem. Then $ lL $ is base point free for $ l\in\mathbb{N}, l\gg 0, lr_A\in \mathbb{N} $.
\end{thm}
\begin{proof}
	We prove it in three cases: a: $  L^2>0$ ; b: $ L^2=0$ but $ L\not\equiv 0 $ and c: $ L\equiv0 $. Since $ r=\frac{p}{q} $ is rational, we can replace $ A $ by $ qA $, then $ r=p $ is an integer, $ L $ is a cartier divisor.
		
	\textbf{case a}: Since $ L^2>0 $ and $ L $ is nef but not ample, by criterion of ampleness, there is an irreducible curve $ E $ such that $ 0=(L.E)=A.E+rE.K_X>rK_X.E $. By Hodge index theorem, $ E^2L^2\leqslant (E.L)^2=0 $, and the equality holds only for $ L\equiv E $, but then $ 0=E^2=L^2>0 $,   hence $ E^2L^2< (E.L)^2=0 $ and $ E^2<0 $. Then $ \deg K_E =(K_X+E).E<0 $, therefore $ E\cong \mathbb{P}^1 $ and $ E^2=L.E=-1 $. Blowing down this curve via a birational morphism $ f:X\to Y $, then
	\begin{itemize}
		\item $ Y $ is a smooth projective surface, and $ f(E) $ is a point, $ \rho (Y)=\rho(X)-1 $;
		\item $ f^*K_Y=K_X-E, f_*K_X=K_Y $;
		\item $ L':=f_*L $ is a cartier divisor, since $ L.E=0 $. Suppose $ f^*L'=L +aE $, then $ 0=L'.f(E)=(L+aE).E=aE^2 $, therefore $ a=0,f^*L'=L $. Furthermore, $ L'  $ is nef: for any curve $ C' $ on $ Y $, let $ C $ be strict transform of $ C' $ on $ X $, then $ L'.C'=L.C\geqslant0 $.
		\item $ A':=f_*A=L'-rK_Y $ is a cartier divisor. $ A' $ is ample: indeed, 
		$$ f^*A'=f^*(L'-rK_Y)=L-r(K_X-E)=A+rE $$
		Any curve $ C' $ on $ Y $, take strict transform $ C $ on $ X $, then $ A'.C'=(A+rE).C>0 $; $ L.E=A.E+rK_X.E=A.E-r=0 $, $ A'^2=A'.f_*A=f^*A'.A=(A+E).A=A^2+r>0 $.
	\end{itemize}   
	If $ L' $ is ample, then $ lL' $ is base point free, and so is $ lL=f^*lL' $; If $ L' $ is nef but not ample, then replace $ X,A,L $ by $ Y,A',L' $ and repeat the agrument. Since picard number goes down, this ends in finitely many steps.
	
	\textbf{case b}: $ L=A+rK_X\equiv 0 $, then $ -K_X\equiv 1/rA $ is ample, and so is $ lL-K_X $ for all integers $ l $. By Riemann-Roch:
	$$ \chi (lL)=\chi+\frac{1}{2}lL.(lL-K_X)=\chi $$
	By Kodaira vanishing, $ lL-K_X $ ample $ \Rightarrow h^i(lL)=0,i>0$; $ K_X $ ample $ \Rightarrow h^i(K_X-K_X)=0,i>0$. Hence $ \chi(lL)=h^0(lL)=\chi=h^0(\mathcal{O}_X)=1 $. This shows $ lL $ is effective, corresponding to an effective $ C $. But $ L\equiv 0 $, for any very ample divisor $ H $ we have $ C.H>0 $ unless $ C=0 $, therefore $ lL=\mathcal{O}_X $ is base point free. 
	
	\textbf{case c}: $ L^2=0$ but $ L\not\equiv 0 $. Take an irreducibe curve $ C $ such that $ L.C>0 $. First we show that $ K_X.L<0 $. Indeed, take $ h\gg0 $ such that $ hA-C $ is effective, then 
	$$ L.hA=L.(hA-C)+L.C\geqslant L.C>0 $$
	so that $ L.A>0 $, but then
	$$ 0=L^2=L.A+rL.K_X>rL.K_X $$
	therefore $ L.K_X<0 $.
	
	Notice that $ lL-K_X=\frac{1}{r}A+\frac{lr-1}{r}L $, which is ample for $ l\gg 0 $, then by Kodaira vanishing and Riemann-Roch
	$$ h^0(lL)=\chi(lL)=\chi+\frac{1}{2}L.(L-K_X)=-\frac{L.K_X}{2}l+\chi>0 $$
	$ lL $ is effective, hence we can write $ |lL|=|M|+F $ where effective divisor  $ F $ is the fixed part and $ |M| $ is movable part ( this means a bijection: $ (D\in |lL|)\mapsto D-F\in |M| $). Claim that $ M $ is nef: assume $ M=\sum_{i=1}^{k}m_iM_i, m_i>0 $, if there is an irreducible curve $ C $ such that $ C.M<0 $, $ C $ must be one of $ M_i $. But $ M $ is the movable part, one can find $ M'\in |M| $ and $ C $ is not a componet of $ M' $, then $ C.M'\geqslant 0 $, therefore $ M $ is nef. Then
	$$ 0\leqslant M^2<M.(M+F)=M.L\leqslant(M+F).L=L^2=0  $$
	implies $ M^2=M.F=F^2=0 $. Notice that $ M^2=0 $ and $ M $ nef, then $ M $ is base point free. Otherwise, there is a point $ x\in X $ such that $ x\in M'  $ for any $ M'\in |M| $. Take $ M'\in |M| $ such that $ \dim M\cap M'=0 $, then $ 0=M^2=M.M'\geqslant\#M\cap M'\geqslant1 $.
	
	$ |M| $ induces a moprhism $ X$ to the projective space $ |M| $. Let $f:X\to Y $ be the closed image,  then there is an ample divisor $ \mathcal{O}_{|M|}(1)|_Y=N $ on $ Y $ such that $ f^*N=M $. Furthermore, $ h^0(X,alL)=h^0(X,aM)=h^0(Y,aN)=-\frac{lL.K_X}{2}a+\chi $ is a polynomial of degree $ 1 $ in $ a $, hence $ \dim Y=1 $.	By Stein factorization theorem, there are  maps $ f':X\to Y' $ and $ g:Y'\to Y $, where $ f'_*\mathcal{O}_X=\mathcal{O}_{Y'} $ and $ Y'\to Y $ is finite. Furthermore, $ Y' $ is normal, hence is smooth.	Replace $ Y $ by $ Y' $, then $ N=f_*M $ is an ample divisor of form $ \sum_{i=1}^{r}n_iy_i, n_i>0 $, i.e. sum of fibres of the pencile. Suppose $ F=\sum f_iF_i, f_i>0 $, if $ F_i $ contained in some fibre of the pencil, then $ M.F=0 $ ($ M $ is sum of fibres); if $ F_i $ is not contained in a fibre, then $ f(F_i)=Y $ and $ F_i.M>0 $. But $ M.F=0 $, this shows $ F=\sum_{j}G_j $ cotained in the union of fibres, such that $ f(G_j)=f(\sum g_{j,k}G_k)=y_j $ for distinct $ y_j\in Y $. Notice that $ G_i.G_j=0, i\neq j $ and $ F^2=0 $, we have $ G_j^2=0 $. By zarisiki lemma, $ G_j=g_jf^{-1}(y_j), g_j>0 $. Therefore, 
	$$ lL=M+F=f^*N+\sum g_jf^*y_j=f^*(N+\sum g_jy_j) $$
	is pull back of an ample divisor on $ Y $, hence is base point free. 
\end{proof}
\end{document}