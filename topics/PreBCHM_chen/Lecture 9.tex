\documentclass{article}

\usepackage{amsfonts}
\usepackage[all]{xy}
\usepackage{amssymb}
\usepackage{amsmath}
\usepackage{mathrsfs}
\usepackage{amsthm}
\usepackage{enumerate}
\usepackage[hidelinks]{hyperref}
\usepackage{ulem}

\newtheorem{defn}{Definition}[section]
\newtheorem{prop}[defn]{Proposition}
\newtheorem{lem}[defn]{Lemma}
\newtheorem{thm}[defn]{Theorem}
\newtheorem{cor}[defn]{Corollary}
\newtheorem{rmk}[defn]{Remark}
\newtheorem{fact}[defn]{Fact}
\newtheorem{problem}{Problem}
\newtheorem*{ques}{Question}

\title{Notes of Seminar of MMP for surfaces}
\author{wyz}
\date{}

\begin{document}
	\maketitle
10.18 Cone theorem, Higher dimensional geometry

First something about contraction:


restatement
\begin{thm}
	Let $ (X,B) $ be a klt pair and $ B\geqslant 0 $, assume $ K=K_X+B $ is not ample. Then
	\begin{enumerate}[(1)]
		\item $ \overline{NE}(X)=\overline{NE}(X)_{K\geqslant0}+\sum R_i $ where $ R_i $ run over all $ K $-negative extremal rays of form $ R_i=F_{D_i} $ with $ D_i $ nef cartier divisor. Furthermore, $ R_i $ is generated by a curve $ C_i $ and every $ K $-negative extremal ray is one of $ R_i $;
		
		\item For any ample cartier divisor $ H $ and $ \epsilon>0 $, denote $ K'=K+\epsilon H $, then 
		$$  \overline{NE}(X)=\overline{NE}(X)_{K'\geqslant0}+\sum_{finite}R_j  $$
		where $ R_j $ is $ K' $-negative (therefore there are at most countablely many extremal rays $ R_i $, and $ \{R_i\} $ is discrete in  $ \overline{NE}(X)_{K\leqslant0}$ );
		\item For any $ K $-negative extremal ray $ R $, there is a contraction $ g_R $ correponding to $ R $. Let $ C $ be a curve on $ X $, then $ g_R(C)=pt $ if and only if $ [C]\in R $. 
		\item Let $ g_R:X\to Y $ be a contraction as above, and $ L $ be a cartier divisor on $ X $. If $ L.R=0 $, then there is a cartier divisor $ L_Y $ on $ Y $ such that $ g_R^*L_Y=L $.
	
	\end{enumerate}
\end{thm}

\begin{rmk}
	\begin{enumerate}
		\item $ R_i $ is generated by a rational curve $ C_i $  satisfying 
		$$ 0<-K.C_i\leqslant2\dim X $$
		\item For all $ K $-negative extremal face $ F $ there is a corresponding contraction $ g_F $. This contraction deponds only on the face $ F $.
	\end{enumerate}
\end{rmk}
\begin{proof}
	We will prove the theorem using following claims:
	\begin{enumerate}[(a)]
		\item Let $ V=\overline{NE}(X)_{K\geqslant0}+\sum R_i $, then $$ \bar{V}=\overline{NE}(X) $$
		\item For any $ K'=K+\epsilon H $, only finitely many $ R_i $ are $ K' $-negative;
		\item $ V $ is closed, thus $ V=\bar{V}=\overline{NE}(X) $ 
	\end{enumerate}

\textbf{Claim (a)}: If $ \bar{V} \neq \overline{NE}(X) $, then there is a cartier divisor $ G $ such that 
$$ G(\bar{V}-0)>0,G(x)<0 $$
for some $ 0\neq x\in\overline{NE}(X)-\bar{V}  $.

blabla

$ D=G-tK $ is  a nef $ \mathbb{Q} $-cartier divisor and $ F_D $ is a $ K $-negative extremal face. If $ \dim F_D>1 $, fix an ample cartier divisor $ H $, and consider the nef threshold for $ mD+H $:
$$ r_m=\max \{r :  mD+H+rK \text{ is nef}  \} $$
 Take $ z\in F_D $ and $ H.z=1 $, since $ mD+H+r_mK $ is nef, 
$ (mD+H+r_mK).z=1+r_mK.z\geqslant 0 $, thus $ r_m\leqslant \frac{1}{-K.z} $ is bounded above. On the other hand, if $ m'>m $, then $ m'D+H+r_mK=(m'-m)D+mD+H+r_mK $ is nef, thus $ r_{m'}\geqslant r_m $. Therefore, $ \lim_{m\to \infty}r_m=r $ exsits. Furthermore, by rationality theorem, $ r_m=\frac{a}{b} $ is rational and has bounded denomitar $ b_0 $, then $ \{  r_m\}\subset \frac{1}{b_0!}\mathbb{N} $ is discrete, thus there is an integer $ m_0 $ such that for all $ m>m_0 $, $ r_m=r_{m_0}=r $. We may choose $ H $ that is linearly independent to  $ K $ as vectors in $ \mathrm{Hom}(\mathrm{Span}\,F_D, \mathbb{R}) $. Let $ D'=D+mD+H+rK $ be a nef divisor for $ m>m_0 $, $ D' $ is not ample, thus $ F_{D'}\neq 0 $. Notice that $ mD+H+rK $ is also nef,  if $ D'.z=0 $, then $ D.z=0 $, therefore $ F_{D'}\subset F_{D} $. By the choice of $ H $, $  F_{D'}\neq F_{D} $, hence $ \dim F_{D'}< \dim F_{D} $. Inductively, we can find a nef $ \mathbb{Q} $- cartier divisor $ D'' $ such that $ R= F_{D''} $ is an extremal ray, therefore $ R\subset V $. But $ G(V)>0 $ and $ G(R)<0 $, this contradiction implies the \textbf{Claim (a)}.

Let $ D $ be a nef $ \mathbb{Q} $-cartier divisor with $ R=F_D $ being an $ K $-negative extremal ray. $ D $ is of form $ D=H-\frac{1}{a}K $ where $ H $ is an ample divisor and $ \frac{1}{a}>0 $ is rational, then $ aD-K=aH $ is ample. By base point free theorem, $ bD $ is base point free and gives a contraction $ g_R:X\to Y $ with $ bD=g_R^*A_Y $ for some ample divisor $ A_Y $ on $ Y $. Since $ D $ is nef but not ample, $ g_R $ is not an isomorphism, thus there is a curve $ C $ on $ X $ contracted by $ g_R $. Then $ bD.C=A_Y.g_R(C)=0 $ implies $ [C]\in R $, and $ R=\mathbb{R}_{\geqslant0}[C] $. This shows \textbf{(3)}


\textbf{Claim (b)}: Let $ \{R_j=F_{D_j}\} $ be all $ K' $-negative extremal ray. Fix an ample divisor $ A_i $, as above, $ D_j'=mD_j+A_i+r_jK $ is a nef $ \mathbb{Q} $-cartier divisor and $ F_{D'_j}=F_{D_j}=R_j $. Let $ N:=\{z\in \overline{NE}(X):z.A_i=1\} $, which is a compact subset, and  $ M= \{z\in \overline{NE}(X)_{K'\leqslant0}:z.A_i=1\}$ is a closed compact subset of $ N $, and $ K(M)<0 $. Notice that $ K(M) $ is bunded, for any $ z_j\in R_j\cap N $,
	$$ 0=z_j.D_j'=1+r_jK.z_j $$
 Thus $ r_j=\frac{1}{-K.z_j} $ is bounded. By rationality theorem, $ r_j $ is rational with bounded denominator, thus $ \{r_j\} $ is a finite set. Take $ x_j\in R_j $ with $ K.x_j=-1 $, then $ D_j'.x_j =0$ implies $ A_i.x_j=r_j $.  Let $ A_1,\ldots, A_{\rho(X)} $ be ample divisors forming basis of $ N^1(X) $, then $ (A_i(z)) $ is a coordinate of $ z $. But each coordinate has only finitely many possiblities, $ \{x_j\} $ is finite. This shows \textbf{Claim (b)}.
 
 \textbf{Claim (c)}: Since a convex closed cone containing no line is the convex hull of all its extremal rays, only need to show any $ K $-negative ray of $ \overline{NE}(X) $ is one of $ R_i $. Indeed, for any  such ray $ R $, assume $ R=\mathbb{R}_{\geqslant0}z $, there is $ K'=K+\epsilon H $ such that $ K'(z)<0 $.
 Since 
 $$  \overline{NE}(X)=\overline{NE}(X)_{K'\geqslant0}+\sum_{finite}R_j  $$
 we may assume $ z=\lim_iz_i $ and $ z_i=r_i+\sum s_i $, where $ r_i\in \overline{NE}(X)_{K'\geqslant0} $ and $ s_i\in \sum_{finite}R_j $. By taking subsequence, we may assume $ \lim r_i=r\in \overline{NE}(X)_{K'\geqslant0} $ and $ \lim s_i=s \in \sum_{finite}R_j$ ( we may assume all $ z_i,r_i,s_i $ are in a nbhd of $ z $ ). Since $ R $ is extremal, $ r_i,s_i\in R $ . But $ K'(\overline{NE}(X)_{K'\geqslant0})\geqslant 0 $ and $ K'(R-0)<0 $, this shows $ r=0 $. Therefore $ R $ is one of $ R_j $. This implies \textbf{Claim (c)}.
 Fow now we've shown \textbf{(1)(2)(3)}.
 
 \textbf{(4)}: Assume $ R=F_D $ and $ g_R $ is induced by $ bD $. By base point free theorem, $ bD $ and $ (b+1)D $ are both base point free.
 
  (Let $ S=\bigoplus_mH^0(X,mD) $ and $ Z=\mathrm{Proj}S $. Suppose $ f_b:X\to \mathrm{Proj}\,S^{(b)}=Z_b $ ( $ f_{b+1}:X\to \mathrm{Proj}\,S^{(b+1)}=Z_{b+1} $) is induced by $ bD $($ (b+1)D $), in fact  $ Z\cong Z_b\cong Z_{b+1} $. Thus by stein factorization,) 
  
  They give same contraction $ g_R $, and $ g_R^*A_b=bD,g_R^*A_{b+1}=(b+1)D $, thus $ g_R^*(A_{b+1}-A_b)=D $. For any other divisor $ L $ with $ L.R=0 $, let
   $$ W_R=\overline{NE}(X)=\overline{NE}(X)_{K\geqslant0}+\sum_{R_i\neq R} R_i \subsetneq\overline{NE}(X) $$
    Since $ D $ is positive on $ W_R-0 $, we can find $ a\gg0 $ such that $ D'=aD+L $ is positive on $ W_R-0 $ . Thus for any extremal ray $ R'\neq R $ , $ D'(R'-0)>0 $, therefore $ D'(\overline{NE}(X))\geqslant0 $ is nef. Let $ A' $ be a cartier divisor on $ Y $ and $ g_R^*A'=D' $, then $ L=g_R^*(D'-a(A_{b+1}-A_b))) $ .
\end{proof}


\begin{rmk}
	uniqueness of contraction:([D],Prop 1.14, lemma 1.15) Let $ X,Y,Z $ be projective varieties, $ f:X\to Y $ and $ g:X\to Z $ are proper morphisms. 
	\begin{enumerate}
		\item $ \overline{NE}(f) $ is extremal subcone of $ \overline{NE}(X) $;
		\item Suppose $ f_*\mathcal{O}_X=\mathcal{O}_Y $, and $ \overline{NE}(f)\subset \overline{NE}(g) $, then there is a unique morphism $ h:Y\to Z $ such that $ g=h\circ f $
	\end{enumerate}

$$ \xymatrix{
	X\ar[d]_{f}\ar[r]^g&Z\\
	Y\ar@{.>}[ur]_{\exists !h}} $$
\end{rmk}

By the remark, the contraction $ g_F $ assoicated to extremal face $ F $ is unique up to an isomorphism.
\end{document}