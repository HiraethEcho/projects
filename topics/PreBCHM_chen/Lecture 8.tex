\documentclass{article}

\usepackage{amsfonts}
\usepackage[all]{xy}
\usepackage{amssymb}
\usepackage{amsmath}
\usepackage{mathrsfs}
\usepackage{amsthm}
\usepackage{enumerate}
\usepackage[hidelinks]{hyperref}
\usepackage{ulem}

\newtheorem{defn}{Definition}[section]
\newtheorem{prop}[defn]{Proposition}
\newtheorem{lem}[defn]{Lemma}
\newtheorem{thm}[defn]{Theorem}
\newtheorem{cor}[defn]{Corollary}
\newtheorem{rmk}[defn]{Remark}
\newtheorem{fact}[defn]{Fact}
\newtheorem{problem}{Problem}
\newtheorem*{ques}{Question}

\newcommand{\Spec}{\mathrm{Spec}\,}
\newcommand{\Proj}{\mathrm{Proj}\,}
\newcommand{\Hom}{\mathrm{Hom}\,}
\newcommand{\NE}{\mathrm{NE}}
\newcommand{\isoto}{\xrightarrow{\sim}}

\title{Notes of Seminar of MMP for surfaces}
\author{wyz}
\date{}

\begin{document}
	\maketitle
10.8 Cone theorem, Lectures on birational geometry, Theorem 7.1
\section{part of proof}
%10.8 start wyz
\begin{lem}
	$ K=K_X+B $ not nef, $ M $ is closure of $  N_{K\geqslant 0}+\sum R $, where $ R $ are extremal rays. Assume $ M\subsetneq N $, and $ 0\neq c\in N-M $, and $ G $ divisor such that $ G(N)>0,G(c)<0 $. Then there is a rational number $ a>0 $ such that $ G-aK $ is ample.
\end{lem}

\begin{proof}
	Let $ H $ be an ample divisor, consider $ N'=N\cap \{H=1\} $ and $ M'=M\cap \{H=1\} $, both closed bounded compact subset, and we can assume $ H.x=1 $. 
	\begin{itemize}
		\item Consider $ L_t=G-tK, t>0 $, claim that for $ t\gg 0 $, $ L_t $ ample on $ N_{K\leqslant 0} $, i.e. $ L_t(N'_2 \cup N'_3)>0 $. Indeed, 
		$$ N'=N'_{K\geqslant 0}\cup N'_{K\leqslant 0,G\geqslant 0}\cup N'_{G\leqslant 0}=N'_1\cup N'_2 \cup N'_3 $$
		Notice that $ M'\subsetneq N'_1\cup N'_2=N'_{G\geqslant 0} $ since $ G(M')>0 $. For all $ t>0 $, we have $ L_t(N'_2)\geqslant0 $. As for $ N'_3 $, since $ G:N_1(X)\to \mathbb{R} $ is continous, and $ N'_3 $ is compact, $ |G(N'_3)|\leqslant b $ is bounded. Same for $ K(N'_3)=[-c,-d] $, and $ d\geqslant 0 $. In fact, $ d>0 $. Otherwise, we have $ x'\in N'_3 $ and $ K(x')=0, G(x')\leqslant 0 $, then $ x'\in N'_1 $, but $ G(N'_1)>0 $, contradiction. Therefore, take $ t> \frac{b}{d} $, then $ L_t(N'_3)\geqslant -b+td>0 $.
		\item 	Consider threshold for nef on $ N_{K\leqslant 0} $:
		$$ s=\min \{t: L_t(N_{K\leqslant 0})\geqslant0  \} $$
		Denote $ D=L_s $, claim that $ D $ is positivite on $ N'_1 $. Indeed, by choice of $ s $, there is $ x'\in N'_{K\leqslant 0} $ such that $ D(x')=0 $ \textbf{Why?}. If $ K(x')=0 $, then $ x'\in M' $, thus $ G(x')>0 $, contradiction. Hence $ x'\in N'_{K<0} $. If $ D $ is not positivite on $ N'_1 $, then there is $ x''\in N_1 $ such that $ D(x'')\leqslant 0 $. If $ K(x'')=0 $, then $ G(x'')\leqslant 0 $, but $ G(x'')>0 $, hence $ K(x'')>0 $. Now $ D(x')=0,K(x')<0 $ and $ D(x'')\leqslant 0,K(x'')>0 $, then there is  a point $ x''' $ on the line $ [x',x''] $ such that  $ K(x''')=0,D(x''')\leqslant 0 $,  therefore $ G(x''')=0 $. However $ K(x''')=0 $ implies $ x'''\in M' $, and $ G(x''')>0 $, contradiction. Hence $ D $ is positivite on $ N'_1 $.
		\item $ D=G-sK $ is nef. Claim that for all $ 0<\epsilon \ll 1 $, $ D'=D-\epsilon K $ is ample. Need to show $ D'(N')>0 $. Since $ D $ is strictly positive on $ N'_1 $, we can take $ \epsilon $ small enough such that $ D' $ is positive on $ N'_1 $. As above, $ D' $ non-negative on $ N'_2 $, if $ D'(x_2)=G(x_2)-(\epsilon +s)K(x_2)=0$ for some $ x_2\in N'_2 $, then $ G(x_2)\geqslant 0,K(x_2)\leqslant 0 $ implies $ G(x_2)=K(x_2)=0 $. However $ K(x_2)=0 $ implies $ x_2\in M' $ and $ G(x_2)>0 $;  Similarily, $ K(N'_3)<0 $, therefore $ D'(N'_3)>0 $. Thus $ D' $ is ample. One can take $ \epsilon $ such that $ s+\epsilon $ rational, and $ H=D-\epsilon K $ is $ \mathbb{Q} $-cartier ample divisor.
		\item If consider nef threshold for $ H $,
		$$ \lambda=\max\{r: H-rK \text{ nef}\} $$
		by rational theorem, $ r=\epsilon $ and $ s $ is also rational.
	\end{itemize}
	
	
	
\end{proof}


(Following is part of proof of cone th and contraction th.)
\begin{prop}
	Let $ (X,B) $ be a projective klt divisor, $ K=K_X+B $ not nef.	Let $ D $ be a nef $ \mathbb{Q} $-cartier divisor, and $ F_D=\{F_D=0\}\cap \overline{NE(X)} $ is an extremay face. Let $ N=\overline{NE(X)} $. Claim that closure $ M $ of $ N_{K\geqslant 0}+\sum F_D $
	is $ N $, where $ F_D $ run over all nef $ D $ such that $ \dim F_D=1 $. 
\end{prop}
\begin{proof}
	Otherwise, there is a point $ c\in N $ and $ c\notin M  $. Assume $ G $ is a cartier divisor such that 
	$$ G(M-0)>0,G(c)<0 $$
	This is a separating hyperplane of $ M $ and $ c $. Claim: there is a rational number $ a>0 $ such that $ G-aK $ is ample.
\end{proof}
%10.8 end wyz



\section{application}
\begin{prop}
	([KM] Theorem 3.17) Let $ (X,B) $ be a projective klt pair, and $ R\subset \overline{NE}(X) $ be a $ (K_X+B) $-negative extremay ray. Let $ g_R:X\to Z $ be the corresponding contraction morphism. Let $ C $ be the curve on $ X $ generating $ R $, then we have an exact sequence
	$$ 0\to \mathrm{Pic}\,Z\xrightarrow{L\mapsto g_R^*L}\mathrm{Pic}\,X\xrightarrow{M\mapsto M.C} \mathbb{Z} $$
	Therefore we have $ \rho(X)=\rho(Z)+1 $. 
\end{prop}
\begin{lem}
	Let $ f:X\to Y $ be a birational morphism of norman varieties. Assume $ D_1,D_2 $ are  Cartier divisors on $ X $, linearly equivalent as cartier divisors. Then $ f_*D_1,f_*D_2 $ are weil divisors on $ Y $, equivalent as weil divisors, i.e. there is a rational functor $ h\in K(Y) $ such that $ (h)= f_*D_1-f_*D_2$.
\end{lem}

\begin{prop}
	Let $ g_R $ be as above. If $ X $ is $ \mathbb{Q} $-factorial and $ g_R $ is either divisorial or Fano, then $ Z $ is also   $\mathbb{Q} $-factorial .
\end{prop}
\end{document}