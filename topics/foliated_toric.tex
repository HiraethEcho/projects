\documentclass{article}

\usepackage{amsfonts}
\usepackage[all]{xy}
\usepackage{amssymb}
\usepackage{amsmath}
\usepackage{mathrsfs}
\usepackage{amsthm}
\usepackage{enumerate}
\usepackage[hidelinks]{hyperref}
\usepackage{ulem}
\usepackage{tikz}  

\usepackage{geometry}
\geometry{a4paper,left=2cm,right=2cm,top=2cm,bottom=2cm}

\newtheorem{definition}{Definition}[section]
\newtheorem{proposition}[definition]{Proposition}
\newtheorem{lemma}[definition]{Lemma}
\newtheorem{theorem}[definition]{Theorem}
\newtheorem{corollary}[definition]{Corollary}
\newtheorem{remark}[definition]{Remark}
\newtheorem{fact}[definition]{Fact}
\newtheorem{assertion}[definition]{Assertion}
\newtheorem{example}[definition]{Example}
\newtheorem{problem}{Problem}
\newtheorem*{ques}{Question}

\setcounter{section}{0}

\title{Foliated toric pairs}
\author{wyz}
%\date{\today}
\begin{document}
\maketitle
%\tableofcontents
%\newpage
\section{Introduction}
\subsection{toric foliation}
\subsection{Main results}
\section{Why toric foliation}
\subsection{Because we can}
\subsection{singularities}
\section{foliation on toric varieties}
\subsection{preliminaries}
\subsection{properties}
\section{Singularities}
\subsection{Description}
\subsection{Relation}
\subsection{Under MMP}
\section{Cone theorem}
\subsection{Extremal}
\subsection{Fibration}

\end{document}
