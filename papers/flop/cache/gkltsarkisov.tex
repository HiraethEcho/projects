\documentclass[11pt]{amsart}


\usepackage{amsfonts, amssymb, amscd}
\numberwithin{equation}{section}

\usepackage[symbol]{footmisc}
\renewcommand{\thefootnote}{\fnsymbol{footnote}}

\usepackage{todonotes}
 \newcommand\liu[1]{\todo[color=green!40]{#1}} %Liu
 \newcommand\liuinline[1]{\todo[inline,color=green!40]{#1}} %Liu inline
 
\usepackage{bm}
\usepackage{verbatim}
\usepackage{amssymb}
\usepackage{amsmath}
\usepackage{mathrsfs}
\usepackage{graphicx}
%\usepackage[nospace,noadjust]{cite}
%\usepackage[all]{xy}
\usepackage{tikz-cd}
\usepackage{subcaption}
\usepackage{listings}
\usepackage{subfiles}
\usepackage[toc,page]{appendix}
\usepackage{mathtools}
\usepackage{comment}
\usepackage{enumerate}
\usepackage{enumitem}
\usepackage[linesnumbered,ruled]{algorithm2e}
\usepackage[all]{xy}

\usepackage{graphicx}
\graphicspath{{images/}}

%\usepackage{fixltx2e}
\usepackage{appendix}
\usepackage{hyperref}
\lstset{
  basicstyle=\ttfamily,
  columns=fullflexible,
  frame=single,
  breaklines=true,
  postbreak=\mbox{\textcolor{red}{$\hookrightarrow$}\space},
}

\newcommand{\bQ}{\mathbb{Q}}
\newcommand{\bP}{\mathbb{P}}
\newcommand{\bA}{\mathbb{A}}
\newcommand{\cA}{\mathcal{A}}
\newcommand{\cO}{\mathcal{O}}
\newcommand{\oE}{\overline{E}}
\newcommand{\cF}{\mathcal{F}}
\newcommand{\bZ}{\mathbb{Z}}
\newcommand{\bb}{\bm{b}}
\newcommand{\oY}{\overline{Y}}
%\newcommand{\oE}{\overline{E}}
\newcommand{\oL}{\overline{L}}
\newcommand{\cI}{\mathcal{I}}
\newcommand{\ind}{\mathrm{ind}}
\newcommand{\Spec}{\mathrm{Spec}}
\newcommand{\id}{\mathrm{id}}



\newcommand{\Cc}{\mathbb{C}}
\newcommand{\KK}{\mathbb{K}}
\newcommand{\Pp}{\mathbb{P}}
\newcommand{\Qq}{\mathbb{Q}}
\newcommand{\QQ}{\mathbb{Q}}
\newcommand{\Rr}{\mathbb{R}}
\newcommand{\RR}{\mathbb{R}}
\newcommand{\Zz}{\mathbb{Z}}
\newcommand{\ZZ}{\mathbb{Z}}

\newcommand{\Nn}{\mathbb{N}}

\newcommand{\PP}{\mathbf{P}}

\newcommand\blfootnote[1]{%
  \begingroup
  \renewcommand\thefootnote{}\footnote{#1}%
  \addtocounter{footnote}{-1}%
  \endgroup
}

\newcommand{\zz}{\mathbf{z}}
\newcommand{\xx}{\mathbf{x}}
\newcommand{\yy}{\mathbf{y}}
\newcommand{\ww}{\mathbf{w}}
\newcommand{\vv}{\mathbf{v}}
\newcommand{\uu}{\mathbf{u}}
\newcommand{\kk}{\mathbf{k}}
\newcommand{\Span}{\operatorname{Span}}
\newcommand{\alct}{a\operatorname{lcT}}
\newcommand{\vol}{\operatorname{vol}}
\newcommand{\Center}{\operatorname{center}}
\newcommand{\Cone}{\operatorname{Cone}}
\newcommand{\Exc}{\operatorname{Exc}}
\newcommand{\Ext}{\operatorname{Ext}}
\newcommand{\Fr}{\operatorname{Fr}}
\newcommand{\glct}{\operatorname{glct}}
\newcommand{\HH}{\operatorname{H}}
\newcommand{\Hom}{\operatorname{Hom}}
\newcommand{\rk}{\operatorname{rank}}
\newcommand{\red}{\operatorname{red}}
\newcommand{\Ker}{\operatorname{Ker}}
\newcommand{\Ima}{\operatorname{Im}}
\newcommand{\mld}{{{mld}}}
\newcommand{\relin}{\operatorname{relin}}

\newcommand{\lcm}{\operatorname{lcm}}
\newcommand{\Weil}{\operatorname{Weil}}
\newcommand{\lct}{\operatorname{lct}}
\newcommand{\lcT}{\operatorname{lcT}}
\newcommand{\proj}{\operatorname{Proj}}
\newcommand{\spec}{\operatorname{Spec}}
\newcommand{\Supp}{\operatorname{Supp}}
\newcommand{\Diff}{\operatorname{Diff}}
\newcommand{\mult}{\operatorname{mult}}

\newcommand{\la}{\langle}
\newcommand{\ra}{\rangle}
\newcommand{\lf}{\lfloor}
\newcommand{\rf}{\rfloor}


\newcommand{\Aa}{\mathcal{A}}
\newcommand{\CC}{\mathcal{C}}
\newcommand{\Bb}{\mathcal{B}}
\newcommand{\Dd}{\mathcal{D}}
\newcommand{\Ff}{\mathcal{F}}
\newcommand{\Oo}{\mathcal{O}}
\newcommand{\Ii}{{\Gamma}}
\newcommand{\Jj}{\mathcal{J}}
\newcommand{\Ee}{\mathcal{E}}
\newcommand{\Hh}{\mathcal{H}}
\newcommand{\Ll}{\mathcal{L}}
\newcommand{\me}{\mathcal{E}}
\newcommand{\mo}{\mathcal{O}}
\newcommand{\nN}{\mathcal{N}}
\newcommand{\anN}{\mathcal{AN}}
\newcommand{\Tt}{\mathcal{T}}
\newcommand{\Ww}{\mathcal{W}}
\newcommand{\Xx}{\mathcal{X}}
\newcommand{\Ss}{\mathcal{S}}
\newcommand{\Yy}{\mathcal{Y}}


\newcommand{\BB}{\mathfrak{B}}

\newcommand{\NE}{\mathrm{NE}}
\newcommand{\Nef}{\mathrm{Nef}}
\newcommand{\Sing}{\mathrm{Sing}}
\newcommand{\Pic}{\mathrm{Pic}}
\newcommand{\reg}{\mathrm{reg}}
%\newcommand\mld{{{mld}}}
\newcommand\MLD{{{MLD}}}
\newcommand\Mod{{{mod}}}



%\theoremstyle{plain}
\newtheorem{thm}{Theorem}[section]
\newtheorem{conj}[thm]{Conjecture}
\newtheorem{cor}[thm]{Corollary}
\newtheorem{lem}[thm]{Lemma}
\newtheorem{prop}[thm]{Proposition}
\newtheorem{ques}[thm]{Question}
\newtheorem{claim}[thm]{Claim}
\theoremstyle{definition}
\newtheorem{rem}[thm]{Remark}
\newtheorem{remdef}[thm]{Remark-Definition}
\newtheorem{ex}[thm]{Example}
\newtheorem{idea}[thm]{Idea}
\newtheorem{exlem}[thm]{Example-Lemma}
\newtheorem{cons}[thm]{Construction}
\newtheorem{code}[thm]{Code}

\theoremstyle{remark}
\newtheorem{defn}[thm]{Definition}

\newtheorem{theorem}{Theorem}[section]
\newtheorem{lemma}[theorem]{Lemma}
\newtheorem{proposition}[theorem]{Proposition}
\newtheorem{corollary}[theorem]{Corollary}
\newtheorem*{notation}{Notation ($\star$)}

\theoremstyle{definition}
\newtheorem{definition}[theorem]{Definition}
\newtheorem{example}[theorem]{Example}
\newtheorem{question}[theorem]{Question}
\newtheorem{remark}[theorem]{Remark}
\newtheorem{conjecture}[theorem]{Conjecture}

\begin{document}


\title{Sarkisov program for generalized pairs}

\author{Jihao Liu}

%\author{Jihao Liu}
\address{Department of Mathematics, The University of Uath, Salt Lake City, UT 84112, USA}
\email{jliu@math.utah.edu}



\begin{abstract}
In this paper we show that any two birational Mori fiber spaces of $\Qq$-factorial gklt g-pairs are connected by a finite sequence of Sarkisov links.
\end{abstract}

\blfootnote{2020 \textit{Mathematics Subject Classification}. Primary 1430, Secondary 14B05.}


\maketitle
\pagestyle{myheadings}\markboth{\hfill  J.Liu \hfill}{\hfill Sarkisov program for generalized pairs\hfill}


\tableofcontents

\section{Introduction}
In this paper we work over the field of complex numbers $\mathbb C$.

The minimal model program is an attempt to classify higher dimensional projective varieties. Let $X$ be a smooth complex projective variety. There are two cases: if $K_X$ is pseudo-effective, i.e. $K_X$ belongs to the closure of the big cone of divisors on $X$, then it is expected that $X$ has a minimal model $X\dashrightarrow X_{min}$. In particular, $K_{X_{min}}$ has terminal singularities, $K_{X_{min}}$ is $\mathbb Q$-Cartier and nef. The existence of such minimal models is known when $K_X$ is big, (cf. \cite{BCHM10}).

Given smooth projective variety $X$ such that $K_X$ is pseudo-effective, it is well-known that $X$ may have more than one minimal model. Nevertheless, the birational map connecting these minimal models are well-studied. For example, Kawamata (cf. \cite{Kaw08}) shows the following: for any smooth projective variety $X$ and any minimal models $X_{min}$ and $X_{min}'$ of $X$, the induced birational map $X_{min}\dashrightarrow X_{min}'$ is a composition of flops. In particular, $X_{min}$ and $X_{min}'$ are isomorphic in codimension $1$.

If $K_X$ is not pseudo-effective, by \cite{BCHM10}, we may also run a $K_X$-MMP. It is not known whether this $K_X$-MMP terminates. However, by \cite[Corollary 1.3.2]{BCHM10}, if this $K_X$-MMP is a MMP with scaling of some ample divisor, then after a sequence of flips and divisorial contractions, it terminates with a Mori fiber space $X_m\rightarrow S$. It is also well-known that the Mori fiber spaces associated to $X$ may not be unique.

\begin{ques}\label{ques: relationship mfs}
Let $X$ be a smooth projective variety such that $K_X$ is not pseudo-effective, and $X_m\rightarrow S$ and $X_m'\rightarrow S'$ two Mori fiber spaces associated to $X$, i.e. there are two $K_X$-MMPs $X\dashrightarrow X_m$ and $X\dashrightarrow X_m'$ which terminates with $X_m\rightarrow S$ and $X_m'\rightarrow S'$ respectively. 

What can we say on the relationship between these two Mori fiber spaces?
\end{ques}

The \emph{Sarkisov program}, which was first introduced by Sarkisov in \cite{Sar80} and \cite{Sar82} in order to study the conic bundles of threefolds and was further developed and generalized by Corti (\cite{Cor95}), Bruno-Matsuki (\cite{BM97}) in order to study the minimal model program of klt pairs, gives us some good understanding on the question above. More precisely, the goal of the Sarkisov program is to decompose the natural birational map $X_m\dashrightarrow X_m'$ as in Question \ref{ques: relationship mfs} into four different types of birational maps, which are called the \emph{Sarkisov links}. This kind of decomposition is particularly useful when calculating the birational automorphism group of Mori fiber spaces of Fano varieties. 

Based on the minimal model program established in \cite{BCHM10}, in \cite{HM09}, Hacon and M\textsuperscript{c}Kernan have established the Sarkisov program for any klt pairs, i.e. they show the existence of such decomposition for any Sarkisov related pairs:

\begin{thm}[{\cite[Theorem 1.3]{HM09}}]\label{thm: existence sarkisov link} 
Let $(Z,\Phi)$ be a $\Qq$-factorial klt pair such that $K_X+\Phi$ is not pseudo-effective. Assume that $\phi: X\rightarrow S$ and $\psi: Y\rightarrow T$ are two Mori fiber spaces which are obtained by running two different $(K_Z+\Phi)$-MMPs. Then the induced birational map $\sigma: X\dashrightarrow Y$ is a composition of Sarkisov links.
\end{thm}

The concept of \emph{generalized pairs} was introduced in \cite{BZ16}, and it has become clear that the study of birational geometry in this category is important. For example, the most important application of generalized pairs appears in the proof of the BAB conjecture and the existence of $n$-complements (\cite{Bir16},~\cite{Bir19}). For simplicity, in this paper, a generalized pair is also called a \emph{g-pair}: see Definition \ref{defn: gpair} for more details.

It is a natural question to ask whether the theory of Sarkisov program holds for generalized pairs. The intuition of this question comes from the famous M$\textsuperscript{c}$Kernan-Shokurov conjecture (cf. \cite[Conjecture 1.2]{Bir16},\cite[Conjecture 1.7]{Bir18}):

\begin{conj}[M$\textsuperscript{c}$Kernan-Shokurov conjecture]\label{conj: ms conjecture}
Let $d>0$ be an integer and $\epsilon>0$ a real number. Then there is a real number $\delta>0$ depending only on $d$ and $\epsilon$ satisfying the following. Assume that $(X,B)$ is a pair and $X\rightarrow Z$ is a contraction, such that 
\begin{itemize}
    \item $(X,B)$ is $\epsilon$-lc of dimension $d$,
    \item $K_X+B\sim_{\Rr,Z}0$, and
    \item $-K_X$ is big$/Z$,
\end{itemize}
then the discriminant $\bb$-divisor $\textbf{\rm\textbf{B}}_Z$ has coefficients in $[-\infty,1-\delta]$.
\end{conj}

In the conjecture above, it was predicted that $(Z,B_Z+M_Z)$ given by the canonical bundle formula has the structure of a generalized pair. The case when $B$ is a $\Qq$-divisor is well-known, but when $B$ is an $\Rr$-divisor, this is only very recently proved by Han and Liu (cf. \cite[Corollary 1.2]{HL19}). Nevertheless, it is now natural to generalize Conjecture \ref{conj: ms conjecture} to the category of generalized pairs, which is only possible after the recent works on the generalized canonical bundle formula by Filipazzi (cf.~\cite[Theroem 1.4]{Fil18}) for generalized $\Qq$-pairs and Han-Liu (cf.~\cite[Corollary 1.2]{HL19}) for any case. Therefore, we have the following conjecture, whose $\Qq$-pair version is stated as in \cite[Conjecture 2.4]{Bir18}:

\begin{conj}\label{conj: generalized ms conjecture}
Let $d>0$ be an integer and $\epsilon>0$ a real number. Then there is a real number $\delta>0$ depending only on $d$ and $\epsilon$ satisfying the following. Assume that $f:X\rightarrow Z$ is a contraction and $(X,B+M_X)$ is a generalized pair$/Z$ with associated nef$/Z$ $\bb$-divisor $M$, such that 
\begin{itemize}
    \item $(X,B+M_X)$ is generalized $\epsilon$-lc of dimension $d$,
    \item $K_X+B+M_X\sim_{\Rr,Z}0$, and
    \item $-K_X$ is big$/Z$,
\end{itemize}
then the generalized pair given by the sub-adjunction
$$K_X+B+M_X\sim_{\Rr}f^*(K_Z+B_Z+M_Z)$$
is generalized $\delta$-lc.
\end{conj}

The most interesting cases of Conjecture \ref{conj: ms conjecture} and Conjecture \ref{conj: generalized ms conjecture} appear when $X\rightarrow Z$ has a Mori fiber space structure. Indeed, Conjecture \ref{conj: ms conjecture} was originally proposed by M\textsuperscript{c}Kernan when $B=0$ and $X\rightarrow Z$ is a $K_X$-Mori fiber space. In this case, we may use the Sarkisov program to attack the conjectures above in the following way:
\begin{enumerate}
    \item We prove Conjecture \ref{conj: ms conjecture} (or Conjecture \ref{conj: generalized ms conjecture}) for Mori fiber spaces with good properties, e.g. when $\dim Z=0$, or in the case of \cite[Theorem 1.8, 1.9]{Bir18}.
    \item We show that if one Mori fiber space satisfies Conjecture \ref{conj: ms conjecture} (or Conjecture \ref{conj: generalized ms conjecture}), then another Mori fiber space associated to a Sarkisov link also satisfies Conjecture \ref{conj: ms conjecture} (or Conjecture \ref{conj: generalized ms conjecture}).
    \item By applying the Sarkisov program, we show that all Mori fiber spaces that are birational to a Mori fiber spaces with good properties satisfy  Conjecture \ref{conj: ms conjecture} (or Conjecture \ref{conj: generalized ms conjecture})
\end{enumerate}


In this paper, we deal with a small part of the outline above, that is, we extend the theory of the Sarkisov program to the theory of generalized pairs:

\begin{thm}\label{thm: existence generalized Sarkisov link}
Assume that 
\begin{itemize}
    \item $W\rightarrow Z$ is a contraction between normal quasi-projective varieties,
    \item $(W,B_W+M_W)$ is a $\mathbb Q$-factorial gklt g-pair with associated nef$/Z$ $\bb$-divisor $M$, such that $K_W+B_W+M_W$ is not pseudo-effective$/Z$,
    \item $\rho_X: W\dashrightarrow X$ and $\rho_Y: W\dashrightarrow Y$ are two $(K_W+B_W+M_W)$-MMP$/Z$ such that $(\rho_X)_*(K_W+B_W+M_W)=K_X+B_X+M_X$ and $(\rho_Y)_*(K_W+B_W+M_W)=K_X+B_Y+M_Y$,
    \item $\phi_X: X\rightarrow S_X$ is a $(K_X+B_X+M_X)$-Mori fiber space$/Z$ and $\phi_Y: X\rightarrow S_Y$ is a $(K_Y+B_Y+M_Y)$-Mori fiber space$/Z$.
\end{itemize}
\begin{center}$\xymatrix{
 & W\ar@{-->}[dl]_{\rho_X}\ar@{-->}[dr]^{\rho_Y}& \\
      X \ar@{->}[d]_{\phi_X}\ar@{-->}[rr]^{f}   &  & Y\ar@{->}[d]^{\phi_Y} \\
    S_X & &S_Y }$
\end{center}
Then
\begin{enumerate}
    \item the induced birational map $f: X\dashrightarrow Y$ is given by a finite sequence of Sarkisov links$/Z$, i.e. $f$ can be written as $X_0\dashrightarrow X_1\dots\dashrightarrow X_n\cong Y$, where each $X_{i}\dashrightarrow X_{i+1}$ is a Sarkisov link$/Z$, and
    \item for any real number $\epsilon>0$, if $(W,B_W+M_W)$ is generalized $\epsilon$-lc, then $(X_i,B_i+M_{X_i})$ is generalized $\epsilon$-lc for every $i$, where each $B_i$ is the birational transform of $B_W$ on $X_i$.
\end{enumerate} 
\end{thm}

There are three additional ingredients in Theorem \ref{thm: existence generalized Sarkisov link} comparing to Theorem \ref{thm: existence sarkisov link}. Firstly, we consider \emph{gklt g-pairs} instead of klt pairs. Secondly, we indeed construct a \emph{relative Sarkisov program} since all the birational maps are over $Z$. Finally, we show that the generalized pairs constructed in the Sarkisov program do not have worse singularities comparing to $(W,B_W+M_W)$. We remark that the last two ingredients are also unknown for the klt pair case previously, and the third ingredient will be important when we apply Theorem \ref{thm: existence generalized Sarkisov link} to attack Conjecture \ref{conj: generalized ms conjecture} as in the outline above.



It is worth to mention that our proof of Theorem \ref{thm: existence generalized Sarkisov link} is quite close to the original ideas of the Sarkisov program in dimension $3$. Comparing to the proof of Hacon and M\textsuperscript{c}Kernan in \cite{HM09}, our proof has not used many detailed results on combinatorics, but also eseentially use the finiteness of weak log canonical models (Theorem \ref{thm: finiteness ltm}) and the uniqueness of log canonical models. We refer the readers to the proof of Theorem \ref{thm: existence sarkisov precise} for more details.

Indeed, the Sarkisov links constructed in our paper can be described very precisely, and can be viewed as a ``Sarkisov program with double scaling". We refer the readers to Section 4 for more details.

\medskip




\noindent\textit{Acknowledgement}. The author would like to thank Prof. Christopher D. Hacon for suggesting the question and his constant support, especially for teaching me the approaches to the Sarkisov program for pairs. He would like to thank Jingjun Han for his many helpful comments, especially for his effort during the winter of 2017--2018 when he read a very early draft of this paper. He would like to thank the anonymous referee(s) for carefully checking the details and many useful suggestions. The author was partially supported by NSF research grants no: DMS-1300750, DMS-1265285 and by a grant from the Simons Foundation (Award number: 256202). 


\section{Notation and conventions} 
We adopt the standard notation and definitions in \cite{Sho92} and \cite{KM98}, and will freely use them.

\begin{defn}[$\bb$-divisors] Let $X$ be a normal quasi-projective variety. A $\bb$-$\Rr$ Cartier $\bb$-divisor ($\bb$-divisor for short) over $X$ is the choice of a projective birational morphism $Y\to X$ from a normal quasi-projective variety $Y$ and an $\Rr$-Cartier $\mathbb R$-divisor $M$ on $Y$ up to the following equivalence: another projective birational morphism $Y'\to X$ from a normal quasi-projective variety and an $\Rr$-Cartier $\Rr$-divisor $M'$ defines the same $\bb$-divisor if there is a common resolution $W\to Y$ and $W\to Y'$ on which the pullback of $M$ and $M'$ coincide. If there is a choice of birational morphism $Y\rightarrow X$ such that the corresponding $\Rr$-Cartier $\Rr$-divisor $M$ is a prime divisor, the $\bb$-divisor is called \emph{prime}.

	Let $E$ be a prime $\bb$-divisor over $X$. The \emph{center} of $E$ on $X$ is the closure of its image on $X$, and is denoted by $c_X(E)$. If $c_X(E)$ is not a divisor, $E$ is called \emph{exceptional}$/X$. If $c_X(E)$ is a divisor, we say that $E$ is \emph{on} $X$. For any $\bb$-divisor $M=\sum a_iE_i$ over $X$, where $E_i$ are prime $\bb$-divisors over $X$, we define $M_X:=\sum a_ic_X(E_i)$ to be the $\Rr$-divisor where the sum is taken over all the prime $\bb$-divisors $E_i$ which are on $X$. If all the $E_i$ are on $X$, we say that $M$ is \emph{on} $X$. 
\end{defn}

\begin{defn}[Multiplicities] Let $X$ be a normal quasi-projective variety, $E$ a prime divisor on $X$ and $D$ an $\Rr$-divisor on $X$. We define $\mult_ED$ to be the multiplicity of $E$ along $D$. 
Let $F$ be a prime $\bb$-divisor over $X$, $B$ an $\Rr$-Cartier $\Rr$-divisor on $X$ and $\phi: Y\to X$ a birational morphism such that $F$ is on $Y$. We define $\mult_FD:=\mult_F\phi^*D$.
\end{defn}

\begin{defn}[Pairs]\label{defn: positivity}
	A \emph{pair} $(X,B)$ consists of a normal quasi-projective variety $X$ and an effective $\Rr$-divisor $B$ on $X$ such that $K_X+B$ is $\Rr$-Cartier.
	Let $\phi:W\to X$
	be any log resolution of $(X,B)$ and let
	$$K_W+B_W:=\phi^{*}(K_X+B).$$
	The \emph{log discrepancy} of a prime divisor $D$ on $W$ with respect to $(X,B)$ is $1-\mult_{D}B_W$ and is denoted by $a(D,X,B).$
	For any real number $\epsilon\geq 0$, we say that $(X,B)$ is \emph{lc} (resp. \emph{klt}, $\epsilon$-\emph{lc}) if $a(D,X,B)\ge0$ (resp. $>0$, $\ge\epsilon$) for every log resolution $\phi:W\to X$ as above and every prime divisor $D$ on $W$. 
	
	We say that $(X,B)$ is $\Qq$-factorial if every $\Qq$-divisor on $X$ is $\Qq$-Cartier. 
	
	For any prime $\bb$-divisor $E$ over $X$, let $Y\rightarrow X$ be a birational morphism such that $E_Y$ is a prime divisor. The \emph{log discrepancy} of $E$ with respect to $(X,B)$ is $a(E_Y,X,B)$. We say that $(X,B)$ is \emph{terminal} if $a(E,X,B)>1$ for every exceptional$/X$ prime $\bb$-divisor $E$. 
\end{defn}

\begin{defn} Let $f: X\dashrightarrow Y$ a birational map between normal quasi-projective varieties, $p: W\rightarrow X$ and $q: W\rightarrow Y$ a resolution of indeterminacy of $f$, and $D$ an $\Rr$-Cartier $\Rr$-divisor on $X$ such that $D_Y:=f_*D$ is an $\Rr$-Cartier $\Rr$-divisor on $Y$. $f$ is called \emph{$D$-non-positive} (resp. \emph{$D$-negative}), if
\begin{itemize}
    \item $f$ does not extract any divisor, and
    \item $p^*D=q^*D_Y+E$, where $E\geq 0$ is exceptional$/Y$ (resp. $E\geq 0$ is exceptional$/Y$, and $\Supp p_*E$ contains all $f$-exceptional divisors). 
\end{itemize}
\end{defn}

\begin{defn} Let $X$ be a normal quasi-projective variety. We define $\Weil_{\Rr}(X)$ to be the $\Rr$-vector space spanned by all Weil divisors on $X$. Let $\mathcal{V}$ be a finite dimensional subspace of $\Weil_{\Rr}(X)$ and $A\in\mathcal{V}$ an $\mathbb R$-divisor. We define 
$$\mathcal{L}_{A}(\mathcal{V}):=\{B\mid(X,B) \text{ is lc, } B=A+B', B'\geq 0, B'\in \mathcal{V}\}\subset\Weil_{\Rr}(X).$$
By \cite[Lemma 3.7.2]{BCHM10}, if $\mathcal{V}$ is a rational subspace, then $\mathcal{L}_{A}(\mathcal{V})$ is a rational polytope.
\end{defn}



\begin{defn}
A \emph{contraction} is a projective morphism $f:X\to Z$ between normal quasi-projective varieties such that $f_{*}\Oo_X=\Oo_Z$. 

For any $\bb$-divisor $M$ over $X$, $M$ is called nef$/Z$ if there is a projective morphism $Y\rightarrow X$ such that $M$ is on $Y$ and $M_Y$ is nef$/Z$.
\end{defn}


\begin{defn}\label{defn: gpair} A \emph{generalized pair} (\emph{g-pair} for short) consists of a normal quasi-projective variety $X$, an effective $\Rr$-divisor $B$ on $X$, a contraction $X\rightarrow Z$, and a $\bb$-divisor $M$ over $X$ such that $M$ is nef$/Z$. If there is no confusion, we usually say that $(X,B+M_X)$ is a generalized pair$/Z$. $M$ is called the \emph{associated nef$/Z$ $\bb$-divisor} of the generalized pair $(X,B+M_X)$. If $Z$ is not important, we may omit $Z$ and say that $(X,B+M_X)$ is a generalized pair.

Let $(X,B+M_X)$ be a generalized pair$/Z$ with associated nef$/Z$ $\bb$-divisor $M$. Let $\phi:W\to X$
	be a log resolution of $(X,B)$ such that $M_W=M$ (i.e. $M$ is the choice of $M_W$ and the morphism $\phi$) and
	$$K_W+B_W+M_W:=\phi^{*}(K_X+B+M_X).$$
	The \emph{generalized log discrepancy} of a prime divisor $D$ on $W$ with respect to $(X,B+M_X)$ is $1-\mult_{D}B_W$ and is denoted by $a(D,X,B+M_X).$ For any prime $\bb$-divisor $E$ over $X$, let $Y\rightarrow X$ be a birational morphism such that $E_Y$ is a prime divisor.  The \emph{generalized log discrepancy} of $E$ with respect to $(X,B+M_X)$ is $a(E_Y,X,B+M_X)$.
	For any real number $\epsilon\geq 0$, we say that
	\begin{itemize}
	    \item $(X,B+M_X)$ is \emph{glc} (resp. \emph{gklt}, $\epsilon$-\emph{glc}) if $a(E,X,B)\ge0$ (resp. $>0$, $\ge\epsilon$) for every prime $\bb$-divisor $E$ over $X$,
	    \item  $(X,B+M_X)$ is \emph{g-terminal} if $a(E,X,B)>1$ for every exceptional$/X$ prime $\bb$-divisor $E$,
	    \item $(X,B+M_X)$ is \emph{gdlt} if $a(D,X,B)>0$ for some log resolution $\phi:W\to X$ as above and every prime divisor $D$ on $W$ that is exceptional$/X$,
	    \item $(X,B+M_X)$ is \emph{$\Qq$-factorial} if every $\Qq$-divisor on $X$ is $\Qq$-Cartier.
	\end{itemize}
A \emph{generalized terminalization} of a glc g-pair $(X,B+M_X)$ is a birational morphism $f: Y\rightarrow X$ satisfying the following.
\begin{itemize}
    \item $K_Y+B_Y+M_Y=f^*(K_X+B+M_X)$,
    \item $(Y,B_Y+M_Y)$ is $\Qq$-factorial g-terminal,
    \item $f$ only extracts prime $\bb$-divisors $E$ over $X$ such that $0\leq a(E,X,B+M)\leq 1$.
\end{itemize}
\end{defn}

\begin{defn}
Assume that
\begin{itemize}
    \item $X\rightarrow Z$ and $Y\rightarrow Z$ are two contractions,
    \item $(X,B+M_X)$ and $(Y,B_Y+M_Y)$ are two g-pairs$/Z$ with the same associated nef$/Z$ $\bb$-divisor $M$, and
    \item $f: X\dashrightarrow Y$ is a birational map$/Z$,
\end{itemize}
such that
\begin{itemize}
    \item $f$ does not extract any divisor, and
    \item $a(E,X,B+M_X)\leq a(E,Y,B_Y+M_Y)$ for every prime $\bb$-divisor $E$ over $X$,
\end{itemize}
then we may write $(X,B+M_X)\geq (Y,B_Y+M_Y)$.
\end{defn}

\begin{defn}[Sarkisov links]
Assume that
\begin{itemize}
    \item $X_1\rightarrow Z$ and $X_2\rightarrow Z$ are two contractions,
    \item $(X_1,B_1+M_{X_1})$ and $(X_2,B_2+M_{X_2})$ are two gklt g-pairs with the same associated nef$/Z$ $\bb$-divisor $M$,
    \item $\phi_1: X_1\rightarrow S_1$ is a $(K_{X_1}+B_1+M_{X_1})$-Mori fiber space$/Z$ and $\phi_2: X_2\rightarrow S_2$ is a $(K_{X_2}+B_2+M_{X_2})$-Mori fiber space$/Z$, 
    \item there are two birational morphisms $W\rightarrow X_1$ and $W\rightarrow X_2$, and an effective $\Rr$-divisor $B_W$ on $W$, such that $B_1$ and $B_2$ are the pushforwards of $B_W$ on $X_1$ and $X_2$ respectively, and
    \item  $f: X_1\dashrightarrow X_2$ is the induced birational map$/Z$.
\end{itemize}
Then 
\begin{itemize}
    \item $f$ is called a $(K_{X_1}+B_1+M_{X_1})$-\emph{Sarkisov link$/Z$ of type I}, or a \emph{Sarkisov link$/Z$ of type I}, if there exists an extraction $g: V\rightarrow X_1$, a sequence of flips $V\dashrightarrow X_2$ over $Z$, and an extremal contraction $S_2\rightarrow S_1$, such that the following diagram commutes:
    \begin{center}$\xymatrix{
 V\ar@{->}[d]_{g}\ar@{-->}[rr]& &X_2\ar@{->}[d]^{\phi_2} \\
      X_1\ar@{-->}[rru]_{f}\ar@{->}[dr]_{\phi_1}   &  & S_2\ar@{->}[dl] \\
    & S_1 &}$
\end{center}
\item $f$ is called a $(K_{X_1}+B_1+M_{X_1})$-\emph{Sarkisov link$/Z$ of type II}, or a \emph{Sarkisov link$/Z$ of type II}, if there exists an extraction $g: V\rightarrow X_1$, a sequence of flips $V\dashrightarrow U$ over $Z$, and a divisorial contraction $U\rightarrow X_2$, such that the following diagram commutes:
\begin{center}$\xymatrix{
 V\ar@{->}[d]_{g}\ar@{-->}[r]& U\ar@{->}[d] \\
      X_1\ar@{-->}[r]_{f}\ar@{->}[d]_{\phi_1}    & X_2\ar@{->}[d]^{\phi_2} \\
    S_1\ar@{->}[r]^{\cong}& S_2 }$
\end{center}
\item $f$ is called a $(K_{X_1}+B_1+M_{X_1})$-\emph{Sarkisov link$/Z$ of type III}, or a \emph{Sarkisov link$/Z$ of type III}, if there exists a sequence of flips $X_1\dashrightarrow U$ over $Z$, a divisorial contraction $U\rightarrow X_2$ and an extremal contraction $S_1\rightarrow S_2$, such that the following diagram commutes:
\begin{center}$\xymatrix{
 X_1\ar@{->}[d]_{\phi_1}\ar@{-->}[drr]^{f}\ar@{-->}[rr]& &U\ar@{->}[d] \\
      S_1\ar@{->}[dr]   &  & X_2\ar@{->}[dl]^{\phi_2} \\
    & S_2 &}$
\end{center}

\item $f$ is called a $(K_{X_1}+B_1+M_{X_1})$-\emph{Sarkisov link$/Z$ of type IV}, or a \emph{Sarkisov link$/Z$ of type IV}, if $f$ is a sequence of flips$/Z$, and there are two extremal contractions $S_1\rightarrow T$ and $S_2\rightarrow T$ over $Z$, such that the following diagram commutes:
\begin{center}$\xymatrix{
 X_1\ar@{-->}[rr]^{f}\ar@{->}[d]_{\phi_1}& &X_2\ar@{->}[d]^{\phi_2} \\
      S_1\ar@{->}[dr]   &  & S_2\ar@{->}[dl] \\
    & T&}$
\end{center}
\item $f$ is called a  $(K_{X_1}+B_1+M_{X_1})$-\emph{Sarkisov link}$/Z$, or a \emph{Sarkisov link}$/Z$, if it is a Sarkisov link$/Z$ of one of the four types above. We remark that we allow $f$ to be the identity map.
\end{itemize}
\end{defn}









\section{Preliminaries} 

\begin{lem}[{\cite[Lemma 4.5 and 4.6]{BZ16}}]\label{lem: gen extraction}
Let $m\geq 0$ be an integer, $(X,B+M_X)$ a gklt g-pair, $E_1,\dots,E_m$ exceptional$/X$ $\bb$-divisors, such that $a(E_i,X,B+M_X)\leq 1$ for each $i$. Then there is an extraction $f: Y\rightarrow X$ satisfying the following.
\begin{itemize}
\item $f$ extracts exactly $E_1,\dots,E_m$,
\item $Y$ is $\Qq$-factorial, and
\item if $m=1$ and $X$ is $\Qq$-factorial, then $f$ is a divisorial contraction.
\end{itemize}
\end{lem}

\begin{cor}\label{cor: gen terminalization}
Let $(X,B+M_X)$ be a gklt g-pair, then there is a generalized terminalization of $(X,B+M_X)$. More precisely, there is an extraction $f: Y\rightarrow X$, such that $Y$ is $\Qq$-factorial, and $f$ extracts exactly all the exceptional$/X$ $\bb$-divisors $E$ such that $a(E,X,B+M_X)\leq 1$. In particular, let $K_Y+B_Y+M_Y:=f^*(K_X+B+M_X)$, then $(Y,B_Y+M_Y)$ is g-terminal.
\end{cor}

\begin{proof}
By Lemma \ref{lem: gen extraction}, we only need to show that there are finitely many prime $\bb$-divisor $E$ over $X$ such that $a(E,X,B+M_X)\leq 1$.

Let $g: W\rightarrow X$ be a log resolution of $(X,B)$ such that $M=M_W$. Let $K_W+B_W+M_W:=g^*(K_X+B+M_X)$. Since $(X,B+M_X)$ is gklt, all the coefficients of $B_W$ are $\leq 1-c$ for some real number $c>0$. Suppose that $B_W=B_W^+-B_W^-$, where $B_W^+,B_W^-\geq 0$ and $B_W^+\wedge B_W^-=0$. Then $(W,B_W^+)$ is klt. Let $h:U\rightarrow W$ be a terminalization of $(W,B_W^+)$, then all the prime $\bb$-divisors $E$ such that $a(E,X,B+M_X)\leq 1$ are on $U$. Therefore there are finitely many prime $\bb$-divisors $E$ over $X$ such that $a(E,X,B+M_X)\leq 1$, and the corollary follows.
\end{proof}


\begin{prop}\label{prop: g terminalization prop} 
Let $W\rightarrow Z$ and $X\rightarrow Z$ be two contractions, $f:W\dashrightarrow X$ a birational map$/Z$, and $(W,B_W+M_W)$ and $(X,B+M_X)$ two g-pairs$/Z$. Assume that
\begin{itemize}
    \item $K_X+B+M_X$ is nef$/Z$,
    \item $f$ does not extract any divisor,
    \item for any prime divisor $D\subset W$, $a(D,X,B+M_X)\geq a(D,W,B_W+M_W)$, and
    \item $(W,B_W+M_W)$ is g-terminal,
\end{itemize}
then
\begin{enumerate}
    \item $a(E,X,B+M_X)\geq a(E,W,B+M_W)$ for any prime $\bb$-divisor $E$ over $X$. In other words, $(W,B+M_W)\geq (X,B+M_X)$.
    \item $(X,B+M_X)$ is gklt,
    \item there is a generalized terminalization $g: Y\rightarrow X$ of $(X,B+M_X)$,
    \item the induced birational map $W\dashrightarrow Y$ does not extract any divisor,
    \item for any exceptional$/X$ $\bb$-divisor $E$ such that $a(E,X,B+M_X)\leq 1$, $E$ is on $W$.
\end{enumerate}
\end{prop}

\begin{proof}
Let 
$p: V\rightarrow W$ and $q: V\rightarrow X$ be any resolution of indeterminacy of $f$ such that
$$p^*(K_W+B_W+M_W)=q^*(K_X+B+M_X)+E_V,$$
then $p_*E_V=\sum_{E\subset W}(a(E,X,B+M_X)-a(E,W,B_W+M_W))E\geq 0$. Since $K_X+B+M_X$ is nef$/Z$, $-E_V$ is nef$/W$. By the negativity lemma, $E_V\geq 0$, which implies (1). Since $(W,B_W+M_W)$ is g-terminal, $\lfloor B _W\rfloor=0$, and (2) follows from (1). (3) follows from (2) and Corollary \ref{cor: gen terminalization}. 

Suppose that $E$ is an exceptional$/W$ prime $\bb$-divisor. Since $(W,B_W+M_W)$ is g-terminal, $a(E,W,B_W+M_W)>1$. Since $f$ does not extract any divisor, $E$ is exceptional$/X$. By construction of generalized terminalization, we deduce (4). (5) follows from (4).
\end{proof}

\begin{thm}[MMP for generalized pairs, {\cite[Lemma 4.4(1)(2)]{BZ16}}]\label{thm: gen pair mmp}
Let $X\rightarrow Z$ be a contraction, $(X,B+M_X)$ a $\Qq$-factorial glc g-pair$/Z$ such that $(X,0)$ is klt, and $A$ a general ample$/Z$ $\Rr$-divisor on $X$. Then there is a $(K_X+B+M_X)$-MMP$/Z$ with scaling of $A$ satisfying the following.
\begin{enumerate}
    \item If $K_X+B+M_X$ is not pseudo-effective$/Z$, then the $(K_X+B+M_X)$-MMP$/Z$ with scaling of $A$ terminates with a Mori fiber space$/Z$.
    \item If
    \begin{itemize}
    \item $K_X+B+M_X$ is pseudo-effective$/Z$, 
        \item $(X,B+M_X)$ is gklt, and
        \item there are real numbers $\alpha,\beta\geq 0$ such that $K_X+(1+\alpha)B+(1+\beta)M_X$ is big$/Z$,
    \end{itemize}
    the $(K_X+B+M_X)$-MMP$/Z$ with scaling of $A$ terminates with a log model$/Z$ $f: X\dashrightarrow Y$, such that $K_Y+B_Y+M_Y:=f_*(K_X+B+M_X)$ is semi-ample$/Z$.
\end{enumerate}
\end{thm}

\begin{lem}\label{lem: terminalization and mmp}
Let $X\rightarrow Z$ be a contraction, $(X,B+M_X)$ a $\Qq$-factorial gklt g-pair$/Z$, and $f: X\dashrightarrow Y$ a $(K_X+B+M_X)$-non-positive map$/Z$ such that $f_*(K_X+B+M_X)=K_Y+B_Y+M_Y$. Then there is 
\begin{itemize}
    \item a resolution of indeterminacy $p: W\rightarrow X$ and $q: W\rightarrow Y$, and
    \item a $\Qq$-factorial g-terminal pair $(W,B_W+M_W)$,
\end{itemize}
such that
\begin{enumerate}
    \item $q$ is $(K_W+B_W+M_W)$-non-positive and $q_*(K_W+B_W+M_W)=K_Y+B_Y+M_Y$,
    \item $(W,B_W+M_W)\geq (Y,B_Y+M_Y)$,
    \item for any real number $\epsilon>0$, if $(X,B+M_X)$ is generalized $\epsilon$-lc, then $(W,B_W+M_W)$ is generalized $\epsilon$-lc, and
    \item $M=M_W$.
\end{enumerate} 
\end{lem}

\begin{proof}
Let $p: W\rightarrow X$ and $q: W\rightarrow Y$ be a resolution of indeterminacy, such that $p$ is a log resolution of $(W,B_W)$ and $M=M_W$. Let $K_W+B'_W+M_W:=p^*(K_X+B+M_X)$, then we may write
$B_W'=B_{1,W}+B_{2,W}-B_{3,W}$,
such that
\begin{itemize}
    \item $B_{1,W}$ is the strict transform of $B$ on $W$, 
    \item $B_{2,W},B_{3,W}$ are exceptional$/X$, and
    \item $B_{2,W},B_{3,W}\geq 0$ and $B_{2,W}\wedge B_{3,W}=0$.
\end{itemize}
Possibly blowing up more, we may assume that all the irreducible components of $B_{1,W}$ and $B_{2,W}$ do not intersect. Since $(X,B+M_X)$ is gklt, $(W,B_W:=B_{1,W}+B_{2,W})$ is terminal. Since $W$ is smooth, $(W,B_W+M_W)$ is $\Qq$-factorial g-terminal. 

Since $p$ is $(K_W+B_W+M_W)$-non-positive, $K_X+B+M_X=p_*(K_W+B_W+M_W)$ and $f$ is $(K_X+B+M_X)$-non-positive, we have that $q=f\circ p$ is $(K_W+B_W+M_W)$-non-positive, which implies (1). (2) follows from (1) and (3)(4) follow from the construction of $(W,B_W+M_W)$.
\end{proof}




\begin{thm}[Finiteness of weak log canonical models]\label{thm: finiteness ltm}
Let 
\begin{itemize}
\item $X\rightarrow Z$ be a projective morphism between normal quasi-projective varieties,
\item $A$ a general ample$/Z$ $\Qq$-divisor on $X$,
\item $\mathcal{V}\subset\Weil_{\Rr}(X)$ a finite dimensional rational subspace, and
\item $\mathcal{C}\subset\mathcal{L}_A(\mathcal{V})$ a rational polytope such that $(X,B)$ is klt for any $B\in\mathcal{C}$,
\end{itemize}
then there exists an integer $k\geq 0$ and birational maps$/Z$ $\phi_i: X\dashrightarrow Y_i$ for each $1\leq i\leq k$, such that for every $B\in\mathcal{C}$,
\begin{enumerate}
    \item there exists an integer $1\leq i\leq k$ such that $\phi_i$ is a weak log canonical model$/Z$ of $(X,B)$,
    \item for any log terminal model $\phi: X\dashrightarrow Y$, there exists an integer $1\leq j\leq k$ such that $\phi_j\circ\phi^{-1}: Y\rightarrow Y_j$ is an isomorphism.
\end{enumerate}
\end{thm}

\begin{proof}
The theorem follows from \cite[Theorem C]{BCHM10} and \cite[Theorem E]{BCHM10}.
\end{proof}

\begin{thm}\label{thm: contraction extremal face gklt}
Let $X\rightarrow Z$ be a contraction, $(X,B+M_X)$ a gklt g-pair$/Z$, and $F\subset\overline{NE}(X/Z)$ a $(K_X+B+M_X)$-negative extremal face. Then there is a unique contraction $f: X\rightarrow Y$ over $Z$, such that any irreducible curve $C\subset X$ is mapped to a point by $f$ if and only if $[C]\in F$.
\end{thm}
\begin{proof}
Since $(X,B+M_X)$ a gklt g-pair$/Z$, there exists an ample$/Z$ $\Rr$-divisor $A$ on $X$ and $\Delta\geq 0$ on $X$, such that 
\begin{itemize}
    \item $F$ is also a $(K_X+B+M_X+A)$-negative extremal face,
    \item $\Delta\sim_{\mathbb R}B+M_X+A$, and
    \item $(X,\Delta)$ is klt.
\end{itemize}
The existence of $f$ follows from the usual cone theorem, c.f. \cite[Theorem 3.25]{KM98}, \cite[Theorem 3-2-1]{KMM87}.
\end{proof}







\section{Sarkisov program with double scaling}

In this section we construct a special type of Sarkisov program, called the ``Sarkisov program with double scaling". As the notation is complicated and technical, we first illustrate our ideas.

\begin{idea}\label{idea: sarkisov scaling}
First, recall the typical structure of the Sarkisov program as in Theorem \ref{thm: existence generalized Sarkisov link}. Possibly replacing $W$, we may assume that $\rho_X$ and $\rho_Y$ are morphisms:
\begin{center}$\xymatrix{
 & W\ar@{->}[dl]_{\rho_X}\ar@{->}[dr]^{\rho_Y}& \\
      X \ar@{->}[d]_{\phi_X}\ar@{-->}[rr]^{f}   &  & Y\ar@{->}[d]^{\phi_Y} \\
    S_X & &S_Y }$
\end{center}
Here $\phi_X: X\rightarrow S_X$ is a $(K_X+B_X+M_X)$-Mori fiber space$/Z$ and $\phi_Y: Y\rightarrow S_Y$ is a $(K_Y+B_Y+M_Y)$-Mori fiber space$/Z$. 

We need to study the difference and similarity between $\phi_X: X\rightarrow S_X$ and $\phi_Y: Y\rightarrow S_Y$. A common strategy in birational geometry is to study the ample divisors on $X$ and $Y$. This works well in our setting, as $-(K_X+B_X+M_X)$ is ample over $S_X$ and  $-(K_Y+B_Y+M_Y)$ is ample over $S_Y$. Therefore, we may pick general ample$/Z$ $\Rr$-divisors $L_X$ and $H_Y$ on $X$ and $Y$ respectively, such that
\begin{itemize}
    \item $L_X\sim_{\Rr,Z}-(K_X+B_X+M_X)+\phi_X^*A_{S_X}$ and
    \item $H_Y\sim_{\Rr,Z}-(K_Y+B_Y+M_Y)+\phi_Y^*A_{S_Y}$, 
\end{itemize}
for some general ample $\Rr$-divisors $A_{S_X}$ and $A_{S_Y}$ on $S_X$ and $S_Y$ respectively. In particular, $L_W:=\rho_X^*L_X$ and $H_W:=\rho_Y^*H_Y$ are big and nef$/Z$, and we may define $H_X:=(\rho_X)_*H_W$ and $L_Y:=(\rho_Y)_*L_W$. We have
\begin{itemize}
    \item $K_X+B_X+L_X+0H_Y+M_X\sim_{\mathbb R,S_X}0$, and
    \item $K_Y+B_Y+0L_Y+H_Y+M_Y\sim_{\mathbb R,S_Y}0$.
\end{itemize}
One key observation is the following: pick any big$/Z$ $\Rr$-divisor $D_W$ on $W$ and let $D_Y$ be its image on $Y$. If we assume that $(K_W+B_W+M_W+H_W)$ is pseudo-effective, then $(Y,B_Y+M_Y+H_Y+\epsilon D_Y)$ is a log canonical model of $(W,B_W+M_W+H_W+\epsilon D_W)$ over $Z$ for an $0<\epsilon\ll 1$. We would like to use this fact and construct the Sarkisov program by using the uniqueness of the log canonical model, and we make it by using the idea for ``double scaling" in the following way:
\begin{enumerate}
    \item First, we let $l:=1$ and $h:=0$. Starting with $X$, we decrease $l$ and increase $h$ linearly, making sure that $K_X+B_X+lL_X+hH_X+M_X\sim_{\mathbb R,S_X}0$ still holds. 
    \item Next, we ``aggressively" reduce $l$ to $0$. In this case, if $h=1$, then by picking a general big$/Z$ $\Rr$-divisor $D_X$ on $X$ and let $D_Y$ be the birational transform of $D_X$ on $Y$, we deduce that $X=Y$ by the uniqueness of the log canonical model. 
    \item However, it is possible that $h<1$ or $h>1$. Nevertheless, 
    \begin{itemize}
        \item if $h<1$, we observe that $K_X+B_X+hH_X+M_X$ is not nef over $Z$ but $K_X+B_X+L_X+M_X$ is nef over $Z$, and
        \item if $h>1$, we observe that $(W,B_W+hH_W+M_W)\geq (X,B_W+hH_W+M_W)$ does not hold, while $(W,B_W+L_W+M_W)\geq (X,B_W+L_W+M_W)$ holds.
    \end{itemize}
    Therefore, instead of directly reducing $l$ to $0$, we may reduce $l$ ``moderately" until reaching a threshold when either case above appears.
    \item Finally, it appears that when $l$ and $h$ reach the threshold, one of the four types of Sarkisov links will naturally appear. 
    \end{enumerate}
\end{idea}

The writing of the rest of this section is based on the idea above, which contains three parts.

In the first part, we construct a Sarkisov link with a double scaling structure by using the threshold which appears in Idea \ref{idea: sarkisov scaling}(3). All four different types of Sarkisov links may appear in this part.

In the second part, we study the behavior of some invariants for every Sarkisov link constructed in the first part. 

In the last part, we use the invariants constructed in the second part to show that any sequence of the Sarkisov links constructed in the first part terminates.

\subsection{Constructing a Sarkisov link}

\begin{cons}[Setting]\label{cons: setting for sarkisov link}
This setting will be used in the rest of this section. We assume that
\begin{itemize}
    \item $X\rightarrow Z$ is a contraction,
    \item $\rho: W\dashrightarrow X$ is a birational map,
    \item $(W,B_W+M_W)$ is a g-pair with associated nef$/Z$ $\bb$-divisor $M$,
    \item $L_W$ and $H_W$ are two general big and nef$/Z$ $\Rr$-divisors on $W$, 
    \item $(X,B+M_X)$ is a g-pair, 
    \item $\phi: X\rightarrow S$ is a $(K_X+B+M_X)$-Mori fiber space$/Z$,
    \item $\Sigma$ is a $\phi$-vertical curve,
    \item $L$ and $H$ are two $\Rr$-Cartier $\Rr$-divisors on $X$, and
    \item $0<l\leq 1$ and $0\leq h\leq 1$ are two real numbers,
\end{itemize} 
such that
\begin{enumerate}
    \item $(W,B_W+2(L_W+H_W)+M_W)$ is $\Qq$-factorial g-terminal,
    \item $K_W+B_W+H_W+M_W$ is pseudo-effective$/Z$,
    \item $(X,B+M_X)$ is $\Qq$-factorial gklt,
    \item $(W,B_W+lL_W+hH_W+M_W)\geq (X,B+lL+hH+M_X)$. In particular, $\rho$ does not extract any divisor,
    \item $B,L$ and $H$ are the birational transforms of $B_W,L_W$ and $H_W$ on $X$ respectively,
    \item $K_X+B+lL+hH+M_X\sim_{\mathbb R,S}0$, and
    \item $K_X+B+lL+hH+M_X$ is nef$/Z$.
\end{enumerate}
We illustrate this setting in the following diagram:
\medskip

\begin{center}$\xymatrix{
  W\ar@{-->}[d]_{\rho}&\supset & B_W\ar@{-->}[d] &lL_W\ar@{-->}[d] &hH_W\ar@{-->}[d] &M_W\ar@{-->}[d] &\\
      X \ar@{->}[d]^{\phi}&\supset & B &lL &hH &M_X  &\Sigma:\phi\text{-vertical}    \\
    S & }$
\end{center}
\end{cons}

\begin{defn}[Auxiliary constants and divisors]\label{defn: auxiliary invariants}
Assumptions and notations as Construction \ref{cons: setting for sarkisov link},
\begin{enumerate}
    \item we define $$r:=\frac{H\cdot\Sigma}{L\cdot\Sigma}.$$
    \item For any real number $t$, we define
    $$B_W(t):=B_W+lL_W+hH_W+t(H_W-rL_W),$$ 
and 
$$B(t):=B+lL+hH+t(H-rL).$$
\item We define $\Ii$ to be the set of all real number $t$ satisfying the following:
\begin{enumerate}
    \item $0\leq t\leq\frac{l}{r}$,
        \item for any prime divisor $E\subset W$,
    $$a(E,W,B_W(t)+M_W)\leq a(E,X,B(t)+M_X),$$
    and
    \item $K_X+B(t)+M_X$ is nef$/Z$.
\end{enumerate}
\item We define $s:=\sup\{t\mid t\in\Ii\}$.
\item We define $l_Y:=l-rs$ and $h_Y:=h+s$.
\end{enumerate}
\end{defn}

\begin{lem}\label{lem: sarkisov h<=1}
Assumptions and notations as Construction \ref{cons: setting for sarkisov link} and Definition \ref{defn: auxiliary invariants}, then 
\begin{enumerate}
\item $r>0$ is well-defined,
    \item either $\Ii=\{0\}$, or $\Ii$ is a closed interval,
    \item $\Ii$ is non-empty and $s\in\Ii$, 
    \item $l_Y=l$ if and only if $h_Y=h$, and
    \item $\Ii\subset [0,1-h]$. In particular, $h_Y\leq 1$.
\end{enumerate}
\end{lem}
\begin{proof}
Since $L_W$ and $H_W$ are general big and nef$/Z$ divisors on $W$, $L$ and $H$ are big$/Z$, hence ample$/S$. Thus $H\cdot\Sigma>0$ and $L\cdot\Sigma>0$, hence $r>0$ is well-defined. This is (1).

By Definition \ref{defn: auxiliary invariants}(3), $0\in\Ii$ and $\Ii$ is closed and connected, which implies (2). (3) follows from (2) and the definition of $s$. (4) follows from (1) and the definitions of $l_Y$ and $h_Y$.

Assume that (5) does not hold. By (2), there exists $t_0\in\Ii$ such that $1<h+t_0<2$. By Construction \ref{cons: setting for sarkisov link}(1), $(W,B_W(t_0)+M_W)$ is g-terminal. By Proposition \ref{prop: g terminalization prop}(1) and the definition of $\Ii$, $(W,B_W(t_0)+M_W)\geq (X,B(t_0)+M_X)$. Therefore $(X,B(t_0)+M_X)$ is gklt. Since $(K_X+B(t_0)+M_X)\cdot\Sigma=0$ and $H$ is big$/Z$, 
$$(K_X+B+(l-t_0r)L+H+M_X)\cdot\Sigma=((K_X+B(t_0)+M_X)-(h+t_0-1)H)\cdot\Sigma<0.$$
Thus $\phi$ is a $(K_X+B+(l-t_0r)L+H+M_X)$-Mori fiber space$/Z$. In particular, $K_X+B+H+M_X$ is not pseudo-effective$/Z$. Since $\rho$ does not extract any divisor, $K_W+B_W+H_W+M_W$ is not pseudo-effective$/Z$, which contradicts Construction \ref{cons: setting for sarkisov link}(2).
\end{proof}

\begin{cons}\label{cons: cases of sarkisov link with scaling}
Assumptions and notations as Construction \ref{cons: setting for sarkisov link} and Definition \ref{defn: auxiliary invariants}. Then there are three possibilities for $s$:
\begin{itemize}
   \item[\textbf{Case 1}] $s=\frac{l}{r}$. In particular, $l_Y=0.$
    \item[\textbf{Case 2}] 
    \begin{itemize}
        \item $s<\frac{l}{r}$. In particular, $l_Y>0$, and 
        \item there exists $0<\epsilon\ll 1$ and a prime divisor $E\subset W$, such that $a(E,W,B_W(s+\epsilon)+M_W)>a(E,X,B(s+\epsilon)+M_X).$
    \end{itemize}
        \item[\textbf{Case 3}] 
    \begin{itemize}
        \item $s<\frac{l}{r}$. In particular, $l_Y>0$, and 
        \item there exists $0<\epsilon\ll 1$, such that 
        \begin{itemize}
            \item $a(E,W,B_W(s+\epsilon)+M_W)\leq a(E,X,B(s+\epsilon)+M_X)$ for any prime divisor $E\subset W$, and
            \item $K_X+B(s+\epsilon)+M_X$ is not nef$/Z$.
        \end{itemize} 
    \end{itemize}
\end{itemize}
\end{cons}


\begin{thm}[Sarkisov link with double scaling]\label{thm: scaling sarkisov}
Assumptions and notations as Construction \ref{cons: setting for sarkisov link} and Definition \ref{defn: auxiliary invariants}. The there exist
\begin{itemize}
    \item a birational map$/Z$ $\rho_Y: W\dashrightarrow Y$ which does not extract any divisor,
    \item three $\Rr$-divisors $B_Y,L_Y$ and $H_Y$ on $Y$,
    \item a $(K_Y+B_Y+M_Y)$-Mori fiber space$/Z$ $\phi_Y:Y\rightarrow S_Y$, and
    \item a Sarkisov link$/Z$ $f: X\dashrightarrow Y$,
\end{itemize}
such that
\begin{enumerate}
\item $(Y,B_Y+M_Y)$ is a $\Qq$-factorial gklt g-pair$/Z$,
    \item $(W,B_W+l_YL_W+h_YH_W+M_W)\geq (Y,B_Y+l_YL_Y+h_YH_Y+M_Y)$. In particular, $\rho_Y$ does not extract any divisor,
    \item $B_Y,L_Y$ and $H_Y$ are the birational transforms of $B_W,L_W$ and $H_W$ on $Y$ respectively,
    \item $K_Y+B_Y+l_YL_Y+h_YH_Y+M_Y\sim_{\Rr,S_Y}0$, 
    \item $K_Y+B_Y+l_YL_Y+h_YH_Y+M_Y$ is nef$/Z$, 
    \item for any $\phi_Y$-vertical curve $\Sigma_Y$ on $Y$, and $r=\frac{H\cdot\Sigma}{L\cdot\Sigma}\geq\frac{H_Y\cdot\Sigma_Y}{L_Y\cdot\Sigma_Y}>0$.
\end{enumerate}
\end{thm}


\begin{proof}

We prove the Theorem by considering the three different cases in Construction \ref{cons: cases of sarkisov link with scaling} separately.

\medskip

\noindent\textbf{Case 1}. In this case, we finish the proof by letting $\rho_Y:=\rho, Y:=X, B_Y:=B, L_Y:=L, H_Y:=H, M_Y:=M_X, \phi_Y:=\phi_X, S_Y:=S$, and $f:=\id_X$.

\medskip

\noindent\textbf{Case 2}. In this case, $a(E,W,B_W(s)+M_W)=a(E,X,B(s)+M_X),$ and $E$ is exceptional$/X$. Since $E\subset W$, $$a(E,X,B(s+\epsilon)+M_X)<a(E,W,B_W(s+\epsilon)+M_W)\leq 1.$$
By Lemma \ref{lem: gen extraction}, there is an extraction $g: V\rightarrow X$ of $E$ such that $V$ is $\Qq$-factorial. By Proposition \ref{prop: g terminalization prop}(4), the induced birational map $W\dashrightarrow V$ does not extract any divisor. We let $B_V,L_V,H_V$ be the birational transforms of $B_W,L_W$ and $H_W$ on $V$ respectively, then we have
\begin{align*}
    &K_V+B_V+(l_Y-r\epsilon)L_V+(h_Y+\epsilon)H_V+M_V\\
    =&g^*(K_X+B+(l_Y-r\epsilon)L+(h_Y+\epsilon)H+M_X).
\end{align*}
Moreover, since $a(E,X,B(s+\epsilon)+M_X)<1$, $\mult_E(B_V+(l_Y-r\epsilon)L_V+(h_Y+\epsilon)H_V)>0$. Thus we may pick a sufficiently small positive real number $0<\delta\ll\epsilon$, such that 
$(V,\Delta_V+M_V)$ is gklt, where
$$K_V+\Delta_V+M_V:=g^*(K_X+B+(l_Y-r\epsilon-\delta)L+(h_Y+\epsilon)H+M_X).$$
We may run a $(K_V+\Delta_V+M_V)$-MMP$/S$ $\psi: V\dashrightarrow Y$ which terminates with a Mori fiber space$/S$ $\phi_Y: Y\rightarrow S_Y$ by Theorem \ref{thm: gen pair mmp}. Since $\rho(V/S)=\rho(V/X)+\rho(X/S)=2$ and $1=\rho(Y/S_Y)\leq\rho(V/S_Y)\leq\rho(V/S)$, there are two possibilities:

\medskip

\noindent\textbf{Case 2.1}. $\rho(V/Y)=0$. In this case $\psi$ is a sequence of flips, and we get a Sarkisov link$/Z$ $f:X\dashrightarrow Y$ of type I. Let $B_Y,L_Y$ and $H_Y$ be the birational transforms of $B_V,L_V$ and $H_V$ on $Y$ respectively and $\rho_Y: W\dashrightarrow Y$ the induced morphism. By our constructions, (1)-(5) are clear, and we only left to show (6).

For any general $\phi_Y$-vertical curve $\Sigma_Y$, $\psi$ is an isomorphism in a neighborhood of $\Sigma_Y$, and we may let $\Sigma_V$ be the birational transform of $\Sigma_Y$ on $V$. Pick any $0<\delta'\ll\delta$ and let
$$K_V+\Delta'_V+M_V:=g^*(K_X+B+(l_Y-r\epsilon-\delta')L+(h_Y+\epsilon)H+M_X),$$
then $\psi$ is also a $(K_V+\Delta'_V+M_V)$-MMP$/S$. Let $\Delta_Y'$ be the birational transform of $\Delta'_V$ on $Y$. Then
\begin{align*}
&g^*(K_X+B+(l_Y-r\epsilon-\delta')L+(h_Y+\epsilon)H+M_X)\cdot\Sigma_V\\
=&(K_Y+\Delta_Y'+M_Y)\cdot\Sigma_Y<0
\end{align*}
Let $\delta'\rightarrow 0$, then we have
$$g^*(K_X+B+(l_Y-r\epsilon)L+(h_Y+\epsilon)H+M_X)\cdot\Sigma_V\leq 0.$$
Since $g^*(K_X+B+l_YL+h_YH+M_X)\sim_{\mathbb R,S}0$, we deduce that
$$g^*(H-rL)\cdot\Sigma_V\leq 0.$$
Moreover, by our assumptions, $g^*(H-rL)=g^{-1}_*(H-rL)+eE$ for some real number $e>0$, and $\Sigma_V\not\subset E$. Thus
\begin{align*}
    (H_Y-rL_Y)\cdot\Sigma_Y&=g^{-1}_*(H-rL)\cdot\Sigma_V=(g^*(H-rL)-eE)\cdot\Sigma_V\\
    &\leq g^*(H-rL)\cdot\Sigma_V\leq 0,
\end{align*}
which implies (6), and the theorem follows in this case.

\medskip

\noindent\textbf{Case 2.2}. $\rho(V/Y)=1$. In this case, suppose that $U\rightarrow U'$ is the first divisorial contraction in $\psi$. Then $\rho(U'/S_Y)=\rho(U'/S)=1$, which implies that $U‘\rightarrow S$ is a Mori fiber space. Thus $U'=Y$ and $S\cong S_Y$, and the induced birational map $f:X\dashrightarrow Y$ is a Sarkisov link$/Z$ of type II. Let $B_Y,L_Y,H_Y$ be the birational transforms of $B_V,L_V$ and $H_V$ on $Y$ respectively and $\rho_Y: W\dashrightarrow Y$ the induced morphism. By our constructions, (1)-(5) are clear, and we only left to show (6).

For any general $\phi_Y$-vertical curve $\Sigma_Y$, $\psi$ is an isomorphism in a neighborhood of $\Sigma_Y$, and we may let $\Sigma_V$ be the birational transform of $\Sigma_Y$ on $V$. Pick any $0<\delta'\ll\delta$ and let
$$K_V+\Delta'_V+M_V:=g^*(K_X+B+(l_Y-r\epsilon-\delta')L+(h_Y+\epsilon)H+M_X),$$
then $\psi$ is also a $(K_V+\Delta'_V+M_V)$-MMP$/S$. Let $\Delta_Y'$ be the birational transform of $\Delta'_V$ on $Y$. Then
\begin{align*}
&g^*(K_X+B+(l_Y-r\epsilon-\delta')L+(h_Y+\epsilon)H+M_X)\cdot\Sigma_V\\
=&(K_Y+\Delta_Y'+M_Y)\cdot\Sigma_Y<0
\end{align*}
Let $\delta'\rightarrow 0$, then we have
$$g^*(K_X+B+(l_Y-r\epsilon)L+(h_Y+\epsilon)H+M_X)\cdot\Sigma_V\leq 0.$$
Since $g^*(K_X+B+l_YL+h_YH+M_X)\sim_{\mathbb R,S}0$, we deduce that
$$g^*(H-rL)\cdot\Sigma_V\leq 0.$$
Moreover, by our assumptions, $g^*(H-rL)=g^{-1}_*(H-rL)+eE$ for some real number $e>0$, and $\Sigma_V\not\subset E$. Thus
\begin{align*}
    (H_Y-rL_Y)\cdot\Sigma_Y&=g^{-1}_*(H-rL)\cdot\Sigma_V=(g^*(H-rL)-eE)\cdot\Sigma_V\\
    &\leq g^*(H-rL)\cdot\Sigma_V\leq 0,
\end{align*}
which implies (6), and the theorem follows in this case.

\medskip

\noindent\textbf{Case 3}. In this case, there exists a $(K_X+B(s+\epsilon)+M_X)$-negative extremal ray $[C]$ on $X$. Since $(K_X+B(s+\epsilon)+M_X)\cdot\Sigma=0$, $[C]\not=[\Sigma]$. Let $P\subset\overline{NE}(X/Z)$ be the extremal face over $Z$ defined by all $(K_X+B(s+\epsilon)+M_X)$-non-positive irreducible curves. Then $P\not=[\Sigma]$, and hence there exists an extremal ray $[\Pi]$ such that $[\Sigma]$ and $[\Pi]$ span a two-dimensional face of $P$. By our construction, $(K_X+B(s+\epsilon)+M_X)\cdot\Pi<0$. Now for $0<\delta\ll 1$, we have
$$(K_X+B+(l_Y-r\epsilon-\delta)L_X+(h_Y+\epsilon)H_X+M_X)\cdot\Sigma<0$$
and
$$(K_X+B+(l_Y-r\epsilon-\delta)L_X+(h_Y+\epsilon)H_X+M_X)\cdot\Pi<0.$$
By Theorem \ref{thm: contraction extremal face gklt}, there exists a contraction $\pi: X\rightarrow T$ of the extremal face of $\overline{NE}(X/Z)$ spanned by $[\Sigma]$ and $[\Pi]$. Then $\pi$ factors through $S$, and  $K_X+B(s)+M_X\sim_{\Rr,T}0$.

Since $L,H$ are big$/Z$, $L,H$ are big$/T$. Therefore, if $K_{X}+B(s+\epsilon)+M_X$ is pseudo-effective$/T$, then $K_X+(1+\alpha)B(s+\epsilon)+M_X$ is big$/T$. By Theorem \ref{thm: gen pair mmp}, we may run a $(K_{X}+B(s+\epsilon)+M_X)$-MMP$/T$ with scaling of some ample$/T$ divisor, and this MMP$/T$ terminates. There are three cases:

\medskip


\noindent\textbf{Case 3.1}. After a sequence of flips $f: X\dashrightarrow Y$, the MMP$/T$ terminates with a Mori fiber space$/T$ $\phi_Y: Y\rightarrow S_Y$. Therefore, $f$ is a Sarkisov link$/Z$ of type IV.  Let $B_Y,L_Y,H_Y$ be the birational transforms of $B,L$ and $H$ on $Y$ respectively and $\rho_Y: W\dashrightarrow Y$ the induced morphism. By our constructions, (1)-(5) are clear, and we only left to show (6).

For any general $\phi_Y$-vertical curve $\Sigma_Y$, $f$ is an isomorphism in a neighborhood of $\Sigma_Y$, and we may let $\Sigma_X$ be the birational transform of $\Sigma_Y$ on $X$. Since $\phi_Y$ is a $(K_Y+B_Y+(l_Y-r\epsilon)L_Y+(h_Y+\epsilon)H_Y+M_Y)$-Mori fiber space$/T$, 
$$-(K_Y+B_Y+(l_Y-r\epsilon)L_Y+(h_Y+\epsilon)H_Y+M_Y)\cdot\Sigma_Y>0,$$
which implies that
$$-(K_X+B(s+\epsilon)+M_X)\cdot\Sigma_X>0.$$
Since $K_X+B(s)+M_X\sim_{\Rr,T}0$, 
$$-(K_X+B(s)+M_X)\cdot\Sigma_X=0,$$
which implies that 
$$(H_Y-rL_Y)\cdot\Sigma_Y=(H-rL)\cdot\Sigma_X<0.$$ 
Thus $r>\frac{H_Y\cdot\Sigma_Y}{L_Y\cdot\Sigma_Y}$, which implies (6), and the theorem follows in this case.

\medskip

\noindent\textbf{Case 3.2}. After a sequence of flips $X\dashrightarrow U$, we get a divisorial contraction$/T$: $U\rightarrow Y$. Therefore $\rho(Y/T)=1$, which implies that the induced morphism $\phi_Y:=Y\rightarrow T$ is a Mori fiber space, and the induced birational map $f: X\dashrightarrow Y$ is a Sarkisov link$/Z$ of type III. Let $B_Y,L_Y,H_Y$ be the birational transforms of $B,L$ and $H$ on $Y$ respectively and $\rho_Y: W\dashrightarrow Y$ the induced morphism. By our constructions, (1)-(5) are clear, and we only left to show (6).

For any general $\phi_Y$-vertical curve $\Sigma_Y$, $f$ is an isomorphism in a neighborhood of $\Sigma_Y$, and we may let $\Sigma_X$ be the birational transform of $\Sigma_Y$ on $X$. Since $-(K_X+B(s+\epsilon)+M_X)$ is nef$/T$ and $K_X+B(s)+M_X\sim_{\Rr,T}0$, we have
$$-(K_X+B(s+\epsilon)+M_X)\cdot\Sigma_X\geq 0=-(K_X+B(s)+M_X)\cdot\Sigma_X,$$
which implies that 
$$(H_Y-rL_Y)\cdot\Sigma_Y=(H-rL)\cdot\Sigma_X\leq 0.$$ 
Thus $r\geq\frac{H_Y\cdot\Sigma_Y}{L_Y\cdot\Sigma_Y}$, which implies (6), and the theorem follows in this case.

\medskip

\noindent\textbf{Case 3.3}. After a sequence of flips $f: X\dashrightarrow Y$, the MMP terminates with a minimal model $Y$ over $T$. Let $B_Y,L_Y,H_Y$ be the birational transforms of $B,L$ and $H$ on $Y$ respectively. Since $\Sigma$ is a general $\phi$-vertical curve, we may let $\Sigma'$ be the birational transform of $\Sigma$ on $Y$. Since $(K_{X}+B(s+\epsilon)+M_X)\cdot\Sigma=0$ and $(K_{X}+B(s)+M_X)\cdot\Sigma=0$, we have
$$(K_Y+B_Y+(l_Y-r\epsilon)L_Y+(h_Y+\epsilon)H_Y+M_Y)\cdot\Sigma'=0$$
and
$$(K_Y+B_Y+l_YL_Y+h_YH_Y+M_Y)\cdot\Sigma'=0$$
which implies that $(K_Y+B_Y+M_Y)\cdot\Sigma'<0$ and $r=\frac{H_Y\cdot\Sigma'}{L_Y\cdot\Sigma'}$. Since $\Sigma$ can be chosen to be any $\phi$-vertical curve, by Theorem \ref{thm: contraction extremal face gklt}, there exists a contraction $\phi_Y:Y\rightarrow S_Y$ of $[\Sigma']$ such that $\phi_Y$ is a $(K_Y+B_Y+M_Y)$-Mori fiber space$/T$. Thus $f$ is a Sarkisov link$/Z$ of type IV.  We finish the proof by letting $\rho_Y: W\dashrightarrow Y$ be the induced birational map.
\end{proof}

\subsection{Behavior of invariants under a Sarkisov link with double scaling}
\begin{lem}\label{lem: sar scaling terminalization}
Assumptions and notations as in Construction \ref{cons: setting for sarkisov link}, Definition \ref{defn: auxiliary invariants}, and Theorem \ref{thm: scaling sarkisov}. Then
\begin{enumerate}
    \item In \textbf{\rm\textbf{Case 2.1}}, $\rho(Y)-\rho(X)=1$.
    \item In \textbf{\rm\textbf{Case 2.2}},
    \begin{enumerate}
        \item $\rho(X)=\rho(Y)$,
        \item there is a prime divisor $F_0$ over $W$, such that 
    \begin{align*}
        &a(F_0,X,B+(l_Y-r\epsilon-\delta)L+(h_Y+\epsilon)H+M_X)\\
        <&a(F_0,Y,B_Y+(l_Y-r\epsilon-\delta)L_Y+(h_Y+\epsilon)H_Y+M_Y),
    \end{align*}
    and
    \item for any prime divisor $F$ over $W$,
      \begin{align*}
        &a(F,X,B+(l_Y-r\epsilon-\delta)L+(h_Y+\epsilon)H+M_X)\\
        \leq &a(F,Y,B_Y+(l_Y-r\epsilon-\delta)L_Y+(h_Y+\epsilon)H_Y+M_Y).
    \end{align*}
    \end{enumerate}
    \item In \textbf{\rm\textbf{Case 3}}, $$a(F,Y,B_Y+(l_Y-r\epsilon)L_Y+(h_Y+\epsilon)H_Y+M_Y)\geq a(F,W,B_W(s+\epsilon)+M_W).$$
    \item In \textbf{\rm\textbf{Case 3.1}}, $\frac{H\cdot\Sigma}{L\cdot\Sigma}>\frac{H_Y\cdot\Sigma_Y}{L_Y\cdot\Sigma_Y}$.
    \item In \textbf{\rm\textbf{Case 3.2}}, $\rho(X)-\rho(Y)=1$.
    \item In \textbf{\rm\textbf{Case 3.3}},
    \begin{enumerate}
        \item $\rho(X)=\rho(Y)$,
        \item there is a prime divisor $F_0$ over $W$, such that 
   $$a(F_0,Y,B_Y+(l_Y-r\epsilon)L_Y+(h_Y+\epsilon)H_Y+M_Y)> a(F_0,X,B(s+\epsilon)+M_X),$$
    and
    \item for any prime divisor $F$ over $W$,
   $$a(F,Y,B_Y+(l_Y-r\epsilon)L_Y+(h_Y+\epsilon)H_Y+M_Y)\geq a(F,X,B(s+\epsilon)+M_X).$$
    \end{enumerate}
\end{enumerate}
\end{lem}
\begin{proof}
(1)(4)(5) immediately follow from the proof of Theorem \ref{thm: scaling sarkisov}. (2) follows from the fact that in \textbf{\rm\textbf{Case 2.2}}, the Sarkisov link$/Z$ is constructed by running a $g^*(K_X+B+(l_Y-r\epsilon-\delta)L+(h_Y+\epsilon)H+M_X)$-MMP$/S$ and $X\not\cong Y$. (3)(6) follow from the fact that in \textbf{\rm\textbf{Case 3}}, the Sarkisov link$/Z$ is constructed by running a $(K_X+B(s+\epsilon)+M_X)$-MMP$/T$ and $X\not\cong Y$ in \textbf{\rm\textbf{Case 3.3}}.
\end{proof}

\subsection{Running the Sarkisov program with double scaling}
\begin{cons}[Sarkisov program with double scaling]\label{cons: sarkisov scaling}
Assume that $W\rightarrow Z$ is a contraction and $(W,B_W+M_W)$ is a $\Qq$-factorial gklt g-pair$/Z$ with nef$/Z$ $\bb$-divisor $M=M_W$, such that $K_W+B_W+M_W$ is not pseudo-effective$/Z$.

Let $\rho: W\dashrightarrow X$ be a $(K_W+B_W+M_W)$-non-positive map$/Z$ and $\phi: X\rightarrow S$ a $(K_X+B+M_X)$-Mori fiber space$/Z$, where $B$ is the birational transform of $B_W$ on $X$. By Theorem \ref{thm: gen pair mmp}, a special choice of $\rho$ is when $\rho$ is a $(K_W+B_W+M_W)$-MMP$/Z$. By Lemma \ref{lem: terminalization and mmp}, possibly taking a resolution of indeterminacy $p:W'\rightarrow W$ and $q:W'\rightarrow X$, we may assume that $W$ is smooth and $\rho$ is a morphism. 
 Then $\phi$ is a $(K_{X}+B+M_{X})$-Mori fiber space$/Z$. In particular, $-(K_{X}+B+M_{X})$ is ample$/S$. Therefore, we may pick a general ample$/Z$ $\Rr$-divisor $A$ on $S$ such that $-(K_{X}+B+M_{X})+\phi^*A$ is ample$/Z$. We let $L$ be a general element of $|-(K_{X}+B+M_{X})+\phi^*A|_{\Rr/Z}$ and $L_W:=\rho^*L=(\rho^{-1})_*L$. Then $L_W$ is big and nef$/Z$, $K_{X}+B+L+M_X\sim_{\Rr,S}0$ and $K_{X}+B+L+M_{X}$ is nef$/Z$.
 
 Finally, we pick a general big and nef$/Z$ $\Rr$-divisor $H_W$ on $W$ such that
\begin{itemize}
\item $(W,B_W+2(L_W+H_W)+M_W)$ is $\Qq$-factorial g-terminal, and
    \item $K_W+B_W+H_W+M_W$ is pseudo-effective$/Z$, 
\end{itemize}
and pick a general $\phi$-vertical curve $\Sigma$ on $X$. We construct the \emph{Sarkisov program$/Z$ of $(X,B+M_{X})$ with scaling of $(L_W,H_W)$} in the following way. 
\begin{itemize}
    \item[\textbf{Step 1}] We define $X_0:=X,B_0:=B,\rho_0:=\rho,\phi_0:=\phi$, $L_0:=L, H_0:=\rho_*H_W$, $r_0:=\frac{H_0\cdot\Sigma}{L_0\cdot\Sigma}$, $\Sigma_0:=\Sigma$, and $(l_0,h_0):=(1,0)$.
    \item[\textbf{Step 2}] For any integer $i\geq 0$, suppose that we already have
    \begin{itemize}
        \item a $\Qq$-factorial gklt g-pair $(X_i,B_i+M_{X_i})$,
        \item a birational map $\rho_i: W\dashrightarrow X_i$,
        \item a $(K_{X_i}+B_i+M_{X_i})$-Mori fiber space$/Z$ $\phi_i: X_i\rightarrow S_i$,
        \item two $\Rr$-Cartier $\Rr$-divisors $L_i$ and $H_i$ on $X_i$, 
            \item two real number $0<l_i\leq 1$ and $0\leq h_i\leq 1$, 
            \item a general $\phi_i$-vertical curve $\Sigma_i$, and
            \item $r_i:=\frac{H_i\cdot\Sigma_i}{L_i\cdot\Sigma_i}>0$
    \end{itemize}
    such that
    \begin{itemize}
        \item $(W,B_W+l_iL_W+h_iH_W+M_W)\geq (X_i,B_i+l_iL_i+h_iH_i+M_{X_i})$,
        \item $B_i,L_i$ and $H_i$ are the birational transforms of $B_i,L_i$ and $H_i$ on $X_i$ respectively,
        \item $K_{X_i}+B_i+l_iL_i+h_iH_i+M_{X_i}\sim_{\Rr,S_i}0$, and
        \item $K_{X_i}+B_i+l_iL_i+h_iH_i+M_{X_i}$ is nef$/Z$,
    \end{itemize}
    then by Theorem \ref{thm: scaling sarkisov}, there exists 
    \begin{itemize}
        \item a $\Qq$-factorial gklt g-pair $(X_{i+1},B_{i+1}+M_{X_{i+1}})$,
        \item a birational map $\rho_{i+1}: W\dashrightarrow X_{i+1}$,
        \item a $(K_{X_{i+1}}+B_{i+1}+M_{X_{i+1}})$-Mori fiber space$/Z$ $\phi_{i+1}: X_{i+1}\rightarrow S_{i+1}$,
        \item two $\Rr$-Cartier $\Rr$-divisors $L_{i+1}$ and $H_{i+1}$ on $X_{i+1}$, 
            \item two real number $0\leq l_{i+1}\leq l_i$ and $h_i\leq h_{i+1}\leq 1$,
            \item a $\phi_{i+1}$-vertical curve $\Sigma_{i+1}$, 
            \item a real number $r_{i+1}:=\frac{H_{i+1}\cdot\Sigma_{i+1}}{L_{i+1}\cdot\Sigma_{i+1}}$, and
            \item a Sarkisov link$/Z$ $f_i: X_i\dashrightarrow X_{i+1}$ as in \textbf{Case 1}, or \textbf{Case 2.1}, or \textbf{Case 2.2}, or \textbf{Case 3.1}, or \textbf{Case 3.2}, or \textbf{Case 3.3} of Theorem \ref{thm: scaling sarkisov},
    \end{itemize}
    such that
     \begin{itemize}
        \item $(W,B_W+l_{i+1}L_W+h_{i+1}H_W+M_W)\geq (X_{i+1},B_{i+1}+l_{i+1}L_{i+1}+h_{i+1}H_{i+1}+M_{X_{i+1}})$,
        \item $B_{i+1},L_{i+1}$ and $H_{i+1}$ are the birational transforms of $B_{i},L_{i}$ and $H_{i}$ on $X_{i+1}$ respectively,
        \item $K_{X_{i+1}}+B_{i+1}+l_{i+1}L_{i+1}+h_{i+1}H_{i+1}+M_{X_{i+1}}\sim_{\Rr,S_{i+1}}0$, 
        \item $K_{X_{i+1}}+B_{i+1}+l_{i+1}L_{i+1}+h_{i+1}H_{i+1}+M_{X_{i+1}}$ is nef$/Z$, and
        \item $r_i\geq r_{i+1}>0$.
    \end{itemize}
    Notice that the assumptions hold when $i=0$.
\item[\textbf{Step 3}] If $l_{i+1}=0$, we stop and let $n:=i+1$. Otherwise, we replace $i$ with $i+1$ and return to \textbf{Step 2}.
\end{itemize}
The following diagram gives the birational maps and Mori fiber spaces in this construction:
\begin{center}$\xymatrix{
 & & &W\ar@{-->}[dlll]_{\rho_0}\ar@{-->}[dll]^{\rho_1}\ar@{-->}[d]^{\rho_i}\ar@{-->}[drr]^{\rho_n}& & & \\
      X_0 \ar@{-->}[r]_{f_0}\ar@{->}[d]^{\phi_0}&X_1\ar@{-->}[r]_{f_1}\ar@{->}[d]^{\phi_1}&\dots\ar@{-->}[r]&X_i\ar@{-->}[r]^{f_i}\ar@{->}[d]^{\phi_i}&\dots\ar@{-->}[r]&X_n\ar@{->}[d]^{\phi_n}\ar@{-->}[r]^{f_n}& \dots \\
    S_0 & S_1 & & S_i& &S_n & }$
\end{center}
\end{cons}

\begin{lem}\label{lem: sarkisov scaling terminate}
Assumptions and notation as in Construction \ref{cons: sarkisov scaling}. Then
\begin{enumerate}
    \item there are only finitely many possibilities of $X_i$ up to isomorphism, and
    \item the Sarkisov program of $(X,B+M_X)$ with scaling of $(L_W,H_W)$ terminates, i.e. there exists an integer $n>0$ such that $l_n=0$.
\end{enumerate}
\end{lem}
\begin{proof}
We construct two a ample$/Z$ $\Rr$-divisors $\Gamma_W$ and $N_W$ on $W$, boundary divisors $\Delta_{i,W}$ for every $i\gg 0$, and a finite dimensional affine subspace $\mathcal{V}$ of $\Weil_{\Rr}(W)$ in the following way:
\begin{itemize}
    \item Assume that $h_k>0$ for some integer $k\geq 0$. In this case, since $H_W$ is big and nef$/Z$, we may write $H_W\sim_{\Rr,Z}C_W+G_W$, such that $C_W$ is ample$/Z$, $G_W$ is effective, and $(W,B_W+L_W+H_W+C_W+G_W+M_W)$ is g-terminal. We let $\Gamma_W\sim_{\Rr,Z}M_W+\frac{h_k}{2}C_W$ be a general ample$/Z$ $\Rr$-divisor on $W$ such that 
    $$(W,B_W+l_iL_W+\frac{h_k}{2}C_W+h_iG_W+\Gamma_W+(h_i-h_k)C_W)$$ 
    is klt for every $i\geq k$. We define $N_W:=\frac{h_k}{2}C_W$, $\Delta_{i,W}:=B_W+l_iL_W+\frac{h_k}{2}C_W+h_iG_W+\Gamma_W+(h_i-h_k)C_W$ for every $i\geq 0$, and let $\mathcal{V}$ be the finite dimensional affine subspace of $\Weil_{\Rr}(W)$ generated by the irreducible components of $B_W,L_W,G_W,\Gamma_W$ and $C_W$.
    \item Assume that $h_i=0$ for every integer $i\geq 0$. In this case, $l_i=1$ for every $i\geq 0$. Since $L_W$ is big and nef$/Z$, we may write $L_W\sim_{\Rr,Z}D_W+J_W$, such that $D_W$ is ample$/Z$, $J_W$ is effective, and $(W,B_W+L_W+D_W+J_W+M_W)$ is g-terminal.  We let $\Gamma_W\sim_{\Rr,Z}M_W+\frac{1}{2}D_W$ be a general ample$/Z$ $\Rr$-divisor on $W$ such that 
     $$(W,B_W+\frac{1}{2}D_W+J_W+\Gamma_W)$$ 
    is klt for every $i\geq k$. We define $N_W:=\frac{1}{2}D_W$, $\Delta_{i,W}:=B_W+\frac{1}{2}D_W+J_W+\Gamma_W)$ for every $i\geq 0$, and $\mathcal{V}$ be the finite dimensional affine subspace of $\Weil_{\Rr}(W)$ generated by the irreducible components of $B_W,J_W$ and $\Gamma_W$.
\end{itemize}
Since  $K_{X_i}+B_i+l_iL_i+h_iH_i+M_{X_i}$ is nef$/Z$ for every $i$ and $K_{W}+B_W+l_iL_W+h_iH_W+M_W\sim_{\Rr,Z}K_W+\Delta_{i,W}$, $\rho_i: W\dashrightarrow X_i$ is a weak log canonical model$/Z$ of $(W,\Delta_{i,W})$. Since $\Delta_{i,W}\in\mathcal{L}_{N_W}(\mathcal{V})$ for every $i$, we deduce (1) by applying Theorem \ref{thm: finiteness ltm}.

Suppose that (2) does not hold. Then $l_i>0$ for every $i>0$. By (1), there exists a strictly increasing sequence $\{i_j\}_{j=1}^{+\infty}$, such that $X_{i_j}\cong X_{i_k}$ for any $j,k\geq 1$.  By construction, we have $l_{i_j}=l_{i_k}$ and $h_{i_j}=h_{i,k}$ for every $j,k\geq 1$. Thus $l_i=l_{i_1}$ and $h_i=h_{i_1}$ for every $i\geq i_1$. In particular, $f_i$ is a Sarkisov link$/Z$ of as in \textbf{Case 2} or \textbf{Case 3} of Construction \ref{cons: cases of sarkisov link with scaling} for every $i\geq i_1$.

Suppose that $f_i$ is a Sarkisov link$/Z$ as in \textbf{Case 3} of Construction \ref{cons: cases of sarkisov link with scaling} for some $i\geq i_1$. By Lemma \ref{lem: sar scaling terminalization}(3), $f_{i+1}$ remains a Sarkisov link$/Z$ as in \textbf{Case 3} of  Construction \ref{cons: cases of sarkisov link with scaling}, which implies that $f_j$ is a Sarkisov link$/Z$ as in \textbf{Case 3} of Theorem \ref{thm: scaling sarkisov} for every $j\geq i$. By Theorem \ref{thm: scaling sarkisov}(6) and Lemma \ref{lem: sar scaling terminalization}(4), $f_j$ is not a Sarkisov link$/Z$ as in \textbf{Case 3.1} of Theorem \ref{thm: scaling sarkisov} for any $j\geq i$. Since $\rho(X_j)>0$ for every $j$, by Lemma \ref{lem: sar scaling terminalization}(5)(6.a), for every $j\gg i$,  $f_j$ is a Sarkisov link$/Z$ as in \textbf{Case 3.3} of Theorem \ref{thm: scaling sarkisov}. This contradicts Lemma \ref{lem: sar scaling terminalization}(6).

Therefore, we may assume that $f_j$ is a Sarkisov link$/Z$ as in \textbf{Case 2} of Theorem \ref{thm: scaling sarkisov} for every $j\geq i_1$. Since $\rho(X_j)\leq\rho(W)$ for every $j$, by Lemma \ref{lem: sar scaling terminalization}(1)(2.a), for every $j\gg i_1$, $f_j$ is a Sarkisov link$/Z$ as in \textbf{Case 2.2} of Theorem \ref{thm: scaling sarkisov}. This contradicts Lemma \ref{lem: sar scaling terminalization}(2).
\end{proof}


\section{Proof of the main theorem}
\begin{thm}\label{thm: existence sarkisov precise}
Assume that 
\begin{itemize}
    \item $W\rightarrow Z$ is a contraction between normal quasi-projective varieties,
    \item $(W,B_W+M_W)$ is a $\mathbb Q$-factorial gklt g-pair with associated nef$/Z$ $\bb$-divisor $M$, such that $K_W+B_W+M_W$ is not pseudo-effective$/Z$,
    \item $\rho_X: W\dashrightarrow X$ and $\rho_Y: W\dashrightarrow Y$ are two $(K_W+B_W+M_W)$-non-positive maps$/Z$ such that $(\rho_X)_*(K_W+B_W+M_W)=K_X+B_X+M_X$ and $(\rho_Y)_*(K_W+B_W+M_W)=K_X+B_Y+M_Y$,
    \item $\phi_X: X\rightarrow S_X$ is a $(K_X+B_X+M_X)$-Mori fiber space$/Z$ and $\phi_Y: X\rightarrow S_Y$ is a $(K_Y+B_Y+M_Y)$-Mori fiber space$/Z$.
\end{itemize}
\begin{center}$\xymatrix{
 & W\ar@{-->}[dl]_{\rho_X}\ar@{-->}[dr]^{\rho_Y}& \\
      X \ar@{->}[d]_{\phi_X}\ar@{-->}[rr]^{f}   &  & Y\ar@{->}[d]^{\phi_Y} \\
    S_X & &S_Y }$
\end{center}
Then
\begin{enumerate}
    \item the induced birational map $f: X\dashrightarrow Y$ is given by a finite sequence of Sarkisov links$/Z$, i.e. $f$ can be written as $X_0\dashrightarrow X_1\dots\dashrightarrow X_n\cong Y$, where each $X_{i}\dashrightarrow X_{i+1}$ is a Sarkisov link,
    \item $f$ is a Sarkisov program$/Z$ with double scaling, i.e., each $X_{i}\dashrightarrow X_{i+1}$ is a Sarkisov link$/Z$ as in Construction \ref{cons: sarkisov scaling},
    \item for any real number $\epsilon>0$, if $(W,B_W+M_W)$ is generalized $\epsilon$-lc, then $(X_i,B_i+M_{X_i})$ is generalized $\epsilon$-lc for every $i$, where each $B_i$ is the birational transform of $B_W$ on $X_i$.
\end{enumerate} 
\end{thm}
\begin{proof}
By Lemma \ref{lem: terminalization and mmp}, possibly replacing $(W,B_W+M_W)$ with a common log resolution of $(W,B_W+M_W)$, $(X,B_X+M_X)$ and $(Y,B_Y+M_Y)$, we may assume that $(W,B_W+M_W)$ is g-terminal, $\rho_X$ and $\rho_Y$ are morphisms, and $M=M_W$. By our assumptions, $-(K_X+B_X+M_X)$ is $\phi_X$-ample and $-(K_Y+B_Y+M_Y)$ is $\phi_Y$-ample. Therefore, there are
\begin{itemize}
    \item two ample$/Z$ $\Rr$-divisors $A_{S_X}$ and $A_{S_Y}$ on $S_X$ and $S_Y$ respectively, 
    \item two general ample$/Z$ $\Rr$-divisors $L_X$ and $H_Y$ on $X$ and $Y$ respectively, and
    \item two general big and nef$/Z$ $\Rr$-divisors $L_W$ and $H_W$ on $W$,
\end{itemize}
such that
\begin{itemize}
    \item $L_X\sim_{\Rr,Z}-(K_X+B_X+M_X)+\phi_X^*A_{S_X}$,
    \item $H_Y\sim_{\Rr,Z}-(K_Y+B_Y+M_Y)+\phi_Y^*A_{S_Y}$, and
    \item $(X,B_X+L_X+M_X)$ and $(Y,B_Y+H_Y+M_Y)$ are both gklt,
    \item $L_W=\rho_X^*L_X=(\rho_X^{-1})_*L_X$ and $H_W=\rho_X^*L_X=(\rho_Y^{-1})_*L_X$, and
    \item $(W,B_W+2(H_W+L_W)+M_W)$ is g-terminal.
\end{itemize}
By Lemma \ref{lem: sarkisov scaling terminate}, we may let $f': X:=X_0\dashrightarrow X_1\dots\dashrightarrow X_n$ be a Sarkisov program of $(X,B_X+M_X)$ with scaling of $(L_W,H_W)$ as in Construction \ref{cons: sarkisov scaling}. 

We show that $f=f'$ and $h_n=1$. By Lemma \ref{lem: sarkisov h<=1}, $h_n\leq 1$. If $h_n<1$, then since $(\rho_Y)_*(K_W+B_W+h_nH_W+M_W)\sim_{\Rr,S_Y}-(1-h_n)H_Y$, $\phi_Y$ is a $(K_Y+B_Y+h_nH_Y+M_Y)$-Mori fiber space$/Z$, which implies that $K_W+B_W+h_nH_W+M_W$ is not pseudo-effective$/Z$. However, since $K_{X_n}+B_{n}+h_nH_{n}+M_{X_n}$ is nef$/Z$ by construction, $K_W+B_W+h_nH_W+M_W$ is pseudo-effective$/Z$, a contradiction. Thus $h_n=1$.

Let $D_n$ be a general ample$/Z$ $\Rr$-divisor on $X_n$ and $D_Y,D_W$ the birational transforms of $D_n$ on $Y$ and $W$ respectively. Since $D_Y$ is big$/Z$, $D_Y$ is ample$/S_Y$. Pick $0<\epsilon\ll 1$ and let $\Delta_W\sim_{\Rr,Z}B_W+H_W+M_W+\epsilon D_n$ be an effective $\Rr$-divisor on $W$ such that $(W,\Delta_W)$ is klt. Then $W\dashrightarrow X_n$ and $W\dashrightarrow Y$ are both log canonical models$/Z$ of $(W,\Delta_W)$, which implies that $X_n\cong Y$.

Let $\Sigma_n$ be a $\phi_n$-vertical curve, then 
$$\phi_Y^*A_{S_Y}\cdot\Sigma_n=(K_{Y}+B_{Y}+H_{Y}+M_{Y})\cdot\Sigma_n=0,$$
which implies that $\Sigma_n$ is $\phi_Y$-vertical. Thus $\phi_n$ and $\phi_Y$ define the same Mori fiber space and the theorem follows.
\end{proof}

\begin{proof}[Proof of Theorem \ref{thm: existence generalized Sarkisov link}]
Since a $(K_W+B_W+M_W)$-MMP$/Z$ is a $(K_W+B_W+M_W)$-non-positive map$/Z$, Theorem \ref{thm: existence generalized Sarkisov link} follows from Theorem \ref{thm: existence sarkisov precise}.
\end{proof}










\begin{thebibliography}{99}

	\bibitem{BCHM10}
	C. Birkar, P. Cascini, C.D. Hacon and J. M\textsuperscript{c}Kernan,  \textit{Existence of minimal models for varieties of log general type}. J. Amer. Math. Soc. {\bf 23} (2010), no. 2, 405--468.		


		\bibitem{Bir16} C. Birkar, \textit{Singularities of linear systems and boundedness of Fano varieties}. arXiv: 1609.05543v1 (2016).

\bibitem{Bir18} C. Birkar, \textit{Log Calabi-Yau fibrations}. arXiv: 1811.10709v2 (2018).

	\bibitem{Bir19} C. Birkar, {Anti-pluricanonical systems on Fano varieties}. Ann. of Math. (2), \textbf{190} (2019), 345--463.

\bibitem{BM97} A. Bruno and K. Matsuki, \textit{Log Sarkisov proram}, Internati. J.Math. \textbf{8}, no.4 (1997), 451--494.

\bibitem{BZ16} C. Birkar and D.-Q. Zhang, \textit{Effectivity of Iitaka fibrations and pluricanonical systems of polarized pairs}. Publ. Math. Inst. Hautes Etudes Sci., \textbf{123} (2016), 283--331.

\bibitem{Cor95} A. Corti,\textit{Factoring  birational  maps  of  threefolds  after  Sarkisov}, J. Algebraic Geom., \textbf{4} (1995) no. 2, 223--254.

\bibitem{Fil18} S, Filipazzi, \textit{On a generalized canonical bundle formula and generalized adjunction}, to appear in Annali della Scuola Normale Superiore di Pisa, arXiv:1807.04847v3 (2018).

\bibitem{HL19}
J. Han and W. Liu, \emph{On a generalized canonical bundle formula for generically finite morphisms}, to appear in The Annales de l’Institut Fourier, arXiv:1905.12542 (2019).

\bibitem{HM09} C.D. Hacon and J. M\textsuperscript{c}Kernan, \textit{The Sarkisov program}, J. Algebraic Geom., \textbf{22}(2) (2013), 389--405.



\bibitem{Kaw08} Y. Kawamata, \textit{Flops connects minimal models}, Publ. Res. Inst. Math. Sci. \textbf{44}, no. 2 (2008), 419--423.

	\bibitem{KM98} J. Koll\'{a}r and S. Mori, {Birational geometry of algebraic varieties}. Cambridge Tracts in Math. \textbf{134}, Cambridge Univ. Press (1998).

\bibitem{KMM87} Y. Kawamata, K. Matsuda, and K. Matsuki, \textit{Introduction to the minimal model problem}, Algebraic geometry, Sendai, 1985, Adv. Stud. Pure Math., vol. 10, North-Holland, Amsterdam (1987).


\bibitem{Sar80} V. G. Sarkisov, \textit{Birational automorphisms of conic bundles}, Izv. Akad. Nauk SSSR Ser. Mat. \textbf{44} (1980), no. 4, 918–-945, 974.

\bibitem{Sar82} V. G. Sarkisov, \textit{On conic bundle structures}, Izv. Akad. Nauk SSSR Ser. Math., \textbf{46}(2) (1982): 371--408, 432.

	\bibitem{Sho92} V.V. Shokurov, {3--fold log flips}. Izv. Ross. Akad. Nauk Ser. Mat., \textbf{56} (1992), 105--203. 
	
\end{thebibliography}




\end{document}
