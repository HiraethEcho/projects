\documentclass[11pt]{amsart}

\usepackage{geometry}
\geometry{a4paper,top=3cm,bottom=3cm,left=2.5cm,right=2.5cm}

%\setcounter{tocdepth}{1}%delete the subsections in the contents


\hyphenpenalty=5000
\tolerance=1000

\usepackage{todonotes}
 \newcommand\liu[1]{\todo[color=green!40]{#1}} %Liu
 \newcommand\liuinline[1]{\todo[inline,color=green!40]{#1}} %Liu inline
 \newcommand\hacon[1]{\todo[color=yellow!40]{#1}} %Hacon
 \newcommand\haconinline[1]{\todo[inline,color=yellow!40]{#1}} %Hacon inline


\usepackage{amsfonts, amssymb, amscd}
\numberwithin{equation}{section}

\usepackage[symbol]{footmisc}
\renewcommand{\thefootnote}{\fnsymbol{footnote}}

\usepackage{bm}
\usepackage{verbatim}
%\usepackage{amssymb}
\usepackage{mathrsfs}
\usepackage{graphicx}
\usepackage{tikz-cd}
\usepackage{subcaption}
\usepackage{listings}
\usepackage{subfiles}
\usepackage[toc,page]{appendix}
\usepackage{mathtools}
\usepackage{comment}
\usepackage{enumerate}
\usepackage{enumitem}
\usepackage[all]{xy}

\usepackage{graphicx}
\graphicspath{{images/}}

\usepackage{appendix}
\usepackage{hyperref}
\hypersetup{
    colorlinks=true,
    citecolor=red,
    linkcolor=blue,
    filecolor=magenta,      
    urlcolor=red,
}
\lstset{
  basicstyle=\ttfamily,
  columns=fullflexible,
  frame=single,
  breaklines=true,
  postbreak=\mbox{\textcolor{red}{$\hookrightarrow$}\space},
}



\newcommand{\bQ}{\mathbb{Q}}
\newcommand{\bP}{\mathbb{P}}
\newcommand{\bA}{\mathbb{A}}
\newcommand{\cA}{\mathcal{A}}
\newcommand{\cO}{\mathcal{O}}
\newcommand{\oE}{\overline{E}}
\newcommand{\cF}{\mathcal{F}}
\newcommand{\bZ}{\mathbb{Z}}
\newcommand{\bb}{\bm{b}}
\newcommand{\Mm}{{\bf{M}}}
\newcommand{\PP}{{\bf{P}}}
\newcommand{\NN}{{\bf{N}}}
\newcommand{\Dd}{{\bf{D}}}
\newcommand{\oY}{\overline{Y}}
\newcommand{\oL}{\overline{L}}
\newcommand{\cI}{\mathcal{I}}
\newcommand{\ind}{\mathrm{ind}}
\newcommand{\Spec}{\mathrm{Spec}}
\newcommand{\id}{\mathrm{id}}
\newcommand{\exc}{\mathrm{exc}}



\newcommand{\Cc}{\mathbb{C}}
\newcommand{\KK}{\mathbb{K}}
\newcommand{\Pp}{\mathbb{P}}
\newcommand{\Qq}{\mathbb{Q}}
\newcommand{\Nn}{\mathbb{N}}
\newcommand{\QQ}{\mathbb{Q}}
\newcommand{\Rr}{\mathbb{R}}
\newcommand{\RR}{\mathbb{R}}
\newcommand{\Zz}{\mathbb{Z}}
\newcommand{\ZZ}{\mathbb{Z}}





\newcommand{\zz}{\mathbf{z}}
\newcommand{\xx}{\mathbf{x}}
\newcommand{\yy}{\mathbf{y}}
\newcommand{\ww}{\mathbf{w}}
\newcommand{\vv}{\mathbf{v}}
\newcommand{\uu}{\mathbf{u}}
\newcommand{\kk}{\mathbf{k}}
\newcommand{\Span}{\operatorname{Span}}
\newcommand{\alct}{a\operatorname{LCT}}
\newcommand{\vol}{\operatorname{vol}}
\newcommand{\Center}{\operatorname{center}}
\newcommand{\Cone}{\operatorname{Cone}}
\newcommand{\Exc}{\operatorname{Exc}}
\newcommand{\Ext}{\operatorname{Ext}}
\newcommand{\Fr}{\operatorname{Fr}}
\newcommand{\Bir}{\operatorname{Bir}}
\newcommand{\Aut}{\operatorname{Aut}}
\newcommand{\glct}{\operatorname{glct}}
\newcommand{\GLCT}{\operatorname{GLCT}}
\newcommand{\HH}{\operatorname{H}}
\newcommand{\Hom}{\operatorname{Hom}}
\newcommand{\rk}{\operatorname{rank}}
\newcommand{\red}{\operatorname{red}}
\newcommand{\Ker}{\operatorname{Ker}}
\newcommand{\Ima}{\operatorname{Im}}
\newcommand{\Nklt}{\operatorname{Nklt}}
\newcommand{\mld}{{\rm{mld}}}
\newcommand{\relin}{\operatorname{relin}}

\newcommand{\loc}{\mathrm{loc}}
\newcommand{\expsing}{\mathrm{exp}}
\newcommand{\lcm}{\operatorname{lcm}}
\newcommand{\Weil}{\operatorname{Weil}}
\newcommand{\lct}{\operatorname{lct}}
\newcommand{\LCT}{\operatorname{LCT}}
\newcommand{\CR}{\operatorname{CR}}
\newcommand{\proj}{\operatorname{Proj}}
\newcommand{\spec}{\operatorname{Spec}}
\newcommand{\Supp}{\operatorname{Supp}}
\newcommand{\Ngklt}{\operatorname{Ngklt}}
\newcommand{\Nlc}{\operatorname{Nlc}}
\newcommand{\Diff}{\operatorname{Diff}}
\newcommand{\codim}{\operatorname{codim}}
\newcommand{\mult}{\operatorname{mult}}
\newcommand{\Rct}{\operatorname{Rct}}
\newcommand{\RCT}{\operatorname{RCT}}
\newcommand{\Div}{\operatorname{Div}}
\newcommand{\cont}{\operatorname{cont}}

\newcommand{\la}{\langle}
\newcommand{\ra}{\rangle}
\newcommand{\lf}{\lfloor}
\newcommand{\rf}{\rfloor}


\newcommand{\Aa}{\mathcal{A}}
\newcommand{\CC}{\mathcal{C}}
\newcommand{\Bb}{\mathcal{B}}
\newcommand{\Ff}{\mathcal{F}}
\newcommand{\LCP}{\mathcal{LCP}}
\newcommand{\Oo}{\mathcal{O}}
\newcommand{\Ii}{\mathcal{I}}
\newcommand{\Jj}{\mathcal{J}}
\newcommand{\Ee}{\mathcal{E}}
\newcommand{\Hh}{\mathcal{H}}
\newcommand{\Ll}{\mathcal{L}}
\newcommand{\me}{\mathcal{E}}
\newcommand{\mo}{\mathcal{O}}
\newcommand{\nN}{\mathcal{N}}
\newcommand{\anN}{\mathcal{AN}}
\newcommand{\Tt}{\mathcal{T}}
\newcommand{\Ww}{\mathcal{W}}
\newcommand{\Xx}{\mathcal{X}}
\newcommand{\Ss}{\mathcal{S}}
\newcommand{\Yy}{\mathcal{Y}}


\newcommand{\BB}{\mathfrak{B}}

\newcommand{\NE}{\mathrm{NE}}
\newcommand{\Nef}{\mathrm{Nef}}
\newcommand{\Sing}{\mathrm{Sing}}
\newcommand{\Pic}{\mathrm{Pic}}
\newcommand{\reg}{\mathrm{reg}}
\newcommand{\creg}{\mathrm{creg}}
\newcommand\MLD{{\rm{MLD}}}
\newcommand\FT{{\rm{FT}}}
\newcommand{\crt}{{\rm{crt}}}
\newcommand{\CRT}{{\rm{CRT}}}
\newcommand{\Coeff}{{\rm{Coeff}}}
\newcommand\coeff{{\rm{coeff}}}



\newtheorem{thm}{Theorem}[section]
\newtheorem{conj}[thm]{Conjecture}
\newtheorem{cor}[thm]{Corollary}
\newtheorem{lem}[thm]{Lemma}
\newtheorem{prop}[thm]{Proposition}
\newtheorem{exprop}[thm]{Example-Proposition}
\newtheorem{claim}[thm]{Claim}

\theoremstyle{definition}
\newtheorem{defn}[thm]{Definition}
\newtheorem{ques}[thm]{Question}
\theoremstyle{definition}
\newtheorem{rem}[thm]{Remark}
\newtheorem{remdef}[thm]{Remark-Definition}
\newtheorem{deflem}[thm]{Definition-Lemma}
\newtheorem{ex}[thm]{Example}
\newtheorem{nota}[thm]{Notation}
\newtheorem{exlem}[thm]{Example-Lemma}
\newtheorem{cons}[thm]{Construction}
\newtheorem{code}[thm]{Code}

\newtheorem{theorem}{Theorem}[section]
\newtheorem{lemma}[theorem]{Lemma}
\newtheorem{proposition}[theorem]{Proposition}
\newtheorem{corollary}[theorem]{Corollary}
\newtheorem*{notation}{Notation ($\star$)}

\theoremstyle{definition}
\newtheorem{definition}[theorem]{Definition}
\newtheorem{example}[theorem]{Example}
\newtheorem{question}[theorem]{Question}
\newtheorem{remark}[theorem]{Remark}
\newtheorem{conjecture}[theorem]{Conjecture}

\begin{document}

\title{Relative Nakayama-Zariski decomposition and minimal models of generalized pairs}
\author{Jihao Liu and Lingyao Xie}


\address{Department of Mathematics, Northwestern University, 2033 Sheridan Rd, Evanston, IL 60208, USA}
\email{jliu@northwestern.edu}


\address{Department of Mathematics, The University of Utah, Salt Lake City, UT 84112, USA}
\email{lingyao@math.utah.edu}


\subjclass[2020]{14E30,14C20.14E05,14J17}
\date{\today}



\begin{abstract}
In this note, we prove some basic properties on the relative Nakayama-Zariski decomposition. We apply them to the study of generalized lc pairs. We prove the existence of log minimal models or Mori fiber spaces for (relative) generalized lc pairs, an analogue of a result of Hashizume-Hu. We also show that, for any generalized lc pair $(X,B+A,\Mm)/Z$ such that $K_X+B+A+\Mm_X\sim_{\Rr,Z}0$ and $B\geq 0,A\geq 0$, $(X,B,\Mm)/Z$ has either a log minimal model or a Mori fiber space. This is an analogue of a result of Birkar/Hacon-Xu and Hashizume in the category of generalized pairs.
\end{abstract}

\maketitle

\tableofcontents
\section{Introduction}

We work over the field of complex numbers $\mathbb C$.

The theory of \emph{generalized pairs} (\emph{g-pairs} for short) was introduced by Birkar and Zhang in \cite{BZ16} to tackle the effective Iitaka fibration conjecture. The structure of g-pairs naturally appears in the canonical bundle formula and sub-adjunction formulas \cite{Kaw98,FM00}. This theory has been used in an essential way in the proof of the Borisov-Alexeev-Borisov conjecture \cite{Bir19,Bir21}. We refer the reader to \cite{Bir20} for a more detailed introduction to the theory of g-pairs.

Recently, there are some progress towards the minimal model program theory for generalized pairs. In particular, in \cite{HL21}, Hacon and the first author proved the cone theorem, contraction theorem, and the existence of flips for $\Qq$-factorial glc g-pairs. However, some related results on the termination of flips and the existence of log minimal models and good minimal models for generalized pairs remain unknown. For example, we have the following results:
\begin{thm}\label{thm: hh20 1.5 and has19 1.1}
\begin{enumerate}
    \item (\cite[Theorem 1.5]{HH20}) Let $(X,B)/Z$ be a pair and $A\geq 0$ an ample$/Z$ $\Rr$-divisor such that $(X,\Delta:=B+A)$ is lc. Then $(X,\Delta)/Z$ has a good minimal model or a Mori fiber space.
    \item (\cite[Theorem 1.1]{Has19}; see \cite{Bir12,HX13} for the $\Qq$-coefficient case) Let $(X,B)/Z$ be a pair and $A\geq 0$ an $\Rr$-divisor such that $(X,B+A)$ is glc and $K_X+B+A\sim_{\Rr,Z}0$. Then:
    \begin{enumerate}
        \item $(X,B)/Z$ has either a Mori fiber space or a log minimal model $(Y,B_Y)/Z$.
        \item If $K_Y+B_Y$ is nef$/Z$, then $K_Y+B_Y$ is semi-ample$/Z$.
        \item If $(X,B)$ is $\Qq$-factorial dlt, then any $(K_X+B)$-MMP$/Z$ with scaling of an ample$/Z$ $\Rr$-divisor terminates.
    \end{enumerate}
\end{enumerate}
\end{thm}

In this paper, we further investigate the minimal model program for generalized pairs. We prove the following results, which can be considered as analogues of Theorem \ref{thm: hh20 1.5 and has19 1.1}:

\begin{thm}\label{thm: has22 3.17 rel ver intro}
Let $(X,B,\Mm)/U$ be an NQC glc g-pair and $A\geq 0$ an ample$/U$ $\Rr$-divisor such that $(X,\Delta:=B+A,\Mm)$ is glc. Then
\begin{enumerate}
    \item $(X,\Delta,\Mm)/U$ has a log minimal model or a Mori fiber space, and
    \item if $\Mm_X$ is $\Rr$-Cartier, then  $(X,\Delta,\Mm)/U$ has a good minimal model or a Mori fiber space.
\end{enumerate}
\end{thm}

\begin{thm}\label{thm: bir12 1.1 gpair}
Let $(X,B,\Mm)/U$ be an NQC glc g-pair, $X\rightarrow Z$ a projective morphism$/U$ between normal quasi-projective varieties, and $A\geq 0$ an $\Rr$-divisor such that $(X,B+A,\Mm)$ is glc and $K_X+B+A+\Mm_X\sim_{\Rr,Z}0$. Then
\begin{enumerate}
    \item $(X,B,\Mm)/Z$ has a log minimal model or a Mori fiber space, and
    \item if $(X,B,\Mm)$ is $\Qq$-factorial gdlt, then any $(K_X+B+\Mm_X)$-MMP$/Z$ with scaling of an ample$/Z$ $\Rr$-divisor terminates.
\end{enumerate}
\end{thm}

We expect Theorems \ref{thm: has22 3.17 rel ver intro} and \ref{thm: bir12 1.1 gpair} to 
play important roles in the minimal model program theory for generalized lc pairs.

Note that when $\Mm=\bf{0}$, Theorem \ref{thm: has22 3.17 rel ver intro} is exactly Theorem \ref{thm: hh20 1.5 and has19 1.1}(1) and Theorem \ref{thm: bir12 1.1 gpair} is exactly Theorem \ref{thm: hh20 1.5 and has19 1.1}(2.a, 2.c). For technical reasons, at the moment, we cannot remove the ``$\Mm_X$ is $\Rr$-Cartier" assumption in Theorem \ref{thm: has22 3.17 rel ver intro}(2).

We still expect the analogue of Theorem \ref{thm: hh20 1.5 and has19 1.1}(2.b) to be true. That is, we expect that any log minimal model of $(X,B,\Mm)/Z$ is a good minimal model under the setting of Theorem \ref{thm: bir12 1.1 gpair}. This is because such $K_X+B+\Mm_X$ is log abundant$/U$ with respect to $(X,B,\Mm)$ by Theorem \ref{thm: has20b 4.1 rel ver} below. However, the following example shows that the question is very subtle as ``log abundant" does not imply semi-ampleness in general for glc g-pairs:

\begin{ex}\label{ex: log abundant not semiample}
Let $C_0$ be a nodal cubic in $\Pp^2$ and $l$ the hyperplane class on $\Pp^2$. Let $P_1,P_2,...,P_{12}$ be twelve distinct points on $C_0$ which are different from the nodal point. Let
$$
\mu:X=\text{Bl}_{\{P_1,...,P_{12}\}}\to\Pp^2
$$
to be the blow-up of $\Pp^2$ at the chosen points with the exceptional divisor $E=\sum_{i=1}^{12}E_i$, where $E_i$ is the prime exceptional divisor over $P_i$ for each $i$. Let $H:=\mu^*l$ and $C:=\mu^{-1}_*C_0$. Then $C\cong C_0$, $C\in|3H-E|$, and $K_X+C=\mu^*(K_{\Pp^2}+C_0)=0$. 


We consider the big divisor $M=4H-E\sim H+C$. Since $H$ is semi-ample and $M\cdot C=0$, $M$ is nef. Notice that $\Oo_C(M)=\Oo_{C_0}(4l-\sum_{i=1}^{12}P_i)$ and $\Pic^0(C)\cong\mathbb G_m$, where $\mathbb G_m$ is the multiplication group of $\Cc^*$.

\begin{enumerate}
    \item Suppose that $P_1,...,P_{12}$ are in general position such that $\Oo_C(M)$ is a non-torsion in $\Pic^0(C)$. Then $M$ can never be semi-ample since $M|_C$ is not. However, the normalization $C^n$ of $C$ is $\Pp^1$, so $M|_{C^n}$ is semi-ample. This gives a glc pair $(X,C,\Mm:=\overline{M})$ such that both $M$ and $K_X+C+M$ are nef and log abundant with respect to $(X,C,\Mm)$, but $K_X+C+M$ is not semi-ample. One can further take the blow-up of the nodal point and take the crepant pullback to make each lc center normal.
    \item Suppose that $P_1,...,P_{12}$ are the intersection points of $C_0$ with a general quartic curve $Q_0\in|4l|$. Let $Q$ be the birational transform of $Q_0$ on $X$. Then $M\sim Q\sim H+C$ is semi-ample and defines a projective birational contraction $f: X\to Y$ which contracts exactly the nodal curve $C$. Let $M'=H-3E_1$, then $\Oo_C(M')=\Oo_{C_0}(l-3P_1)$ is a non-torsion since $Q_0$ is general. Therefore $M'$ is not $\Qq$-linearly equivalent to 0 over $Y$ (which also implies that $f(M')$ is not $\Qq$-Cartier). This gives a glc pair $(X,C,\Mm':=\overline{M'})/Y$ such that  both $M'$ and $K_X+C+M'$ are log abundant and numerically trivial over $Y$ but $K_X+C+M'$ is not semi-ample over $Y$.
\end{enumerate}
\end{ex}

We refer the reader to \cite{BH22} for some other interesting examples on the failure of positivity results for generalized pairs.

\medskip



To prove our main theorems, the central idea is to combine the methods in \cite{Has22} (some originated in \cite{Has20a,Has20b,HH20}) and \cite{HL21}. In particular, we need to generalize many results in \cite{Has22} for projective varieties $X$ to normal quasi-projective varieties $X$ equipped with projective morphisms $\pi: X\rightarrow U$. Despite their similarities, a major difficulty is the use of Nakayama-Zariski decomposition \cite[III. \S 1]{Nak04}, which is usually applied to projective varieties only. It is important to remark that the relative Nakayama-Zariski decomposition \cite[III. \S 4]{Nak04} does not always behave as good as the global Nakayama-Zariski decomposition (cf. \cite{Les16}), and we are lack of references for even the most basic properties of them. In this note, we will study the behavior and basic properties on the relative Nakayama-Zariski decomposition. We refer the reader to \cite{LT21} for further applications of the relative Nakayama-Zariski decomposition on the minimal model theory for generalized pairs.

\medskip

\noindent\textit{Idea of the proof}. It is important to notice that, Theorems \ref{thm: has22 3.17 rel ver intro} and \ref{thm: bir12 1.1 gpair} both have some ``$\bb$-log abundant" conditions:
\begin{enumerate}
    \item In Theorem \ref{thm: has22 3.17 rel ver intro}, possibly replacing $(X,B,\Mm)$ with $(X,B,\Mm+\frac{1}{2}\bar A)$ and $A$ with $\frac{1}{2}A$, we may assume that $\Mm$ is $\bb$-log abundant with respect to $(X,B,\Mm)$.
    \item In Theorem \ref{thm: bir12 1.1 gpair}, $K_X+B+A+\Mm_X$ is automatically $\bb$-log abundant$/Z$ as it is $\Rr$-linearly trivial over $Z$.
\end{enumerate}
Therefore, one important goal of this paper is to study the minimal model program for g-pairs $(X,B,\Mm)$ with $\bb$-log abundant nef part $\Mm$ or with log abundant $K_X+B+\Mm_X$. Despite the technicality, the condition ``$\bb$-log abundant" is actually a very natural condition as it is preserved under adjunction. The key idea to study the minimal model program for such g-pairs is the following:
\begin{itemize}
    \item By applying the Iitaka fibration and the generalized canonical bundle formula, we reduce the questions to the cases when either $\kappa_{\iota}(X/U,K_X+B+\Mm_X)=0$ or when $\kappa_{\iota}(X/U,K_X+B+\Mm_X)=\dim X$ (see Section 4).
    \item When the invariant Iitaka dimension is $0$, by abundance, the minimal model program behaves well (cf. Lemma \ref{lem: has19 3.2 step 3 abu ver}). So we can reduce the question to the case when $K_X+B+\Mm_X$ is big$/U$.
    \item If $(X,B,\Mm)$ is gklt then we can apply \cite[Lemma 4.4(2)]{BZ16}. Otherwise, by induction on dimension, we can apply the the special termination results near $\Ngklt(X,B,\Mm)$.
\end{itemize}


\noindent\textit{Structure of the paper}. In section 2, we introduce some preliminary results. In particular, we will recall some results on the the minimal model program for generalized pairs that are already included in \cite[Version 2, Version 3]{HL21} (but may not appear in the published version). In section 3, we study the basic behavior of the relative Nakayama-Zariski decomposition. In section 4, we use the Iitaka fibration and the generalized canonical bundle formula to simplify the question. In section 5, we prove a special termination result for generalized pairs. In section 6,7,8, we use the relative Nakayama-Zariski decomposition to prove analogues of most results in \cite[Section 3]{Has22} (section 6), \cite[Theorem 3.14]{Has22} (section 7), and \cite[Theorem 4.1]{Has20b} (section 8) respectively. In section 9, we prove Theorems \ref{thm: has22 3.17 rel ver intro} and \ref{thm: bir12 1.1 gpair}.


\medskip

\noindent\textbf{Acknowledgement}. The authors would like to thank Christopher D. Hacon, Jingjun Han, Junpeng Jiao, Vladimir Lazi\'c, Yuchen Liu, Yujie Luo, Fanjun Meng, Nikolas Tsakanikas, and Qingyuan Xue for useful discussions. The authors would like to thank Kenta Hashizume for useful comments on an earlier manuscript of this paper. Part of the work is done during the visit of the first author to the University of Utah in March and April 2022, and the first author would like to thank their hospitality. The second author is partially supported by NSF research grants no: DMS-1801851, DMS-1952522 and by a grant from the Simons Foundation; Award Number: 256202.

Some parts of this note has overlap with results in \cite[Version 2 or Version 3]{HL21}. Since these results are not expected to be published in the final version of \cite{HL21} due to the simplification of the proofs of the main theorems of \cite{HL21}, for the reader's convenience, we include some of the results of \cite[Version 2 or Version 3]{HL21} in this paper and provide detailed proofs. The authors would like to thank Christopher D. Hacon for granting the text overlap.

We thank Xiaowei Jiang for useful comments for the first version of the paper.

\medskip


\section{Preliminaries}

We adopt the same notation as in \cite{KM98,BCHM10}. For g-pairs, we adopt the same notation as in \cite{HL21}, which is the same as \cite{Has20b,Has22} except that we use ``$a(E,X,B,\Mm)$ instead of ``$a(E,X,B+\Mm_X)$" to represent log discrepancies. This is because $(X,B+\Mm_X)$ is a sub-pair and the log discrepancies of this sub-pair may be different with the log discrepancies of the generalized pair $(X,B,\Mm)$.



\subsection{Iitaka dimensions}

\begin{lem}[{\cite[Version 2, Lemma 2.9]{HL21}, cf. \cite[V. 2.6(5) Remark]{Nak04}}]\label{lem: num dimension >=0 imply pe}
Let $X$ be a normal projective variety and $D$ an $\Rr$-Cartier $\Rr$-divisor on $X$ such that $\kappa_{\sigma}(D)\geq 0$. Then $D$ is pseudo-effective.
\end{lem}
\begin{proof}
By definition, there exists a Cartier divisor $A$ on $X$ such that $\sigma(D;A)\geq 0$. In particular, there exists a sequence of strictly increasing positive integers $m_i$, such that $\dim H^0(X,\lfloor m_iD\rfloor+A)>0$, hence $\lfloor m_iD\rfloor+A$ is effective for any $i$. Thus $m_iD+A$ is effective for any $i$, hence $D+\frac{1}{m_i}A$ is effective for any $i$. Thus $D$ is the limit of the effective $\Rr$-divisors $D+\frac{1}{m_i}A$, hence $D$ is pseudo-effective.
\end{proof}

\begin{lem}[{cf. \cite[Version 2, Lemma 2.10]{HL21}}]\label{lem: property of numerical and Iitaka dimension} Let $\pi: X\rightarrow U$ be a projective morphism from a normal variety to a variety, and $D$ an $\Rr$-Cartier $\Rr$-divisor on $X$. Then:
\begin{enumerate}
    \item $D$ is big$/U$ if and only if $\kappa_{\sigma}(X/U,D)=\dim X-\dim U$.
    \item Let $D_1,D_2$ be two $\Rr$-Cartier $\Rr$-divisors on $X$. Suppose that $D_1\sim_{\mathbb R,U}E_1\geq 0$ and $D_2\sim_{\mathbb R,U}E_2\geq 0$ for some $\Rr$-divisors $E_1,E_2$ such that $\Supp E_1=\Supp E_2$. Then $\kappa_{\sigma}(X/U,D_1)=\kappa_{\sigma}(X/U,D_2)$ and $\kappa_{\iota}(X/U,D_1)=\kappa_{\iota}(X/U,D_2)$.
    \item Let $f: Y\rightarrow X$ be a surjective birational morphism and $D_Y$ an $\Rr$-Cartier $\Rr$-divisor on $Y$ such that $D_Y=f^*D+E$ for some $f$-exceptional $\Rr$-divisor $E\geq 0$. Then $\kappa_{\sigma}(Y/U,D_Y)=\kappa_{\sigma}(X/U,D)$ and $\kappa_{\iota}(Y/U,D_Y)=\kappa_{\iota}(X/U,D)$. 
    \item Let $g: Z\rightarrow X$ be a surjective morphism from a normal variety such that $Z$ is projective over $U$. Then $\kappa_{\sigma}(Z/U,f^*D)=\kappa_{\sigma}(X/U,D)$ and $\kappa_{\iota}(Z/U,f^*D)=\kappa_{\iota}(X/U,D)$.
    \item  Let $\bar D$ be an $\Rr$-Cartier $\Rr$-divisor on $X$ such that $D\equiv_U\bar D$. Then $\kappa_{\sigma}(X/U,D)=\kappa_{\sigma}(X/U,\bar D)$.
    \item  Let $\phi: X\dashrightarrow X'$ be a partial $D$-MMP$/U$ and let $D':=\phi_*D$. Then $\kappa_{\sigma}(X/U,D)=\kappa_{\sigma}(X'/U,D')$ and $\kappa_{\iota}(X/U,D)=\kappa_{\iota}(X'/U,D')$
\end{enumerate}
\end{lem}
\begin{proof}
For (1)-(5), let $F$ be a very general fiber of the Stein factorization of $\pi$. Possibly replacing $X$ with $F$, $U$ with $\{pt\}$, and $D,D_1,D_2,\bar D$ with $D|_F,D_1|_F,D_2|_F,\bar D|_F$ respectively, we may assume that $X$ is projective and $U=\{pt\}$. (2) follows from \cite[Remark 2.8(1)]{HH20} and (3)(4) follow from \cite[Remark 2.8(2)]{HH20}.

To prove (1)(5), let $h: \tilde X\rightarrow X$ be a resolution of $X$. By (4), we may replace $X$ with $\tilde X$, $D$ with $h^*D$, and $\bar D$ with $h^*\bar D$, and assume that $X$ is smooth. 

If $D$ is big, then  $\kappa_{\sigma}(D)=\dim X$ by definition. If $\kappa_{\sigma}(D)=\dim X$, then $D$ is pseudo-effective by Lemma \ref{lem: num dimension >=0 imply pe}, hence $D$ is big by \cite[V. 2.7(3) Proposition]{Nak04}. This gives (1).

To prove (5), notice that $D$ is pseudo-effective if and only if $\bar D$ is pseudo-effective. If $D$ is not pseudo-effective, then $\kappa_{\sigma}(D)=\kappa_{\sigma}(\bar D)=-\infty$ by Lemma \ref{lem: num dimension >=0 imply pe}. If $D$ is pseudo-effective, then (5) follows from from \cite[V. 2.7(1) Proposition]{Nak04}.


To prove (6), let $p: W\rightarrow X$ and $q: W\rightarrow X'$ be a common resolution such that $q=\phi\circ p$, then $p^*D=q^*D'+F$ for some $F\geq 0$ that is $q$-exceptional. By (3), we have
$$\kappa_{\sigma}(X/U,D)=\kappa_{\sigma}(W/U,p^*D)=\kappa_{\sigma}(W/U,q^*D'+F)=\kappa_{\sigma}(X'/U,D')$$
and
$$\kappa_{\iota}(X/U,D)=\kappa_{\iota}(W/U,p^*D)=\kappa_{\iota}(W/U,q^*D'+F)=\kappa_{\iota}(X'/U,D').$$
\end{proof}




\subsection{Preliminaries on the MMP for generalized pairs}

This subsection includes results in \cite[Version 2, Version 3]{HL21} that are not included in its final version.

\begin{lem}[{\cite[Version 3, Lemma 2.20]{HL21}, cf. \cite[Proposition 3.8]{HL18}}]\label{lem: rlinear version of hl18 3.8}
Let $(X,B,\Mm)/U$ be a $\Qq$-factorial glc g-pair such that $X$ is klt and $K_X+B+\Mm_X\equiv_{U}D_1-D_2$ (resp.  $\sim_{\mathbb R,U}D_1-D_2$) where $D_1\geq 0$, $D_2\geq 0$ have no common components. Suppose that $D_1$ is very exceptional over $U$. Then any $(K_X+B+\Mm_X)$-MMP$/U$ with scaling of an ample$/U$ $\Rr$-divisor either terminates with a Mori fiber space or contracts $D_1$ after finitely many steps. Moreover, if $D_2=0$, then this MMP terminates with a model $Y$ such that  $K_Y+B_Y+\Mm_Y\equiv_{U}0$ (resp. $\sim_{\mathbb R,U}0$), where $B_Y$ is the strict transform of $B$ on $Y$. 
\end{lem}
\begin{proof}
The numerical equivalence part of the lemma is exactly \cite[Proposition 3.8]{HL18}. Thus we can assume that $K_X+B+\Mm_X\sim_{\mathbb R,U}D_1-D_2$, $D_2=0$, and we only need to show that the MMP terminates with a model $Y$ such that $K_Y+B_Y+\Mm_Y\sim_{\mathbb R,U}0$, where $B_Y$ is the strict transform of $B$ on $Y$.

By the numerical equivalence part of the lemma, the MMP contracts $D_1$ after finitely many steps. We may let $\phi: X\dashrightarrow Y$ be the birational map corresponding to this partial MMP. Since $K_X+B+\Mm_X\sim_{\mathbb R,U}D_1$ and $\phi$ contracts $D_1$, we have $K_Y+B_Y+\Mm_Y\sim_{\mathbb R,U}0$, where $B_Y$ is the strict transform of $B$ on $Y$, and the lemma is proved.
\end{proof}

\begin{lem}[{\cite[Version 3, Lemma 2.25]{HL21}}]\label{lem: still an mmp under perturbation}
Let $X\rightarrow U$ be a projective morphism such that $X$ is normal quasi-projective. Let $D,A$ be two $\Rr$-Cartier $\Rr$-divisors on $X$ and let $\phi: X\dashrightarrow X'$ be a partial $D$-MMP$/U$. Then there exists a positive real number $t_0$, such that for any $t\in (0,t_0]$, $\phi$ is also a partial $(D+tA)$-MMP$/U$. Note that $A$ is not necessarily effective.
\end{lem}
\begin{proof}
We let
$$X:=X_0\dashrightarrow X_1\dashrightarrow\dots\dashrightarrow X_n=X'$$
be this partial MMP, and $D_i,A_i$ the strict transforms of $D$ and $A$ on $X_i$ respectively. Let $X_i\rightarrow Z_i$ be the $D_i$-negative extremal contraction of a $D_i$-negative extremal ray $R_i$ in this MMP for each $i$, then $D_i\cdot R_i<0$ for each $i$. Thus there exists a positive real number $t_0$, such that $(D_i+t_0A_i)\cdot R_i<0$ for each $i$. In particular, $(D_i+tA_i)\cdot R_i<0$ for any $i$ and any $t\in (0,t_0]$. Thus $\phi$ is a partial $(D+tA)$-MMP$/U$ for any $t\in (0,t_0]$.
\end{proof}

\begin{lem}[{\cite[Version 2, Lemma 2.37]{HL21}}]\label{lem: limit movable r divisors gpairs}
Let $(X,B,\Mm)/U$ be a $\Qq$-factorial NQC glc g-pair such that $X$ is klt, and $A\geq 0$ an ample$/U$ $\Rr$-divisor on $X$ such that $(X,B+A,\Mm)$ is glc and $K_X+B+A+\Mm_X$ is nef$/U$. Let $$(X,B,\Mm):=(X_0,B_0,\Mm)\dashrightarrow (X_1,B_1,\Mm)\dashrightarrow\dots\dashrightarrow (X_i,B_i,\Mm)\dashrightarrow\dots$$
be a $(K_X+B+\Mm_X)$-MMP$/U$ with scaling of $A$, and $A_i$ the strict transform of $A$ on $X_i$ for each $i$. Then there exists a positive integer $n$ and a positive real number $\epsilon_0$, such that $K_{X_j}+B_j+\epsilon A_j+\Mm_{X_j}$ is movable$/U$ for any $\epsilon\in (0,\epsilon_0)$ and $j\geq n$. In particular, $K_{X_j}+B_j+\Mm_{X_j}$ is a limit of movable$/U$ $\Rr$-divisors.
\end{lem}
\begin{proof}
Let $\lambda_i$ be the $i$-th scaling number of this MMP for each $i$, i.e.
$$\lambda_i:=\inf\{t\mid t\geq 0, K_{X_i}+B_i+tA_i+\Mm_{X_i}\text{ is nef/}U\}.$$
We may assume that this MMP does not terminate. By \cite[Theorem 2.8]{HL21} ($=$\cite[Version 3,Theorem 2.24]{HL21}), we have $\lim_{i\rightarrow+\infty}\lambda_i=0$.

Let $n$ be the minimal positive integer such that $X_{i}\dashrightarrow X_{i+1}$ is a flip for any $i\geq n$. For any $i$, $X\dashrightarrow X_{i}$ is a $(K_X+B+tA+\Mm_X)$-MMP$/U$ with scaling of $(1-t)A$ for any $t\in [\lambda_i,\lambda_{i-1})$. Since $X$ is $\Qq$-factorial klt, there exists $\Delta_t\sim_{\Rr,U}B+tA+\Mm_X$ such that $(X,\Delta_t)$ is klt and $\Delta_t$ is big for any $t\in [0,1]$. Thus $K_{X_i}+B_i+tA_i+\Mm_{X_i}$ is semi-ample$/U$ for any $i$ and any $t\in [\lambda_i,\lambda_{i-1})$. Let $\epsilon_0:=\lambda_n$, then for any $\epsilon\in (0,\epsilon_0)$, there exists $i\geq n$ such that $\epsilon\in [\lambda_i,\lambda_{i-1})$, and $K_{X_i}+B_i+\epsilon A_i+\Mm_{X_i}$ is semi-ample$/U$. Since $X_i\dashrightarrow X_j$ is small for any $i,j\geq n$, $K_{X_j}+B_j+\epsilon A_j+\Mm_{X_j}$ is movable$/U$ for any $j\geq n$ and $\epsilon\in (0,\epsilon_0)$, and $K_{X_j}+B_j+\Mm_{X_j}$ is a limit of movable$/U$ $\Rr$-divisors.
\end{proof}

\begin{lem}[{\cite[Version 2, Lemma 2.38]{HL21}}]\label{lem: limit of movable divisors mmp only contain flips}
Let $X\rightarrow U$ be a projective morphism such that $X$ is quasi-projective. Assume that $D$ is an $\Rr$-Cartier $\Rr$-divisor on $X$ such that $D$ is a limit of movable$/U$ $\Rr$-divisors on $X$, and let $\phi: X\dashrightarrow X'$ be a partial $D$-MMP$/U$. Then $\phi$ only contains flips.
\end{lem}
\begin{proof}
Since $D$ is a limit of movable$/U$ $\Rr$-divisors, $D$ is pseudo-effective$/U$, so $\phi$ only contains flips and divisorial contractions. 

If $\phi$ contains a divisorial contraction, let $\psi: X_1\rightarrow X_1'$ be the first divisorial contraction in $\phi$. Let $D_1$ be the strict transform of $D$ on $X_1$, then since $X\dashrightarrow X_1$ only contains flips, $D_1$ is a limit of movable$/U$ $\Rr$-divisors on $X_1$. Let $D_1':=\psi_*D_1$, then
$$D_1=\psi^*D_1'+F$$
for some $F\geq 0$ that is exceptional over $X_1'$.

Since $D_1$ is a limit of movable$/U$ divisors,  $D_1$ is also a limit of movable$/X_1'$ divisors. 
Thus for any very general $\psi$-exceptional curve $C$, $D_1\cdot C\geq 0$. By the general negativity lemma \cite[Lemma 3.3]{Bir12}, $-F\geq 0$. Thus $F=0$, and $\psi$ cannot be a $D_1$-negative extremal contraction, a contradiction. Thus $\phi$ only contains flips.
\end{proof}


\begin{lem}[{\cite[Version 2, Lemma 2.39]{HL21}}]\label{lem: termination over open subset}
Let $(X,B,\Mm)/U$ be a $\Qq$-factorial NQC gdlt g-pair. Let $U^0\subset U$ be a non-empty open subset such that $(X^0:=X\times_UU^0,B^0:=B\times_UU^0,\Mm^0:=\Mm\times_UU^0)/U^0$ has a log minimal model. Then any $(K_X+B+\Mm_X)$-MMP$/U$ with scaling of an ample$/U$ $\Rr$-divisor terminates over $U^0$ with a log minimal model of $(X^0,B^0,\Mm^0)/U^0$.
\end{lem}
\begin{proof}
We run a $(K_X+B+\Mm_X)$-MMP$/U$ with scaling of an ample$/U$ $\Rr$-divisor $A\geq 0$, such that $(X,B+A,\Mm)$ is glc and $K_X+B+A+\Mm_X$ is nef$/U$. If this MMP terminates then we are done. Otherwise, we may assume that this MMP does not terminate. Let
$$(X,B,\Mm):=(X_1,B_1,\Mm)\dashrightarrow (X_2,B_2,\Mm)\dashrightarrow\dots\dashrightarrow (X_i,B_i,\Mm)\dashrightarrow\dots$$
be this MMP. Let $A_i$ be the strict transform of $A$ on $X_i$ for each $i$ and let $\lambda_i$ the $i$-th scaling number of this MMP for each $i$, i.e.
$$\lambda_i:=\inf\{t\mid t\geq 0, K_{X_i}+B_i+tA_i+\Mm_{X_i}\text{ is nef/}U\}.$$
By \cite[Remark 3.21]{HL18} $\lambda_i\geq\lambda_{i+1}$ for each $i$, and $\lim_{i\rightarrow+\infty}\lambda_i=0$. We let $X_i^0:=X_i\times_UU^0,$ $B_i^0:=B_i\times_UU^0,$ and $A_i^0:=A_i\times_UU^0$. Let 
$$\mathcal{N}:=\{i\in\mathbb N^+\mid X^0_i\dashrightarrow X^0_{i+1} \text{ is not the identity map over }U^0\}.$$ 
There are two cases:

\medskip

\noindent\textbf{Case 1}. $\mathcal{N}$ is a finite set. In this case, let $n:=\max\{j\mid j\in\mathcal{N}\}+1$. Then $X^0_i\dashrightarrow X^0_{i+1}$ is the identity map over $U^0$ for any $i\geq n$. In this case, $K_{X_i^0}+B_i^0+\lambda_iA_i^0+\Mm^0_{X_i^0}$ is nef$/U^0$ for any $i\geq n$, hence $K_{X_n^0}+B_n^0+\lambda_iA_n^0+\Mm^0_{X_n^0}$ is nef$/U^0$ for any $i\geq n$. Since $\lambda=0, $ $K_{X_n^0}+B_n^0+\Mm^0_{X_n^0}$ is nef$/U^0$. Thus $(X_n^0,B_n^0,\Mm^0)/U^0$ is a log minimal model of $(X^0,B^0,\Mm^0)/U^0$.

\medskip

\noindent\textbf{Case 2}. $\mathcal{N}$ is not a finite set. We may write $\mathcal{N}=\{n_i\}_{i=1}^{+\infty}$ such that $n_i<n_{i+1}$ for each $i$, then we get a sequence of induced birational maps
$$(X^0,B^0,\Mm^0)=(X^0_{n_1},B^0_{n_1},\Mm^0)\dashrightarrow (X^0_{n_2},B^0_{n_2},\Mm^0)\dashrightarrow\dots\dashrightarrow (X^0_{n_i},B^0_{n_i},\Mm^0)\dashrightarrow\dots,$$
which is a sequence of the $(K_{X^0}+B^0+\Mm^0_{X^0})$-MMP$/U^0$ with scaling of $A^0:=A\times_UU^0$. Since $(X^0,B^0,\Mm^0)/U^0$ has a log minimal model, by \cite[Theorem 2.8]{HL21} ($=$\cite[Version 3,Theorem 2.24]{HL21}), this MMP terminates, a contradiction.
\end{proof}


\begin{lem}[{\cite[Version 2, Lemma 2.40]{HL21}}]\label{lem: termination along a very general fiber}
Let $(X,B,\Mm)/U$ be a $\Qq$-factorial NQC glc g-pair, $X$ is klt ,and the induced morphism $\pi: X\rightarrow U$ is a contraction. Let $F$ be a very general fiber of $\pi$, and $(F,B_F,\Mm^F)$ the projective generalized pair given by the adjunction
$$K_F+B_F+\Mm^F_F:=(K_X+B+\Mm_X)|_F.$$
Assume that $(F,B_F,\Mm^F)$ has a log minimal model. Then any $(K_X+B+\Mm_X)$-MMP$/U$ with scaling of an ample$/U$ $\Rr$-divisor terminates along $F$ with a log minimal model of $(F,B_F,\Mm^F)$.
\end{lem}
\begin{proof}
We may assume that $F=\pi^{-1}(z)$ for some very general point $z\in U$. We run a $(K_X+B+\Mm_X)$-MMP$/U$ with scaling of an ample$/U$ $\Rr$-divisor $A\geq 0$, such that $(X,B+A,\Mm)$ is glc and $K_X+B+A+\Mm_X$ is nef$/U$. If this MMP terminates then we are done. Otherwise, we may assume that this MMP does not terminate. Let
$$(X,B,\Mm):=(X_0,B_0,\Mm)\dashrightarrow (X_1,B_1,\Mm)\dashrightarrow\dots\dashrightarrow (X_i,B_i,\Mm)\dashrightarrow\dots$$
be this MMP. Let $A_i$ be the strict transform of $A$ on $X_i$ for each $i$ and let $\lambda_i$ the $i$-th scaling number of this MMP for each $i$, i.e.
$$\lambda_i:=\inf\{t\mid t\geq 0, K_{X_i}+B_i+tA_i+\Mm_{X_i}\text{ is nef/}U\}.$$
By \cite[Remark 3.21, Theorem 4.1]{HL18}, $\lambda_i\geq\lambda_{i+1}$ for each $i$, and $\lambda:=\lim_{i\rightarrow+\infty}\lambda_i=0$. 

Let $\pi_i: X_i\rightarrow U$ be the induced morphism for each $i$. Since $z$ is a very general point, we may let $F_i:=\pi_i^{-1}(z)$, $(F_i,B_{F_i},\Mm^{F})$  the projective generalized pair given by the adjunction
$$K_{F_i}+B_{F_i}+\Mm^{F}_{F_i}:=(K_{X_i}+B_i+\Mm_{X_i})|_{F_i},$$
and $A_{F_i}:=A_i|_{F_i}$. Note that the nef part $\Mm^F$ does not depend on $i$ by the construction of generalized adjunction \cite[Definition 4.7]{BZ16}. Let 
$$\mathcal{N}:=\{i\in\mathbb N\mid F_i\dashrightarrow F_{i+1} \text{ is not the identity map}\}.$$ 
There are two cases:

\medskip

\noindent\textbf{Case 1}. $\mathcal{N}$ is a finite set. In this case, let $n:=\max\{j\mid j\in\mathcal{N}\}+1$. Then $F_i\dashrightarrow F_{i+1}$ is the identity map for any $i\geq n$. In this case, $K_{F_i}+B_{F_i}+\lambda_iA_{F_i}+\Mm^F_{F_i}$ is nef for any $i\geq n$, hence $K_{F_n}+B_{F_n}+\lambda_iA_{F_n}+\Mm^F_{F_n}$ is nef for any $i\geq n$. Since $\lambda=0, $ $K_{F_n}+B_{F_n}+\Mm^F_{F_n}$ is nef. Thus $(F_n,B_{F_n},\Mm^F)$ is a log minimal model of $(F,B_F,\Mm^F)$.

\medskip

\noindent\textbf{Case 2}. $\mathcal{N}$ is not a finite set. We may write $\mathcal{N}=\{n_i\}_{i=1}^{+\infty}$ such that $n_i<n_{i+1}$ for each $i$, then we get a sequence of induced birational maps
$$(F,B_F,\Mm^F)=(F_{n_1},B_{F_{n_1}},\Mm^0)\dashrightarrow (F_{n_2},B_{F_{n_2}},\Mm^0)\dashrightarrow\dots\dashrightarrow (F_{n_i},B_{F_{n_i}},\Mm^0)\dashrightarrow\dots,$$
which is a sequence of the $(K_{F}+B_F+\Mm^F_F)$-MMP with scaling of $A_F:=A|_F$. Since $(F,B_F,\Mm^F)$ has a log minimal model, by \cite[Theorem 2.8]{HL21} ($=$\cite[Version 3,Theorem 2.24]{HL21}), this MMP terminates, a contradiction.
\end{proof}


\begin{lem}[{\cite[Version 2, Lemma 2.43]{HL21}}]\label{lem: gmmp scaling numbers go to 0}
Let $(X,B,\Mm)/U$ be a $\Qq$-factorial NQC glc g-pair such that $X$ is klt. Let $H\geq 0$ be an $\Rr$-divisor on $X$ such that $(X,B+H,\Mm)$ is glc and $K_X+B+H+\Mm_X$ is nef$/U$. Assume that $(X,B+\mu H,\Mm)/U$ has a log minimal model for any $\mu\in (0,1]$. Then we can construct a $(K_X+B+\Mm_X)$-MMP$/U$ with scaling of $H$:
$$(X,B,\Mm):=(X_0,B_0,\Mm)\dashrightarrow (X_1,B_1,\Mm)\dashrightarrow\dots\dashrightarrow (X_i,B_i,\Mm)\dashrightarrow\dots.$$
Let $H_i$ be the strict transform of $H$ on $X_i$ for each $i$, and let
$$\lambda_i:=\inf\{t\mid t\geq 0, K_{X_i}+B_i+tH_i+\Mm_{X_i}\text{ is nef/}U\}$$
be the $i$-th scaling number of this MMP for each $i$. Then this MMP
\begin{enumerate}
    \item either terminates after finitely many steps, or
    \item does not terminate and $\lim_{i\rightarrow+\infty}\lambda_i=0$.
\end{enumerate}
\end{lem}
\begin{proof}
If $\lambda_0=0$ then there is nothing left to prove. So we may assume that $\lambda_0>0$. By \cite[Lemma 3.17]{HL18}, we may pick $\lambda_0'\in (0,\lambda_0)$ such that any sequence of the $(K_X+B+\lambda_0'H+\Mm_X)$-MMP$/U$ is $(K_X+B+\lambda_0H+\Mm_X)$-trivial. 

By \cite[Theorem 2.8]{HL21} ($=$\cite[Version 3, Theorem 2.24]{HL21}), we may run a $(K_X+B+\lambda_0'H+\Mm_X)$-MMP$/U$ with scaling of a general ample$/U$ divisor, which terminates with a log minimal model. We let
$$(X,B,\Mm):=(X_0,B_0,\Mm)\dashrightarrow (X_1,B_1,\Mm)\dashrightarrow\dots\dashrightarrow (X_{k_1},B_{k_1},\Mm)$$
be this sequence of the MMP$/U$. Then this sequence consists of finitely many steps of a $(K_X+B+\Mm_X)$-MMP$/U$ with scaling of $H$, with scaling numbers $\lambda_0=\lambda_1=\dots=\lambda_{k_1-1}$. Since
$$K_{X_{k_1}}+B_{k_1}+\lambda'_1H_{k_1}+\Mm_{X_{k_1}}$$
is nef$/U$, we have $\lambda_{k_1}\leq\lambda_1'<\lambda_1$. 

We may replace $(X,B,\Mm)/U$ with $(X_{k_1},B_{k_1},\Mm)/U$ and continue this process. If this MMP does not terminate, then we may let $\lambda:=\lim_{i\rightarrow+\infty}\lambda_i$. By our construction, $\lambda\not=\lambda_i$ for any $i$, and the lemma follows from \cite[Remark 3.21, Theorem 4.1]{HL18}. 
\end{proof}


\begin{defn}[{\cite[Version 3, Definition 3.1]{HL21}}]\label{defn: log smooth models}
Let $(X,B,\Mm)/U$ be a glc g-pair and $(W,B_W,\Mm)$ a log smooth model of $(X,B,\Mm)$. If any exceptional$/X$ prime divisor $D$ on $W$ such that $a(D,X,B,\Mm)>0$ is a component of $\{B_W\}$, then  $(W,B_W,\Mm)$ is called a \emph{proper log smooth model} of $(X,B,\Mm)$.
\end{defn}


\begin{lem}[{\cite[Version 3, Lemma 3.6]{HL21}}]\label{lem: existence of proper log smooth model}
Let $(X,B,\Mm)/U$ be a glc g-pair and $h: W\rightarrow X$ a log resolution of $(X,\Supp B)$ such that $\Mm$ descends to $W$. Then $(X,B,\Mm)$ has a proper log smooth model $(W,B_W,\Mm)$ for some $\Rr$-divisor $B_W$ on $W$.
\end{lem}
\begin{proof}
Assume that
$$K_W+h^{-1}_*B+\Gamma+\Mm_W=h^*(K_X+B+\Mm_X),$$
then $\Gamma$ is $h$-exceptional. Let $E=\Exc(h)$ be the reduced $h$-exceptional divisor. Then there exists a real number $\epsilon\in (0,1)$, such that for any component $D$ of $E$, if $\mult_D\Gamma<1$, then $\mult_D\Gamma<1-\epsilon$. We let $$B_W:=h^{-1}_*B+\epsilon\Gamma^{=1}+(1-\epsilon)E,$$
then $(W,B_W,\Mm)$ is a proper log smooth model of $(X,B,\Mm)$.
\end{proof}

\begin{lem}[{\cite[Version 3, Lemma 3.7]{HL21}}]\label{lem: proper log smooth model keep lc center}
Let $(X,B,\Mm)/U$ be a glc g-pair and $(W,B_W,\Mm)$ a proper log smooth model of $(X,B,\Mm)$ with induced morphism $h: W\rightarrow X$. Assume that
$$K_W+B_W+\Mm_W=h^*(K_X+B+\Mm_X)+E,$$
then:
\begin{enumerate}
    \item $\Supp B_W=\Supp h^{-1}_*B\cup\Exc(h)$.
    \item For any prime divisor $D$ on $W$ that is exceptional over $X$, $D$ is a component $E$ if and only if $a(D,X,B,\Mm)>0$.
    \item Any glc place of $(W,B_W,\Mm)$ is a glc place of $(X,B,\Mm)$. In particular, the image of any glc center of $(W,B_W,\Mm)$ on $X$ is a glc center of $(X,B,\Mm)$.
\end{enumerate}
\end{lem}
\begin{proof}
First we prove (1). By construction, $\Supp B_W\subset\Supp h^{-1}_*B\cup\Exc(h)$ and $\Supp h^{-1}_*B\subset\Supp B_W$. Let $D$ be a component of $\Exc(h)$. If $a(D,X,B,\Mm)=0$, then since $E\geq 0$, $D$ is a component of $B_W$. If $a(D,X,B,\Mm)>0$, by Definition \ref{defn: log smooth models}, $E$ is a component of $\{B_W\}$, hence a component of $B_W$. Thus $\Exc(h)\subset\Supp B_W$, and we have (1).

We prove (2). Let $D$ be a prime divisor on $W$. If $a(D,X,B,\Mm)>0$, then $D$ is a component of $E$ by the definition of log smooth models. If $a(D,X,B,\Mm)=0$, then
$$0=a(D,X,B,\Mm)=a(D,W,B_W-E,\Mm)\geq a(D,W,B_W,\Mm)\geq 0,$$
which implies that $a(D,W,B_W-E,\Mm)=a(D,W,B_W,\Mm)$, hence $\mult_DE=0$. Thus we have (2).

We prove (3). Let $D$ be a glc place of $(W,B_W,\Mm)$. Then the center of $D$ on $W$ is a stratum of $\lfloor B_W\rfloor$. If $\Center_WD\subset\Supp E$, then since $B_W+E$ is simple normal crossing, there exists a prime divisor $F$ that is a component of $\lfloor B_W\rfloor$ such that $\Center_WD\subset F$ and $F$ is a component of $E$. By (2), $a(F,X,B,\Mm)>0$. By Definition \ref{defn: log smooth models}, $F$ is a component of $\{B_W\}$, so $F$ cannot be a component of $\lfloor B_W\rfloor$, a contradiction. Thus $\Center_WD\not\subset\Supp E$. Therefore, any glc place of $(W,B_W,\Mm)$ is a glc place of  of $(W,B_W-E,\Mm)$, hence a glc place of $(X,B,\Mm)$, and we have (3).
\end{proof}



\begin{lem}[{\cite[Version 3, Lemma 3.19]{HL21}}]\label{lem: existence good minimal model under pullbacks weak glc case}
Let $(X,B,\Mm)/U$ and $(Y,B_Y,\Mm)/U$ be two $\Qq$-factorial NQC gdlt g-pairs, and $f: Y\rightarrow X$ a projective birational morphism such that
$$K_Y+B_Y+\Mm_Y=f^*(K_X+B+\Mm_X)+E$$
for some $E\geq 0$ that is exceptional over $X$. Assume that
\begin{enumerate}
\item $\Mm$ descends to $Y$,
\item $(Y,B_Y+\Exc(f))$ is log smooth, and
\item $(Y,B_Y,\Mm)/U$ has a weak glc model.
\end{enumerate}
Then $(X,B,\Mm)/U$ has a weak glc model.
\end{lem}
\begin{proof}
By our assumption, $K_Y+B_Y+\Mm_Y$ and $K_X+B+\Mm_X$ are pseudo-effective$/U$. Since $(X,B,\Mm)$ is gdlt, we may pick an ample$/U$ $\Rr$-divisor $A\geq 0$ on $X$ such that $(X,B+A,\Mm)$ is glc, and $K_X+B+A+\Mm_X$ and $A+\lfloor B\rfloor$ are ample. Since $(X,B,\Mm)$ is gdlt, $(X,\{B\},\Mm)$ is gklt, so we may pick an ample$/U$ $\Rr$-divisor $0\leq A'\sim_{\mathbb R,U}A+\lfloor B\rfloor$ such that $(X,\Delta:=\{B\}+A',\Mm)$ is gklt and $f$ is a log resolution of $(X,B+A)$. Since $\Delta\sim_{\Rr,U}B+A$, $K_X+\Delta+\Mm_X$ is big$/U$. We may write
$$K_Y+\Gamma+\Mm_Y=f^*(K_X+\Delta+\Mm_X)+F$$
for some $\Gamma\geq 0$, $F\geq 0$ such that $\Gamma\wedge F=0$. By our construction, $\Mm$ descends to $Y$, $(Y,B_Y+\Exc(f))$ is log smooth, $(Y,\Gamma)$ is log smooth, $(Y,\Gamma,\Mm)$ is gklt, and $K_Y+\Gamma+\Mm_Y$ is big$/U$. We let
$$\Delta_t:=t\Delta+(1-t)B\sim_{\mathbb R,U}B+tA$$
and
$$\Gamma_t:=t\Gamma+(1-t)B_Y$$
for any real number $t$. Then $(X,\Delta_t,\Mm)$ and $(Y,\Gamma_t,\Mm)$ are gklt for any $t\in (0,1]$, and $K_X+\Delta_t+\Mm_X$ and $K_Y+\Gamma_t+\Mm_Y$ are big$/U$ for any $t\in (0,1]$. 



Since $(Y,B_Y,\Mm)/U$ has a weak glc model, by \cite[Lemma 3.8]{HL21} ($=$\cite[Version 3, Lemma 3.15]{HL21}),  $(Y,B_Y,\Mm)/U$ has a log minimal model. Since $Y$ is klt, by \cite[Theorem 2.8]{HL21} ($=$\cite[Version 3,Theorem 2.24]{HL21}), we may run a $(K_Y+B_Y+\Mm_Y)$-MMP$/U$ with scaling of a general ample$/U$ divisor $H$, which terminates with a log minimal model $(Y',B_{Y'},\Mm)/U$ with induced birational map $\phi: Y\dashrightarrow Y'$. 

We let $\Gamma'_t$ be the strict transform of $\Gamma_t$ on $Y'$ for any $t$. By Lemmas \ref{lem: still an mmp under perturbation} and \cite[Lemma 3.17]{HL18}, there exists $t_0\in (0,1)$, such that 
\begin{itemize}
    \item $\phi$ is also a $(K_Y+\Gamma_{t_0}+\Mm_Y)$-MMP$/U$, and
    \item for any $t\in (0,t_0]$, any partial $(K_{Y'}+\Gamma'_{t}+\Mm_{Y'})$-MMP$/U$ is $(K_{Y'}+B_{Y'}+\Mm_{Y'})$-trivial.
\end{itemize}
Thus $(Y',\Gamma_{t_0}',\Mm)$ is gklt and $K_{Y'}+\Gamma'_{t_0}+\Mm_{Y'}$ is big$/U$. By \cite[Theorem 4.4(2)]{BZ16}, we may run a $(K_{Y'}+\Gamma'_{t_0}+\Mm_{Y'})$-MMP$/U$, which terminates with a log minimal model $(Y'',\Gamma''_{t_0},\Mm)/U$ of $(Y',\Gamma'_{t_0},\Mm)/U$. We let $\Gamma''_t$ be the strict transform of $\Gamma_t$ on $Y''$ for any $t$ and $B_{Y''}$ the strict transform of $B_Y$ on $Y''$. Since the induced birational map $\phi': Y'\dashrightarrow Y''$ is $(K_{Y'}+B_{Y'}+\Mm_{Y'})$-trivial, $K_{Y''}+B_{Y''}+\Mm_{Y''}$ is nef$/U$. Moreover, the induced map $\phi'\circ\phi: Y\dashrightarrow Y''$ does not extract any divisor, and is both $(K_Y+B_Y+\Mm_Y)$-non-positive and $(K_Y+\Gamma_{t_0}+\Mm_Y)$-non-positive. Thus $(Y'',B_{Y''},\Mm)/U$ is a weak glc model of $(Y,B_Y,\Mm)/U$ and $(Y'',\Gamma''_{t_0},\Mm)/U$ is a weak glc model of $(Y,\Gamma_{t_0},\Mm)/U$, hence $(Y'',\Gamma''_t,\Mm)/U$ is a weak glc model of $(Y,\Gamma_t,\Mm)/U$ for any $t\in [0,t_0]$.


By \cite[Lemma 3.5]{HL18}, we can run a $(K_X+B+\Mm_X)$-MMP$/U$ with scaling of $A$:
$$(X,B,\Mm):=(X_0,B_0,\Mm)\dashrightarrow (X_1,B_1,\Mm)\dashrightarrow\dots\dashrightarrow (X_i,B_i,\Mm)\dashrightarrow\dots.$$
Let $A_i,\Delta_i,\Delta_{t,i}$ be the strict transforms of $A,\Delta,\Delta_t$ on $X_i$ for any $t,i$ respectively, and let
$$\lambda_i:=\inf\{t\mid t\geq 0, K_{X_i}+B_i+tA_i+\Mm_{X_i}\text{ is nef/}U\}$$
be the scaling numbers. If this MMP terminates, then there is nothing left to prove as we already get a log minimal model for $(X,B,\Mm)/U$. Thus we may assume that this MMP does not terminate. By \cite[Theorem 2.8]{HL21} ($=$\cite[Version 3, Theorem 2.24]{HL21}), $\lim_{i\rightarrow+\infty}\lambda_i=0$. 

In particular, there exists a positive integer $n$ such that $\lambda_n<\lambda_{n-1}\leq t_0$. Since $\Delta_{t,i}\sim_{\mathbb R,U}B_i+tA_i$ for any $t$, $(X_n,\Delta_{\lambda_{n-1},n},\Mm)/U$ is a weak glc model of $(X,\Delta_{\lambda_{n-1}},\Mm)/U$ and $(X_n,\Delta_{\lambda_{n},n},\Mm)/U$ is a weak glc model of $(X,\Delta_{\lambda_{n}},\Mm)/U$. Since
$$K_Y+\Gamma_t+\Mm_Y=f^*(K_X+\Delta_t+\Mm_X)+tF+(1-t)E$$
for any $t$, by \cite[Lemma 3.10(1)]{HL21} ($=$\cite[Version 3, Lemma 3.17]{HL21}), $(X_n,\Delta_{\lambda_{n-1},n},\Mm)/U$ is a weak glc model of $(Y,\Gamma_{\lambda_{n-1}},\Mm)/U$ and $(X_n,\Delta_{\lambda_{n},n},\Mm)/U$ is a weak glc model of $(Y,\Gamma_{\lambda_{n}},\Mm)/U$. By our construction, $(Y'',\Gamma''_{\lambda_{n-1}},\Mm)/U$ is a weak glc model of $(Y,\Gamma_{\lambda_{n-1}},\Mm)/U$ and $(Y'',\Gamma''_{\lambda_{n}},\Mm)/U$ is a weak glc model of $(Y,\Gamma_{\lambda_{n}},\Mm)/U$.

We let $p: W\rightarrow X_n$ and $q: W\rightarrow Y''$ be a resolution of indeterminacy. 
\begin{center}$\xymatrix{
Y\ar@{->}[d]_{f}\ar@{-->}[r]^{\phi}& Y'\ar@{-->}[r]^{\phi'} & Y''& W\ar@{->}[d]^{p}\ar@{->}[l]_{q}\\
 X\ar@{-->}[r]& X_2\ar@{-->}[r] & \dots\ar@{-->}[r] & X_n
}$
\end{center}
By \cite[Lemma 3.5(1)]{HL21} ($=$\cite[Version 3, Lemma 3.9(1)]{HL21}),
$$p^*(K_{X_n}+\Delta_{\lambda_{n-1},n}+\Mm_{X_n})=q^*(K_{Y''}+\Gamma''_{\lambda_{n-1}}+\Mm_{Y''}).$$
and
$$p^*(K_{X_n}+\Delta_{\lambda_{n},n}+\Mm_{X_n})=q^*(K_{Y''}+\Gamma''_{\lambda_{n}}+\Mm_{Y''}).$$
Thus
$$p^*(K_{X_n}+t\Delta_i+(1-t)B_i+\Mm_{X_n})=q^*(K_{Y''}+t\Gamma''_1+(1-t)B_{Y''}+\Mm_{Y''})$$
when $t\in\{\lambda_{n-1},\lambda_{n}\}$. Since $\lambda_{n-1}\not=\lambda_n$, we have
$$p^*(K_{X_n}+t\Delta_i+(1-t)B_i+\Mm_{X_n})=q^*(K_{Y''}+t\Gamma''_1+(1-t)B_{Y''}+\Mm_{Y''})$$
for any $t$. In particular,
$$p^*(K_{X_n}+B_n+\Mm_{X_n})=q^*(K_{Y''}+B_{Y''}+\Mm_{Y''})$$
is nef$/U$, hence $K_{X_n}+B_n+\Mm_{X_n}$ is nef$/U$, and $\lambda_n=0$, a contradiction.
\end{proof}

\begin{thm}[{\cite[Version 3, Theorem 3.14]{HL21}}]\label{thm: existence good minimal model under pullbacks}
Let $(X,B,\Mm)/U$ and $(Y,B_Y,\Mm)/U$ be two NQC glc g-pairs and let $f: Y\rightarrow X$ be a projective birational morphism such that
$$K_Y+B_Y+\Mm_Y=f^*(K_X+B+\Mm_X)+E$$
for some $E\geq 0$ that is exceptional over $X$. Then $(X,B,\Mm)/U$ has a weak glc model (resp. log minimal model, good minimal model) if and only if $(Y,B_Y,\Mm)/U$ has a weak glc model (resp. log minimal model, good minimal model).
\end{thm}
\begin{proof}[Proof of Theorem \ref{thm: existence good minimal model under pullbacks}]
First we prove the weak glc model case. By \cite[Lemma 3.10(1)]{HL21} ($=$\cite[Version 3, Lemma 3.17]{HL21}), we only need to prove that if $(Y,B_Y,\Mm)/U$ has a weak glc model, then $(X,B,\Mm)/U$ has a weak glc model. Let $g: \bar X\rightarrow X$ be a gdlt modification of $(X,B,\Mm)$ such that
$$K_{\bar X}+\bar B+\Mm_{\bar X}=g^*(K_X+B+\Mm_X),$$
and let $p: W\rightarrow Y$ and $q: W\rightarrow\bar X$ be a resolution of indeterminacy, such that $\Mm$ descends to $W$, $p$ is a log resolution of $(Y,\Supp(B_Y+E))$, and $q$ is a log resolution of $(\bar X,\Supp\bar B)$.  By Lemma \ref{lem: existence of proper log smooth model}, we may find a proper log smooth model $(W,B_W,\Mm)$ of $(Y,B_Y,\Mm)$. We have
$$K_W+B_W+\Mm_W=p^*(K_Y+B_Y+\Mm_Y)+F=(p\circ f)^*(K_X+B+\Mm_X)+p^*E+F$$
for some $p$-exceptional $\Rr$-divisor $F\geq 0$.

Let $D$ be a component of $p^*E+F$. Then $a(D,W,B_W,\Mm)<a(D,X,B,\Mm)$ and $D$ is exceptional over $X$. If $D$ is not exceptional over $\bar X$, then $a(D,W,B_W,\Mm)<a(D,X,B,\Mm)=0$, which is not possible. Thus $p^*E+F$ is exceptional over $\bar X$. 

By \cite[Lemma 3.10(1)]{HL21} ($=$\cite[Version 3, Lemma 3.17]{HL21}), $(W,B_W,\Mm)/U$ has a weak glc model. Since $p^*E+F$ is exceptional over $\bar X$, $\Mm$ descends to $W$, $(W,B_W+p^*E+F)$ is log smooth, by Lemma \ref{lem: existence good minimal model under pullbacks weak glc case}, we have that $(\bar X,\bar B,\Mm)/U$ has a weak glc model. By \cite[Lemma 3.11]{HL21} ($=$\cite[Version 3, Lemma 3.13]{HL21}), $(X,B,\Mm)/U$ has a weak glc model, and we have proven the weak glc model case.

Now we prove the general case. By \cite[Lemma 3.10(2)]{HL21} ($=$\cite[Version 3, Lemma 3.18]{HL21}), we only need to prove that if $(Y,B_Y,\Mm)/U$ has a weak glc (resp. log minimal model, good minimal model), then $(X,B,\Mm)/U$ has a weak glc (resp. log minimal model, good minimal model). The weak glc case has just been proven, and the log minimal model case follows from the weak glc model case and \cite[Lemma 3.8]{HL21} ($=$\cite[Version 3, Lemma 3.15]{HL21}). Assume that $(Y,B_Y,\Mm)/U$ has a good minimal model. By the log minimal model case, we may assume that $(X',B',\Mm)/U$ is a log minimal model of $(X,B,\Mm)/U$. By \cite[Lemma 3.10(1)]{HL21} ($=$\cite[Version 3, Lemma 3.17]{HL21}), $(X',B',\Mm)/U$ is also a weak glc model of $(Y,B_Y,\Mm)/U$. By \cite[Lemma 3.5(2)]{HL21} ($=$\cite[Version 3, Lemma 3.9(2)]{HL21}), $K_{X'}+B'+\Mm_{X'}$ is semi-ample$/U$, hence $(X',B',\Mm)/U$ is a  good minimal model of $(X,B,\Mm)/U$, and the proof is concluded.
\end{proof}

\begin{thm}[{\cite[Version 2, Theorem 4.11]{HL21}}]\label{thm: has19 weak semistable reduction}
Let $(X,B)$ be a dlt pair and $\pi: X\rightarrow U$ a projective surjective morphism over a normal variety $U$. Then there exists a commutative diagram of projective morphisms
\begin{center}$\xymatrix{
Y\ar@{->}[r]^{f}\ar@{->}[d]_{\pi'} & X\ar@{->}[d]^{\pi}\\
V\ar@{->}[r]^{\varphi} & U
}$
\end{center}
such that
\begin{enumerate}
    \item $f,\varphi$ are birational morphisms, $\pi'$ is an equidimensional contraction, $Y$ only has $\Qq$-factorial toroidal singularities, and $V$ is smooth, and
    \item there exist two $\Rr$-divisors $B_Y$ and $E$ on $Y$, such that
    \begin{enumerate}
    \item $K_Y+B_Y=f^*(K_X+B)+E$,
    \item $B_Y\geq 0$, $E\geq 0$, and $B_Y\wedge E=0$,
    \item $(Y,B_Y)$ is lc quasi-smooth, and any lc center of $(Y,B_Y)$ on $X$ is an lc center of $(X,B)$.
    \end{enumerate}
\end{enumerate}
\end{thm}
\begin{proof} This result follows from \cite{AK00}, see also \cite[Theorem B.6]{Hu20}, \cite[Theorem 2]{Kaw15} and \cite[Step 2 of Proof of Lemma 3.2]{Has19}.
\end{proof}

\begin{thm}[{\cite[Version 2, Theorem 5.1]{HL21}}]\label{thm: numerical generalized canonical bundle formula}
Let $(X,B,\Mm)/U$ be an NQC glc g-pair such that $U$ is quasi-projective, and let $\pi: X\rightarrow V$ be a surjective morphism over $U$. Assume that $K_X+B+\Mm_X\sim_{\mathbb R,V}0$, then there exists an NQC glc g-pair $(V,B_V,\Mm^V)/U$, such that
\begin{enumerate}
    \item  $K_X+B+\Mm_X\sim_{\mathbb R}\pi^*(K_V+B_V+\Mm^V_V)$,
    \item any glc center of $(V,B_V,\Mm^V)$ is the image of a glc center of $(X,B,\Mm)$ in $V$, and
    %\footnote{JL: check the proof of (2). I do not see any trouble for the proof, but I do not think this is even mentioned in the usual canonical bundle formula. CH: It is good to mention for us, because restricting to non klt locus plays an important role in our paper.}
    \item if all glc centers of $(X,B,\Mm)$ dominate $V$, then $(V,B_V,\Mm^V)$ is gklt. 
\end{enumerate}
\end{thm}
\begin{proof}
By the theory of Shokurov-type rational polytopes (cf. \cite[Proposition 3.16]{HL18}) and the theory of uniform rational polytopes (cf. \cite[Lemma 5.3]{HLS19}, \cite[Therem 1.4]{Che20}), we may assume that $(X,B,\Mm)/U$ is a $\Qq$-g-pair. 

\medskip

\noindent\textbf{Step 1}. In this step, we prove the case when $X\rightarrow V$ is a generically finite morphism. Within this step, we assume that $X\rightarrow V$ is a generically finite morphism.

By \cite[Theorem 4.5, (4.3),(4.4)]{HL20}, there exists a glc $\Qq$-g-pair $(V,B_V,\Mm^V)/U$, such that $K_X+B+\Mm_X\sim_{\mathbb Q}\pi^*(K_V+B_V+\Mm^V_V)$, and $B_V$ and $\Mm^V$ are defined in the following way: 

Let $V^0$ be the smooth locus of $V$, $X^0:=X\times_VV^0$, and $\pi|_{X^0}: X^0\rightarrow V^0$ the restriction of $\pi$. Then we have the Hurwitz formula
$$K_{X^0}=(\pi|_{X^0})^*K_{V^0}+R^0$$
where $R^0$ is the effective ramification divisor of $f|_{X^0}$. Let $R$ be the closure of $R^0$ in $X$, and let $B_V:=\frac{1}{\deg \pi}\pi_*(R+B)$. For any proper birational morphism $\mu: V'\rightarrow V$, let $X'$ be the main component of $X\times_VV'$ with induced birational map $\pi': X'\rightarrow V'$. We let $\Mm^V_{V'}=\frac{1}{\deg\pi}\pi'_*\Mm_{X'}$.

(1) follows immediately. 

Since $(V,B,\Mm^V)/U$ is a g-pair, for any prime divisor $E$ over $V$, there exists a birational morphism $h_V: \tilde V\rightarrow V$ such that $\Mm^V$ descends to $\tilde V$ and $E$ is on $\tilde V$. We let $h: \tilde X\rightarrow X$ be a birational morphism such that $h$ descends to $\tilde X$, $\Mm$ descends to $\tilde X$, and the induced map $\tilde\pi: \tilde X\rightarrow\tilde V$ is a  morphism. 
\begin{center}$\xymatrix{
X'\ar@{->}[d]_{\pi'}\ar@{->}[r]& X\ar@{->}[d]_{\pi} & \tilde X\ar@{->}[l]_{h}\ar@{->}[d]^{\tilde\pi}\\
V'\ar@{->}[r]^{\mu}& V & \tilde V\ar@{->}[l]_{h_V}
}$
\end{center}
There are two cases:

\medskip

\noindent\textbf{Case 1}. $E$ is exceptional over $V$. In this case, we let $F\subset\tilde\pi^{-1}(E)$ be a prime divisor, and let $r\leq\deg f$ be the ramification index of $\tilde\pi$ along $F$. Near the generic point of $F$, we have
$$K_{\tilde X}=h^*(K_X+B+\Mm_X)+(a(F,X,B,\Mm)-1)F\sim_{\Qq}h^*\pi^*(K_V+B_V+\Mm_V)+(a(F,X,B,\Mm)-1)F$$
and
\begin{align*}
 K_{\tilde X}&=\tilde\pi^*K_{\tilde Z}+(r-1)F=\tilde\pi^*h_V^*(K_V+B_V+\Mm_V)+r(a(E,V,B_V,\Mm^V)-1)F+(r-1)F\\
 &=h^*\pi^*(K_V+B_V+\Mm_V)+(ra(E,V,B_V,\Mm^V)-1)F.
\end{align*}
Let $\tilde X\rightarrow \bar X\rightarrow V$ be the Stein factorization of $\pi\circ h=h_V\circ\tilde\pi$. Since $E$ is exceptional over $V$, $F$ is exceptional over $\bar X$. By the negativity lemma, we have
$$a(F,X,B,\Mm)-1=ra(E,V,B_V,\Mm^V)-1,$$
hence $a(F,X,B,\Mm)\geq 0$ if and only if $a(E,V,B_V,\Mm^V)\geq 0$ and $a(F,X,B,\Mm)>0$ if and only if $a(E,V,B_V,\Mm^V)>0$. Moreover, since $F\subset\tilde\pi^{-1}(E)$, if $E$ is a glc place of $(V,B_V,\Mm^V)$, then $F$ is a glc place of $(X,B,\Mm)$ and $\Center_VE$ is contained in the image of $\Center_XF$ in $V$. 

\medskip

\noindent\textbf{Case 2}. $E$ is not exceptional over $V$. In this case, if $E$ is not a component of $B_V$, then $a(E,V,B_V,\Mm^V)=1>0$. If $E$ is a component of $B_V$, then we may let $B_1,\dots,B_m\subset\pi^{-1}(E)$ be the prime divisors on $X$ lying over $V$ and let $d_i$ be the degree of the induced morphism $\pi|_{B_i}: B_i\rightarrow E$. By our construction of $B_V$, 
$$a(E,V,B_V,\Mm^V)=1-\mult_EB_V=1-\frac{\sum_{i=1}^md_i\mult_{B_i}B}{\deg\pi}.$$
Since $\sum_{i=1}^m d_i\leq\deg\pi$, $a(E,V,B_V,\Mm^V)\geq 0$ if $\mult_{B_i}B\leq 1$ for each $i$, and $a(E,V,B_V,\Mm^V)>0$ if $\mult_{B_i}B<1$ for each $i$. Moreover, since $B_i\subset\pi^{-1}(E)$ for each $i$, if $E$ is a glc place of $(V,B_V,\Mm^V)$, then $B_i$ is a glc place of $(X,B,\Mm)$ for some $i$ and $E$ is contained in the image of $B_i$ in $V$.

\medskip

By our discussions above, we finish the proof in the case when $X\rightarrow V$ is a generically finite morphism.

\medskip

\noindent\textbf{Step 2}. In this step, we prove the case when $X\rightarrow V$ is a contraction. Within this step, we assume that $X\rightarrow V$ is a contraction.

By \cite[Theorem 2.20]{FS20}, there exists a glc $\Qq$-g-pair $(V,B_V,\Mm^V)/U$, such that $K_X+B+\Mm_X\sim_{\mathbb Q}\pi^*(K_V+B_V+\Mm^V_V)$. Moreover, for any birational morphism $h_V: \tilde V\rightarrow V$, we have an $\Rr$-divisor $B_{\tilde V}$ satisfies that $K_{\tilde V}+B_{\tilde V}+\Mm^V_{\tilde V}=h_V^*(K_V+B_V+\Mm^V_V)$ and defined in the following way: let $\tilde X$ be the main component of $X\times_{V}\tilde V$, and $h: \tilde X\rightarrow X$ and $\tilde\pi: \tilde X\rightarrow\tilde V$ the induced morphisms. Let $K_{\tilde X}+\tilde B+\Mm_{\tilde X}:=h^*(K_X+B+\Mm_X)$. For any prime divisor $E$ on $\tilde V$, $\mult_{E}B_{\tilde V}=1-t_E$, where $$t_E:=\sup\{s\mid (\tilde X,\tilde B+s\tilde \pi^*E,\Mm)\text{ is glc over the generic point of }E\}.$$
Note that $E$ may not be $\Qq$-Cartier but $\tilde \pi^*E$ is always defined over the generic point of $E$. 

(1) follows immediately. 

If $E$ is a glc place of $(V,B_V,\Mm^V)$ on $\tilde V$, then $t_E=0$, hence $\tilde \pi^*E$ contains a glc center $F$ of $(\tilde X,\tilde B,\Mm)$ over the generic point of $E$. We have $F\subset\Supp\tilde\pi^*E$ and $\tilde\pi(F)\subset E$, hence $\tilde\pi(F)=E$. Thus $E$ is the image of a glc center of $(\tilde X,\tilde B,\Mm)$ on $\tilde V$, hence $\Center_{V}E$ is the image of a glc center of $(X,B,\Mm)$ in $V$. 

By our discussions above, we finish the proof in the case when $X\rightarrow V$ is a contraction.

\medskip

\noindent\textbf{Step 3}. In this step we prove the general case. 

We let $X\xrightarrow{f}Y\xrightarrow{g}V$ be the Stein factorization of $\pi$. Then $K_X+B+\Mm_X\sim_{\Qq,Y}0$, $f: X\rightarrow Y$ is a contraction and $g: Y\rightarrow V$ is a finite morphism. By Step 2, $K_X+B+\Mm_X\sim_{\Qq}f^*(K_Y+B_Y+\Mm^Y_Y)$ for some glc $\Qq$-g-pair $(Y,B_Y,\Mm^Y)/U$ such that any glc center of $(Y,B_Y,\Mm^Y)$ is the image of a glc center of $(X,B,\Mm)$ in $Y$. Moreover, $K_Y+B_Y+\Mm^Y_Y\sim_{\Qq,V}0$. By Step 1, $K_Y+B_Y+\Mm^Y_Y\sim_{\Qq}g^*(K_V+B_V+\Mm^V_V)$ for some glc g-pair $(V,B_V,\Mm^V)/U$ such that any glc center of $(V,B_V,\Mm^V)$ is the image of a glc center of $(Y,B_Y,\Mm^Y)$ in $V$, hence the image of a glc center of $(X,B,\Mm)$ in $V$. We immediately get (1)(2) and (3) follows from (2).
\end{proof}



\begin{lem}[{\cite[Version 2, Lemma 8.2]{HL21}}]\label{lem: relative subaddivitiy iitaka dimensions}
Let $f: X\rightarrow Y$ and $g: Y\rightarrow Z$ be two contractions between normal quasi-projective varieties such that general fibers of $Y\rightarrow Z$ are smooth and $Y$ is $\Qq$-Gorenstein. Let $(X,B)$ be a pair that is lc over a non-empty open subset of $Y$. Let $D$ be an $\Rr$-Cartier $\Rr$-divisor on $X$ such that $D-(K_{X/Y}+B)$ is nef$/Z$. Then for any $\Rr$-Cartier $\Rr$-divisor $Q$ on $Y$,
$$\kappa_{\sigma}(X/Z,D+f^*Q)\geq\kappa_{\sigma}(X/Y,D)+\kappa(Y/Z,Q).$$
\end{lem}
\begin{proof}
Let $z\in Z$ be a very general point and let $X_z:=(g\circ f)^{-1}(z),Y_z:=g^{-1}(z)$ be the fibers of $X$ and $Y$ over $z$ respectively. We have an induced contraction $f_z: X_z\rightarrow Y_z$. Let $F$ be a very general fiber of $f_z$, then $F$ is also a very general fiber of $f$. 

First assume that $\dim Y>\dim Z$. By our assumption, $Y_z$ is smooth, $(X_z,B|_{X_z})$ is lc over a non-empty open subset of $Y_z$, and
$$D|_{X_z}-(K_{X_z/Y_z}+B|_{X_z})=(D-(K_{X/Y}+B))|_{X_z}$$
is nef. By \cite[(3.3)]{Fuj19},
\begin{align*}
    \kappa_{\sigma}(X/Z,D+f^*Q)&=\kappa_{\sigma}(X_z,D|_{X_z}+f_z^*Q|_{Y_z})\geq \kappa_{\sigma}(X_z/Y_z,D|_{X_z})+\kappa(Y_z,Q|_{Y_z})\\
    &=\kappa_{\sigma}(F,D|_F)+\kappa(Y/Z,Q)=\kappa_{\sigma}(X/Y,D)+\kappa(Y/Z,Q).
\end{align*}
Now assume that $\dim Y=\dim Z$. If $\dim X=\dim Y$ then there is nothing left to prove, so we may assume that $\dim X>\dim Y$. In this case, $f^*Q|_{X_z}=0$, so we have
\begin{align*}
\kappa_{\sigma}(X/Z,D+f^*Q)&=\kappa_{\sigma}(X_z,D|_{X_z}+f^*Q|_{X_z})=\kappa_{\sigma}(X_z,D|_{X_z})= \kappa_{\sigma}(X/Z,D)\\
    &\geq\kappa_{\sigma}(X/Y,D)=\kappa_{\sigma}(X/Y,D)+\kappa(Y/Z,Q).
\end{align*}
\end{proof}


\begin{lem}[{\cite[Version 2, Lemma 8.3]{HL21}}]\label{lem: iitaka fibration numerical abundant divisor gpair dimension}
Let $(X,B,\Mm)/U$ be a glc g-pair such that $K_X+B+\Mm_X\equiv_{U}G$ for some $\Rr$-divisor $G\geq 0$, such that $U$ is quasi-projective and $G$ is abundant over $U$. Let $X\dashrightarrow V$ be the Iitaka fibration over $U$ associated to $G$, and $(W,B_W,\Mm)$ a log smooth model of $(X,B,\Mm)$ such that the induced map $\psi: W\rightarrow V$ is a morphism over $U$. Then
\begin{enumerate}
    \item $\kappa_{\sigma}(W/U,K_W+B_W+\Mm_W)=\dim V-\dim U$, and
    \item $\kappa_{\sigma}(W/V,K_W+B_W+\Mm_W)=0$.
\end{enumerate}
\end{lem}
\begin{proof}
Let $h_V: \bar V\rightarrow V$ be a resolution of $V$. By Lemmas \ref{lem: property of numerical and Iitaka dimension}(3) and \ref{lem: existence of proper log smooth model} possibly replacing $(W,B_W,\Mm)/U$ with a higher model, we may assume that the induced map $\bar\psi: W\rightarrow\bar V$ is a morphism. Since $(W,B_W,\Mm)$ a log smooth model of $(X,B,\Mm)$, we have
$$K_W+B_W+\Mm_W=h^*(K_X+B+\Mm_X)+E$$ where $h: W\to X$ is the induced morphism, $\Mm$ descends to $W$, and $E\geq 0$.
\begin{center}$\xymatrix{
W\ar@{->}[d]_{\bar\psi}\ar@{->}[drr]^{\psi}\ar@{->}[rr]^{h} &  & X\ar@{-->}[d] \\
\bar V\ar@{->}[dr]\ar@{->}[rr]^{h_V}&    & V\ar@{->}[dl] \\
 & U &
}$
\end{center}
Since $G\geq 0$ is abundant over $U$, by \cite[Proposition 2.2.2(1)]{Cho08}, $$\dim V-\dim U=\kappa(X/U,G)=\kappa_{\iota}(X/U,G)=\kappa_{\sigma}(X/U,G)\geq 0.$$
Since $X\dashrightarrow V$ is the Iitaka fibration associated to $G$ over $U$, there exists an ample$/U$ $\Rr$-divisor $A$ on $V$ and an $\Rr$-divisor $F\geq 0$ on $W$ such that $h^*G=\psi^*A+F$ 
for some $h$-exceptional $\Rr$-divisor $F\geq 0$ on $W$. Then for any real number $k$, we have
$$K_W+B_W+\Mm_W+k\psi^*A\equiv_{U}(1+k)\psi^*A+E+F.$$
By Lemma \ref{lem: property of numerical and Iitaka dimension}(2)(3)(5), for any $k\geq 0$ we have \begin{align*}
    \kappa_{\sigma}(W/U,K_W+B_W+\Mm_W+k\psi^*A)&=\kappa_{\sigma}(W/U,(1+k)\psi^*A+E+F)=\kappa_{\sigma}(W/U,\psi^*A+E+F)\\
    &=\kappa_{\sigma}(W/U,K_W+B_W+\Mm_W)=\kappa_{\sigma}(X/U,K_X+B+\Mm_X)\\
    &=\kappa_{\sigma}(X/U,G)=\kappa(X/U,G)=\dim V-\dim U
\end{align*}
for any non-negative real number $k$. In particular, we get (1). Since $A$ is ample$/U$, $h_V^*A$ is big$/U$, and we may pick a sufficiently large positive integer $k$ such that $K_{\bar V}+kh_V^*A$ is big$/U$. 


Since $(W,B_W,\Mm)$ is a log smooth model of $(X,B,\Mm)$, $(W,B_W)$ is lc. Since $\bar V$ is smooth, any very general fiber of the induced morphism $\bar V\rightarrow U$ is smooth. Let $D:=K_W+B_W+\Mm_W-\bar\psi^*K_{\bar V}$ and $Q:=K_{\bar V}+kh_V^*A$, then $D-(K_{W/\bar V}+B_W)=\Mm_W$ is nef$/U$. By Lemma \ref{lem: relative subaddivitiy iitaka dimensions},
 \begin{align*}
    \dim V-\dim U&=\kappa_{\sigma}(W/U,K_W+B_W+\Mm_W+k\psi^*A)=\kappa_{\sigma}(W/U,K_W+B_W+\Mm_W+k\bar\psi^*h_V^*A)\\
    &=\kappa_{\sigma}(W/U,D+\bar\psi^*Q)\geq \kappa_{\sigma}(W/\bar V,D)+\kappa(\bar V/U,Q)\\
    &=\kappa_{\sigma}(W/\bar V,K_W+B_W+\Mm_W-\bar\psi^*K_{\bar V})+\kappa(\bar V/U,K_{\bar V}+kh_V^*A)\\
    &=\kappa_{\sigma}(W/\bar V,K_W+B_W+\Mm_W)+(\dim V-\dim U).
\end{align*}
Thus $\kappa_{\sigma}(W/\bar V,K_W+B_W+\Mm_W)\leq 0$, hence $\kappa_{\sigma}(W/V,K_W+B_W+\Mm_W)\leq 0$. Since $K_W+B_W+\Mm_W\equiv_U h^*G+E\geq 0$, $\kappa_{\sigma}(W/V,K_W+B_W+\Mm_W)\geq 0$. Thus $\kappa_{\sigma}(W/V,K_W+B_W+\Mm_W)=0$, and we get (2).
\end{proof}


\begin{lem}[{\cite[Version 2, Lemma 8.4]{HL21}}]\label{lem: special proper log smooth model}
Let $(X,B,\Mm)/U$ be a glc g-pair. Then there exists a proper log smooth model $(W,B_W=B_W^h+B_W^v,\Mm)$ of $(X,B,\Mm)$, such that
\begin{enumerate}
    \item $B_W^h\geq 0$ and $B_W^v$ is reduced,
    \item $B_W^v$ is vertical over $U$, and
    \item for any real number $t\in (0,1]$, all glc centers of $(W,B_W-tB_W^v,\Mm)$ dominate $U$.
\end{enumerate}

\end{lem}
\begin{proof}
By Lemma \ref{lem: existence of proper log smooth model}, possibly replacing $(X,B,\Mm)$ with a proper log smooth model, we may assume that $(X,\Supp B)$ is log smooth and $\Mm$ descends to $X$. By \cite[Lemma 2.10]{Has18}, there exists a proper log smooth model $(W,B_W=B_W^h+B_W^v)$ of $(X,B)$, such that
\begin{itemize}
    \item $B_W^h\geq 0$ and $B_W^v$ is reduced,
    \item $B_W^v$ is vertical over $U$, and
    \item for any real number $t\in (0,1]$, all lc centers of $(W,B_W-tB_W^v)$ dominate $U$.
\end{itemize}
Since $\Mm$ descends to $X$, $(W,B_W,\Mm)$ is a proper log smooth model of $(X,B,\Mm)$, and for any real number $t\in (0,1]$, any glc center of $(W,B_W-tB_W^v,\Mm)$ is an lc center of $(W,B_W-tB_W^v)$ and dominates $U$. Thus $(W,B_W=B_W^h+B_W^v,\Mm)$ satisfies our requirements.
\end{proof}




\section{Relative Nakayama-Zariski decomposition}

\begin{defn}\label{defn: rel nz decomposition}
Let $\pi: X\rightarrow U$ be a projective morphism from a normal quasi-projective variety to a variety, $A$ an ample$/U$ $\Rr$-divisor on $X$, $D$ a pseudo-effective$/U$ $\Rr$-Cartier $\Rr$-divisor on $X$, and $P$ a prime divisor on $X$. For any big$/U$ $\Rr$-Cartier $\Rr$-divisor $B$, we define
$$\sigma_P(X/U,B):=\inf\{\mult_PB'\mid 0\leq B'\sim_{\Rr,U}B\}.$$
We define
$$\sigma_P(X/U,D):=\lim_{\epsilon\rightarrow0^+}\sigma_P(X/U,D+\epsilon A),$$
where we allow $+\infty$ be a limit as well. We let
$$N_{\sigma}(X/U,D):=\sum_{C\text{ is a prime divisor on }X}\sigma_C(X/U,D)$$
and
$$P_{\sigma}(X/U,D):=D-N_{\sigma}(X/U,D).$$
\end{defn}

Definition \ref{defn: rel nz decomposition} is the same as the one adopted in \cite{HX13,HMX18}. The following lemma shows that relative Nakayama-Zariski decomposition defined in Definition \ref{defn: rel nz decomposition} is the same as the $\sigma$-decomposition defined in \cite[III. \S 4.a]{Nak04}:

\begin{lem}\label{lem: nz decomposition definitions are equivalent}
Notation as in Definition \ref{defn: rel nz decomposition}. If $X$ is smooth, then $\sigma_P(X/U,D)$ is the same as $\sigma_P(D,X/U)$, where the latter is the value defined as in Nakayama's original relative $\sigma$-decomposition \cite[III. \S 4.a]{Nak04}. 
\end{lem}
\begin{proof}
By definition, we only need to deal with the case when $D$ is big. We may pick an affine open subset $U^0$ of $U$ such that $P$ intersects $X^0:=X\times_UU^0$. Let $P^0:=P\times_UU^0$ and $D^0:=D\times_UU^0$. Then
$$\sigma_P(X/U,D)=\sigma_{P^0}(X^0/U^0,D^0).$$
Possibly replacing $(X/U,D)$ and $P$ with $(X^0/U^0,D^0)$ and $P^0$ respectively, we may assume that $U$ is affine. Thus for any Cartier divisor $Q$ on $U$, there exists a principle divisor $Q'$ on $U$ such that $Q'=Q$ in a neighborhood of the generic point of $\pi(P)$. In particular, we have
$$\sigma_P(X/U,D)=\inf\{\mult_{P^0}B'\mid 0\leq B'\sim_{\Rr}B^0\}.$$
For any Cartier divisor $F$ on $X$, let
$$m_F:=\inf\{+\infty,\mult_PF'\mid 0\leq F'\sim F\}.$$
If $m_F<+\infty$, then by definition,
$$m_F=\max\{m\mid m\in\mathbb N, H^0(X,F-mP)\hookrightarrow H^0(X,F)\text{ is isomorphic}\}.$$
Moreover, since $U$ is affine and $H^0(X,\mathcal{O}_X(F))=H^0(U,\pi_*\mathcal{O}_X(F))$, if $m_F<+\infty$, then
$$m_F=\max\{m\mid m\in\mathbb N, \pi_*\mathcal{O}_X(F-mP)\hookrightarrow\pi_*\mathcal{O}_X(F)\text{ is isomorphic}\}.$$
The lemma follows by the construction in \cite[III. \S 4.a]{Nak04}.
\end{proof}






\begin{lem}\label{lem: nz keep under pullback}
Let $\pi: X\rightarrow U$ be a projective morphism from a normal quasi-projective variety to a variety and $D$ a pseudo-effective$/U$ $\Rr$-Cartier $\Rr$-divisor on $X$. Let $f: Y\rightarrow X$ be a birational morphism. Then:
\begin{enumerate}
    \item For any prime divisor $P$ on $X$, we have
$$\sigma_P(X/U,D)=\sigma_{f^{-1}_*P}(Y/U,f^*D).$$
    \item For any exceptional$/X$ $\Rr$-Cartier $\Rr$-divisor $E\ge0$ and prime divisor $P$ on $Y$, $$\sigma_P(Y/U,f^*D+E)=\sigma_P(Y/U,f^*D)+\mult_PE.$$
\item For any exceptional$/X$ $\Rr$-Cartier $\Rr$-divisor $E\ge0$, $N_{\sigma}(X/U,D)$ is well-defined if $N_{\sigma}(Y/U,f^*D+E)$ is well-defined. Moreover, $N_{\sigma}(X/U,D)=f_*N_{\sigma}(Y/U,f^*D)$ and $P_{\sigma}(X/U,D)=f_*P_{\sigma}(Y/U,f^*D)$.
\end{enumerate}
\end{lem}
\begin{proof}
Let $g=\pi\circ f$ and $A$ (resp. $A'$) be an ample$/U$ divisor on $X$ (resp. $Y$). Fix a real number $a>0$ such that $aA'+f^*A$ is ample$/U$. For any prime divisor $P$ on $Y$, since $f^*A$ is semi-ample, we have
\begin{align*}
    \sigma_P(X/U,f^*D)&=\lim_{\epsilon\rightarrow 0^+}\sigma_P(X/U,f^*D+\epsilon(aA'+f^*A))\leq \lim_{\epsilon\rightarrow 0^+}\sigma_P(X/U,f^*D+\epsilon f^*A)\\
    &\leq\sigma_P(X/U,f^*D),
\end{align*}
hence $\lim_{\epsilon\to0^+}\sigma_P(Y/U,f^*D+\epsilon f^*A)=\sigma_P(Y/U,f^*D)$.


We prove (1). Since $\pi_*\Oo_X(F)=g_*\Oo_Y(f^*F)$ for any Cartier divisor $F$ on $X$. Then by definition we have $\sigma_P(X/U,D+\epsilon A)=\sigma_{f_*^{-1}P}(Y/U,f^*D+\epsilon f^*A)$ for any $\epsilon>0$. Thus we have 
$$
\sigma_P(X/U,D)=\lim_{\epsilon\to0^+}\sigma_{f_*^{-1}P}(Y/U,f^*D+\epsilon f^*A)=\sigma_{f^{-1}_*P}(Y/U,f^*D)
$$ 
which is (1).

We prove (2). Since $\lim_{\epsilon\to0^+}\sigma_P(Y/U,f^*D+\epsilon f^*A)=\sigma_P(Y/U,f^*D)$, we may assume that $D$ is a big$/U$. (2) follows from the fact that $\pi_*\Oo_Y(f^*F+E)=g_*\Oo_X(F)$ for any Cartier divisor $F$ on $X$ and any exceptional$/X$ divisor $E\ge0$.


(3) is an immediate consequence of (1) and (2).
\end{proof}

\begin{lem}\label{lem: nz finite is well-defined}
Let $\pi: X\rightarrow U$ be a projective morphism from a normal quasi-projective variety to a variety and $D$ a pseudo-effective$/U$ $\Rr$-Cartier $\Rr$-divisor on $X$. Then there are only finitely many prime divisors $P$ on $X$ such that $\sigma_P(X/U,D)\not=0$. In particular, if $\sigma_P(X/U,D)<+\infty$ for any prime divisor $P$ on $X$, then $N_{\sigma}(X/U,D)$ and $P_{\sigma}(X/U,D)$ are well-defined.
\end{lem}
\begin{proof}
We show that there are at most $\dim N^1(X/U)_{\Rr}$ prime divisors $P$ on $X$ such that $\sigma_P(X/U,D)\not=0$. By Lemma \ref{lem: nz keep under pullback} we may assume that $X$ is smooth. Let $P_1,P_2,...,P_l$ be distinct prime divisors of $X$ such that $\sigma_{P_i}(X/U,D)>0$ for each $i$.  If $l\leq\dim N^1(X/U)_{\Rr}$ then we are done. Otherwise, by Lemma \ref{lem: nz decomposition definitions are equivalent} and \cite[III, Lemma 4.2(2)]{Nak04},
$$\sigma_{P_i}(X/U,\sum^{l}_{j=1}x_jP_j)=x_i$$
for any $x_1,x_2,...,x_l\in\Rr_{\ge0}$, and possibly reordering indices, we have
$$\sum_{i=1}^{s}x_iP_i\equiv_U\sum_{j=s+1}^{l}x_jP_j\in\mathrm{N}^1(X/U)$$
for some $1\le s\le l$. By Lemma \ref{lem: nz decomposition definitions are equivalent} and \cite[III, Lemma 4.2(2)]{Nak04} again,
$$
x_1=\sigma_{P_1}(X/U,\sum^{s}_{i=1}x_iP_i)=\sigma_{P_1}(X/U,\sum^{l}_{j=s+1}x_jP_j)=0,
$$
a contradiction.
\end{proof}



\begin{defn}\label{defn: sigma over X}
Let $\pi: X\rightarrow U$ be a projective morphism from a normal quasi-projective variety to a variety, $D$ a pseudo-effective$/U$ $\Rr$-Cartier $\Rr$-divisor on $X$, and $P$ a prime divisor over $X$. Let $f: Y\rightarrow X$ be a birational morphism such that $P$ descends to $Y$. We define
$$\sigma_P(X/U,D):=\sigma_P(Y/U,f^*D).$$
By Lemma \ref{lem: nz keep under pullback}, $\sigma_P(X/U,D)$ is well-defined and is independent of the choice of $Y$.
\end{defn}

\begin{lem}\label{lem: nz basic properties}
Let $\pi: X\rightarrow U$ be a projective morphism from a normal quasi-projective variety to a variety, $D,D'$ two pseudo-effective$/U$ $\Rr$-Cartier $\Rr$-divisors on $X$, and $P$ a prime divisor over $X$. Then
\begin{enumerate}
    \item $\sigma_P(X/U,D+D')\leq\sigma_P(X/U,D)+\sigma_P(X/U,D')$.
    \item If $D'\geq 0$, then $\lim_{\epsilon\rightarrow 0^+}\sigma_P(X/U,D+\epsilon D')=\sigma_P(X/U,D)$.
    \item If $D$ is a limit of movable$/U$ $\Rr$-Cartier $\Rr$-divisors, then $P_{\sigma}(X/U,D)=D$ and $N_{\sigma}(X/U,D)=0$.
    \item If $N_{\sigma}(X/U,D)$ is well-defined, then $\Supp N_{\sigma}(X/U,D)$ conincides with the divisorial part of ${\bf{B}}_{-}(D/U)$.
    \item If $0\leq D'\leq N_{\sigma}(X/U,D)$, then $P_{\sigma}(X/U,D-D')=P_{\sigma}(X/U,D)$.
    \item If $D'\geq 0$ and $\Supp D'\subset\Supp N_{\sigma}(X/U,D)$, then $P_{\sigma}(X/U,D+D')=P_{\sigma}(X/U,D)$.
\end{enumerate}
\end{lem}
\begin{proof}
By Lemma \ref{lem: nz keep under pullback}, possibly replacing $X$ with a resolution, we may assume that $X$ is smooth and $P$ is a prime divisor on $X$. Let $A$ be an ample$/U$ divisor on $X$. \par

(1) follows from the fact that $\sigma_P(X/U,D+D'+\epsilon A)\le\sigma_P(X/U,D+\frac{\epsilon}{2}A)+\sigma_P(X/U, D'+\frac{\epsilon}{2}A)$.

There exists $a>0$ such that $A-aD'$ is ample$/U$. Thus by (1), $$\sigma_P(X/U,D)+\sigma_P(X/U,a\epsilon D')\ge\sigma_P(X/U, D+a\epsilon D')\ge\sigma_P(X/U,D+\epsilon A),$$
and (2) follows after taking $\epsilon\to 0^+$.


For (3), if this is not true, then we have $\sigma_P(X/U,D)>0$ for some $P$. By definition, there exist an $\epsilon>0$ such that $\sigma_P(X/U,D+\epsilon A)>0$. Assume $D=\lim_iD_i$, where $D_i$ is a movable divisor for each $i\ge1$. Then $\epsilon A-(D_i-D)$ is ample for any $i\gg0$. Thus $0=\sigma_P(X/U,D_i)=\sigma_P(X/U,D_i+\epsilon A-(D_i-D))=\sigma_P(X/U,D+\epsilon A)>0$, which is a contradiction.


(4) follows from the definition of ${\bf{B}}_-(D/U)$.

(5) and (6) follows from Lemmma \ref{lem: nz decomposition definitions are equivalent} and \cite[III, Lemma 4.2]{Nak04}.
\end{proof}

\begin{lem}[{cf. \cite[Remark 2.4]{Has20a}}]\label{lem: has20a 2.4 rel ver}
Let $\pi: X\rightarrow U$ be a projective morphism from a normal quasi-projective variety to a variety, $D$ a pseudo-effective$/U$ $\Rr$-Cartier $\Rr$-divisors on $X$ such that $N_{\sigma}(X/U,D)$ is well-defined, and $D'\geq 0$ an $\Rr$-Cartier $\Rr$-divisor. Then there exists $t_0>0$, such that $\Supp N_{\sigma}(X/U,D+tD')$ is independent of $t$ for any $t\in (0,t_0]$.
\end{lem}
\begin{proof}
Since $N_{\sigma}(X/U,D)$ is well-defined and $D'\geq N_{\sigma}(X/U,D')$ is well-defined, by Lemma \ref{lem: nz basic properties}(1), we may let $D_1,\dots,D_k$ be the irreducible components of $\Supp N_{\sigma}(X/U,D)\cup\Supp D'$. Let $$J_i:=\{s\mid s\in (0,1],\sigma_{D_i}(X/U,D+sD')=0\}$$
for any $1\leq i\leq k$. For each $i$, we define
$$s_i=\begin{cases} 
      \text{some number in }J_i & J_i\not=\emptyset, \inf J_i=0 \\
      \frac{1}{2}\inf J_i & J_i\not=\emptyset, \inf J_i>0 \\
      1 & J_i=\emptyset. 
   \end{cases}
   $$
   Let $t_0:=\min_{1\leq i\leq k}\{s_i\}$. Then by Lemma \ref{lem: nz basic properties}(1), for any $1\leq i\leq k$, \begin{itemize}
       \item if $J_i\not=\emptyset$ and $\inf J_i=0$, then $\sigma_{D_i}(X/U,D+tD')=0$ for any $t\in (0,t_0]$, and
       \item $\sigma_{D_i}(X/U,D+tD')>0$ for any $t\in (0,t_0]$ otherwise.
   \end{itemize}
   Thus $t_0$ satisfies our requirement.
\end{proof}

\begin{lem}\label{lem: nz for glc divisor}
Let $(X,B,\Mm)/U$ be an NQC glc g-pair, $D:=K_X+B+\Mm_X$, and $P$ a prime divisor over $X$. Then
\begin{enumerate}
    \item $\sigma_P(X/U,D)<+\infty$. In particular, $N_{\sigma}(X/U,D)$ is well-defined and $N_{\sigma}(X/U,f^*D)$ is well-defined for any birational morphism $f: \tilde X\rightarrow X$, and
    \item if $(X,B,\Mm)$ is $\Qq$-factorial gdlt, then for any partial $D$-MMP$/U$ $\phi: X\dashrightarrow \bar X$, 
    \begin{enumerate}
    \item the divisors contracted by $\phi$ are contained in $\Supp N_{\sigma}(X/U,D)$, and
    \item let $\bar B$ be the strict transform of $B$ on $\bar X$. If $K_{\bar X}+\bar B+\Mm_{\bar X}$ is a limit of movable$/U$ $\Rr$-divisors, then $\Supp N_{\sigma}(X/U,D)$ is the set of all $\phi$-exceptional divisors.
    \end{enumerate}
\end{enumerate} 
\end{lem}
\begin{proof}
First we show that $\sigma_P(X/U,D)<+\infty$ for any prime divisor $P$ on $X$ and also prove (2). Let $(Y,B_Y,\Mm)$ be a gdlt model of $(X,B,\Mm)$. By Lemma \ref{lem: nz keep under pullback}, we may replace $(X,B,\Mm)$ with  $(Y,B_Y,\Mm)$  and assume that $(X,B,\Mm)$ is $\Qq$-factorial gdlt. By Lemma \ref{lem: limit movable r divisors gpairs}, we may run a partial $(K_X+B+\Mm_X)$-MMP$/U$ $\psi: X\dashrightarrow X'$, such that
\begin{itemize}
    \item $\psi$ is the composition of $\phi$ and a partial $(K_{\bar X}+\bar B+\Mm_{\bar X})$-MMP$/U$ with scaling of an ample$/U$ $\Rr$-divisor, where $\bar B$ is the strict transform of $B$ on $\bar X$, and
    \item $D':=K_{X'}+B'+\Mm_{X'}$ is a limit of movable$/U$ $\Rr$-divisors, where $B'$ is the strict transform of $B$ on $X'$.
\end{itemize} 
Let $p: W\rightarrow X$ and $q: W\rightarrow X'$ be a resolution of indeterminacy, then
$$p^*D=q^*D'+E$$
for some $E\geq 0$ that is exceptional$/X'$. Moreover, $q^*D'$ is a limit of movable$/U$ $\Rr$-divisors. By Lemmas \ref{lem: nz basic properties}(3) and \ref{lem: nz keep under pullback}(2)(3), 
$$N_{\sigma}(X/U,D)=p_*(N_{\sigma}(W/U,q^*D'+E))=p_*(E+N_{\sigma}(W/U,q^*D'))=p_*E$$
is well-defined. Thus $\sigma_P(X/U,D)<+\infty$ for any prime divisor $P$ on $X$, and we also get (2.b). Since the divisors contracted by $\psi$ are contained in $\Supp p_*E=\Supp N_{\sigma}(X/U,D)$, the divisors contracted by $\phi$ are contained in $\Supp N_{\sigma}(X/U,D)$, and we get (2.a).

For any prime divisor $P$ over $X$, let $(\tilde B,\tilde B,\Mm)/U$ be a log smooth model of $(X,B,\Mm)$. Then $\sigma_P(\tilde X/U, K_{\tilde X}+\tilde B+\Mm_{\tilde X})<+\infty$ and (1) follows from Lemma \ref{lem: nz keep under pullback}.
\end{proof}

\begin{lem}\label{lem: hmx18 2.7.3 gpair rel}
Let $(X,B,\Mm)/U$ be a $\Qq$-factorial NQC gdlt g-pair such that $K_X+B+\Mm_X$ is pseudo-effective$/U$. Let $\phi: X\dashrightarrow X'$ be a birational map$/U$ which does not extract any divisor and $B'$ the strict transform of $B$ on $X'$, such that
\begin{enumerate}
\item $K_{X'}+B'+\Mm_{X'}$ is nef$/U$, and
\item $\phi$ only contract divisors contained in $\Supp N_{\sigma}(X/U,K_X+B+\Mm_X)$,
\end{enumerate}
then $(X',B',\Mm)/U$ is a log minimal model of $(X,B,\Mm)/U$.
\end{lem}
\begin{proof}
Let $p: W\rightarrow X$ and $q: W\rightarrow X'$ be a resolution of indeterminacy of $\phi$, such that
$$p^*(K_X+B+\Mm_X)+E=q^*(K_{X'}+B'+\Mm_{X'})+F$$
where $E\geq 0,F\geq 0$, and $E\wedge F=0$. Then $E$ and $F$ are $q$-exceptional. By Lemmas \ref{lem: nz basic properties}(3) and \ref{lem: nz keep under pullback}(2)(3), $F=N_{\sigma}(W/U,q^*(K_{X'}+B'+\Mm_{X'})+F)$. 

We may write $E=E_1+E_2$ such that $E_1$ is $p$-exceptional and every component of $E_2$ is not $p$-exceptional. For any component $D$ of $E_1$, by Lemma \ref{lem: nz keep under pullback}(2), $D\subset\Supp N_{\sigma}(W/U,p^*(K_X+B+\Mm_X)+E)$. Thus $N_{\sigma}(W/U,p^*(K_X+B+\Mm_X)+E)=N_{\sigma}(W/U,p^*(K_X+B+\Mm_X)+E_2)$.  For any component $D$ of $E_2$, since $p_*D$ is exceptional$/X'$, $p_*D$ is contained in $\Supp N_{\sigma}(X/U,K_X+B+\Mm_X)$. By Lemmas \ref{lem: nz keep under pullback}(3) and \ref{lem: nz basic properties}(6), $D$ is a component of $N_{\sigma}(W/U,p^*(K_X+B+\Mm_X)+E)$. Therefore, $$\Supp E\subset\Supp N_{\sigma}(W/U,p^*(K_X+B+\Mm_X)+E)=\Supp F,$$
hence $E=0$. By Lemma \ref{lem: nz for glc divisor}(2.b), $\phi$ contracts all components of $\Supp N_{\sigma}(X/U,K_X+B+\Mm_X)=\Supp p_*F$, and the lemma follows.
\end{proof}


\section{Reduction via Iitaka fibration}

This section is similar to \cite[Version 2, Section 4]{HL21}.

\begin{lem}[{cf. \cite[Version 2, Lemma 4.9]{HL21}}]\label{lem: has19 3.2 step 3 abu ver}
Let $(X,B,\Mm)/U$ be a $\Qq$-factorial NQC glc g-pair with $X$ klt and $\pi: X\rightarrow U$ the induced morphism, such that
\begin{enumerate}
\item $\pi$ is an equidimensional contraction,
\item $U$ is quasi-projective and $\Qq$-factorial, and
\item $\kappa_{\sigma}(X/U,K_X+B+\Mm_X)=\kappa_{\iota}(X/U,K_X+B+\Mm_X)=0$.
\end{enumerate}
Let $A\geq 0$ be an ample$/U$ $\Rr$-divisor on $X$ such that $(X,B+A,\Mm)$ is glc and $K_X+B+A+\Mm_X$ is nef$/U$, and run a $(K_X+B+\Mm_X)$-MMP$/U$ with scaling of $A$. Then this MMP terminates with a good minimal model $(X',B',\Mm)/U$ of $(X,B,\Mm)/U$. Moreover, $K_{X'}+B'+\Mm_{X'}\sim_{\Rr,U} 0$.
\end{lem}
\begin{proof}
If $\dim X=\dim U$, then since $\pi$ is an equidimensional contraction, $\pi$ is the identity map, and there is nothing left to prove. In the following, we assume that $\dim X>\dim U$.

Since $\kappa_{\iota}(X/U,K_X+B+\Mm_X)=0$, $K_X+B+\Mm_X\sim_{\Rr,U}E\geq 0$ for some $\Rr$-divisor $E$ on $X$. We may write $E=E^{h}+E^{v}$, such that $E^h\geq 0,E^v\geq 0$, each component of $E^h$ is horizontal over $U$, and $E^v$ is vertical over $U$. Since $\pi$ is equidimensional, the image of any component of $E^v$ on $U$ is a divisor. Since $U$ is $\Qq$-factorial, for any prime divisor $P$ on $U$, we may define
$$\nu_P:=\sup\{\nu\mid \nu\geq 0, E^v-\nu\pi^*P\geq 0\}.$$
Then $\nu_P>0$  for only finitely many prime divisors $P$ on $U$. Possibly replacing $E^v$ with $E^v-\pi^*(\sum_P\nu_PP)$, we may assume that $E^v$ is very exceptional over $U$. 

Let $F$ be a very general fiber of $\pi$, and $(F,B_F,\Mm^F)$ the projective g-pair induced by the adjunction to the fiber
$$K_F+B_F+\Mm^F_F:=(K_X+B+\Mm_X)|_F.$$
Then $\kappa_{\sigma}(K_F+B_F+\Mm^F_F)=\kappa_{\iota}(K_F+B_F+\Mm^F_F)=0$. By \cite[Lemma 3.10]{Has22}, $(F,B_F,\Mm^F)$ has a good minimal model $(F',B_{F'},\Mm^F)$ such that $K_{F'}+B_{F'}+\Mm^F_{F'}\sim_{\Rr}0$. By Lemma \ref{lem: termination along a very general fiber}, any $(K_X+B+\Mm_X)$-MMP$/U$ with scaling of $A$ terminates along $F$ with a log minimal model of $(F,B_F,\Mm^F)$. In particular, let 
$$(X,B,\Mm):=(X_0,B_0,\Mm)\dashrightarrow (X_1,B_1,\Mm)\dashrightarrow\dots\dashrightarrow (X_i,B_i,\Mm)\dashrightarrow\dots$$
be our $(K_X+B+\Mm_X)$-MMP$/U$ with scaling of $A$, and let $A_i,E^h_i,E^v_i,F_i$ be the strict transforms of $A,E^h,E^v,F$ on $X_i$ respectively, then there exists a positive integer $n$ such that 
$$E_n^h|_{F_n}=(E_n^h+E_n^v)|_{F_n}\sim_{\Rr} (K_{X_n}+B_n+\Mm_{X_n})|_{F_n},$$
and the projective g-pair $(F_n,B_{F_n},\Mm^F)$ given by the adjunction
$$K_{F_n}+B_{F_n}+\Mm^F_{F_n}:=(K_{X_n}+B_n+\Mm_{X_n})|_{F_n}$$
is a log minimal model of $(F,B_F,\Mm^F)$.

By \cite[Lemma 3.5(1)]{HL21} ($=$\cite[Version 3, Lemma 3.9(1)]{HL21}), $K_{F_n}+B_{F_n}+\Mm^F_{F_n}\sim_{\Rr}0$. Thus $E^h_n|_{F_n}\sim_{\Rr}0$. Since $E^h_n\geq 0$ is horizontal over $U$, $E^h_n=0$, and $K_{X_n}+B_n+\Mm_{X_n}\sim_{\Rr,U}E^v_n$.
Since this MMP$/U$ is also a $(E^h+E^v)$-MMP$/U$ and $E^v_n$ is very exceptional over $U$, by Lemma \ref{lem: rlinear version of hl18 3.8}, this MMP terminates with a log minimal model $(X',B',\Mm)/U=(X_m,B_m,\Mm)/U$ of $(X,B,\Mm)/U$ for some positive integer $m$, such that $K_{X'}+B'+\Mm_{X'}\sim_{\Rr,U}0$. In particular, $(X',B',\Mm)/U$ is a good minimal model of $(X,B,\Mm)/U$.
\end{proof}


\begin{thm}[{cf. \cite[Version 2, Theorem 4.1]{HL21}}]\label{thm: hl21 v2 4.1 abu ver}
Let $(X,B,\Mm)/U$ be an NQC glc g-pair and $\pi: X\rightarrow V$ a contraction over $U$ such that $V$ is quasi-projective. Assume that $\kappa_{\sigma}(X/V,K_X+B+\Mm_X)=\kappa_{\iota}(X/V,K_X+B+\Mm_X)=0$. Then there exists a $\Qq$-factorial NQC gdlt g-pair $(X',B',\Mm)/U$, a contraction $\pi': X'\rightarrow V'$ over $U$, and a birational projective morphism $\varphi: V'\rightarrow V$ over $U$ satisfying the following:
\begin{center}$\xymatrix{
X'\ar@{->}[d]_{\pi'}\ar@{-->}[rr]& & X\ar@{->}[d]^{\pi}\\
V'\ar@{->}[rr]^{\varphi}\ar@{->}[dr] & & V\ar@{->}[dl]\\
& U &
}$
\end{center}
\begin{enumerate}
    \item $X'$ is birational to $X$ and $V'$ is smooth,
    \item $K_{X'}+B'+\Mm_{X'}\sim_{\Rr,V'}0$.
    \item $(X,B,\Mm)/U$ has a good minimal model if and only if $(X',B',\Mm)/U$ has a good minimal model.
    \item Any weak glc model of $(X,B,\Mm)/U$ is a weak glc model of $(X',B',\Mm)/U$, and any weak glc model of $(X',B',\Mm)/U$ is a weak glc model of $(X,B,\Mm)/U$.
    \item If all glc centers of $(X,B,\Mm)$ dominate $V$, then all glc centers of $(X',B',\Mm)$ dominate $V'$.
    \item $\kappa_{\sigma}(X/U,K_X+B+\Mm_X)=\kappa_{\sigma}(X'/U,K_{X'}+B'+\Mm_{X'})$ and $\kappa_{\iota}(X/U,K_X+B+\Mm_X)=\kappa_{\iota}(X'/U,K_{X'}+B'+\Mm_{X'})$
\end{enumerate}
\end{thm}

\begin{proof} Let $h: W\rightarrow X$ be a log resolution of $(X,\Supp B)$ such that $\Mm$ descends to $W$. 
By Lemma \ref{lem: existence of proper log smooth model}, $(X,B,\Mm)$ has a proper log smooth model $(W,B_W,\Mm)$ for some $\Rr$-divisor $B_W$ on $W$. By Lemmas \ref{lem: property of numerical and Iitaka dimension}(3) and \ref{lem: proper log smooth model keep lc center}(3), Theorem \ref{thm: existence good minimal model under pullbacks}, and \cite[Lemmas 3.6, 3.10]{HL21} (($=$\cite[Version 3, Lemmas 3.10, 3.17]{HL21}), we may replace $(X,B,\Mm)$ with $(W,B_W,\Mm)$, and assume that $(X,B)$ is log smooth dlt and $\Mm$ descends to $X$. 

By Theorem \ref{thm: has19 weak semistable reduction}, there exists a commutative diagram of projective morphisms
\begin{center}$\xymatrix{
Y\ar@{->}[r]^{f}\ar@{->}[d]_{\pi_Y} & X\ar@{->}[d]^{\pi}\\
V'\ar@{->}[r]^{\varphi} & V
}$
\end{center}
such that
\begin{itemize}
    \item $f,\varphi$ are birational morphisms, $\pi_Y$ is an equidimensional contraction, $Y$ only has $\Qq$-factorial toroidal singularities, and $V'$ is smooth, and
    \item there exist two $\Rr$-divisors $B_Y$ and $E$ on $Y$, such that
    \begin{itemize}
    \item $K_Y+B_Y+\Mm_Y=f^*(K_X+B+\Mm_X)+E$,
    \item $B_Y\geq 0$, $E\geq 0$, and $B_Y\wedge E=0$,
    \item $(Y,B_Y)$ is lc quasi-smooth, and any glc center of $(Y,B_Y,\Mm)$ on $X$ is a glc center of $(X,B,\Mm)$.
    \end{itemize}
\end{itemize}
In particular, $(Y,B_Y,\Mm)$ is $\Qq$-factorial NQC glc and $Y$ is klt. Since $\varphi$ is birational, by Lemma \ref{lem: property of numerical and Iitaka dimension}(3), $$\kappa_{\sigma}(Y/V',K_Y+B_Y+\Mm_Y)=\kappa_{\sigma}(Y/V,K_Y+B_Y+\Mm_Y)=\kappa_{\sigma}(X/V,K_X+B+\Mm_X)=0$$
and
$$\kappa_{\iota}(Y/V',K_Y+B_Y+\Mm_Y)=\kappa_{\iota}(Y/V,K_Y+B_Y+\Mm_Y)=\kappa_{\iota}(X/V,K_X+B+\Mm_X)=0.$$
By Lemma \ref{lem: has19 3.2 step 3 abu ver}, we may run a $(K_Y+B_Y+\Mm_Y)$-MMP$/V'$ with scaling of a general ample$/V'$ divisor $A$ on $Y$, which terminates with a good minimal model $(X',B',\Mm)/V'$ of $(Y,B_Y,\Mm)/V'$ such that $K_{X'}+B'+\Mm_{X'}\sim_{\Rr,V'}0$. Let $\pi': X'\rightarrow V'$ be the induced contraction. 
\begin{center}$\xymatrix{
X'\ar@{->}[dr]_{\pi'}& Y\ar@{-->}[l]\ar@{->}[r]^{f}\ar@{->}[d]_{\pi_Y} & X\ar@{->}[d]^{\pi}\\
& V'\ar@{->}[r]^{\varphi} & V
}$
\end{center}
We show that $(X',B',\Mm)/U,\pi',\varphi$ satisfy our requirements. (1)(2) follow from our construction. 

Let $p: W'\rightarrow Y$ and $q: W'\rightarrow X'$ be a resolution of indeterminacy of the induce map $Y\dashrightarrow X'$ such that $p$ is a log resolution of $(Y,B_Y)$. Then we have
$$p^*(K_Y+B_Y+\Mm_Y)=q^*(K_{X'}+B'+\Mm_{X'})+F$$
for some $F\geq 0$ that is exceptional over $X'$. Let $B_{W'}:=p^{-1}_*B_Y+\Exc(p)$, then $(W',B_{W'},\Mm)$ is a log smooth model of $(Y,B_Y,\Mm)$ and $(X',B',\Mm)$.

Since $K_Y+B_Y+\Mm_Y=f^*(K_{X}+B+\Mm_{X})+E$, by Theorem \ref{thm: existence good minimal model under pullbacks}, $(X,B,\Mm)/U$ has a good minimal model if and only if $(Y,B_Y,\Mm)/U$ has a good minimal model, if and only if $(W',B_{W'},\Mm)/U$ has a good minimal model, if and only if $(X',B',\Mm)/U$ has a good minimal model, hence (3).




By \cite[Lemmas 3.6, 3.10(1)]{HL21} ($=$\cite[Version 3, Lemmas 3.10, 3.17]{HL21}), a g-pair $(X'',B'',\Mm)/U$ is a weak glc model of $(X,B,\Mm)/U$ if and only if $(X'',B'',\Mm)/U$ is a weak glc model of $(W',B_{W'},\Mm)/U$, if and only if $(X'',B'',\Mm)/U$ is a weak glc model of $(X',B',\Mm)/U$, hence (4).

Let $D$ be a glc place of $(X',B',\Mm)$. Since $Y\dashrightarrow X'$ is a $(K_{Y}+B_Y+\Mm_Y)$-MMP$/V'$, $D$ is a glc place of $(Y,B_Y,\Mm)$, hence a glc place of $(X,B,\Mm)$. Thus if all glc centers of $(X,B,\Mm)$ dominate $V$, then all glc centers of $(X',B',\Mm)$ dominate $V$, hence all glc centers of $(X',B',\Mm)$ dominate $V'$ as $\varphi$ is birational, and we have (5). 

Finally, by Lemma \ref{lem: property of numerical and Iitaka dimension}(3), 
\begin{align*}
    \kappa_{\sigma}(X/U,K_X+B+\Mm_X)&=\kappa_{\sigma}(Y/U,K_Y+B_Y+\Mm_Y)=\kappa_{\sigma}(W'/U,p^*(K_Y+B_Y+\Mm_Y))\\
    &=\kappa_{\sigma}(W'/U,q^*(K_{X'}+B'+\Mm_{X'}))=\kappa_{\sigma}(X'/U,K_{X'}+B'+\Mm_{X'})
\end{align*}
and
\begin{align*}
    \kappa_{\iota}(X/U,K_X+B+\Mm_X)&=\kappa_{\iota}(Y/U,K_Y+B_Y+\Mm_Y)=\kappa_{\iota}(W'/U,p^*(K_Y+B_Y+\Mm_Y))\\
    &=\kappa_{\iota}(W'/U,q^*(K_{X'}+B'+\Mm_{X'}))=\kappa_{\iota}(X'/U,K_{X'}+B'+\Mm_{X'}),
\end{align*}
and we get (6).
\end{proof}



\begin{prop}\label{prop: prop 3.4 has19 abu ver}
Let $(X,B,\Mm)/U$ be an NQC glc g-pair and $\pi: X\rightarrow V$ a contraction over $U$, such that
\begin{itemize}
    \item $V$ is normal quasi-projective,
    \item $\kappa_{\sigma}(X/V,K_X+B+\Mm_X)=\kappa_{\iota}(X/V,K_X+B+\Mm_X)=0$ and $\kappa_{\sigma}(X/U,K_X+B+\Mm_X)=\dim V-\dim U$, and 
    \item all glc centers of $(X,B,\Mm)$ dominate $V$.
\end{itemize}
Then:
\begin{enumerate}
    \item $(X,B,\Mm)/U$ has a good minimal model, and
    \item Let $(\bar X,\bar B,\Mm)/U$ be a good minimal model of $(X,B,\Mm)/U$ and $\bar X\rightarrow\bar V$ is the contraction over $U$ induced by $K_{\bar X}+\bar B+\Mm_{\bar X}$. Then all glc centers of $(\bar X,\bar B,\Mm)$ dominate $\bar V$.
\end{enumerate}
\end{prop}
\begin{proof}
By Theorem \ref{thm: hl21 v2 4.1 abu ver}, there exists a $\Qq$-factorial NQC gdlt g-pair $(X',B',\Mm)/U$, a contraction $\pi': X'\rightarrow V'$ over $U$, and a birational projective morphism $\varphi: V'\rightarrow V$ over $U$, such that 
\begin{itemize}
    \item $X'$ is birational to $X$ and $V'$ is smooth,
    \item $K_{X'}+B'+\Mm_{X'}\sim_{\mathbb R,V'}0$. In particular, $\kappa_{\sigma}(X'/V',K_{X'}+B'+\Mm_{X'})=0$ by Lemma \ref{lem: property of numerical and Iitaka dimension}(5),
    \item $(X,B,\Mm)/U$ has a good minimal model if and only if $(X',B',\Mm)/U$ has a good minimal model,
    \item any weak glc model of $(X,B,\Mm)/U$ is a weak glc model of $(X',B',\Mm)/U$, and any weak glc model of $(X',B',\Mm)/U$ is a  weak glc model of $(X,B,\Mm)/U$,
    \item all glc centers of $(X',B',\Mm)$ dominate $V'$, and
    \item $\kappa_{\sigma}(X'/U,K_{X'}+B'+\Mm_{X'})=\kappa_{\sigma}(X/U,K_X+B+\Mm_X)=\dim V-\dim U=\dim V'-\dim U$.
\end{itemize} 
\begin{claim}\label{claim: check last part of prop 3.4 has19 for g pair}
Assume that $(X',B',\Mm)/U$ has a good minimal model $(\bar X',\bar B',\Mm)/U$, $\bar X'\rightarrow \bar V'$ is the contraction over $U$ induced by $K_{\bar X'}+\bar B'+\Mm_{\bar X'}$, and all glc centers of $(\bar X',\bar B',\Mm)$ dominate $\bar V'$. Then Proposition \ref{prop: prop 3.4 has19 abu ver}(2) holds for $(X,B,\Mm)/U$.
\end{claim}
\begin{proof}
Let $(\bar X,\bar B,\Mm)/U$ be a good minimal model of $(X,B,\Mm)/U$. Then $(\bar X,\bar B,\Mm)/U$ is a weak glc model of $(X',B',\Mm)/U$. Since $(\bar X',\bar B',\Mm)/U$ is also a weak glc model of  $(X',B',\Mm)/U$, by \cite[Lemma 3.5(1)]{HL21} ($=$\cite[Version 3, Lemma 3.9(1)]{HL21}), we may take a resolution of indeterminacy $p: W\rightarrow\bar X$ and $q: W\rightarrow\bar X'$ of the induced birational map $\bar X\dashrightarrow\bar X'$, such that 
$$p^*(K_{\bar X}+\bar B+\Mm_{\bar X})=q^*(K_{\bar X'}+\bar B'+\Mm_{\bar X'}).$$
Then $K_{\bar X}+\bar B+\Mm_{\bar X}$ is semi-ample$/U$, and if we let $\bar X\rightarrow\bar V$ be the contraction over $U$ induced by $K_{\bar X}+\bar B+\Mm_{\bar X}$, then $\bar V=\bar V'$. Since all glc centers of $(\bar X',\bar B',\Mm)$ dominate $\bar V'=\bar V$, all glc centers of  $(\bar X,\bar B,\Mm)$ dominate $\bar V$, and the claim is proved.
\end{proof}
\noindent\textit{Proof of Proposition \ref{prop: prop 3.4 has19 abu ver} continued}. By Claim \ref{claim: check last part of prop 3.4 has19 for g pair}, we may replace $(X,B,\Mm),V$ and $\pi$ with $(X',B',\Mm),V'$ and $\pi'$ respectively, and assume that $V$ is smooth and $K_X+B+\Mm_X\sim_{\mathbb R,V}0$. By Theorem \ref{thm: numerical generalized canonical bundle formula}, there exists an NQC gklt g-pair $(V,B_V,\Mm^V)/U$ such that 
$$K_X+B+\Mm_X\sim_{\mathbb R}\pi^*(K_V+B_V+\Mm^V_V).$$
By Lemma \ref{lem: property of numerical and Iitaka dimension}(4)(5), we have
$$\kappa_{\sigma}(V/U,K_V+B_V+\Mm^V_V)=\kappa_{\sigma}(X/U,K_X+B+\Mm_X)=\dim V-\dim U.$$
By Lemma \ref{lem: property of numerical and Iitaka dimension}(1), $K_V+B_V+\Mm^V_V$ is big$/U$. By \cite[Lemma 4.4(2)]{BZ16}, we may run a $(K_V+B_V+\Mm^V_V)$-MMP$/U$ with scaling of some general ample$/U$ divisor $A$, which terminates with a good minimal model $(\widehat V,B_{\widehat V},\Mm^V)/U$ of $(V,B_V,\Mm^V)/U$. Let $\phi: V\dashrightarrow\widehat V$ be the induced morphism, and let $g: \tilde V\rightarrow V$ and $\widehat g: \tilde V\rightarrow\widehat V$ be a common resolution such that $\widehat g=\phi\circ g$. Let $h: W\rightarrow X$ be a log resolution of $(X,\Supp B)$ such that $\Mm$ descends to $W$ and the induced map $\pi_W: W\rightarrow\tilde V$ is a morphism. By Lemma \ref{lem: existence of proper log smooth model}, there exists a proper log smooth model $(W,B_W,\Mm)$ of $(X,B,\Mm)$. In particular,
$$K_W+B_W+\Mm_W=h^*(K_X+B+\Mm_X)+E$$
for some $h$-exceptional $\Rr$-divisor $E\geq 0$. Assume that
$$g^*(K_V+B_V+\Mm^V_V)=\widehat g^*(K_{\widehat V}+B_{\widehat V}+\Mm^V_{\widehat V})+F.$$
Then
\begin{align*}
    K_W+B_W+\Mm_W&=h^*(K_X+B+\Mm_X)+E\sim_{\mathbb R}(\pi\circ h)^*(K_V+B_V+\Mm^V_V)+E\\
    &=\pi_W^*g^*(K_V+B_V+\Mm^V_V)+E=\pi_W^*\widehat g^*(K_{\widehat V}+B_{\widehat V}+\Mm^V_{\widehat V})+\pi_W^*F+E.
\end{align*}
Since $E$ is exceptional over $X$, $E$ is very exceptional over $V$. Since $\phi$ is a birational contraction, $E$ is very exceptional over $\widehat V$. Since $F$ is exceptional over $\widehat V$, $\pi_W^*F$ is very exceptional over $\widehat V$. Thus $\pi_W^*F+E$ is very exceptional over $\widehat V$. In particular, $$K_W+B_W+\Mm_W\sim_{\mathbb R,\widehat V}\pi_W^*F+E$$
is very exceptional over $\widehat V$. By Lemma \ref{lem: rlinear version of hl18 3.8}, we may run a $(K_W+B_W+\Mm_W)$-MMP$/\widehat V$ with scaling of a general ample$/\widehat V$ divisor which terminates with a good minimal model $(\widehat W,B_{\widehat W},\Mm)/\widehat V$ such that $K_{\widehat W}+B_{\widehat W}+\Mm_{\widehat W}\sim_{\Rr,\widehat V}0$ and the induced birational map $W\dashrightarrow\widehat W$ exactly contracts $\Supp(\pi_W^*F+E)$. In particular, let $\pi_{\widehat W}: \widehat W\rightarrow\widehat  V$ be the induced morphism, then $$K_{\widehat W}+B_{\widehat W}+\Mm_{\widehat W}\sim_{\Rr}\pi_{\widehat W}^*(K_{\widehat V}+B_{\widehat V}+\Mm^V_{\widehat V}).$$
Since $(\widehat V,B_{\widehat V},\Mm^V)/U$ is a good minimal model of $(V,B_V,\Mm^V)/U$, $K_{\widehat V}+B_{\widehat V}+\Mm^V_{\widehat V}$ is semi-ample$/U$, hence $K_{\widehat W}+B_{\widehat W}+\Mm_{\widehat W}$ is semi-ample$/U$. Thus $(\widehat W,B_{\widehat W},\Mm)/U$ is a good minimal model of $(W,B_W,\Mm)/U$. By \cite[Lemma 3.6]{HL21} ($=$\cite[Version 3, Lemma 3.10]{HL21}), $(\widehat W,B_{\widehat W},\Mm)/U$ is a good minimal model of $(X,B,\Mm)/U$, which implies (1).

Let $(\bar X,\bar B,\Mm)/U$ be a good minimal model of $(X,B,\Mm)/U$. By \cite[Lemma 3.5(1)]{HL21} ($=$\cite[Version 3, Lemma 3.9(1)]{HL21}), there exists a resolution $f: Z\rightarrow \bar X$ and $\widehat f: Z\rightarrow\widehat W$ of indeterminacy of the induced birational map $\bar X\dashrightarrow\widehat W$, such that 
$$f^*(K_{\bar X}+\bar B+\Mm_{\bar X})=\widehat f^*(K_{\widehat W}+B_{\widehat W}+\Mm_{\widehat W}).$$ In particular, any glc place of $(\bar X,\bar B,\Mm)$ is a glc place of $(\widehat W,B_{\widehat W},\Mm)$, hence a glc place of $(W,B_W,\Mm)$, and hence a glc place of $(X,B,\Mm)$ by Lemma \ref{lem: proper log smooth model keep lc center} as $(W,B_W,\Mm)$ is a proper log smooth model of $(X,B,\Mm)$. In particular, any glc place of $(\bar X,\bar B,\Mm)$ dominates $V$. Moreover, the contraction $\bar X\rightarrow\bar V$ induced by  $K_{\bar X}+\bar B+\Mm_{\bar X}$ factors through $\widehat V$, and the induced morphism $\widehat V\rightarrow\bar V$ is birational as $K_{\widehat V}+B_{\widehat V}+\Mm^V_{\widehat V}$ is big$/U$. In particular, the induced map $V\dashrightarrow\bar V$ is birational. Thus all glc places of $(\bar X,\bar B,\Mm)$ dominate $\bar V$, hence all glc centers of  $(\bar X,\bar B,\Mm)$ dominate $\bar V$, which implies (2).
\begin{center}$\xymatrix{
Z\ar@/^2pc/[rr]^{\widehat{f}}\ar@{->}[d]_{f} & W\ar@{->}[d]_{h}\ar@{->}[dr]^{\pi_W}\ar@{-->}[r] & \widehat{W}\ar@/^2pc/[dd]^{\pi_{\widehat W}}\\
\bar X\ar@{->}[d]& X\ar@{->}[d]_{\pi}\ar@{-->}[l]& \tilde V\ar@{->}[dl]_{g}\ar@{->}[d]^{\widehat g} \\
\bar V& V\ar@{-->}[r]^{\phi}\ar@{-->}[l] & \widehat V\ar@/^2pc/[ll]
}$
\end{center}
\end{proof}


\section{Special termination}

This section is adopted from \cite[Version 2, Section 6]{HL21}. 

\begin{defn}[{\cite[Definition 6.1]{HL21}}]\label{defn: gpair set for difficulty}
Let $\Ii\subset [0,1]$ and $\Ii'\subset [0,+\infty)$ be two sets. We define
$$\mathbb S(\Ii,\Ii'):=\{1-\frac{1}{m}+\sum_j\frac{r_jb_j}{m}+\sum_i\frac{s_i\mu_i}{m}\mid m\in\mathbb N^+,r_i,s_i\in\mathbb N, b_j\in\Ii,\mu_j\in\Ii'\}\cap ( 0,1].$$
\end{defn}

\begin{prop}[{\cite[Proposition 2.8]{HL18}}]\label{prop: hl18 2.8}
Let $\Ii\subset [0,1]$ and $\Ii'\subset [0,+\infty)$ be two sets. Let $(X,B,\Mm)/U$ be a $\Qq$-factorial NQC gdlt g-pair such that $B\in\Ii$ and $\Mm=\sum\mu_i\Mm_i$, where $\mu_i\in\Ii'$ for each $i$ and each $\Mm_i$ is nef$/U$ $\bb$-Cartier. Then for any glc center $S$ of $(X,B,\Mm)$, the g-pair $(S,B_S,\Mm^S)/U$ given by the adjunction
$$K_S+B_S+\Mm^S_S:=(K_X+B+\Mm_X)|_S$$
is gdlt, and $B_S\in\mathbb S(\Ii,\Ii')$.
\end{prop}


\begin{defn}[Difficulty, {\cite[Definition 4.3]{HL18}}]\label{defn: gpair difficulty}
Let $\Ii$ and $\Ii'$ be two finite sets of non-negative real numbers. Let $(X,B,\Mm)/U$ be a $\Qq$-factorial NQC gdlt g-pair such that $B\in\Ii$ and $\Mm=\sum\mu_i\Mm_i$, where $\mu_i\in\Ii'$ for each $i$ and each $\Mm_i$ is nef$/U$ $\bb$-Cartier. For any glc center $S$ of $(X,B,\Mm)$ of dimension $\geq 1$, let $(S,B_S,\Mm^S)$ be the g-pair given by the generalized adjunction
$$K_S+B_S+\Mm^S_S:=(K_X+B+\Mm_X)|_S,$$
then we define
\begin{align*}
    d_{\Ii,\Ii'}(S,B_S,\Mm^S):=&\sum_{\alpha\in\mathbb S(\Ii,\Ii')}\#\{E\mid a(E,B_S,\Mm^S)<1-\alpha,\Center_SE\not\subset\lfloor B_S\rfloor\}\\
    &+\sum_{\alpha\in\mathbb S(\Ii,\Ii')}\#\{E\mid a(E,B_S,\Mm^S)\leq 1-\alpha,\Center_SE\not\subset\lfloor B_S\rfloor\}.
\end{align*}
%\footnote{JL: This follows from \cite{HL18}, but I have no idea why we define difficulty in this way. Isn't the first sum already enough? I am fine with this, but it just looks little weird. CH: I think both are necessary. CH: I think it depends on how you set up the proof. When we are trying to show that we have isom in codim 1 on $S$, then we have to worry about contracting and extracting divisors. Contracting improves the first sum and extracting improves the second. One can also argue that we never extract divisors by using the second and then argue that by some Picard number argument that we can't contract infinitely many times. I think we should just quote him. JL: I believe your philosophy is correct, but I guess the second way is what we are doing now. The previous proof I wrote goes through the first way and you figured out that it is wrong. Actually, it seems to me that the proof of \cite[Theorem 4.5]{HL18} has some mistakes (they try to use the second sum to show that contractions will increase the difficulty). I also checked first version of \cite{HL18}, but they are not emphasizing this place (just saying that the definition of difficulty implies that it will strictly decrease if $S_i$ and $S_{i+1}$ are not isomorphic in codimension $1$). Anyway, as long as it is correct, I agree that we can leave it in this way.}
\end{defn}


\begin{lem}\label{lem: special termination reduce to flip lemma}
Let $(X,B,\Mm)/U$ be a $\Qq$-factorial NQC gdlt g-pair and let
$$(X,B,\Mm):=(X_0,B_0,\Mm)\dashrightarrow (X_1,B_1,\Mm)\dashrightarrow\dots\dashrightarrow (X_i,B_i,\Mm)\dashrightarrow\dots$$
be a $(K_X+B+\Mm_X)$-MMP$/U$. Let $\phi_{i,j}: X_i\dashrightarrow X_{j}$ be the induced birational maps for each $i$. For any $i\geq 0$ and any glc center $S_i$ of $(X_i,B_i,\Mm)$ of dimension $\geq 1$, we let $(S_i,B_{S_i},\Mm^{S_i})/U$ be the generalized pair given by the adjunction
$$K_{S_i}+B_{S_i}+\Mm^{S_i}_{S_i}:=(K_{X_i}+B_i+\Mm_{X_i})|_{S_i}.$$
Then we have the following.
\begin{enumerate}
\item For any $i\gg 0$, $j\geq i$, and any glc center $S_i$ of $(X_i,B_i,\Mm)$, $\phi_{i,j}$ induces an isomorphism near the generic point of $S_i$. In particular, for any $i,j\gg 0$ and any glc center $S_i$ of $(X_i,B_i,\Mm)$, we may let $S_{i,j}$ be the strict transform of $S_i$ on $X_j$.
\item Fix $i\gg 0$ and a glc center $S_i$ of $(X_i,B_i,\Mm)$ such that $\phi_{i,j}$ induces an isomorphism for every glc center of $(S_i,B_{S_i},\Mm^{S_i})/U$ for any $j\geq i$. Then
\begin{enumerate}
    \item $\phi_{j,k}|_{S_{i,j}}: S_{i,j}\dashrightarrow S_{i,k}$ is an isomorphism in codimension $1$ for any $j,k\gg i$, and
    \item $B_{S_{i,j}}$ is the strict transform of $B_{S_{i,k}}$ for any $j,k\gg i$.
\end{enumerate}
\item Suppose that this $(K_X+B+\Mm_X)$-MMP$/U$ is a MMP with scaling of an $\Rr$-divisor $A\geq 0$ on $X$. Let $$\lambda_j:=\inf\{t\mid t\geq 0, K_{X_j}+B_j+tA_j+\Mm_{X_j}\text{ is nef/}U\}$$
be the scaling numbers, where $A_j$ is the strict transform of $A$ on $X_j$ for each $j$. Fix $i\gg 0$ and a glc center $S_i$ of $(X_i,B_i,\Mm)$ such that $\phi_{j,k}|_{S_{i,j}}: S_{i,j}\dashrightarrow S_{i,k}$ is an isomorphism in codimension $1$ and $B_{S_{i,j}}$ is the strict transform of $B_{S_{i,k}}$ for any $k,j\geq i$. Let $T$ be the normalization of the image of $S_i$ on $U$, $(S_{i}',B_{S_{i}'},\Mm^{S_i})$ a gdlt model of $(S_{i},B_{S_{i}},\Mm^{S_i})$, and $A_{S_{i}'}$ the pullback of $A_i$ on $S_{i}'$. Then this $(K_X+B+\Mm_X)$-MMP$/U$ induces the following commutative diagram$/T$
 \begin{center}$
 \xymatrixrowsep{0.135in}
\xymatrixcolsep{0.03in}
\xymatrix{
 (S_{i}',B_{S_{i}'},\Mm^{S_i})\ar@{-->}[rr]\ar@{->}[d] & &  (S_{i,i+1}',B_{S_{i,i+1}'},\Mm^{S_i})\ar@{-->}[rr]\ar@{->}[d] & &  \dots\ar@{-->}[rr] & & (S_{i,j}',B_{S_{i,j}'},\Mm^{S_i})\ar@{-->}[rr]\ar@{->}[d] & & \dots\\
 (S_{i},B_{S_{i}},\Mm^{S_i})\ar@{-->}[rr] & &  (S_{i,i+1},B_{S_{i,i+1}},\Mm^{S_i})\ar@{-->}[rr]& &  \dots\ar@{-->}[rr] & & (S_{i,j},B_{S_{i,j}},\Mm^{S_i})\ar@{-->}[rr] & & \dots\\
}$
\end{center}
such that
\begin{enumerate}
    \item $$(S_{i}',B_{S_{i}'},\Mm^{S_i})\dashrightarrow  (S_{i,i+1}',B_{S_{i,i+1}'},\Mm^{S_i})\dashrightarrow\dots\dashrightarrow (S_{i,j}',B_{S_{i,j}'},\Mm^{S_i})\dashrightarrow\dots$$
is a $(K_{S_{i}'}+B_{S_{i}'}+\Mm^{S_i}_{S_{i}'})$-MMP$/T$ with scaling of $A_{S_i'}$. Note that it is possible that $(S_{i,j}',B_{S_{i,j}'},\Mm^{S_i})\dashrightarrow (S_{i,j+1}',B_{S_{i,j+1}'},\Mm^{S_i})$ is the identity morphism or a composition of several steps of the  $(K_{S_{i,j}'}+B_{S_{i,j}'}+\Mm^{S_i}_{S_{i,j}'})$-MMP$/T$ for some $j$,
\item for any $j\geq i$, $(S_{i,j}',B_{S_{i,j}'},\Mm^{S_i})$ is a gdlt model of $(S_{i,j},B_{S_{i,j}},\Mm^{S_i})$, and
\item let
$$\mu_j:=\inf\{t\mid t\geq 0, K_{S_{i,j}'}+B_{S_{i,j}'}+tA_{S_{i,j}'}+\Mm^{S_i}_{S_{i,j}'}\text{ is nef}/T\}$$
for each $j\geq i$, where $A_{S_{i,j}'}$ is the pullback of $A_j$ on $S_{i,j}'$. Then $\mu_j\leq\lambda_j$ for each $j\geq i$.
\end{enumerate}

\end{enumerate}
\end{lem}
\begin{proof}
Let $\Ii\subset [0,1]$ be a finite set such that $B\in\Ii$, and let $\Ii'\subset [0,+\infty)$ be a finite set such that $\Mm=\sum \mu_i\Mm_i$, where each $\Mm_i$ is nef$/U$ $\bb$-Cartier and each $\mu_i\in\Ii'$. Let $\phi_i:=\phi_{i,i+1}$ for each $i$.

We may assume that the MMP does not terminate, otherwise there is nothing left to prove. Possibly replacing $X$ with $X_i$ for $i\gg 0$, we may assume that each $\phi_i$ is a flip. Since the number of glc centers of $(X,B,\Mm)$ is finite, possibly replacing $X$ with $X_i$ for $i\gg 0$, we may assume that the flipping locus of $\phi_i$ does not contain any glc centers. This proves (1). 

We prove (2). We let $S:=S_i$. By (1), we may let $S_j:=S_{i,j}$ for any $j\geq i$. Possibly replacing $X$ we $X_i$, we may assume that $i=0$. By \cite[Proposition 2.8]{HL18}, for any $j$, the g-pair $(S_j,B_{S_j},\Mm^{S})$ given by the adjunction
$$K_{S_j}+B_{S_j}+\Mm_{S_j}^S:=(K_{X_j}+B_j+\Mm_{X_j})|_{S_j}$$
is gdlt, and $B_{S_j}\in\mathbb S(\Ii,\Ii')$. By assumption, $\phi_{j,k}$ induces an isomorphism on $\lfloor B_{S_j}\rfloor$ for any $j,k$. Thus for any $j$ and any prime divisor $E$ over $S_j$, $\Center_{S_j}E\subset\lfloor B_{S_j}\rfloor$ if and only if $\Center_{S_{j+1}}E\subset\lfloor B_{S_{j+1}}\rfloor$. By the negativity lemma, $a(E,S_j,B_{S_j},\Mm^{S})\leq a(E,S_{j+1},B_{S_{j+1}},\Mm^{S})$ for each $j$ and any prime divisor $E$ over $S_j$. Thus
$$d_{\Ii,\Ii'}(S_j,B_{S_j},\Mm^{S})\geq d_{\Ii,\Ii'}(S_{j+1},B_{S_{j+1}},\Mm^{S})$$
for each $j$. Moreover, for any $j$ such that $S_j$ and $S_{j+1}$ are not isomorphic in codimension $1$,  if there exists a prime divisor $E$ on $S_{j+1}$ that is exceptional over $S_j$, then $$1-\alpha=a(E,S_{j+1},B_{S_{j+1}},\Mm^{S})>a(E,S_j,B_{S_j},\Mm^{S})$$ for some $\alpha\in\mathbb S(\Ii,\Ii')$, and hence $$d_{\Ii,\Ii'}(S_j,B_{S_j},\Mm^{S})> d_{\Ii,\Ii'}(S_{j+1},B_{S_{j+1}},\Mm^{S}).$$
By \cite[Remark 4.4]{HL18}, $d_{\Ii,\Ii'}(S_j,B_{S_j},\Mm^{S})<+\infty$. Thus possibly replacing $X$ with $X_j$ for some $j\gg 0$, we may assume that
$d_{\Ii,\Ii'}(S_j,B_{S_j},\Mm^{S})=d_{\Ii,\Ii'}(S_k,B_{S_k},\Mm^{S})$ for any $j,k$. Thus $S_j\dashrightarrow S_{j+1}$ does not extract any divisor for any $j$. In particular, $\rho(S_{j+1})\leq \rho(S_j)$, and $\rho(S_{j+1})<\rho(S_j)$ if $S_j\dashrightarrow S_{j+1}$ contracts a divisor. Thus possibly replacing $X$ with $X_j$ for some $i\gg 0$, we may assume that  $S_j$ and $S_{j+1}$ are isomorphic in codimension $1$ for each $j$, which implies (2.a). Since $d_{\Ii,\Ii'}(S_j,B_{S_j},\Mm^{S})=d_{\Ii,\Ii'}(S_k,B_{S_k},\Mm^{S})$ for any $j,k$, (2.b) follows from (2.a).

We prove (3). Since $i\gg 0$, possibly replacing $X$ with $X_i$, we may assume that $i=0$ and $\phi_{j}$ is a flip for every $j$. We let $S:=S_0,S':=S_0',S_j:=S_{0,j}$, and $S_j':=S_{0,j}'$ for every $j$. We let $X_j\rightarrow Z_j\leftarrow X_{j+1}$ be each flip and let $T_j$ be the normalization of the image of $S_{j}$ on $Z_j$ for each $j$. Then we have an induced birational map $S_{j}\dashrightarrow S_{j+1}$ for each $j$.

Since $\phi_{0}$ is a $(K_{X_0}+B_0+\Mm_{X_0})$-flip$/U$, $X_{1}\rightarrow Z_0$ is $(K_{X_{1}}+B_1+\Mm_{X_1})$-positive and $K_{S_1}+B_{S_1}+\Mm^S_{S_1}$ is ample$/T_0$. In particular, $(S_1,B_{S_1},\Mm^S)/T_0$ is a weak glc model of $(S_0,B_{S_0},\Mm^S)$. By \cite[Lemmas 3.5, 3.8]{HL21} ($=$\cite[Version 3, Lemmas 3.9, 3.15]{HL21}) and Theorem \ref{thm: existence good minimal model under pullbacks}, we may run a $(K_{S_0'}+B_{S_0'}+\Mm^{S}_{S_0'})$-MMP$/T_0$ with scaling of an ample$/T_0$ divisor, which terminates with a good minimal model of $(S_0',B_{S_0'},\Mm^S)/T_0$. By \cite[Lemma 3.5]{HL21} ($=$\cite[Version 3, Lemma 3.9]{HL21}), $(S_0',B_{S_0'},\Mm^S)$ is a gdlt model of $(S_1,B_{S_1},\Mm^S)$. Since 
$$K_{S_0'}+B_{S_0'}+\lambda_0A_{S_0'}+\Mm^{S}_{S_0'}\equiv_{T_0}0,$$
 this MMP is also a $(K_{S_0'}+B_{S_0'}+\Mm^{S}_{S_0'})$-MMP$/T_0$ with scaling of $\lambda_0A_{S_0'}$. 
 We may replace $(S_0,B_{S_0},\Mm^S)/T$ with $(S_1,B_{S_1},\Mm^S)/T$ and continue this process. This gives us the desired $(K_{S_0'}+B_{S_0'}+\Mm^{S}_{S_0'})$-MMP$/T$ with scaling of $A_{S_0'}$, which gives the commutative diagram, and proves (3.a) and (3.b). For each $j$, since $K_{S_j'}+B_{S_j'}+\lambda_jA_{S_j'}+\Mm^{S}_{S_j'}\equiv_{T_j}0$, 
 $K_{S_j'}+B_{S_j'}+\lambda_jA_{S_j'}+\Mm^{S}_{S_j'}$ is nef, hence $\mu_j\leq\lambda_j$, and we get (3.c).
\end{proof}


\begin{comment}

\begin{thm}\label{thm: special termination}
Let $(X,B,\Mm)/U$ be a $\Qq$-factorial NQC gdlt g-pair and $A\geq 0$ an $\Rr$-divisor on $X$, such that $(X,B+A,\Mm)$ is glc and $K_X+B+A+\Mm_X$ is nef$/U$. Let
$$(X,B,\Mm):=(X_0,B_0,\Mm)\dashrightarrow (X_1,B_1,\Mm)\dashrightarrow\dots\dashrightarrow (X_i,B_i,\Mm)\dashrightarrow\dots$$
be a $(K_X+B+\Mm_X)$-MMP$/U$ with scaling of $A$ with scaling of $A$, such that the scaling numbers
$$\lambda_i:=\inf\{t\mid t\geq 0, K_{X_i}+B_i+tA_i+\Mm_{X_i}\text{ is nef/}U\}$$
satisfies that $\lim_{i\rightarrow+\infty}\lambda_i=0$. Here $A_i$ is the strict transform of $A$ on $X_i$ for each $i$. For any $i\geq 0$ and any glc center $S_i$ of $(X_i,B_i,\Mm)$ of dimension $\geq 1$, we let $(S_i,B_{S_i},\Mm^{S_i})/U$ be the generalized pair given by the adjunction
$$K_{S_i}+B_{S_i}+\Mm^{S_i}_{S_i}:=(K_{X_i}+B_i+\Mm_{X_i})|_{S_i}.$$

Assume that for any $i$ and  any glc center $S_i$ of $(X_i,B_i,\Mm)$ of dimension $\geq 1$, $(S_i,B_{S_i},\Mm^{S_i})/U$ has either a log minimal model or a Mori fiber space. Then the MMP terminates near the strict transform of $\lfloor B\rfloor$.
\end{thm}
\begin{proof}
Let $\phi_{i,j}: X_i\dashrightarrow X_j$ be the induced birational maps for any $i,j$ and $\phi_i:=\phi_{i,i+1}$.



We only need to show that $\phi_{i}$ induces an isomorphism for every glc center of $X_i$ for any $i\gg 0$. We may assume that the MMP does not terminate, otherwise there is nothing left to prove. Possibly replacing $X$ with $X_i$ for $i\gg 0$, we may assume that each $\phi_i$ is a flip. We let $X_i\rightarrow Z_i\leftarrow X_{i+1}$ be the flip given by $\phi_i$, where $X_i\rightarrow Z_i$ is the flipping contraction. 

We apply induction on the dimension of glc centers. Since the number of glc centers of $(X,B,\Mm)$ is finite, possibly replacing $X$ with $X_i$ for $i\gg 0$, we may assume that the flipping locus of $\phi_i$ does not contain any glc centers. In particular, $\phi_i$ is an isomorphism of $0$-dimensional glc places for each $i$. Now we only need to prove that, for any $1\leq k\leq\dim X-1$ and any $k$-dimensional glc center $S$, suppose that $\phi_i$ induces an isomorphism for every $d$-dimensional glc center for each $d\leq k-1$ and each $i$, then $\phi_j$ induces an isomorphism near $S_j:=(\phi_{0,i})_*S$ for any $j\gg 0$. By Lemma \ref{lem: special termination reduce to flip lemma}, possibly replacing $X$ with $X_i$ for some $i\gg 0$, we may assume that $\phi_{j,k}|_{S_j}$ is an isomorphism in codimension $1$ for any $j,k$. Then
$$K_{S_i}+B_{S_i}+\Mm^S_{S_i}=\lim_{j\rightarrow+\infty}(\phi_{i,j}^{-1})_*((K_{X_j}+B_j+\lambda_jA_j+\Mm_{X_j})|_{S_j})$$
for each $i$. Since $K_{X_j}+B_j+\lambda_jA_j+\Mm_{X_j}$ is nef$/U$, $K_{S_i}+B_{S_i}+\Mm^S_{S_i}$ is pseudo-effective$/U$ for each $i$. 
%By our assumption, $(S_1,B_{S_1},\Mm^S)/U$ has a log minimal model.

Let $T_i$ be the normalization of the image of $S_i$ in $Z_i$ for each $i$, and let $T$ be the image of $S$ on $U$. Then we have an induced birational map $S_i\dashrightarrow S_{i+1}$ over $T_i$ for each $i$. Let $(S_0',B_{S_0'},\Mm^{S})$ be a gdlt model of $(S_0,B_{S_0},\Mm^{S})$, and let $A_{S_0'}$ be the pullback of $A$ on $S_0'$. We prove the following claim:
\begin{claim}\label{claim: lift sequence gmmp adjunction first}
\begin{enumerate}
    \item The $(K_X+B+\Mm_X)$-MMP$/U$ with scaling of $A$ induces a $(K_{S_0'}+B_{S_0'}+\Mm^{S}_{S_0'})$-MMP$/T$ with scaling of $A_{S_0'}$. 
    \item The $(K_{S_0'}+B_{S_0'}+\Mm^{S}_{S_0'})$-MMP$/T$ with scaling of $A_{S_0'}$ terminates.
\end{enumerate}
\end{claim}
\begin{proof}
Since $\phi_{0}$ is a $(K_{X_0}+B_0+\Mm_{X_0})$-flip$/U$, $X_{1}\rightarrow Z_0$ if $(K_{X_{1}}+B_1+\Mm_{X_1})$-positive and $K_{S_1}+B_{S_1}+\Mm^S_{S_1}$ is ample$/T_0$. In particular, $(S_1,B_{S_1},\Mm^S)/T_0$ is a weak glc model of $(S_0,B_{S_0},\Mm^S)$. By \cite[Lemmas 3.5, 3.8]{HL21} ($=$\cite[Version 3, Lemmas 3.9, 3.15]{HL21}) and Theorem \ref{thm: existence good minimal model under pullbacks}, we may run a $(K_{S_0'}+B_{S_0'}+\Mm^{S}_{S_0'})$-MMP$/T_0$ with scaling of an ample$/T_0$ divisor, which terminates with a good minimal model of $(S_0',B_{S_0'},\Mm^S)/T_0$. By \cite[Lemma 3.5]{HL21} ($=$\cite[Version 3, Lemma 3.9]{HL21}), $(S_0',B_{S_0'},\Mm^S)$ is a gdlt model of $(S_1,B_{S_1},\Mm^S)$.  Since 
$$K_{S_0'}+B_{S_0'}+\lambda_0A_{S_0'}+\Mm^{S}_{S_0'}\equiv_{T_0}0,$$
 this MMP is also a $(K_{S_0'}+B_{S_0'}+\Mm^{S}_{S_0'})$-MMP$/T_0$ with scaling of $\lambda_0A_{S_0'}$. 
 We may replace $(S_0,B_{S_0},\Mm^S)/T$ with $(S_1,B_{S_1},\Mm^S)/T$ and continue this process. This gives us the desired $(K_{S_0'}+B_{S_0'}+\Mm^{S}_{S_0'})$-MMP$/T$ with scaling of $A_{S_0'}$, which proves (1).
 \begin{center}$
 \xymatrixrowsep{0.135in}
\xymatrixcolsep{0.09in}
\xymatrix{
 (S_0',B_{S_0'},\Mm^S)\ar@{-->}[rr]\ar@{->}[d] & &  (S_1',B_{S_1'},\Mm^S)\ar@{-->}[rr]\ar@{->}[d] & &  \dots\ar@{-->}[rr] & & (S_i',B_{S_i'},\Mm^S)\ar@{-->}[rr]\ar@{->}[d] & & 
 (S_{i+1},B_{S_{i+1}},\Mm^S)\ar@{-->}[rr]\ar@{->}[d] & & \dots\\
 (S_0,B_{S_0},\Mm^S)\ar@{-->}[rr]\ar@{->}[dr] & &  (S_1,B_{S_1},\Mm^S)\ar@{-->}[rr]\ar@{->}[dl] & &  \dots\ar@{-->}[rr] & &(S_i,B_{S_i},\Mm^S)\ar@{-->}[rr]\ar@{->}[dr] & & (S_{i+1},B_{S_{i+1}},\Mm^S)\ar@{-->}[rr]\ar@{->}[dl] & & \dots\\
 & T_0 &  & & & &  & T_i & & & \\
}$
\end{center}
Let $\mu_i$ be the scaling numbers of the  $(K_{S_0'}+B_{S_0'}+\Mm^{S}_{S_0'})$-MMP$/T$ with scaling of $A_{S_0'}$ we constructed. Then $\mu_i\leq\lambda_i$ for each $i$, hence $\lim_{i\rightarrow+\infty}\mu_i=0$. By \cite[Remark 3.21, Theorem 4.1]{HL18}, this MMP terminates.
\end{proof}
\noindent\textit{Proof of Theorem \ref{thm: special termination} continued.} By Claim \ref{claim: lift sequence gmmp adjunction first}, $S_i\dashrightarrow S_{i+1}$ is the identity map for $i\gg 0$. Thus the original $(K_X+B+\Mm_X)$-MMP$/U$ with scaling of an ample$/U$ $\Rr$-divisor $A$ on $X$ terminates near $S_i$. We apply this for every $k$-dimensional glc center, and the theorem follows by induction on $k$.
\end{proof}
\end{comment}






\section{Apply Nakayama-Zariski decomposition}

This similar is similar to \cite[Section 3, before Theorem 3.14]{Has22}.


\begin{lem}[{cf. \cite[Lemma 3.5]{Has22}}]\label{lem: has22 3.5 rel} 
Let $(X,B,\Mm)/U$ and $(X',B',\Mm)/U$ be NQC gdlt g-pairs with a birational map $\phi: X\dashrightarrow X'$ over $U$ such that $\phi_*\Mm=\Mm$. Let $S$ and $S'$ be glc centers of $(X,B,\Mm)$ and $(X',B',\Mm)$ respectively, such that $\phi$ is an isomorphism near the generic point of $S$, and $\phi|_S: S\dashrightarrow S'$ defines a birational map$/U$. Suppose that
\begin{enumerate}
    \item $K_{X}+B+\Mm_X$ is pseudo-effective$/U$,
    \item for any prime divisor $D'$ on $X'$, $a(D',X',B',\Mm)\leq a(D',X,B,\Mm)$, and
    \item for every prime divisor $P$ over $X$ such that $a(P,X,B,\Mm)<1$ and $\Center_{X}(P)\cap S\not=\emptyset$, then $\sigma_{P}(X/U,K_X+B+\Mm_X)=0$.
\end{enumerate}
Let $(S,B_S,\Mm^S)/U$ and $(S',B_{S'},\Mm^S)/U$ be the g-pairs induced by the adjunctions $K_S+B_S+\Mm^S_S:=(K_X+B+\Mm_X)|_S$ and $K_{S'}+B_{S'}+\Mm^S_{S'}:=(K_{X'}+B'+\Mm_{X'})|_{S'}$. Then $a(Q,S',B_{S'},\Mm^S)\leq a(Q,S,B_S,\Mm^S)$ for all prime divisors $Q$ on $S'$. 
\end{lem} 


\begin{proof}
We follow \cite[Proof of Lemma 3.5]{Has22}. Let $p: W\rightarrow X$ and $q: W\rightarrow X'$ be a resolution of indeterminacy such that $\Mm$ descends to $W$, and let $S_W$ be the strict transform of $S$ on $W$. Then $p_S:=p|_{S_W}$ and and $q_S:=q|_{S_W}$ is a common resolution of $\phi|_S$. We may write
$$p^*(K_{X}+B+\Mm_{X})=q^*(K_{X'}+B'+\Mm_{X'})+G_{+}-G_{-}$$
such that $G_{+}\geq0,G_{-}\geq0$, and $G_+\wedge G_-=0$. By (2), $G_{+}$ is $q$-exceptional.
Since $K_{X}+B+\Mm_{X}$ is pseudo-effective$/U$, $K_{X'}+B'+\Mm_{X'}$ is pseudo-effective$/U$. Since $S$ and $S'$ are glc centers of $(X,B,\Mm)$ and $(X',B',\Mm)$ respectively, $S_W$ is not contained in $\Supp G_+$ and $\Supp G_-$.

We can write $G_{+}=G_{0}+G_{1}$, where all components of $G_{0}$ intersect $S_W$ and $G_{1}|_{S_W}=0$. For any component $E$ of $\Supp G_0$, we have
\begin{equation*}
\begin{split}
&\sigma_{E}(X/U,K_{X}+B+\Mm_{X})\\
=&\sigma_{E}(W/U,p^*(K_{X}+B+\Mm_{X})) \qquad \qquad \qquad \qquad \qquad \qquad (\text{Definition } \ref{defn: sigma over X})\\
=&\sigma_{E}(W/U,p^*(K_{X}+B+\Mm_{X}))+\mult_E(G_-) \qquad \qquad \ \ \ \ \ \ (G_-\wedge G_+=0, E\subset\Supp G_+)\\
\geq&\sigma_{E}(W/U,p^*(K_{X}+B+\Mm_{X})+G_-) \qquad \qquad \qquad \qquad \quad \ (\text{Lemma }\ref{lem: nz basic properties}(1))\\
=&\sigma_{E}(W/U,q^{*}(K_{X'}+B'+\Mm_{X'})+G_{+}) \\
=& \sigma_{E}(W/U,q^{*}(K_{X'}+B'+\Mm_{X'}))+\mult_{E}(G_{+})>0 \qquad \ \ \ \ (\text{Lemma }\ref{lem: nz keep under pullback}(2), E\subset\Supp G_+).
\end{split}
\end{equation*}
By (3), $a(E,X,B,\Mm)\geq 1$. Since $E\subset\Supp G_+$, $a(E,X',B',\Mm)>a(E,X,B,\Mm)\geq 1$.

\begin{claim}\label{claim: has 22 3.5 lemma followed proof}
$E|_{S_W}$ is exceptional over $S'$.
\end{claim}

Grant the claim for the time being. We have $G_0|_{S_W}$ is exceptional over $S'$, hence for any prime divisor $Q$ on $S'$, $\mult_Q(G_0|_{S_W})=0$. Since we have $G_{1}|_{\bar{S}}=0$ and
\begin{align*}
    &p_S^{*}(K_S+B_S+\Mm^S_S)-q_S^{*}(K_{S'}+B_{S'}+\Mm^S_{S'})\\
    =&p_S^{*}((K_X+B+\Mm_X)|_{S})-q_S^{*}((K_{X'}+B'+\Mm_{X'})|_{S'})=
    (G_{0}+G_{1})|_{S_W}-G_{-}|_{S_W},
\end{align*}
we have
$$a(Q,S',B_{S'},\Mm)-a(Q,S,B_S,\Mm^S)=\mult_Q((G_{0}+G_{1})|_{S_W}-G_{-}|_{S_W})=\mult_Q(-G_-|_{S_W})\leq 0$$
for any prime divisor $Q$ on $S'$, and the lemma follows.
\end{proof}
\begin{proof}[Proof of Claim \ref{claim: has 22 3.5 lemma followed proof}]
We may write 
$$K_W+B_W+\Mm_W=q^*(K_{X'}+B'+\Mm_{X'})+G'$$
such that $B_W\geq 0, G'\geq 0,$ $G'$ is $q$-exceptional, and $B_W\wedge G'=0$. By our construction, $(B_W-S_W)|_{S_W}$ and $G'|_{S_W}$ does not have any common component. By \cite[Definition 4.7]{BZ16}, $$0\leq B_S=(q_S)_*((B_W-S_W-G')|_{S_W})=(q_S)_*(B_W-S_W)-(q_S)_*G'|_{S_W},$$
hence $(q_S)_*G'|_{S_W}=0$. Thus $G'|_{S_W}$ is exceptional over $S'$.  Since $a(E',X',B',\Mm)>1$, $E$ is a component of $G'$, hence $E|_{S_W}$ is exceptional over $S'$, and the claim follows.
\end{proof}

\begin{lem}[{cf. \cite[Lemma 3.6]{Has22}, \cite[Lemma 5.3]{HMX18}}]\label{lem: has22 3.6 rel gdlt} Let $(X,B_1,\Mm)/U$ and $(X,B_2,\Mm)/U$ be $\Qq$-factorial NQC gdlt g-pairs, such that $K_X+B_1+\Mm_X$ is pseudo-effective$/U$ and $0\leq B_1-B_2\leq N_{\sigma}(X/U,K_X+B_1+\Mm_X)$. Then $(X,B_1,\Mm)/U$ has a log minimal model (resp. good minimal model) if and only if $(X,B_2,\Mm)/U$ has a log minimal model (resp. good minimal model).  
\end{lem}
\begin{proof}
$N_{\sigma}(X/U,K_X+B_1+\Mm_X)$ is well-defined by Lemma \ref{lem: nz finite is well-defined}(1). 

First we assume that $(X,B_1,\Mm)/U$ has a log minimal model (resp. good minimal model). By \cite[Theorem 2.8]{HL21} ($=$\cite[Version 3, Theorem 2.24]{HL21}), we may run a $(K_X+B_1+\Mm_X)$-MMP$/U$ which terminates with a log minimal model (resp. good minimal model) $(X',B',\Mm)/U$ with induced birational map $\phi: X\dashrightarrow X'$ over $U$. By Lemma \ref{lem: nz for glc divisor}(2.b), $f$ contracts every component of $N_{\sigma}(X/U,K_X+B_1+\Mm_X)$. Thus $B'$ is also the strict transform of $B$ on $X'$.

Let $p: W\rightarrow X$ and $q: W\rightarrow X'$ be a resolution of indeterminacy of $\phi$, and let
$$p^*(K_X+B_1+\Mm_X)=q^*(K_{X'}+B'+\Mm_{X'})+E,$$
then by Lemmas \ref{lem: nz basic properties}(3) and \ref{lem: nz keep under pullback}(2)(3), $p_*E=N_{\sigma}(X/U,K_X+B_1+\Mm_X)$. Suppose that
$$p^*(K_X+B_2+\Mm_X)=q^*(K_{X'}+B'+\Mm_{X'})+F,$$
then $$F=E-p^*(B_1-B_2)\geq E-p^*N_{\sigma}(X/U,K_X+B_1+\Mm_X)=E-p^*p_*E.$$
Since $E-p^*p_*E$ is $p$-exceptional, $p_*F\geq 0$. By the negativity lemma, $F\geq 0$. Thus $(X',B',\Mm)/U$ is a weak glc model of $(X,B_2,\Mm)/U$. By \cite[Lemmas 3.5(2), 3.8]{HL21} ($=$\cite[Version 3, Lemmas 3.9(2), 3.15]{HL21}), $(X,B_2,\Mm)/U$ has a log minimal model (resp. good minimal model).

Now we assume that $(X,B_2,\Mm)/U$ has a log minimal model (resp. good minimal model). By \cite[Theorem 2.8]{HL21} ($=$\cite[Version 3, Theorem 2.24]{HL21}), we may run a $(K_X+B_2+\Mm_X)$-MMP$/U$ which terminates with a log minimal model (resp. good minimal model) $(X',B',\Mm)/U$ with induced birational map $\phi: X\dashrightarrow X'$ over $U$. Let $C:=B_1-B_2$, then $\phi$ is also a  $(K_X+B_2+\epsilon C+\Mm_X)$-MMP$/U$ for any $0<\epsilon\ll 1$. Let $C'$ be the strict transform of $C$  on $X'$. By Lemma \ref{lem: nz basic properties}(5) and \cite[Lemma 3.17]{HL18}, we may pick $0<\epsilon\ll 1$, such that
\begin{itemize}
    \item $\Supp N_{\sigma}(X/U,K_X+B_2+\epsilon C+\Mm_X)=\Supp N_{\sigma}(X/U,K_X+B_1+\Mm_X)$, and
    \item any partial $(K_{X'}+B'+\epsilon C'+\Mm_{X'})$-MMP$/U$ is $(K_{X'}+B'+\Mm_{X'})$-trivial$/U$.
\end{itemize}
We run a $(K_{X'}+B'+\epsilon C'+\Mm_{X'})$-MMP$/U$ with scaling of an ample$/U$ $\Rr$-divisor. By Lemma \ref{lem: limit movable r divisors gpairs}, after finitely many steps we get a birational map $\psi: X'\dashrightarrow X''$ such that $K_{X''}+B''+\epsilon C''+\Mm_{X''}$ is a limit of movable$/U$ $\Rr$-divisors, where $B''$ and $C''$ are the strict transforms of $B'$ and $C'$ on $X''$ respectively. By Lemma \ref{lem: nz for glc divisor}(2.b), the set of $(\psi\circ\phi)$-exceptional divisors is exactly $\Supp N_{\sigma}(X/U,K_X+B_2+\epsilon C+\Mm_X)=\Supp N_{\sigma}(X/U,K_X+B_1+\Mm_X)$. Thus $C''=0$, $B''$ is also the strict transform of $B_1$ on $X''$, and $K_{X''}+B''+\Mm_{X''}$ is nef$/U$ (resp. semi-ample$/U$). By Lemma \ref{lem: hmx18 2.7.3 gpair rel}, $(X'',B'',\Mm)/U$ is a log minimal model of $(X,B_1,\Mm)/U$. The lemma follows from \cite[Lemma 3.5(2)]{HL21} ($=$\cite[Version 3, Lemma 3.9(2)]{HL21}).
\end{proof}


\begin{lem}[{cf. \cite[Lemma 3.6]{Has22}}]\label{lem: has22 3.6 rel} Let $(X,B,\Mm)/U$ and $(Y,B_Y,\Mm)/U$ be NQC glc g-pairs and $f: Y\rightarrow X$ a birational morphism, such that
\begin{enumerate}
    \item $K_X+B+\Mm_X$ is pseudo-effective$/U$, and
    \item for any prime divisor $D$ on $Y$,
    $$0\leq a(D,Y,B_Y,\Mm)-a(D,X,B,\Mm)\leq\sigma_D(X/U,K_X+B+\Mm_X).$$
\end{enumerate}
Then $(X,B,\Mm)/U$ has a log minimal model (resp. good minimal model) if and only if $(Y,B_Y,\Mm)/U$ has a log minimal model (resp. good minimal model).  
\end{lem}
\begin{proof}
Let $g: W\rightarrow Y$ be a log resolution of $(Y,\Supp B_Y)$ such that $\Mm$ descends to $W$ and $h:=f\circ g$ is a log resolution of $(X,\Supp B)$. Let $B_W:=h^{-1}_*B+\Supp\Exc(h)$ and $B'_W:=g^{-1}_*B_Y+\Supp\Exc(g)$, then we have
$$K_W+B_W+\Mm_W=h^*(K_X+B+\Mm_X)+E$$
for some $E_W\geq 0$ that is exceptional$/X$. By Lemma \ref{lem: nz keep under pullback}(1)(2), 
$$\sigma_P(W/U,K_W+B_W+\Mm_W)=\sigma_P(X/U,K_X+B+\Mm_X)+\mult_PE$$
for any prime divisor $P$ on $W$.
\begin{claim}\label{claim: has 3.6 change to gdlt}
For any prime divisor $P$ on $W$,
$$0\leq a(P,W,B_W',\Mm)-a(P,W,B_W,\Mm)\leq\sigma_P(W/U,K_W+B_W+\Mm_W).$$
\end{claim}
Grant Claim \ref{claim: has 3.6 change to gdlt} for the time being. By Claim \ref{claim: has 3.6 change to gdlt} and Theorem \ref{thm: existence good minimal model under pullbacks}, possibly replacing $(X,B,\Mm)/U$ and $(Y,B_Y,\Mm)/U$ with $(W,B_W,\Mm)/U$ and $(W,B_W',\Mm)/U$ respectively, we may assume that $(X,B,\Mm)$ and $(Y,B_Y,\Mm)$ are $\Qq$-factorial gdlt and $X=Y$. The lemma follows from Lemma \ref{lem: has22 3.6 rel gdlt}.
\end{proof}


\begin{proof}[Proof of Claim \ref{claim: has 3.6 change to gdlt}]
For any prime divisor $P$ on $W$, one of the following cases holds:

\medskip

\noindent\textbf{Case 1}. $P$ is not exceptional over $X$. In this case, 
$$a(P,W,B_W',\Mm)-a(P,W,B_W,\Mm)=a(P,Y,B_Y,\Mm)-a(P,X,B,\Mm)$$
and the claim follows.

\medskip

\noindent\textbf{Case 2}. $P$ is exceptional over $X$ but not exceptional over $Y$. In this case, $a(P,W,B_W,\Mm)=0$, $a(P,W,B_W',\Mm)=a(P,Y,B_Y,\Mm)$, and $a(P,X,B,\Mm)=\mult_PE$, so
$$0\leq a(P,Y,B_Y,\Mm)=a(P,W,B_W',\Mm)-a(P,W,B_W,\Mm)=a(P,Y,B_Y,\Mm)-a(P,X,B,\Mm),$$
and
\begin{align*}
    a(P,Y,B_Y,\Mm)&\leq\sigma_P(X/U,K_X+B+\Mm_X)+a(P,X,B,\Mm)\\
    &=\sigma_P(X/U,K_X+B+\Mm_X)+\mult_PE=\sigma_P(W/U,K_W+B_W+\Mm_W)
\end{align*}
and the claim follows.

\medskip

\noindent\textbf{Case 3}. $P$ is exceptional over $Y$. In this case, $a(P,W,B_W,\Mm)=a(P,W,B'_W,\Mm)=0$, and the claim follows.
\end{proof}

\begin{lem}[{cf. \cite[Lemma 3.8]{Has22}}]\label{lem: has22 3.8 rel}
Let $(X,B,\Mm)/U$ be an NQC glc g-pair with induced morphism $\pi: X\rightarrow U$ such that $U$ is quasi-projective. Let $S$ be a subvariety of $X$, and
$$(X,B,\Mm):=(X_0,B_0,\Mm)\dashrightarrow (X_1,B_1,\Mm)\dashrightarrow\dots\dashrightarrow (X_n,B_n,\Mm)\dashrightarrow\dots$$
a $(K_X+B+\Mm_X)$-MMP$/U$ with scaling of an ample$/U$ $\Rr$-divisor $A\geq 0$. Let
$$\lambda_i:=\inf\{t\mid t\geq 0, K_{X_i}+B_i+\Mm_{X_i}+tA_i\text{ is nef}/U\}$$
be the scaling numbers, where $A_i$ is the strict transform of $A$ on $X_i$. Suppose that
\begin{itemize}
    \item each step of this MMP is an isomorphism on a neighborhood of $S$, and
    \item $\lim_{i\rightarrow+\infty}\lambda_i=0$,
\end{itemize}
then 
\begin{enumerate}
    \item for any $\pi$-ample $\Rr$-divisor $H$ on $X$ and any closed point $x\in S$, there exists an $\Rr$-divisor $E$ such that $0\leq E\sim_{\mathbb{R},U}K_X+B+\Mm_X+H$ and $x\not\in\Supp E$, and
    \item  for any prime divisor $P$ over $X$ such that $\Center_XP\cap S\not=\emptyset$, $\sigma_P(X/U,K_X+B+\Mm_X)=0$.
\end{enumerate} 
\end{lem}
\begin{proof}
(1) follows from \cite[Lemma 3.8]{Has22} and (2) follows from (1) and Lemma \ref{lem: nz basic properties}(4).
\end{proof}


\begin{lem}[{\cite[Lemma 3.9]{Has22}}]\label{lem: has22 3.9 rel} Let $(X,B,\Mm)/U$ and $(X',B',\Mm)/U$ be two NQC glc g-pairs and $\phi: X\dashrightarrow X'$ a birational map such that $\phi_*\Mm=\Mm$. Suppose that
\begin{itemize}
    \item $a(P,X,B,\Mm)\leq a(P,X',B',\Mm)$ for any prime divisor $P$ on $X$, and
    \item $a(P',X',B',\Mm)\leq a(P',X,B,\Mm)$ for any prime divisor $P'$ on $X'$, 
\end{itemize}
then
\begin{enumerate}
    \item $K_X+B+\Mm_{X}$ is abundant$/U$ if and only if $K_{X'}+B'+\Mm_{X'}$ is abundant$/U$, and
    \item $(X,B,\Mm)/U$ has a log minimal model (resp. good minimal model) if and only if $(X',B',\Mm)/U$ has a log minimal model (resp. good minimal model).
\end{enumerate}
\end{lem}

\begin{proof}
Let $p: W\rightarrow X$ and $q: W\rightarrow X'$ be a resolution of indeterminacy such that $\Mm$ descends to $W$, $p$ is a log resolution of $(X,\Supp B)$, and $q$ is a log resolution of $(X',\Supp B')$. Let
$$B_W:=\sum_{D\text{ is a prime divisor on }W}\max\{1-a(D,X,B,\Mm),1-a(D,X',B',\Mm),0\}D.$$
Then $(W,B_W,\Mm)$ is glc and $(W,B_W)$ is log smooth. By construction, there exists a $p$-exceptional $\Rr$-divisor $E\geq 0$ and a $q$-exceptional $\Rr$-divisor $F\geq 0$, such that
$$E+p^*(K_X+B+\Mm_X)=K_W+B_W+\Mm_W=q^*(K_{X'}+B'+\Mm_{X'})+F.$$
(1) follows from Lemma \ref{lem: property of numerical and Iitaka dimension}(3) and (2) follows from Theorem \ref{thm: existence good minimal model under pullbacks}.
\end{proof}

\section{A special log minimal model}

The purpose of this section is to prove Theorem \ref{thm: has22 3.14 rel ver}, an analogue of \cite[Theorem 3.14]{Has22}. 

\begin{thm}[{cf. \cite[Theorem 3.14]{Has22}}]\label{thm: has22 3.14 rel ver}
Let $(X,B,\Mm)/U$ be an NQC gdlt g-pair such that
\begin{itemize}
    \item $K_X+B+\Mm_X$ is pseudo-effective$/U$ and abundant$/U$,
    \item for any glc center $S$ of $(X,B,\Mm)$, $(K_X+B+\Mm_X)|_S$ is nef$/U$, and
    \item for any prime divisor $P$ over $X$ such that $a(P,X,B,\Mm)<1$ and $\Center_XP\cap\Ngklt(X,B,\Mm)\not=\emptyset$, $\sigma_P(X/U,K_X+B+\Mm_X)=0$.
\end{itemize}
Then $(X,B,\Mm)/U$ has a log minimal model.
\end{thm}

\begin{proof}

\noindent\textbf{Step 1}. In this step, we show that we may replace $(X,B,\Mm)$ with a $\Qq$-factorial gdlt model and find two $\Rr$-divisors $G\geq 0,H\geq 0$, and a real number $1>t_0>0$, such that
\begin{itemize}
    \item[(I)] $K_X+B+\Mm_X\sim_{\Rr,U}G+H$,
    \item[(II)] $\Supp G\subset\Supp\lfloor B\rfloor$, and
    \item[(III)] for any $t\in (0,t_0]$,
    \begin{enumerate}
        \item[(III.1)] $(X,B+tH,\Mm)/U$ is gdlt, $\Supp N_{\sigma}(X/U,K_X+B+tH+\Mm_X)$ is well-defined and does not depend on $t$, and
        \item[(III.2)] $(X,B-tG,\Mm)/U$ has a good minimal model.
    \end{enumerate}
\end{itemize}

Since $K_X+B+\Mm_X$ is pseudo-effective$/U$ and abundant$/U$,  $K_{X}+B+\Mm_{X}\sim_{\Rr,U}D\geq 0$ for some $\Rr$-divisor $D$. Let $X\dashrightarrow V$ be the Iitaka fibration$/U$ associated to $D$, then $\dim V-\dim U=\kappa_{\sigma}(X/U,K_X+B+\Mm_X)$. Let $h: W\rightarrow X$ be a log resolution of $(X,\Supp B)$ such that $\Mm$ descends to $W$ and the induced map $\psi: W\dashrightarrow V$ is a morphism. Then we may write
$$K_W+B_W+\Mm_W=h^*(K_X+B+\Mm_X)+E$$
such that $B_W\geq 0,E\geq 0$, and $B_W\wedge E=0$. By Lemma \ref{lem: iitaka fibration numerical abundant divisor gpair dimension}, 
\begin{itemize}
    \item[(i)]  $\kappa_{\sigma}(W/U,K_{W}+B_W+\Mm_W)=\dim V-\dim U$ and $\kappa_{\sigma}(W/V,K_{W}+B_W+\Mm_{W})=0$. 
\end{itemize}
Thus by construction $K_{W}+B_W+\Mm_{W}$ is $\Rr$-linearly equivalent$/U$ to the sum of an effective $\Rr$-divisor and the pullback of an ample$/U$ $\Rr$-divisor on $V$. In particular, we may find $0\leq D_W\sim_{\Rr,U}K_W+B_W+\Mm_W$ such that $\Supp D_W$ contains all glc centers of $(W,B_W,\Mm)$ that are vertical over $V$. 

Let $(\bar X,\bar B,\Mm)$ be a proper log smooth model of $(W,B_W,\Mm)$ with induced morphism $g: \bar X\rightarrow W$, such that $g$ is a log resolution of $(W,B_W+D_W)$, and
$$K_{\bar X}+\bar B+\Mm_{\bar X}=g^*(K_W+B_W+\Mm_W)+\bar E$$
for some $\bar E\geq 0$.  By Lemma \ref{lem: special proper log smooth model}, possibly replacing $(\bar X,\bar B,\Mm)$ with a higher model, we may assume that $\bar B=\bar B^h+\bar B^v$, such that
\begin{itemize}
    \item[(ii)] $\bar B^h\geq 0$ and $\bar B^v$ is reduced,
    \item[(iii)] $\bar B^v$ is vertical over $V$, and
    \item[(iv)] for any $t\in (0,1]$, all glc centers of $(\bar X,\bar B-t\bar B^v,\Mm)$ dominate $V$.
\end{itemize}
Let $\bar D:=g^*D_W+\bar E$, then $(\bar X,\bar B+\bar D)$ is log smooth and $\bar D\sim_{\Rr,U}K_{\bar X}+\bar B+\Mm_{\bar X}$. By Lemma \ref{lem: proper log smooth model keep lc center}, $\Supp\bar B^v\subset\Supp\bar D$. Thus we may way write $\bar D=\bar G+\bar H$, such that
\begin{itemize}
    \item[(v)] $K_{\bar X}+\bar B+\Mm_{\bar X}\sim_{\Rr,U}\bar G+\bar H$,
    \item[(vi)] $\Supp\bar B^v\subset\Supp\bar G\subset\Supp\lfloor\bar B\rfloor$, and
    \item[(vii)] no component of $\bar H$ is contained in $\lfloor\bar B\rfloor$ and $(\bar X,\bar B+\bar H)$ is log smooth.
\end{itemize}
We fix a real number $t_1\in(0,1)$ such that $\bar B-t_0\bar G\geq 0$. For any $t\in (0,t_1]$, by (ii)(iii)(iv)(vi), any glc center of $(\bar X,\bar B-t\bar G,\Mm)$ dominates $V$. By (i)(v) and Lemma \ref{lem: property of numerical and Iitaka dimension}(2), $\kappa_{\sigma}(\bar X/U,K_{\bar X}+\bar B-t\bar G+\Mm_{\bar X})=\dim V-\dim U$ and $\kappa_{\sigma}(\bar X/V,K_{\bar X}+\bar B-t\bar G+\Mm_{\bar X})=\kappa_{\iota}(\bar X/V,K_{\bar X}+\bar B-t\bar G+\Mm_{\bar X})=0$. By Proposition \ref{prop: prop 3.4 has19 abu ver},
\begin{itemize}
    \item[(viii)] $(\bar X,\bar B-t\bar G,\Mm)/U$ has a good minimal model for any $t\in (0,t_1]$.
\end{itemize}

Since $(\bar X,\bar B,\Mm)$ is a log smooth model of $(X,B,\Mm)$, we may run a $(K_{\bar X}+\bar B+\Mm_{\bar X})$-MMP$/X$ which terminates with a gdlt model $(Y,B_Y,\Mm)$ of $(X,B,\Mm)$ with induced morphism $f: Y\rightarrow X$ and birational map $\phi: \bar X\dashrightarrow Y$. Let $G_Y$ and $H_Y$ be the strict transforms of $\bar G$ and $\bar H$ on $Y$ respectively, then $K_Y+B_Y+\Mm_Y\sim_{\Rr,U}G_Y+H_Y$. By (vii) and Lemma \ref{lem: still an mmp under perturbation}, there exists $0<t_2<t_1$ such that $(\bar Y,\bar B+t_2\bar H,\Mm)$ is gdlt and $\phi$ is a $(K_{\bar X}+\bar B+t\bar H+\Mm_{\bar X})$-MMP$/X$ as well as a $(K_{\bar X}+\bar B-t\bar G+\Mm_{\bar X})$-MMP$/X$ for any $t\in (0,t_2]$. Then $(Y,B_Y+t_2H_Y,\Mm)$ is gdlt, and by (viii) and \cite[Theorem 2.8, Lemma 3.5(2)]{HL21} ($=$\cite[Version 3, Theorem 2.24, Lemma 3.9(2)]{HL21}), $(Y,B_Y-tG,\Mm))/U$ has a good minimal model for any $t\in (0,t_2]$. By Lemma \ref{lem: has20a 2.4 rel ver}, we may pick $0<t_0<t_2$ such that $\Supp N_{\sigma}(Y/U,K_Y+B_Y+tH_Y+\Mm_Y)$ is well-defined and does not depend on $t$ for any $t\in (0,t_0]$.

We may replace $(X,B,\Mm)$ with $(Y,B_Y,\Mm)$ and let $G:=G_Y$ and $H:=H_Y$, and assume that $(X,B,\Mm),G,H$ and $t_0$ satisfy (I)(II)(III). In the following, we forget all other auxiliary varieties and divisors constructed in this step.

\medskip

\noindent\textbf{Step 2}. For any $t\in (0,t_0]$, by (III.2), $(X,B-\frac{t}{1+t}G,\Mm)/U$ has a good minimal model. Since
$$K_X+B+tH+\Mm_X\sim_{\Rr,U}(1+t)(K_X+B-\frac{t}{1+t}G+\Mm_X),$$
by (III.1) and \cite[Theorem 2.8, Lemmas 3.5(2), 4.3]{HL21} ($=$\cite[Version 3, Theorem 2.24, Lemma 3.9(2), 4.2]{HL21}), we may run a $(K_X+B+tH+\Mm_X)$-MMP$/U$ $\phi_t: X\dashrightarrow X_t$ which terminates with a good minimal model $(X_t,B_t+tH_t,\Mm)/U$ of $(X,B+tM,\Mm)/U$. By (III.1) and Lemma \ref{lem: nz for glc divisor}(2.b), divisors contracted by $\phi_t$ is $\Supp N_{\sigma}(X/U,K_X+B+t_0H+\Mm_X)$ and is independent of $t$. We let $X_0:=X_{t_0},B_0:=B_{t_0}$, and $H_0:=H_{t_0}$. Then $X_0$ and $X_t$ are isomorphic in codimension $1$, and $K_{X_0}+B_0+\Mm_{X_0}$ is a limit of movable$/U$ $\Rr$-divisors. By the negativity lemma, $(X_t,B_t+tH_t,\Mm)/U$ is a good minimal model of $(X_0,B_0+tH_0,\Mm)/U$ for any $t\in (0,t_0]$. 

\begin{claim}\label{claim: has20a 3.3 rel gpair}
We may run a $(K_{X_0}+B_0+\Mm_{X_0})$-MMP$/U$ with scaling of $H_0$, such that we either get a log minimal model of $(X_0,B_0,\Mm)/U$ or a sequence of flips
$$(X_0,B_0,\Mm)\dashrightarrow (X_1,B_1,\Mm)\dashrightarrow\dots (X_i,B_i,\Mm)\dashrightarrow\dots$$
with scaling numbers 
$$\lambda_i:=\inf\{t\mid t\geq 0, K_{X_i}+B_i+tH_i+\Mm_{X_i}\text{ is nef}/U\}$$
where $H_i$ is the strict transform of $H$ on $X_i$, such that
\begin{enumerate}
    \item either the MMP$/U$ terminates, or $\lim_{i\rightarrow+\infty}\lambda_i=0$,
    \item for any $i\geq 1$ and $\lambda\in [\lambda_i,\lambda_{i-1}]$, $(X_i,B_i+\lambda H_i,\Mm)/U$ is a good minimal model of both $(X,B+\lambda H,\Mm)$ and $(X_0,B_0+\lambda H_0,\Mm)/U$, and
    \item the MMP only contracts sub-varieties of $\Supp\lfloor B_0\rfloor$.
\end{enumerate}
\end{claim}
\begin{proof}
Since $K_{X_0}+B_0+\Mm_{X_0}$ is a limit of movable$/U$ $\Rr$-divisors, by Lemma \ref{lem: limit of movable divisors mmp only contain flips}, any $(K_{X_0}+B_0+\Mm_{X_0})$-MMP$/U$ only contains flips. (1) follows from Lemma \ref{lem: gmmp scaling numbers go to 0}. For any $i\geq 1$ and $\lambda\in [\lambda_i,\lambda_{i-1}]$, $(X_i,B_i+\lambda H_i,\Mm)$ is gdlt and $K_{X_i}+B_i+\lambda H_i+\Mm_{X_i}$ is nef$/U$. Since the induced birational maps $X_0\dashrightarrow X_\lambda$ and $X_i\rightarrow X_\lambda$ are both small, by Lemma \ref{lem: hmx18 2.7.3 gpair rel} and \cite[Lemma 3.5(2)]{HL21} ($=$\cite[Version 3, Lemma 3.9(2)]{HL21}), we get (2).

Let $X_i\rightarrow Z_i\leftarrow X_{i+1}$ be the $i$-th step of the MMP where $X_i\rightarrow Z_i$ the flipping contraction. Then for any flipping curve $C_i$ of $X_i\rightarrow Z_i$, we have $(K_{X_i}+B_i+\Mm_{X_i})\cdot C_i<0$ and $H_i\cdot C_i>0$. Let $G_i$ be the strict transform of $G$ on $X_i$, then $0>(K_{X_i}+B_i+\Mm_{X_i}-H_i)\cdot C_i=G_i\cdot C_i$. Thus $C_i\subset\Supp G_i$. Since $\Supp G\subset\Supp\lfloor B\rfloor$, $\Supp G_i\subset\Supp\lfloor B_i\rfloor$, and we get (3).
\end{proof}

\begin{claim}\label{claim: has22 3.14 step 2 reduction}
Let
$$(X_0,B_0,\Mm)\dashrightarrow (X_1,B_1,\Mm)\dashrightarrow\dots (X_i,B_i,\Mm)\dashrightarrow\dots,$$
$\lambda_i$, and $H_i$ be the MMP$/U$, the scaling numbers, and the strict transform of $H$ on $X_i$ for each $i$ as in Claim \ref{claim: has20a 3.3 rel gpair} respectively. If the MMP$/U$ terminates, then Theorem \ref{thm: has22 3.14 rel ver} holds.
\end{claim}
\begin{proof}
Let $\lambda_{-1}:=t_0$. If the MMP$/U$ terminates, then $\lambda_{l-1}>\lambda_l=0$ for some $l\in\mathbb N$. By Claim \ref{claim: has20a 3.3 rel gpair}(2), for any $t\in (0,\lambda_{l-1}]$, $K_{X_l}+B_l+tH_l+\Mm_{X_l}$ is nef$/U$, and $a(P,X,B+tH,\Mm)\leq a(P,X_l,B_l+tH_l,\Mm)$ for any prime divisor $P$ on $X$ that is exceptional$/X_l$. Let $t\rightarrow 0$, we have that  $K_{X_l}+B_l+\Mm_{X_l}$ is nef$/U$ and $a(P,X,B,\Mm)\leq a(P,X_l,B_l,\Mm)$ for any prime divisor $P$ on $X$ that is exceptional$/X_l$. Thus $(X_l,B_l,\Mm)/U$ is a weak glc model of $(X,B,\Mm)/U$. The Claim follows from \cite[Lemma 3.8]{HL21} ($=$\cite[Version 3, Lemma 3.15]{HL21}).
\end{proof}

\noindent\textit{Proof of Theorem \ref{thm: has22 3.14 rel ver} continued}. In the following, we let
$$(X_0,B_0,\Mm)\dashrightarrow (X_1,B_1,\Mm)\dashrightarrow\dots (X_i,B_i,\Mm)\dashrightarrow\dots,$$
$\lambda_i$, and $H_i$ be the MMP$/U$, the scaling numbers, and the strict transform of $H$ on $X_i$ for each $i$ as in Claim \ref{claim: has20a 3.3 rel gpair} respectively.

\medskip

\noindent\textbf{Step 3}. For every $i$ and glc center $S_{i}$ of $(X_{i},B_{i},\Mm)$, we let $(S_i,B_{S_i},\Mm^{S_i})$ be the g-pair induced by the adjunction
$$K_{S_i}+B_{S_i}+\Mm^{S_i}_{S_i}:=(K_{X_i}+B_i+\Mm_{X_i})|_{S_i},$$
and let $H_{S_i}:=H_i|_{S_i}$ for each $i$. For every glc center of $(X,B,\Mm)$ we let $(S,B_S,\Mm^S)$ be the g-pair induced by the adjunction
$$K_S+B_S+\Mm^S_S:=(K_X+B+\Mm_X)|_S$$
and let $H_S:=H|_S$. By Lemma \ref{lem: special termination reduce to flip lemma}, \cite[Remark 3.21, Theorem 4.1]{HL18}, Claim \ref{claim: has20a 3.3 rel gpair}(1), and induction on the dimension of glc centers, we only need to show that for any $m\gg 0$ and any glc center $S_m$ of $(X_m,B_m,\Mm)$ of dimension $\geq 1$, $(S_m,B_{S_m},\Mm^S)/U$ has a log minimal model. For each $i$, we let $S_i,S$ be the strict transforms of $S_m$ on $X_i,X$ respectively. We let $\phi^S_{i,j}: S_i\dashrightarrow S_j$  and $\phi^S_i: S\dashrightarrow S_i$ be the induced birational maps. By Lemma \ref{lem: special termination reduce to flip lemma} and induction on the dimension of glc centers, possibly replacing $m$, we may assume that $\phi_{m,i}$ is small for any $i\geq m$.

\medskip\noindent\textbf{Step 4}. We prove the following claim.

\begin{claim}\label{claim: has22 3.14 step 4 abcd}
There exists a $\Qq$-factorial glc g-pair $(T,B_T,\Mm^S)/U$ and a birational morphism $\psi: T\rightarrow S_m$ satisfying the following:
\begin{enumerate}
    \item For any prime divisor $D$ on $S$ such that $a(D,S_m,B_{S_m},\Mm^S)<a(D,S,B_S,\Mm^S)$, $D$ is on $T$ and is a $\psi$-exceptional.
    \item $$B_T=\sum_{D\text{ is a prime divisor on }T}(1-a(D,S,B_S,\Mm^S))D.$$
    \item For any $i\geq m$ and any prime divisor $Q$ over $S$, $$a(Q,S,B_S+\lambda_iH_S,\Mm^S)\leq a(Q,S_i,B_{S_i}+\lambda_iH_{S_i},\Mm^S).$$
    \item For any prime divisor $Q'$ over $S_m$,
    $$a(Q',S_m,B_{S_m},\Mm^S)\leq a(Q',T,B_T,\Mm^S).$$
\end{enumerate}
\end{claim}
\begin{proof}
By Claim \ref{claim: has20a 3.3 rel gpair}(2), $(X_i,B_i+\lambda_iH_i,\Mm)/U$ is a good minimal model of $(X,B+\lambda_iH,\Mm)$, hence (3) holds.

Since $\phi_i$ does not extract any divisor, $a(P,X_i,B_i,\Mm)\leq a(P,X,B,\Mm)$ for any prime divisor $P$ on $X_i$. Since  $\sigma_P(X/U,K_X+B+\Mm_X)=0$ for any prime divisor $P$ over $X$ such that $a(P,X,B,\Mm)<1$ and $\Center_XP\cap\Ngklt(X,B,\Mm)\not=\emptyset$, by Lemma \ref{lem: has22 3.5 rel} and since $\phi_{m,i}$ is small for any $i\geq m$, $a(D,S_i,B_{S_i},\Mm^S)\leq a(D,S,B_S,\Mm^S)$ for any prime divisor $D$ on $S_m$ and $i\geq m$. Thus  $a(D,S_i,B_{S_i}+\lambda_iH_{S_i},\Mm^S)\leq a(D,S,B_S,\Mm^S)$ for any prime divisor $D$ on $S_m$ and $i\geq m$. By (3), 
\begin{align*}
    a(D,S,B_S+\lambda_iH_S,\Mm^S)&\leq a(D,S_i,B_{S_i}+\lambda_iH_{S_i},\Mm^S)\\
    &=a(D,S_m,B_{S_m}+\lambda_iH_{S_m},\Mm^S)\leq a(D,S,B_S,\Mm^S).
\end{align*}
for any $i\geq m$. 

Let $i\rightarrow+\infty$, we have
$$a(D,S,B_S,\Mm^S)=a(D,S_m,B_{S_m},\Mm^S)$$
for any prime divisor $D$ on $S_m$. We define
$$\mathcal{C}:=\{D\mid D\text{ is a prime divisor on }S, a(D,S_m,B_{S_m},\Mm^S)<a(D,S,B_S,\Mm^S)\},$$
then any element of $\mathcal{C}$ is exceptional over $S_m$. Thus for any $D\in\mathcal{C}$, by (3),
\begin{align*}
    a(D,S,B_S+\lambda_mH_S,\Mm^S)&\leq a(D,S_m,B_{S_m}+\lambda_mH_{S_m},\Mm^S)\\
    &\leq a(D,S_m,B_{S_m},\Mm^S)<a(D,S,B_S,\Mm^S)\leq 1.
\end{align*}
Since any element of $\mathcal{C}$ is a prime divisor on $S$, any element of $\mathcal{C}$ is a component of $H_S$. Thus $\mathcal{C}$ is a finite set, and for every $D\in\mathcal{C}$, since $\lambda_m<t_0$,
\begin{align*}
  0\leq &a(D,S,B_S+t_0H_S,\Mm^S)<a(D,S,B_S+\lambda_mH_S,\Mm^S)\\ \leq &a(D,S_m,B_{S_m}+\lambda_mH_{S_m},\Mm^S)\leq  a(D,S_m,B_{S_m},\Mm^S)<a(D,S,B_S,\Mm^S)\leq 1.  
\end{align*}
Thus $0<a(D,S_m,B_{S_m},\Mm^S)<1$ for any $D\in\mathcal{C}$. By \cite[Lemma 3.4]{Has22}, there exists a birational morphism $\psi: T\rightarrow S_m$ from a $\Qq$-factorial variety $T$ which extracts exactly divisors contained in $\mathcal{C}$. (1) follows immediately from the construction of $\mathcal{C}$. Since $(S,B_S,\Mm^S)$ is glc, there are only finitely many divisors $D$ on $T$ such that $a(D,S,B_S,\Mm^S)\not=1$, hence $B_T\geq 0$ is well-defined, and we get (2).

For any prime divisor $D$ on $T$, if $D$ is $\psi$-exceptional, then $$a(D,S_m,B_{S_m},\Mm^S)< a(D,S,B_S,\Mm^S)\leq 1$$ as $D\in\mathcal{C}$, and if $D$ is not $\psi$-exceptional, then $\Center_{S_m}$ is a divisor, hence $a(D,S,B_S,\Mm^S)=a(D,S_m,B_{S_m},\Mm^S)\leq 1$. In either case,
$$a(D,S_m,B_{S_m},\Mm^S)\leq a(D,S,B_S,\Mm^S)\leq 1.$$
Since $T$ is $\Qq$-factorial, $K_T+B_T+\Mm^S_T$ is $\Rr$-Cartier, and
$$K_T+B_T+\Mm^S_T\leq\psi^*(K_{S_m}+B_{S_m}+\Mm^S_{S_m}).$$
Thus
$$0\leq a(Q',S_m,B_{S_m},\Mm^S)\leq a(Q',T,B_T,\Mm^S)$$
for any prime divisor $Q'$ over $S_m$, and we get (4). In particular, $(T,B_T,\Mm^S)$ is glc, and the proof is concludes.
\end{proof}

\noindent\textit{Proof of Theorem \ref{thm: has22 3.14 rel ver} continued}. \noindent\textbf{Step 5}. In this step, we show that $(T,B_T,\Mm^S)/U$ has a log minimal model. By our assumption, $(S,B_S,\Mm^S)/U$ is a log minimal model of itself. We prove the following claim:

\begin{claim}\label{lem: has22 3.14 claim rel version}
For any prime divisor $D$ over $S$,
\begin{enumerate}
    \item if $D$ is on $S$, then $a(D,S,B_S,\Mm^S)\leq a(D,T,B_T,\Mm^S)$, and
    \item if $D$ is on $T$, then $a(D,T,B_T,\Mm^S)= a(D,S,B_S,\Mm^S)$.
\end{enumerate}
\end{claim}
\begin{proof}
By Claim \ref{claim: has22 3.14 step 4 abcd}(2), we only need to show that for any prime divisor $D$ on $S$ that is exceptional over $T$, $a(D,S,B_S,\Mm^S)\leq a(D,T,B_T,\Mm^S)$. By Claim \ref{claim: has22 3.14 step 4 abcd}(1)(4), $$a(D,S,B_S,\Mm^S)\leq a(D,S_m,B_m,\Mm^S)\leq a(D,T,B_T,\Mm^S),$$
and we get (1).
\end{proof}

\noindent\textit{Proof of Theorem \ref{thm: has22 3.14 rel ver} continued}. By Lemma \ref{lem: has22 3.9 rel}, $(T,B_T,\Mm^S)/U$ has a log minimal model.

\medskip

\noindent\textbf{Step 6}. We conclude the proof in this step. Recall that we only need to show that $(S_m,B_{S_m},\Mm^S)/U$ has a good minimal model.

For any $i\geq m$, since $K_{X_i}+B_i+\lambda_iH_i+\Mm_{X_i}$ is nef$/U$, $K_{S_i}+B_{S_i}+\lambda_iH_{S_i}+\Mm^S_{S_i}=(K_{X_i}+B_i+\lambda_iH_i+\Mm_{X_i})|_{S_i}$ is nef$/U$. Since $\phi_{m,i}$ is small, by the negativity lemma, $(S_i,B_{S_i}+\lambda_iH_{S_i},\Mm^S)/U$ is a weak glc model of $(S_m,B_{S_m}+\lambda_iH_{S_m},\Mm^S)/U$. Let $h_m: W\rightarrow S_m$ and $h_i: W\rightarrow S_i$ be a resolution of indeterminacy of $\phi_{m,i}$. By Lemmas \ref{lem: nz keep under pullback}(2), \ref{lem: nz basic properties}(3), and \ref{lem: nz for glc divisor}(1) and the negativity lemma, for any prime divisor $D$ on $T$, $\sigma_D(S_m/U,K_{S_m}+B_{S_m}+\lambda_iH_{S_m}+\Mm^S_{S_m})$ is well-defined, and
\begin{align*}
    0&\leq a(D,S_i,B_{S_i}+\lambda_iH_{S_i},\Mm^S)-a(D,S_m,B_{S_m}+\lambda_iH_{S_m},\Mm^S)\\
    &=\sigma_D(S_m/U,K_{S_m}+B_{S_m}+\lambda_iH_{S_m}+\Mm^S_{S_m}).
\end{align*}
By Claim \ref{claim: has22 3.14 step 4 abcd}(3), 
$$\sigma_D(S_m/U,K_{S_m}+B_{S_m}+\lambda_iH_{S_m}+\Mm^S_{S_m})\geq  a(D,S,B_{S}+\lambda_iH_{S},\Mm^S)-a(D,S_m,B_{S_m}+\lambda_iH_{S_m},\Mm^S).$$
By Claim \ref{claim: has22 3.14 step 4 abcd}(2), $a(D,S,B_S,\Mm^S)=a(D,T,B_T,\Mm^S)$. By Lemma \ref{lem: nz basic properties}(2) and Claims \ref{claim: has20a 3.3 rel gpair}(1) and \ref{claim: has22 3.14 step 4 abcd}(4), for any prime divisor $D$ on $T$,
\begin{align*}
    &\sigma_D(S_m/U,K_{S_m}+B_{S_m}+\Mm^S_{S_m})\\
    =&\lim_{i\rightarrow+\infty}\sigma_D(S_m/U,K_{S_m}+B_{S_m}+\lambda_iH_{S_m}+\Mm^S_{S_m})\\
    \geq&\lim_{i\rightarrow+\infty}(a(D,S,B_S+\lambda_iH_S,\Mm^S)-a(D,S_m,B_{S_m}+\lambda_iH_{S_m},\Mm^S))\\
    =&a(D,S,B_S,\Mm^S)-a(D,S_m,B_{S_m},\Mm^S)\\
    =&a(D,T,B_T,\Mm^T)-a(D,S_m,B_{S_m},\Mm^S)\geq 0.
\end{align*}
Since $(T,B_T,\Mm^S)/U$ has a log minimal model, by Lemma \ref{lem: has22 3.6 rel}, $(S_m,B_{S_m},\Mm^S)/U$ has a log minimal model, and we are done.
\end{proof}


\begin{thm}[{cf. \cite[Theorem 3.15]{Has22}}]\label{thm: has22 3.15 rel ver}
Let $(X,B,\Mm)/U$ be a $\Qq$-factorial NQC gdlt g-pair and $A\geq 0$ an $\mathbb{R}$-divisor on $X$ such that $K_{X}+B+\Mm_{X}+A$ is nef$/U$. The for any $(K_{X}+B+\Mm_{X})$-MMP$/U$ with scaling of $A$
$$(X,B, \Mm)=:(X_{0},B_{0},\Mm) \dashrightarrow (X_{1}, B_{1},\Mm) \dashrightarrow \dots \dashrightarrow (X_{i}, B_{i},\Mm) \dashrightarrow \dots,$$
with scaling numbers
$$\lambda_i:=\inf\{t\mid t\geq 0, K_{X_i}+B_i+tA_i+\Mm_{X_i}\text{ is nef}/U\}$$
where $A_i$ is the strict transform of $A$ on $X_i$, if $\lambda_i>0$ for each $i$ and $\lim_{i\rightarrow+\infty}\lambda_i=0$, then there are only finitely many $i$ such that $K_{X_i}+B_i+\Mm_{X_i}$ is log abundant$/U$ with respect to $(X_i,B_i,\Mm)$.
\end{thm}
\begin{proof} We apply induction on dimensions. Suppose that the theorem holds in dimension $\leq \dim X-1$ but the theorem does not hold. Then there exists a $(K_{X}+B+\Mm_{X})$-MMP$/U$ with scaling of $A$ as in the statement of the theorem such that  $K_{X_i}+B_i+\Mm_{X_i}$ is log abundant$/U$ with respect to $(X_i,B_i,\Mm)$ for infinitely many $i$. Let $\phi_{i,j}: X_i\dashrightarrow X_j$ be the induced birational map. Possibly replacing $(X,B,\Mm)$ with $(X_m,B_m,\Mm)$ for some $m\gg 0$, we may assume that $\phi_{i,j}$ are small for any $i,j$. 

We prove the following claim.

\begin{claim}\label{claim: has20 3.4 step 2 gpair rel}
If there exists $m\gg 0$ such that $\phi_{m,i}|_S$ is an isomorphism for any glc center $S$ of $(X_m,B_m,\Mm)$ and $i\geq m$, then Theorem \ref{thm: has22 3.15 rel ver} holds.
\end{claim}
\begin{proof}
Possibly replacing $(X,B,\Mm)$ we may assume that $\phi_{i,j}|_{\Ngklt(X_i,B_i,\Mm)}$ is an isomorphism for any $i,j$ and $K_X+B+\Mm_X$ is abundant$/U$. Since $\lim_{i\rightarrow+\infty}\lambda_i=0$ and $\phi_{i,j}$ are small for any $i,j$, $K_X+B+\Mm_X$ is a limit of movable$/U$ $\Rr$-divisors, hence $K_X+B+\Mm_X$ is pseudo-effective$/U$. Notice that $(X_i,B_i,\Mm)$ are all gdlt and $\Qq$-factorial, let $D$ be a component of $\lf B_i\rf$, then $\phi_{i,i+1}|_D$ being an isomorphism implies that the flip $\phi_{i,i+1}$ is an isomorphism near $D$. Therefore $\phi_{i,i+1}$ is an isomorphism on a neighborhood of $\lf B_i\rf$. By Lemma \ref{lem: has22 3.8 rel}, ${\bf B}_{-}(K_{X}+B+\Mm_{X}/U)$ does not intersect $\Supp\lfloor B\rfloor$, and $\sigma_P(X/U,K_X+B+\Mm_X)=0$ for any prime divisor $P$ over $X$ such that $\Center_XP\cap\Supp\lfloor B\rfloor\not=\emptyset$. In particular, $(K_X+B+\Mm_X)|_S$ is nef$/U$ for any glc center $S$ of $(X,B,\Mm)$. By Theorem \ref{thm: has22 3.14 rel ver}, $(X,B,\Mm)/U$ has a log minimal model, but this contradicts \cite[Theorem 4.1]{HL18} so we are done.
\end{proof}

\noindent\textbf{Proof of Theorem \ref{thm: has22 3.15 rel ver} continued}. We let $\phi_i:=\phi_{i,i+1}$ for every $i$ and $X_i\rightarrow Z_i\leftarrow X_{i+1}$ the flip defined by $\phi_i$. By Claim \ref{claim: has20 3.4 step 2 gpair rel}, we only need to show that for any glc center $S$ of $(X,B,\Mm)$, the MMP terminates along $S$ after finitely many steps. By induction on the dimension of glc centers, we may assume that $\phi_i$ induces an isomorphism for every $\leq d-1$-dimensional glc centers and $i\gg 0$ where $d=\dim S$. We may let $S_i$ be the strict transform of $S$ on $X_i$ and $(S_i,B_{S_i},\Mm^{S})$ the g-pair given by the adjunction
$$K_{S_i}+B_{S_i}+\Mm_{S}^{S_i}:=(K_{X_i}+B_i+\Mm_{X_i})|_{S_i}.$$
Let $(S_i',B_{S_i'},\Mm^S)$ be a gdlt model of $(S_i,B_{S_i},\Mm^{S})$. By Lemma \ref{lem: special termination reduce to flip lemma}, for $i\gg0$, the $(K_X+B+\Mm_X)$-MMP$/U$ with scaling of $A$ induces a $(K_{S'_i}+B_{S'_i}+\Mm_{S}^{S'_i})$-MMP$/T$ with scaling of $A_{S_i'}$ such that the limit of the scaling numbers is $0$, where $A_{S_i'}$ is the pullback of $A_i$ on $S_i'$ and $T$ is the normalization of the image of $S_i$ in $U$. Since $K_{X_j}+B_j+\Mm_{X_j}$ is log abundant$/U$ with respect to $(X_j,B_j,\Mm)$ for infinitely many $j$, $K_{S_j'}+B_{S_j'}+\Mm^S_{S'_j}$ is log abundant$/T$ with respect to $(S_j',B_{S_j'},\Mm^S)$ for infinitely many $j$. By Theorem \ref{thm: has22 3.15 rel ver} in dimension $<\dim X$, the $(K_{S'_j}+B_{S'_j}+\Mm_{S}^{S'_j})$-MMP$/T$ terminates, and the claim follows.
\end{proof}

\section{Log abundance under the MMP}

This section is similar to \cite[Section 3 and Theorem 4.1]{Has20b}.

\begin{comment}
\begin{prop}[{cf. \cite[Proposition 3.4]{Has20b}}]\label{prop: has20b 3.4 rel ver}
Let $(X,B,\Mm)/U$ be an NQC gklt g-pair and $\pi: X\rightarrow Z$ a projective morphism between normal quasi-projective varieties over $U$, such that $\kappa_{\iota}(X/Z,K_X+B+\Mm_X)=\kappa_{\sigma}(X/Z,K_X+B+\Mm_X)=0$ for some non-empty open subset $Z^0$ of $Z$. Then for any big$/U$ and nef$/U$ $\Rr$-divisor $A$ on $Z$, $K_X+B+\pi^*A+\Mm_X$ is abundant$/U$.
\end{prop}
\begin{proof}
Possibly replacing $\pi$ with the contraction in the Stein factorization of $\pi$, we may assume that $\pi$i s a contraction. By Lemma \ref{lem: property of numerical and Iitaka dimension}(3), possibly replacing $(X,B,\Mm)$ with a proper log smooth model and shrinking $Z^0$, we may assume that $\Mm$ descends to $X$ and $(X,B)$ is log smooth. By Theorem \ref{thm: has19 weak semistable reduction} and Lemma \ref{lem: property of numerical and Iitaka dimension}(3)(4), possibly replacing $(X,B,\Mm),Z,Z^0,A$ and $\pi$ and probably losing the log smoothness of $(X,B)$, we may assume that $\pi$ is an equi-dimensional and $Z$ is smooth.

Since $\kappa_{\iota}(X/Z,K_X+B+\Mm_X)\geq 0$, by \cite[Lemma 3.2.1]{BCHM10}, $K_X+B+\Mm_X\sim_{\Rr,Z}E^h+E^v$ where $E^v\geq 0$ is vertical over $Z$ and $E^h\geq 0$ is horizontal over $Z$. 
Since $\pi$ is equi-dimensional and $Z$ is smooth, we may define
$$\nu_D:=\sup\{t\mid t\geq 0, E^v-\pi^*D\geq 0\}$$
for any prime divisor $D$ on $Z$. Possibly replacing $E^v$ with $E^v-\sum_{D\text{ is a prime divisor on }Z}\nu_D\pi^*D$, we may assume that $E^v$ is very exceptional$/Z$. 

 By Lemmas \ref{lem: has19 3.2 step 3 abu ver}, \ref{lem: termination over open subset}, and \ref{lem: rlinear version of hl18 3.8}, , we may run a $(K_X+B+\Mm_X)$-MMP$/Z$ with scaling of an ample$/Z$ divisor which terminates with a log minimal model $(X',B',\Mm)/Z$ of $(X,B,\Mm)/Z$, such that $K_{X'}+B'+\Mm_{X'}\sim_{\Rr,Z}0$. By Lemma \ref{lem: property of numerical and Iitaka dimension}(6), possibly replacing $(X,B,\Mm)$ with $(X',B',\Mm)$, we may assume that $K_{X'}+B'+\Mm_{X'}\sim_{\Rr,Z}0$. By \cite[Theorem 1.2]{HL19}, there exists an NQC gklt g-pair $(Z,B_Z,\Mm^Z)$ on $Z$ such that
$$K_{X'}+B'+\Mm_{X'}\sim_{\Rr}\pi^*(K_Z+B_Z+\Mm^Z_Z).$$
Since $A$ is big and nef, by \cite[Lemma 4.4(2)]{BZ16}, $(Z,B_Z,\Mm^Z+\bar A)/U$ has a good minimal model$/U$. Thus $K_Z+B_Z+\Mm^Z_Z+\bar A$ is abundant$/U$, and the proposition follows by Lemma \ref{lem: property of numerical and Iitaka dimension}(4).
\end{proof}

\end{comment}


\begin{thm}[{cf. \cite[Theorem 3.5]{Has20b}}]\label{thm: has20b 3.5 rel ver}
Let $(X,B,\Mm)/U$ be an NQC glc g-pair and $\pi: X\rightarrow Z$ a projective morphism$/U$ such that $Z$ is normal quasi-projective. Let $C\geq 0$ be an $\Rr$-divisor on $X$, $A_Z$ an ample$/U$ $\Rr$-divisor on $Z$, and $0\leq A\sim_{\Rr,U}\pi^*A_Z$ and $\Rr$-divisor on $X$, such that
\begin{enumerate}
    \item $C$ does not contain any glc center of $(X,B,\Mm)$,
    \item $K_X+B+C+\Mm_X\sim_{\Rr,Z}0$, and
    \item $(X,B+A,\Mm)$ is glc.
\end{enumerate}
Then $K_X+B+A+\Mm_X$ is abundant$/U$.
\end{thm}
\begin{proof}
Possibly replacing $\pi$ with the contraction in the Stein factorization of $\pi$, we may assume that $\pi$ is a contraction. Possibly replacing $Z\rightarrow U$ with the Stein factorization of $Z\rightarrow U$, we may assume that $Z\rightarrow U$ is a contraction. Let $F$ be a very general fiber of $X\rightarrow U$ and $F_Z:=\pi(F)$, then $F_Z$ is a very general fiber of $Z\rightarrow U$. Possibly replacing $(X,B,\Mm),A,C,Z,A_Z,\pi,U$ with $(F,B|_F,\Mm|_F),A|_F,C|_F,F_Z,A_Z|_{F_Z},\pi|_F,\{pt\}$, we may assume that $U=\{pt\}$. The theorem follows from \cite[Theorem 3.5]{Has20b}.  Note that we remove the $\Rr$-Cartier assumption of $C$ as it is immediate from (2).
 \end{proof}
 
\begin{lem}[{cf. \cite[Lemma 3.6]{Has20b}}]\label{lem: has20b 3.6 rel ver}
Let $(X,B,\Mm)/U$ be an NQC glc g-pair and $\pi: X\rightarrow Z$ a projective morphism$/U$ such that $Z$ is normal quasi-projective. Let $C\geq 0$ be an $\Rr$-divisor on $X$, $A_Z$ an ample$/U$ $\Rr$-divisor on $Z$, and $0\leq A\sim_{\Rr,U}\pi^*A_Z$ and $\Rr$-divisor on $X$, such that
\begin{enumerate}
    \item $C$ does not contain any glc center of $(X,B,\Mm)$,
    \item $K_X+B+C+\Mm_X\sim_{\Rr,Z}0$, and
    \item $(X,B+A,\Mm)$ is glc.
\end{enumerate}
Let $h: W\rightarrow X$ be a log resolution of $(X,\Supp B)$ such that $\Mm$ descends to $W$, and $B_W\geq 0$ an $\Rr$-divisor on $W$ such that $(W,B_W+h^*A)$ is lc and $(K_W+B_W+\Mm_W-h^*(K_X+B+\Mm_X))^{\geq 0}$ is $h$-exceptional. Then $K_W+B_W+h^*A+\Mm_W$ is abundant$/U$. 
\end{lem}
\begin{proof}
Possibly replacing $\pi$ with the contraction in the Stein factorization of $\pi$, we may assume that $\pi$ is a contraction. Possibly replacing $Z\rightarrow U$ with the Stein factorization of $Z\rightarrow U$, we may assume that $Z\rightarrow U$ is a contraction. Let $F_w$ be a very general fiber of $W\rightarrow U$, $F:=h(F_W)$, and $F_Z:=\pi(F)$, then $F$ and $F_Z$ are a very general fibers of $X\rightarrow U$ and $Z\rightarrow U$ respectively. Possibly replacing $(X,B,\Mm),A,C,Z,A_Z,\pi,U,W,h,B_W$ with $(F,B|_F,\Mm|_F),A|_F,C|_F,F_Z,A_Z|_{F_Z},\pi|_F,\{pt\},F_W,h|_{F_W},B_W|_{F_W}$, we may assume that $U=\{pt\}$. The theorem follows from \cite[Theorem 3.5, Lemma 3.6]{Has20b}. Note that we remove the $\Rr$-Cartier assumption of $C$ as it is immediate from (2).
\end{proof}

\begin{lem}[{cf. \cite[3.5]{HL18}}]\label{lem: lift mmp}
Let $(X,B,\Mm)/U$ be an NQC glc g-pair, $S$ a glc center of $(X,B,\Mm)$,  $(Y,B_Y,\Mm)$ a gdlt model of $(X,B,\Mm)$ with induced birational morphism $f: Y\rightarrow X$, and $S_Y$ a component of $\lfloor B_Y\rfloor$ such that $f(S_Y)=S$. Let
$$\phi: (X,B,\Mm)\dashrightarrow (X',B',\Mm)$$
be a partial $(K_X+B+\Mm_X)$-MMP$/U$ and $S'$ a glc center of $(X',B',\Mm)$, such that $\phi|_S: S\dashrightarrow S'$ is a birational map. Then there exists a partial $(K_Y+B_Y+\Mm_Y)$-MMP$/U$
$$\psi: (Y,B_Y,\Mm)\dashrightarrow (Y',B_Y',\Mm),$$
such that
\begin{enumerate}
    \item $(Y',B_Y',\Mm)$ is a gdlt model of $(X',B',\Mm)$, and
    \item the strict transform of $S_Y$ on $Y'$ is a component of $\lfloor B_Y'\rfloor$. 
\end{enumerate}
\end{lem}
\begin{proof}
We only need to prove the lemma when $\phi$ is a divisorial contraction or a flip. If $\phi$ is a flip, then we let $X\rightarrow Z$ be the flipping contraction and let $X'\rightarrow Z$ be the flipped contraction. If $\phi$ is a divisorial contraction, then we let $Z=X'$. Then $(X',B',\Mm)/Z$ is a log minimal model of $(X,B,\Mm)/Z$ such that $K_{X'}+B'+\Mm_{X'}$ is ample$/Z$. By \cite[Lemmas 3.5, 3.8]{HL21} ($=$\cite[Version 3, Lemmas 3.9, 3.15]{HL21}) and Theorem \ref{thm: existence good minimal model under pullbacks}, we may run a $(K_{Y}+B_Y+\Mm_Y)$-MMP$/Z$ with scaling of an ample$/Z$ divisor which terminates with a good minimal model $(Y',B_Y',\Mm)/Z$. By \cite[Lemma 3.5]{HL21} ($=$\cite[Version 3, Lemma 3.9]{HL21}), $(Y',B_Y',\Mm)$ is a gdlt model of $(X',B',\Mm)$, and we get (1).

We let $p: W\rightarrow Y$ and $q: W\rightarrow Y'$ be a resolution of indeterminacy of the induced birational map $\phi_Y: Y\dashrightarrow Y'$. By \cite[Lemma 3.4]{HL21} ($=$\cite[Version 3, Lemma 3.8]{HL21}), $p^*(K_Y+B_Y+\Mm_Y)=q^*(K_{Y'}+B_{Y'}+\Mm_{Y'})+F$ where $F\geq 0$ is exceptional$/Y'$, and $\Supp p_*F$ contains all $\phi_Y$-exceptional divisors. By (1), $a(S_Y,Y,B_{Y'},\Mm)=0$, hence $S_Y$ is not a component of $\Supp p_*F$, and we get (2).
\end{proof}


\begin{thm}[{cf. \cite[Theorem 4.1]{Has20b}}]\label{thm: has20b 4.1 rel ver}
Let $(X,B,\Mm)/U$ be an NQC glc g-pair and $\pi: X\rightarrow Z$ a projective morphism$/U$ such that $Z$ is normal quasi-projective. Let $C\geq 0$ be an $\Rr$-divisor on $X$, $A_Z$ an ample$/U$ $\Rr$-divisor on $Z$, and $0\leq A\sim_{\Rr,U}\pi^*A_Z$ and $\Rr$-divisor on $X$, such that
\begin{enumerate}
    \item $C$ does not contain any glc center of $(X,B,\Mm)$,
    \item $K_X+B+C+\Mm_X\sim_{\Rr,Z}0$, and
    \item $(X,\Delta:=B+A,\Mm)$ is glc and $\Ngklt(X,B,\Mm)=\Ngklt(X,\Delta,\Mm)$.
\end{enumerate}
Then for any $(K_{X}+\Delta+\Mm_{X})$-MMP$/U$
$$(X,\Delta,\Mm):=(X_0,\Delta_0,\Mm)\dashrightarrow (X_1,\Delta_1,\Mm)\dashrightarrow\dots\dashrightarrow (X_i,\Delta_i,\Mm)\dashrightarrow\dots,$$
$K_{X_{i}}+\Delta_{i}+\Mm_{X_{i}}$ is log abundant$/U$ with respect to $(X_{i},\Delta_{i},\Mm)$ for every $i$. 
\end{thm}

\begin{proof}
For each $i$, we let $\phi_i: X\dashrightarrow X_i$ be the induced birational map. 

By Theorem \ref{thm: has20b 3.5 rel ver}, $K_{X}+B+A+\Mm_{X}$ is abundant$/U$. By Lemma \ref{lem: property of numerical and Iitaka dimension}(6), $K_{X_{i}}+\Delta_{i}+\Mm_{X_{i}}$ is abundant$/U$ for any $i$. Thus we only need to prove that $(K_{X_i}+\Delta_i+\Mm_{X_i})|_{S_i}$ is abundant$/U$ for any glc center $S_i$ of $(X_i,\Delta_i,\Mm)$.

Fix $i$ and a glc center $S_i$ of $(X_i,\Delta_i,\Mm)$. Then there exists a glc center $S$ of $(X,\Delta,\Mm)$ such that $\phi_i|_S: S\dashrightarrow S_i$ is a birational map. Let $(X',B',\Mm)$ be a gdlt model of $(X,B,\Mm)$ with induced birational morhpism $f: X'\rightarrow X$, such that there exists a component $S'$ of $\lfloor\Delta'\rfloor$ which dominates $S$. Let $C':=f^*C,A':=f^*A$, and $\Delta':=B'+A'$. Since $\Ngklt(X,B,\Mm)=\Ngklt(X',\Delta',\Mm)$, $(X',\Delta',\Mm)$ is a gdlt model of $(X,\Delta,\Mm)$. By Lemma \ref{lem: lift mmp}, we may run a $(K_{X'}+\Delta'+\Mm_{X'})$-MMP$/U$ and get a gdlt model $(X_i',\Delta_i',\Mm)$ of $(X_i,\Delta_i,\Mm)$, such that the strict transform of $S'$ on $X_i'$ is a component of $\lfloor\Delta_i'\rfloor$. We let $S_i'$ be the strict transform of $S'$ on $X_i'$, then $(K_{X_i}+\Delta_i+\Mm_{X_i})|_{S_i}$ is abundant$/U$ if and only if $(K_{X'_i}+\Delta'_i+\Mm_{X'_i})|_{S'_i}$ is abundant$/U$. Moreover, $C'$ does not contain any glc center of $(X',B',\Mm)$, $K_{X'}+B'+C'+\Mm_{X'}\sim_{\Rr,Z}0$, $(X',\Delta',\Mm)$ is glc, and $\Ngklt(X',B',\Mm)=\Ngklt(X',\Delta',\Mm)$. Thus possibly replacing $(X,\Delta,\Mm)\dashrightarrow (X_i,\Delta_i,\Mm)$ with $(X',\Delta',\Mm)\dashrightarrow (X'_i,\Delta',\Mm)$ and $A,B,C$ with $A',B',C'$, we may assume that $(X,\Delta,\Mm)$ is $\Qq$-factorial gdlt, $S$ is a component of $\lfloor\Delta\rfloor=\lfloor B\rfloor$, and $S_i$ is a component of $\lfloor\Delta_i\rfloor=\lfloor B_i\rfloor$.

Let $(S,B_S,\Mm^S)/U$ and $(S_i,B_{S_i},\Mm^S)/$ be the gdlt g-pairs induced by the adjunction formulas
$$K_S+B_S+\Mm^S:=(K_X+B+\Mm_X)|_S$$
and
$$K_{S_i}+B_{S_i}+\Mm_{S_i}:=(K_{X_i}+B_i+\Mm_{X_i})|_{S_i}.$$
Let $p: W\rightarrow S$ and $q: W\rightarrow S_i$ be a resolution of indeterminacy such that $\Mm^S$ descends to $W$, $p$ is a log resolution of $(S,\Supp B_S)$, and $q$ is a log resolution of $(S_i,\Supp B_{S_i})$. Since $A$ is semi-ample$/U$, possibly replacing $A$ with a general member of $|A/U|_{\Rr}$ (i.e. write $A=\sum r_iA_i$ where $r_i\in (0,1)$ are real numbers and $A_i$ are base-point-free$/U$ Cartier divisors. We replace $A$ with $\sum r_iH_i$ where $H_i\in |A_i/U|$ are general members) and let $A_S:=A|_S$ and $A_{S_i}:=(\phi_i)_*A)|_{S_i}$, we may assume that 
\begin{itemize}
\item $A_S\geq 0, A_{S_i}\geq 0$,
    \item $(S,\Delta_S:=B_S+A_S,\Mm)$ and $(S_i,\Delta_{S_i}:=B_{S_i}+A_{S_i},\Mm)$ are gdlt, and
    \item $p$ is a log resolution of $(S,\Supp\Delta_S)$ and $q$ is a log resolution of $(S_i,\Supp\Delta_{S_i})$.
\end{itemize}
Moreover, since $A$ is general in $|A/U|_{\Rr}$, $p^*A_S=p^{-1}_*A_S$, hence $A_W:=p^*A\leq q^*A_{S_i}$. We may write
$$K_W+B_W+A_W+\Mm^S_W=q^*(K_{S_i}+\Delta_W+\Mm^S_{S_i})+E$$
and let $\Delta_W:=B_W+A_W$, such that $B_W\geq 0$, $E\geq 0$, and $\Delta_W\wedge E=0$. Then $(W,\Delta_W)$ is log smooth. We may write
$$K_W+B_W+\Mm^S_W=p^*(K_S+\Delta_S+\Mm^S_S)+G_+-G_-$$
where $G_+\geq 0, G_-\geq 0$, and $G_+\wedge G_-=0$. 

For any component $D$ of $G_+$, 
$$a(D,S,\Delta_S,\Mm^S)>a(D,W,\Delta_W,\Mm^S)=\min\{a(D,S_i,\Delta_{S_i},\Mm^S),1\}.$$
Since $\phi_i$ is $(K_X+\Delta+\Mm_X)$-non-positive, $$a(D,S_i,\Delta_{S_i},\Mm^S)\leq a(D,S,\Delta_{S},\Mm^S).$$
Thus $a(D,S,\Delta_S,\Mm^S)>1$, hence $D$ is $p$-exceptional, and $G_+$ is $p$-exceptional.

Let $\pi_S:=\pi|_S$ and $C_S:=C|_S$. Since $C$ does not contain and glc center of $(X,B,\Mm)$, $S$ is not a component of $C$, hence $C_S\geq 0$. Then $(S,B_S,\Mm^S)/U$ is an NQC glc g-pair, $\pi_S: S\rightarrow Z$ is a projective morphism$/U$, $Z$ is normal quasi-projective, $C_S\geq 0$ is an $\Rr$ divisor on $X$, $0\leq A_S\sim_{\Rr,U}\pi_S^*A_Z$, such that
\begin{itemize}
    \item $C_S$ does not contain any glc center of $(S,B_S,\Mm^S)$,
    \item $K_S+B_S+C_S+\Mm^S_S\sim_{\Rr,Z}0$,
    \item $(S,\Delta_S=B_S+A_S,\Mm^S)$ is glc and $\Ngklt(S,B_S,\Mm^S)=\Ngklt(S,\Delta_S,\Mm^S)$,
    \item $p: W\rightarrow S$ is a log resolution of $(S,\Supp B_S)$ such that $\Mm^S$ descends to $W$, $B_W\geq 0$ is an. $\Rr$-divisor on $W$ such that $(W,B_W+p^*A_S)$ is lc and $(K_W+B_W+\Mm^S_W-p^*(K_S+B_S+\Mm^S_S))^{\geq 0}=G_+$ is $p$-exceptional. 
\end{itemize}
By Lemma \ref{lem: has20b 3.6 rel ver}, $K_W+\Delta_W+\Mm^S_W$ is abundant$/U$. By Lemma \ref{lem: property of numerical and Iitaka dimension}(3), $K_{S_i}+\Delta_{S_i}+\Mm^S_{S_i}=(K_{X_i}+\Delta_i+\Mm_{X_i})|_{S_i}$ is abundant$/U$ and we are done.
\end{proof}


\begin{thm}[{cf. \cite[Theroem 1.3]{Has20b}}]\label{thm: has20b 1.3 rel ver}
Let $(X,B,\Mm)/U$ be an NQC glc g-pair and $A\geq 0$ an ample$/U$ $\Rr$-divisor such that $(X,\Delta:=B+A,\Mm)$ is glc and $\Ngklt(X,B,\Mm)=\Ngklt(X,B+A,\Mm)$. Let $(Y,\Delta_Y,\Mm)$ be a gdlt model of $(X,\Delta,\Mm)$. The for any partial $(K_Y+\Delta_Y+\Mm_Y)$-MMP$/U$
$$\phi:(Y,\Delta_Y,\Mm)\dashrightarrow (Y',\Delta_Y',\Mm),$$
$K_{Y'}+\Delta_Y'+\Mm_{Y'}$ is log abundant$/U$ with respect to $(Y',\Delta_Y',\Mm)$.
\end{thm}
\begin{proof}
It follows from Theorem \ref{thm: has20b 4.1 rel ver}.
\end{proof}

\section{Proof of the main theorems}

\begin{proof}[Proof of Theorem \ref{thm: has22 3.17 rel ver intro}]
By \cite[Lemma 4.3]{HL21} ($=$\cite[Version 3, Lemma 4.2]{HL21}), possibly replacing $A$, we may assume that $\Ngklt(X,B,\Mm)=\Ngklt(X,\Delta,\Mm)$.

First we prove (1). Let $(Y,\Delta_Y,\Mm)$ be a gdlt model of $(X,\Delta,\Mm)$. By Theorem \ref{thm: existence good minimal model under pullbacks}, we only need to show that $(Y,\Delta_Y,\Mm)/U$ has a log minimal model. We run a $(K_Y+\Delta_Y+\Mm_Y)$-MMP$/U$ 
$$(Y,\Delta_Y,\Mm):=(Y_0,\Delta_0,\Mm)\dashrightarrow(Y_1,\Delta_1,\Mm)\dashrightarrow\dots\dashrightarrow (Y_i,\Delta_i,\Mm)\dashrightarrow\dots$$
with scaling of a general ample$/U$ divisor $H\geq 0$, and let
$$\lambda_i:=\inf\{t\mid t\geq 0, K_{Y_i}+\Delta_i+\lambda_iH_i+\Mm_{Y_i}\text{ is nef}/U\}$$
be the scaling numbers. If $\lambda_i=0$ for some $i$ then $(Y_i,\Delta_i,\Mm)/U$ is a log minimal model of $(Y,\Delta,\Mm)$ and we are done. Thus we may assume that $\lambda_i>0$ for each $i$. By \cite[Theorem 2.8]{HL21} ($=$\cite[Version 3, Theorem 2.24]{HL21}), $\lim_{i\rightarrow+\infty}\lambda_i=0$. By Theorem \ref{thm: has20b 1.3 rel ver}, $K_{Y_i}+\Delta_i+\Mm_{Y_i}$ is log abundant$/U$ for each $i$, which contradicts Theorem \ref{thm: has22 3.15 rel ver}.

Now we prove (2). We may pick $0<\epsilon\ll 1$ such that $\frac{1}{2}A+\epsilon\Mm_X$ is ample$/U$. Possibly replacing $(X,B,\Mm)$ with $(X,B,(1-\epsilon)\Mm)$ and $A$ with a general member in $|A+\epsilon\Mm_X|_{\Rr}$, we may assume that $\Nklt(X,B)=\Ngklt(X,B,\Mm)=\Ngklt(X,\Delta,\Mm)$. By \cite[Lemma 5.9]{HL21} ($=$\cite[Version 3, Lemma 5.18]{HL21}), there exists a birational morphism $h: W\rightarrow X$ such that $\Mm$ descends to $W$ and $\Supp(h^*\Mm_X-\Mm_W)=\Exc(h)$. We may pick an $h$-exceptional $\Rr$-divisor $E\geq 0$ such that $-E$ is ample$/X$. Let $K_W+B_W:=h^*(K_X+B)$. Since $\Nklt(X,B)=\Ngklt(X,B,\Mm)$, there exists $0<\delta\ll 1$ such that $(W,B_W+\delta E)$ is sub-lc and $\frac{1}{2}h^*A-\delta E$ is ample$/U$. Thus $\Mm_W+\frac{1}{2}h^*A-\delta E$ is ample$/U$, and we may pick $0\leq H_W\sim_{\Rr,U}\Mm_W+\frac{1}{2}h^*A-\delta E$ such that $(W,B_W+H_W+\delta E)$ is sub-lc. Let $\Delta':=h_*(B_W+H_W+\delta E)$, then $(X,\Delta')$ is lc and $\Delta'\sim_{\Rr,U}B+\Mm_X+\frac{1}{2}A$. Possibly replacing $A$ we may assume that $(X,\Delta'+\frac{1}{2}A)$ is lc. By \cite[Theorem 1.5]{HH20},  $(X,\Delta'+\frac{1}{2}A)$ has a good minimal model. By \cite[Lemma 4.3]{HL21} ($=$\cite[Version 3, Lemma 4.2]{HL21}), we get (2).
\end{proof}


\begin{proof}[Proof of Theorem \ref{thm: bir12 1.1 gpair}]
If $K_X+B+\Mm_X$ is not pseudo-effective$/Z$, then the theorem follows from \cite[Lemma 4.4(1)]{BZ16}. So we may assume that $K_X+B+\Mm_X$ is pseudo-effective$/Z$. By Theorem \ref{thm: existence good minimal model under pullbacks}, we only need to prove (2), so we may assume that $(X,B,\Mm)$ is $\Qq$-factorial gdlt. We run a $(K_X+B+\Mm_X)$-MMP$/Z$ with scaling of an ample$/U$ $\Rr$-divisor $H\geq 0$:
$$(X,B,\Mm):=(X_0,B_0,\Mm)\dashrightarrow (X_1,B_1,\Mm)\dashrightarrow\dots\dashrightarrow (X_i,B_i,\Mm)\dashrightarrow\dots.$$
By Theorem \ref{thm: has20b 4.1 rel ver} ($U$ and $Z$ in Theorem \ref{thm: has20b 4.1 rel ver} both correspond to our $Z$, $A_Z$ and $A$ of Theorem \ref{thm: has20b 4.1 rel ver} correspond to $0$, and $C$ corresponds to our $A$), 
$K_{X_i}+B_i+\Mm_{X_i}$ is log abundant$/Z$ with respect to $(X_i,B_i,\Mm)$ for every $i$. By \cite[Theorem 2.8]{HL21} ($=$\cite[Version 3, Theorem 2.24]{HL21}) and Theorem \ref{thm: has22 3.15 rel ver}, this MMP terminates with a log minimal model of $(X,B,\Mm)/Z$.
\end{proof}
\begin{thebibliography}{99}

\bibitem[AK00]{AK00} D. Abramovich and K. Karu, \textit{Weak semistable reduction in characteristic 0}, Invent. Math. \textbf{139} (2000), no. 2, 241--273.

%\bibitem[AHK07]{AHK07} V. Alexeev, C. D. Hacon, and Y. Kawamata, \textit{Termination of (many) $4$-dimensional log flips}, Invent. Math. \textbf{168} (2007), no. 2, 433--448.

%\bibitem[Amb03]{Amb03} F. Ambro, \textit{Quasi-log varieties}, Tr. Mat. Inst. Steklova \textbf{240} (2003), Biratsion. Geom. Linein. Sist. Konechno Porozhdennye Algebry, 220--239; translation in Proc. Steklov Inst. Math. 2003, no. 1 (240), 214--233.

%\bibitem[Bir07]{Bir07} C. Birkar, \textit{Ascending chain condition for log canonical thresholds and termination of log flips}, Duke Math. J. \textbf{136} (2007), no. 1, 173--180. 

\bibitem[Bir12]{Bir12} C. Birkar, \textit{Existence of log canonical flips and a special LMMP}, Pub. Math. IHES., \textbf{115} (2012), 325--368.



%\bibitem[Bir12b]{Bir12b} C. Birkar, \textit{On existence of log minimal models and weak Zariski decompositions}, Math. Ann. \textbf{354} (2012), no. 2, 787--799.

%\bibitem[Bir18]{Bir18} C. Birkar, \textit{Log Calabi-Yau fibrations}, arXiv: 1811.10709v2.
	
\bibitem[Bir19]{Bir19} C. Birkar, \textit{Anti-pluricanonical systems on Fano varieties}. Ann. of Math. (2), \textbf{190} (2019), 345--463.
	
%\bibitem[Bir20a]{Bir20a} C. Birkar, \textit{Geometry and moduli of polarised varieties}, arXiv: 2006.11238v1.

\bibitem[Bir20]{Bir20} C. Birkar, \textit{Generalised pairs in birational geometry}, arXiv: 2008.01008v2.

%\bibitem[Bir20c]{Bir20c} C. Birkar, \textit{On connectedness of non-klt loci of singularities of pairs}, arXiv: 2010.08226v1.

\bibitem[Bir21]{Bir21} C. Birkar, \textit{Singularities of linear systems and boundedness of Fano varieties}, Ann. of Math. \textbf{193} (2021), no. 2, 347--405.


%\bibitem[Bir21b]{Bir21b} C. Birkar, \textit{Boundedness and volume of generalised pairs}, arXiv: 2103.14935v2.

\bibitem[BCHM10]{BCHM10}
C. Birkar, P. Cascini, C. D. Hacon and J. M\textsuperscript{c}Kernan, \textit{Existence of minimal models for varieties of log general type}, J. Amer. Math. Soc. \textbf{23} (2010), no. 2, 405--468.

%\bibitem[BDCS20]{BDCS20} C. Birkar, G. Di Cerbo, and R. Svaldi, \textit{Boundedness of elliptic Calabi-Yau varieties with a rational section}, arXiv: 2010.09769v1.

\bibitem[BH22]{BH22} C. Birkar and C. D. Hacon, \textit{Variations of generalised pairs}, arXiv:2204.10456v1.

%\bibitem[BH14]{BH14} C. Birkar and Z. Hu, \textit{Polarized pairs, log minimal models, and Zariski decompositions}, Nagoya Math. J. \textbf{215} (2014), 203--224.

\bibitem[BZ16]{BZ16} C. Birkar and D.-Q. Zhang, \textit{Effectivity of Iitaka fibrations and pluricanonical systems of polarized pairs}, Pub. Math. IHES., \textbf{123} (2016), 283--331.

\bibitem[Che20]{Che20} G. Chen, \textit{Boundedness of $n$-complements for generalized pairs}, arXiv: 2003.04237v2.

%\bibitem[CT20]{CT20} G. Chen and N. Tsakanikas, \textit{On the termination of flips for log canonical generalized pairs}, arXiv: 2011.02236v1.

%\bibitem[CX20]{CX20} G. Chen and Q. Xue, \textit{Boundedness of $(\epsilon,n)$-Complements for projective generalized pairs of Fano type}, arXiv: 2008.07121v1.

\bibitem[Cho08]{Cho08} R. Choi, \textit{The geography of log models and its applications}, PhD Thesis, Johns Hopkins University (2008).

%\bibitem[Fil18a]{Fil18a} S. Filipazzi, \textit{Boundedness of Log Canonical Surface Generalized Polarized Pairs}, Taiwanese J. Math. \textbf{22} (2018), no.4, 813--850.

%\bibitem[Fil18b]{Fil18b} S. Filipazzi, \textit{On a generalized canonical bundle formula and generalized adjunction}, arXiv: 1807.04847v3.

%\bibitem[Fil20]{Fil20} S. Filipazzi, \textit{On the boundedness of $n$-folds with $\kappa(X)=n-1$}, arXiv: 2005.05508v2. 

%\bibitem[FS20a]{FS20a} S. Filipazzi and R. Svaldi, \textit{Invariance of Plurigenera and boundedness for Generalized Pairs}, arXiv: 2005.04254v2.



\bibitem[FS20]{FS20} S. Filipazzi and R. Svaldi, \textit{On the connectedness principle and dual complexes for generalized pairs}, arXiv: 2010.08018v2.

%\bibitem[FW20]{FW20} S. Filipazzi and J. Waldron, \textit{Connectedness principle in characteristic $p>5$}, arXiv: 2010.08414v2.

%\bibitem[Fuj04]{Fuj04} O. Fujino, \textit{Termination of $4$-fold canonical flips}, Publ. Res. Inst. Math. Sci, \textbf{40} (2004), no. 1, 231--237.

%\bibitem[Fuj05]{Fuj05} O. Fujino, \textit{Addendum to “Termination of $4$-fold canonical flips”}, Publ. Res. Inst. Math. Sci. \textbf{41} (2005), no. 1, 252--257.

%\bibitem[Fuj10]{Fuj10} O. Fujino, \textit{On Kawamata’s theorem}, Classification of Algebraic Varieties, EMS Ser. of Congr. Rep., Eur. Math. Soc., Z\"urich (2010), 305--315.

%\bibitem[Fuj11]{Fuj11} O. Fujino, \textit{Foundations of the minimal model program}, MSJ Memoirs, \textbf{35}. Mathematical Society of Japan, Tokyo (2017).

%\bibitem[Fuj12]{Fuj12} O. Fujino, \textit{Base point free theorems: saturation, B-divisors, and canonical bundle formula}, Algebra Number Theory \textbf{6} (2012), no. 4, 797--823.

%\bibitem[Fuj13]{Fuj13} O. Fujino, \textit{A transcendental approach to Koll\'ar’s injectivity theorem II},  J. Reine Angew. Math. \textbf{681} (2013), 149--174.

%\bibitem[Fuj18]{Fuj18} O. Fujino, \textit{Fundamental properties of basic slc-trivial fibrations I}, to appear in Publ. Res. Inst. Math. Sci., arXiv: 1804.11134v3.

\bibitem[Fuj19]{Fuj19} O. Fujino, \textit{Corrigendum: On subadditivity of the logarithmic Kodaira dimension}, arXiv: 1904.11639v3.

%\bibitem[Fuj21]{Fuj21} O. Fujino, \textit{Cone theorem and Mori hyperbolicity},  arXiv:2102.11986v1.

%\bibitem[FG14]{FG14} O. Fujino and Y. Gongyo, \textit{Log pluricanonical representations and abundance conjecture}, Compos. Math. \textbf{150} (2014), no. 4, 593--620.

%\bibitem[FH21]{FH21} O. Fujino and K. Hashizume, \textit{Existence of log canonical modifications and its applications}, arXiv: 2103.01417.

\bibitem[FM00]{FM00} O. Fujino and S. Mori, \textit{A canonical bundle formula}, J. Differential Geom. \textbf{56} (2000), no. 1, 167--188.

%\bibitem[Fuk96]{Fuk96} S. Fukuda, \textit{On base point free theorem}, Kodai Math. J.19 (1996), no. 2,191--199.

%\bibitem[Gon11]{Gon11} Y. Gongyo, \textit{On the minimal model theory for dlt pairs of numerical kodaira dimension zero}, Math. Rest. Lett. \textbf{18} (2011), no. 5, 991--1000.

%\bibitem[HH19]{HH19}  C. D. Hacon and J. Han, \textit{On a connectedness principle of Shokurov-Koll\'ar type,} Sci. China Math. \textbf{62} (2019), no. 3, 411--416.

\bibitem[HL21]{HL21} C. D. Hacon and J. Liu, \textit{Existence of generalized lc flips}, submitted version, preprint. arXiv: 2105.13590.

%\bibitem[HMX14]{HMX14} C. D. Hacon, J. M\textsuperscript{c}Kernan, and C. Xu, \textit{ACC for log canonical thresholds}, Ann. of Math. \textbf{180} (2014), no. 2, 523--571.

\bibitem[HMX18]{HMX18} C. D. Hacon, J. M\textsuperscript{c}Kernan, and C. Xu, \textit{Boundedness of moduli of varieties of general type}, J. Eur. Math. Soc. \textbf{20} (2018), 865--901.


\bibitem[HX13]{HX13} C. D. Hacon and C. Xu, \textit{Existence of log canonical closures}, Invent. Math. \textbf{192} (2013), no. 1, 161--195.

%\bibitem[HX15]{HX15} C. D. Hacon and C. Xu, \textit{Boundedness of log Calabi-Yau pairs of Fano type}, Math. Res. Lett. \textbf{22} (2015), no. 6, 1699--1716.

%\bibitem[HX16]{HX16} C. D. Hacon and C. Xu, \textit{On finiteness of B-representations and semi-log canonical abundance} in Minimal Models and Extremal Rays (Kyoto, 2011), Adv. Stud. Pure Math. \textbf{70} (2016), Math. Soc. Japan, Tokyo, 361--378. 
\bibitem[HL18]{HL18} J. Han and Z. Li, \textit{Weak Zariski decompositions and log terminal models for generalized polarized pairs}, arXiv: 1806.01234v2.

%\bibitem[HL20a]{HL20a} J. Han and Z. Li, \textit{On Fujita’s conjecture for pseudo-effective thresholds}, Math. Res. Lett. \textbf{27} (2020), no. 2, 377--396.

%\bibitem[HL20b]{HL20b} J. Han and Z. Li, \textit{On accumulation points of pseudo-effective thresholds}, Manuscripta math (2020).

%\bibitem[HL20c]{HL20c} J. Han and J. Liu, \textit{Effective birationality for sub-pairs with real coefficients}, arXiv: 2007.01849v1.

\bibitem[HLS19]{HLS19} J. Han, J. Liu, and V. V. Shokurov, \textit{ACC for minimal log discrepancies of exceptional singularities}, arXiv: 1903.04338v2.

\bibitem[HL19]{HL19} J. Han and W. Liu, \textit{On a generalized canonical bundle formula for generically finite morphisms}, arXiv: 1905.12542v3,  to appear in Ann. Inst. Fourier (Grenoble).

\bibitem[HL20]{HL20} J. Han and W. Liu, \textit{On numerical nonvanishing for generalized log canonical pairs}, Doc. Math. \textbf{25} (2020), 93--123.

\bibitem[Has18]{Has18} K. Hashizume, \textit{Minimal model theory for relatively trivial log canonical pairs}, Ann. Inst. Fourier (Grenoble) \textbf{68} (2018), no. 5, 2069--2107.

\bibitem[Has19]{Has19} K. Hashizume, \textit{Remarks on special kinds of the relative log minimal model program}, Manuscripta Math. \textbf{160} (2019), no. 3, 285--314.


\bibitem[Has20a]{Has20a} K. Hashizume, \textit{Finiteness of log abundant log canonical pairs in log minimal model program with scaling}, arXiv:2005.12253v3.

\bibitem[Has20b]{Has20b} K. Hashizume, \textit{Non-vanishing theorem for generalized log canonical pairs with a polarization}, arXiv: 2012.15038v1.

\bibitem[Has22]{Has22} K. Hashizume, \textit{Iitaka fibrations for dlt pairs polarized by a nef and log big divisor}, arXiv: 2203.05467v1.


\bibitem[HH20]{HH20}  K. Hashizume and Z. Hu, \textit{On minimal model theory for log abundant lc pairs}, J. Reine Angew. Math., \textbf{767} (2020), 109--159. 

\bibitem[Hu20]{Hu20} Z. Hu, \textit{Log abundance of the moduli b-divisors for lc-trivial fibrations}, arXiv: 2003.14379v3.

%\bibitem[Hu21]{Hu21} Z. Hu, \textit{An abundance theroem for generalised pairs}, arXiv: 2103.11813v1.

%\bibitem[Jia21]{Jia21} J. Jiao, \textit{On the Boundedness of Canonical Models}, arXiv: 2103.13609v1.

%\bibitem[Kaw84]{Kaw84} Y. Kawamata, \textit{The cone of curves of algebraic varieties}, Ann. of Math. \textbf{119} (1984), 603--633.

%\bibitem[Kaw92]{Kaw92} Y. Kawamata, \textit{Termination  of  log  flips  for  algebraic $3$-folds}, Internat. J. Math. \textbf{3} (1992), no. 5, 653--659.

\bibitem[Kaw98]{Kaw98} Y. Kawamata, \textit{Subadjunction of log canonical divisors, II}, Amer. J. Math. \textbf{120} (1998), 893--899.

\bibitem[Kaw15]{Kaw15} Y. Kawamata, \textit{Variation of mixed Hodge structures and the positivity for algebraic fiber spaces}, Advanced Studies in Pure Mathematics, \textbf{65} (2015), 27--57.

%\bibitem[KMM87]{KMM87} Y. Kawamata, K. Matsuda, and K. Matsuki, \textit{Introduction to the minimal model problem}, Algebraic geometry, Sendai, 1985, 283--360, Adv. Stud. Pure Math., \textbf{10}, North-Holland, Amsterdam, 1987.

%\bibitem[Kol84]{Kol84} J. Koll\'ar, \textit{The cone theorem}, Ann. of Math., \textbf{120} (1984), 1--5.

%\bibitem[Kol07]{Kol07} J. Koll\'ar, \textit{“Kodaira’s canonical bundle formula and adjunction}. In: \textit{Flips for 3-folds and 4-folds}. Ed. by A. Corti. Vol. 35. Oxford Lecture Series in Mathematics and its Applications. Oxford: Oxford University Press, 2007. Chap. 8, 134--162.

%\bibitem[Kol14]{Kol14} J. Koll\'ar, \textit{Semi-Normal Log Centres and Deformations of Pairs}, Proc. Edinburgh Math. Soc., \textbf{57} (2014), no. 1, 191--199.

\bibitem[KM98]{KM98} J. Koll\'{a}r and S. Mori, \textit{Birational geometry of algebraic varieties}, Cambridge Tracts in Math. \textbf{134} (1998), Cambridge Univ. Press.

%\bibitem[LP20a]{LP20a} V. Laz\'ic and T. Peternell, \textit{On generalised abundance, I}, Publ. Res. Inst. Math. Sci. \textbf{56} (2020), no. 2, 353--389.

%\bibitem[LP20b]{LP20b} V. Laz\'ic and T. Peternell, \textit{On generalised abundance, II}, Peking Mathematical Journal \textbf{3} (2020), 1--46.

%\bibitem[LT19]{LT19} V. Lazi\'c and N. Tsakanikas, \textit{On the existence of minimal models for log canonical pairs}, to appear in Publ. Res. Inst. Math. Sci., arXiv: 1905.05576v3.

\bibitem[LT21]{LT21} V. Lazi\'c and N. Tsakanikas, \textit{Special MMP for log canonical generalised pairs (with an appendix joint with Xiaowei Jiang)}, arXiv:2108.00993v4.


%\bibitem[Li20]{Li20} Z. Li, \textit{Boundedness of the base varieties of certain fibrations}, arXiv: 2002.06565v2.

%\bibitem[Li21]{Li21} Z. Li, \textit{Fujita’s conjecture on iterated accumulation points of pseudo-effective thresholds}, Selecta Mathematica \textbf{27} (2021), no. 9.

%\bibitem[Liu21]{Liu21} J. Liu, \textit{Sarkisov program for generalized pairs}, Osaka J. Math., \textbf{58} (2021), no. 4.

%\bibitem[LX21]{LX21} J. Liu and L. Xie, \textit{Number of singular points on projective surfaces}, arXiv: 2103.04522v1.

\bibitem[Les16]{Les16} J. Lesieutre, \textit{A pathology of asymptotic multiplicity in the relative setting}, Math. Res. Lett. \textbf{23}(5), 1433--1451.

%\bibitem[Nak86]{Nak86} N. Nakayama, \textit{Invariance of the plurigenera of algebraic varieties under minimal model conjectures}, Topology \textbf{25} (1986), no. 2, 237--251.

\bibitem[Nak04]{Nak04} N. Nakayama, \textit{Zariski-decomposition and abundance}, MSJ Memoirs, vol. 14, Mathematical Society of Japan, Tokyo, 2004.


%\bibitem[Nak16]{Nak16} Y. Nakamura, \textit{On minimal log discrepancies on varieties with fixed Gorenstein index}. Michigan Math. J., 65 (1), 165--187, 2016.

%\bibitem[Sho96]{Sho96} V.V. Shokurov, \textit{3-fold log models}, J. Math. Sci. \textbf{81} (1996), no. 3, 2667--2699.

%\bibitem[Sho20]{Sho20} V.V. Shokurov, \textit{Existence and boundedness of $n$-complements}, arXiv: 2012.06495v1.

\end{thebibliography}












    












































\end{document}




