\chapter{更多模问题}
在代数几何中还有许多其他模问题,这一章举例其中一些,以及GIT和叠理论在这些问题中的作用。
\subsection{标记曲线}
无论GIT还是叠的构造,考虑DM稳定曲线作为被分类对象的一个重要原因就是其自同构群有限。在GIT构造中,曲线的Hilbert点稳定需要群作用的稳定子群有限,也就意味着曲线本身的自同构群有限。而在叠理论中,形如$ [X/G] $的群胚是DM叠的一个判定条件同样是稳定子群有限。在$ g\geqslant2 $时光滑射影曲线的自同构群是有限的,但低亏格时就不成立。回顾DM稳定曲线的等价定义时可以发现,对于DM稳定曲线的有理不可约分支,它的自同构群是被其上标记点(与其他部分的交点,即结点)限制的;反之,可以在曲线上附加标记点的结构,使得低亏格时也可以限制自同构。具体来说,一个\textbf{$ n $-标记曲线($ n $ pointed curve)}$ (C;P_1,\ldots,P_n) $是指一条结点曲线$ C $,和$ n $个不同的光滑点$ P_1,\ldots,P_n $,或者记为$ C $上的一个除子$ D=\sum_{i=1}^{n}P_i $,这时曲线记为$ (C;D) $。这些标记点的地位和结点曲线$ C $的结点相同,其间态射$ f:(C;D)\to (C':D') $必须满足$ f(P_i)=(P_i') $。用$ \omega_C(D) $代替对偶化层来定义DM稳定性,或者等价的把有理不可约分支与其他部分的交点个数是少为3(2)改为交点和标记点一共至少为3(2),这就得到了\textbf{DM稳定标记曲线}。在稳定曲线族
$$ \mathcal{C}\to S $$
也附加上额外的截影(section)$ \sigma_i:S\to \mathcal{C} $,使得在每个纤维通过这些截影给出标记点,这就是DM稳定$ n $-标记曲线族。
$ n $-标记光滑射影曲线的模问题记为$ M_{g,n} $,而DM稳定$ n $-标记曲线的模问题记为$ \overline{M_{g,n}} $。在一般的DM稳定曲线的模空间构造中用到的大部分预备构造都有对应的结论:DM稳定$ n $-标记曲线通过$ (\omega_C(D))^{\otimes \nu} $嵌入到$ \mathbb{P}^r $中,其中$ r=(2\nu-1)(g-1)+\nu n-1 $,并且有Hilbert多项式
$$ p(t)=(2\nu t-1)(g-1)+\nu nt $$
可以构造Hilbert概型参数化$ \mathbb{P}^r $中全部有$ n $个标记点且以$ p(t) $为Hilbert多项式的曲线,记作$ \mathcal{H} $;进一步用相应的形变理论可以得到对应的$ \mathcal{K} $来参数化全部被$ (\omega_C(D))^{\otimes \nu} $嵌入的DM稳定$ n $-标记曲线,并且
$$ \dim \mathcal{K}=3g-3+n+(r+1)^2-1 $$

叠意义的模空间构造可以相对轻松地推广,群胚$ [\mathcal{K}/G] $就是$ \overline{\mathscr{M}_{g,n}} $,光滑曲线的部分同理。而如果用GIT方法构造模空间,需要证明相应版本的潜稳定性定理等,来说明$ n $-标记曲线稳定性与对应Hilbert点的稳定性的关系。但这部分直到2006才被Elizabeth Baldwin补上\upcite{GITandModuliofstablecurve},其证明思路是通过归纳,将亏格$ g $且有$ n $个标记点的曲线的稳定性,与亏格$ g+n $而没有标记点的曲线的稳定性联系起来。后来又被David Swinarski推广到加权曲线(weighted pointed curve)上\upcite{GITandModuliofPointedCurves},即标记点可以附加有理数$ 0\leqslant a_i \leqslant 1 $的权重。

\subsection{vectorbundle over sm curve}

\subsection{曲面}
在曲面的分类时,GIT的思路遇到了困难。



\section{模问题分析}

关于模空间的问题:例如是否紧?是否有合适的紧化?维数怎样?是否连通、不可约?是否有奇点?是否有上同调环/chow环?等···

与构造模空间、研究其性质同等重要的,是提出一个好的模问题,并定义一个好的期望的回答(moduli space as scheme/algebraic space/stack/corase space/fine space/good space etc):模问题确实处理了值得关心的对象,模空间确实很好的反映了这些对象的性质。

实际上几乎没有回答这个模问题,而是更好的提出模问题,以及分析模问题本身。

模问题$ M $的groupoid即使是一个DM-stack,也和函子本身的信息完全一致,只是可以更“类似”scheme。DM-stack的diagonal的unramified,因此自同构群blabla。这说明模问题的对象的自同构群blabla


