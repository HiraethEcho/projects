% !Mode:: "TeX:UTF-8"

% 中英文摘要
\begin{cabstract}	
在代数几何中,几何对象的等价类分类问题一直是核心问题之一,被称为模问题。在诸多模问题中,关于曲线的模问题开始较早、有更成熟的技术和结果。众多著名数学家如Riemann、Deligne和Mumford在曲线模空间领域中做了杰出的工作。

这篇读书报告的主要目的是总结几何不变量理论(GIT)和叠理论两种构造模空间的思路,以及它们的紧化。首先介绍了被分类的对象,即光滑射影曲线和DM稳定曲线;然后介绍了一些一般理论,作为构造模空间的工具,包括范畴论、叠理论、Hilbert概型和GIT,这些是本文的核心内容;接着结合一般理论和具体曲线的性质,构造了两种曲线模空间,以及它们的紧化。在最后指出进一步研究的问题,并讨论了其他模问题。
\end{cabstract}

\begin{eabstract}
In algebraic geometry, the classification of isomorphic classes of certain kinds of  geometric objects is one of the major problem, and is usually called moduli problem. Among all kinds of moduli problems, the moduli of curves started early, and has more mature tools and results. Many famous mathematicians have made many outstanding works, such as Rieamann, Deligne and Mumford.

The main purpose of this report is to construct  moduli space of curves by geometric invariant theory (GIT) and stack theory, and their compactifications. First I introduce smooth curves and DM stable curves,  which are the geometric objects to be classify. Then I give some general theories as tools constructing mouduli space, including category theory, stack theory, Hilbert scheme and GIT. These form the major part of this article. Next I construct two kinds of moduli spaces of curves, using the  properties of curves and the tools I've showed earlier. At last I give some further questions, and discuss some other modui spaces. 
\end{eabstract}