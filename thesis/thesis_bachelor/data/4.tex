\chapter{曲线模空间的构造}
在这一章中,将使用之前引入的各种工具和结论,用两种方法赋予曲线模空间$ \mathrm{M}_g $几何结构和紧化,并且简单讨论它们的性质与关系。
\section{GIT思路的构造}
GIT思路的构造最终得到的是曲线模问题的粗模空间。为了整体线索清晰,首先给出构造中需要的定理和得到的结论,而把定理证明的部分细节放在后面。这一节主要内容来自\cite{ModuliofCurves,LecturesonModuliofCurves}。
\subsection{主要思路}
记$ \mathcal{H}=\mathcal{H}_{d,g,r} $,其中$ g\geqslant2 $,$ d=2n(g-1)$ , $r=(2n-1)(g-1)-1 $,且$ n\geqslant 5 $。在Hilbert概型$ \mathcal{H} $的构造中,$ m $的选取并不是唯一的,因此$ G=\mathrm{SL}(r+1) $在$ \mathbb{P}^r $的作用(代替自同构),可以诱导在不同$ m $对应的$ \mathbb{P}(\bigwedge^{p(m)}S_m) $上的作用,进而诱导在$ \mathcal{H} $上不同的作用。记$ \mathcal{H} $中亏格$ g $的DM稳定曲线的$ n $-典范嵌入对应的部分为$ \mathcal{K} $,在合适的$ m $的选取,即合适的$ G $的作用下,对应的商就是所求的粗模空间$ \overline{\mathcal{M}_g} $。其中合适是指使曲线的DM稳定性与曲线的$ m $-Hilbert点的GIT意义下的稳定性等价。证明等价性的关键思路在于通过一系列过渡的稳定性结论,将不同的稳定性联系起来,其中最重要(证明也最长、最需技巧性)的一条是潜稳定性:
\begin{theorem}[\cite{ModuliofCurves}Theorem4.45]\label{potentiallystable}	
	存在正整数$ m' $,使得任意$ m\geqslant m' $,如果$ \mathbb{P}^r $中亏格$ g $,次数$ d $的连通曲线$ C $的$ m $-Hilbert点$ [C]_m\in \mathcal{H} $在对应$ G $的作用下是半稳定的,则$ C $是潜稳定的,特别的,是DM半稳定的。
\end{theorem}
通过这个结论,以及光滑曲线的稳定性,可以选定一个$ m $和对应的作用,有集合的包含关系
$$  \{[C]_m|C\text{是光滑射影曲线}\}\subset \mathcal{K}^{s}\subset \mathcal{K}^{ss}\subset \{[C]_m|C\text{是DM半稳定曲线}\} $$
这些集合的关系通过下述断言有更精确的描述:
	\begin{enumerate}
		\item $ \mathcal{K}^{ss}$ 在 $ \mathcal{H}^{ss} $是闭的;
		\item 如果$ [C]_m\in \mathcal{K}^{ss} $,那么$ C $是DM稳定的;
		\item $ \mathcal{K}^{ss}=\mathcal{K}^s $;
		\item 每一个亏格$ g $的DM稳定曲线$ C $,都有一个$ n $-典范模型,使得$ [C]_m\in \mathcal{K}^{ss} $,即$ \mathcal{K}=\mathcal{K}^{ss} $。
	\end{enumerate}

从断言得到 $ \mathcal{K}=\mathcal{K}^s=\mathcal{K}^{ss}$ ,并且在$ \mathcal{H}_{d,g,r}^{ss} $中是闭的。从GIT的理论知道,可以做商$ \pi:\mathcal{H}_{d,g,r}^{ss}\to\mathcal{H}_{d,g,r}^{ss}/G $,并且闭集$ \pi:\mathcal{K}^{ss}\to \mathcal{K}^{ss}/G $是几何商,记$\mathcal{K}^{ss}/G= \overline{\mathcal{M}_{g}} $,$ \mathcal{K} $中光滑曲线的部分是开集,其像记作 $ \mathcal{M}_g\subset \overline{\mathcal{M}_{g}} $。接下来只需验证这两者就是曲线模问题$ M_g $和$ \overline{M_g} $的粗模空间。我们对$ \overline{\mathcal{M}_g} $验证,$ \mathcal{M}_g $的性质是随之成立的。

取$ \overline{M_g}(B) $中一个等价类的代表元,即一个$ B $上的亏格$ g $的DM稳定曲线族$ \phi:\mathcal{C}\to B $。由$ \phi_*(\omega_{\mathcal{C}/B}^{\otimes n}) $在$ B $上是局部自由层,因此可以将$ \mathcal{C} $嵌入在$ \mathbb{P}^r \times B$中,不同的嵌入就取遍$ \phi:\mathcal{C}\to B $的等价类的全部代表元。由$ \mathcal{H} $的泛性质可以得到$ f: B\to \mathcal{H} $。事实上,可以进一步限制在$ f:B\to \mathcal{K} $上。再复合上$ \pi: \mathcal{K}\to \overline{\mathcal{M}_g} $,因为这时$ G $-不变的,所以$ \phi:\mathcal{C}\to B $所在等价类的代表元会给出相同的态射$ B\to  \overline{\mathcal{M}_g} $,这就定义了函子$ p:\overline{M_g}\to \mathrm{Mor}(-,\overline{\mathcal{M}_g}) $。借助$ \mathcal{H} $的泛性质不难验证$ \overline{\mathcal{M}_g} $和函子$ p $确实构成了粗模空间。

注意到群$ G=\mathrm{SL}(r+1) $是光滑仿射的,并且$ \dim G=(r+1)^2-1 $,而$ \dim \mathcal{K}=3g-3+(r+1)^2-1 $,因此通过概型间态射和维数的关系,容易得到
$$ \dim \overline{\mathcal{M}_g}=\dim \mathcal{K}-\dim G=3g-3 $$
这和最初黎曼得到的结论一致。

\subsection{部分证明细节} 
现在我们给出潜稳定性定理\ref{potentiallystable}和断言的部分细节:

潜稳定性定理的证明非常复杂,Gieseker用了50页的篇幅来证明它,这里只列出主要步骤,并以其中一步的证明说明如何使用关于稳定性的数值判断准则。定理的证明分两部分,第一部分用反证法说明$ \mathbb{P}^r $中性质较差的曲线$ C $的Hilbert点$ [C]_m $不稳定,第二部分通过$ C\hookrightarrow \mathbb{P}^r $上的不等式,进一步说明$ C $的性质。在第一部分需要依次证明如果$ [C]_m $半稳定,则$ C\hookrightarrow \mathbb{P}^r $是
\begin{enumerate}
	\item 非退化的,即$ C $不含在任意一个超平面中;
	\item $ C $的每个分支是一般约化的(generically reduced),即$ C $在每个一般点(generic point)的局部环是约化的;
	\item $ C_{red} $的每个奇异点是二重点;
	\item $ C_{red} $的奇异点不会是尖点(cusp singularity);
	\item $ C_{red} $的奇异点不会是自切点(tacnode);
\end{enumerate}
第二部分证明了$ C $上的不等式:令$ X $是$ C $的约化连通完全子曲线,记$ Y=\overline{C-X} $,并赋予既约诱导子概型结构(reduced induced structure),如果$ X \cap  Y $上的点$ P_1,\ldots,P_k $满足$ Y $的每个不可约分支$ Y_j $都有
$$ \deg L_C|_{Y_j}(-D)\geqslant0 $$
其中$ L_C $是$ \mathscr{O}_{\mathbb{P}^r}(1) $的拉回,$ D=\sum_{i=1}^{k}P_i $,则有不等式
$$ \frac{h^0(X,L_X)}{d-g+1}\geqslant\frac{\deg L_X -\frac{k}{2}}{d} $$
通过这个不等式可以依次证明
\begin{enumerate}
	\item $ C $是约化的曲线;
	\item $ C $的每个光滑有理不可约分支与其他部分交点至少2个,即$ C $是DM半稳定的。
	\item 潜稳定性曲线定义中第六条不等式成立;
	\item 潜稳定性曲线定义中第四、第五条成立。
\end{enumerate}
现在以第一部分第一条为例,说明定理\ref{smstable}的证明技巧如何在潜稳定性定理中应用。使用反证法,证明对退化曲线$ C $的Hilbert点$ [C]_m $可以构造出一个1-ps $ \lambda \to \mathrm{SL}(r+1) $,使$ [C]_m $不稳定。$ C $在$ \mathbb{P}^r $中退化,因此
$$ \bar{\phi}_1:H^0(\mathbb{P}^r,\mathscr{O}_{\mathbb{P}^r}(1))\to H^0(C_{red},L) $$
的核不为$ 0 $,记之为$ W_0 $,并记$ W_1=H^0(\mathbb{P}^r,\mathscr{O}_{\mathbb{P}^r}(1)) $且$ \dim W_0=r_0 $,取$ W_0 $的一组基$ w_0,\ldots,w_{r_0-1} $,并扩充为$ W_1 $的一组基$ w_0,\ldots,w_r $。这里用$ \mathrm{GL}(r+1) $版本的判别法,定义
\begin{equation*}
	\lambda (t) w_i=\left\{
	\begin{array}{rcl}
	 w_i&,&0\leqslant i \leqslant r_0-1\\
		tw_i&,&  r_0\leqslant i \leqslant r
	\end{array}\right.
\end{equation*}
即权重$ \rho_0=\cdots=\rho_{r_0-1}=0 $,$ \rho_{r_0}=\cdots=\rho_r=1 $,并且$ \sum_{i=0}^{r}\rho_i=\dim W_1-\dim W_0\leqslant d-g $。
由引理\ref{uniform m},存在与$ C $无关的$ m' $,使得任意$ m\geqslant m' $,有$ H^1(C,L^m)=0 $和
$$ \phi_m:H^0(\mathbb{P}^r,\mathscr{O}_{\mathbb{P}^r}(m))\to H^0(C,L^m) $$
是满射。记$ M_1,\ldots,M_N $是$ w_i $的$ m $次单项式,成为$ H^0(\mathbb{P}^r,\mathscr{O}_{\mathbb{P}^r}(m)) $的一组基。记$ N_C $是$ \mathscr{O}_C $的幂零元的理想层,整数$ q $使得$ N_C^q=0 $。因此可以构造滤链
$$ 0\subset W_0^qW_1^{m-q}\subset W_0^{q-1}W_1^{m-q+1}\subset\cdots\subset W_0^1W_1^{m-1}\subset  W_0^0W_1^m=H^0(\mathbb{P}^r,\mathscr{O}_{\mathbb{P}^r}(m)) $$
将这些通过$ \phi_m $映入$ H^0(C,L^m) $中,记$ V_s=\phi_m( W_0^{q-s}W_1^{m-q+s} ),s=0,\ldots,q $,这样得到$ H^0(C,L^m)  $中滤链
$$ V_0\subset \cdots \subset V_q $$
并记$ \dim V_s=v_s $。和定理\ref{smstable}的证明完全一致,可以得到$ H^0(C,L^m) $的一组基的权重$ w $的一个估计:
$$ w\geqslant (m-q+1)(dm-g+1) $$
如果点$ [C]_m $半稳定,则$ H^0(C,L^m) $有一组基,满足不等式
$$ \frac{w}{mP(m)}\leqslant \frac{\sum_{i=0}^{r}\rho_i}{d-g+1}$$
而这不等式只在$ m\leqslant (d-g+1)(q-1) $时成立。上述证明中的$ q $和$ m' $都可以对全部$ C $统一的选取,所以当所取$ m $充分大时,对于退化的$ C $,$ [C]_m $就不稳定。

这就是在潜稳定性定理证明时用到的主要技巧,通过曲线的性质在$ H^0(\mathbb{P}^r,\mathscr{O}_{\mathbb{P}^r}(1)) $上构造恰当的滤链和1-ps: $ \lambda \to \mathrm{GL}(r+1) $,进而得到恰当的$ H^0(\mathbb{P}^r,\mathscr{O}_{\mathbb{P}^r}(m)) $和$ H^0(C,L^m) $的滤链,再用不等式估计基的权重。

然后是证明断言:
\begin{enumerate}
	\item $ \mathcal{K}^{ss}$ 在 $ \mathcal{H}^{ss} $是闭的;
	\item 如果$ [C]_m\in \mathcal{K}^{ss} $,那么$ C $是DM稳定的;
	\item $ \mathcal{K}^{ss}=\mathcal{K}^s $;
	\item 每一个亏格$ g $的DM稳定曲线$ C $,都有一个$ n $-典范模型,使得$ [C]_m\in \mathcal{K}^{ss} $,即$ \mathcal{K}=\mathcal{K}^{ss} $。
\end{enumerate}

对于第一条断言,已经知道$ \mathcal{K} $在$ \mathcal{H} $是局部闭的,而$ \mathcal{H}^{ss} $在$ \mathcal{H} $中是开的,因此$ \mathcal{K}^{ss}\subset \mathcal{H}^{ss} $至少是局部闭的。将$ \mathcal{K}^{ss} $分为不可约分支$ \mathcal{Y}_j $和嵌入$ g_j:\mathcal{Y}_j\to \mathcal{H}^{ss} $,只需证明嵌入$ g_j $是本征的。根据本征性的赋值判断准则,任意离散赋值环$ R $和分式域$ K=\mathrm{Frac}\,R $,记一般点为$ \eta $,如果有交换图
$$ \xymatrix{
	\mathrm{Spec}\,K \ar[d] \ar[r] & \mathcal{K}^{ss} \ar[d] \\
	\mathrm{Spec}\,R \ar[r]^-{f} & \mathcal{H}^{ss}
} $$
则通过$ f $拉回$ \mathcal{H}^{ss} $的万有族,得到曲线族$ \mathcal{D}\to \mathrm{Spec}\,R $,并记$ \omega=\omega_{\mathcal{D}/\mathrm{Spec}\,R} $;态射$ \mathrm{Spec}\,K \to \mathcal{K}^{ss} $意味着$ \mathcal{D} $的一般纤维$ \mathcal{D}_{\eta} $的Hilbert点在$ \mathcal{K}^{ss} $中,要构造唯一的$ \mathrm{Spec}\,R \to \mathcal{K}^{ss} $,只需证明$ \mathcal{D} $的特殊纤维$ C=\mathcal{D}_0 $的Hilbert点也在$ \mathcal{K}^{ss} $中。从$ f $知$ C $的Hilbert点是半稳定的,只需要说明它是$ n $-典范嵌入的。由于一般纤维是$ n $-典范嵌入的,因此$ \omega_{\mathcal{D}/\mathrm{Spec}\,R}^{\otimes n}\cong \mathscr{O}_{\mathcal{D}}(1) $,只需证明这个同构在特殊纤维上也成立。记$ C $的不可约分支为$ C_i $,则
$$ \mathscr{O}_{\mathcal{D}}(1)|_{\mathcal{D}_0}\cong \omega^{\otimes n}(\sum a_iC_i) $$
根据潜稳定性定理,$ C $是潜稳定的,利用第六条的不等式可以说明$ C $是不可约的,因此$$ \mathscr{O}_{\mathcal{D}}(1)|_{\mathcal{D}_0}\cong \omega^{\otimes n} $$
这样就证明了$ \mathcal{K}^{ss} $在$ \mathcal{H}^{ss} $是闭的。

对于第二条断言,要说明如果$ [C]_m\in\mathcal{K}^{ss} $,则$ C $是DM稳定曲线,只需说明$ C $的自同构群有限。由于$ C $是潜稳定的,$ C $的自同构不有限只在它的光滑有理不可约分支$ X\cong \mathbb{P}^1 $上发生,即$ Y $与其他部分只有两个交点$ P,Q $。但是$ Y $在$ \mathbb{P}^r $中是直线,故$ \omega_C|_{Y}\cong \mathscr{O}_{\mathbb{P}^1}(-2)(P+Q)\cong \mathscr{O}_{\mathbb{P}^1} $,这样$ \omega_C^{\otimes n}|_{Y}\cong \mathscr{O}_{\mathbb{P}^1} $不是极丰层,矛盾!因此$ C $必然是DM稳定的。

第三条断言,如果$ \mathcal{K}^{ss} $中有一个点$ x=[C]_m $不是稳定的,则$ Gx $不是闭轨道,且$ \overline{Gx} $中有唯一的闭轨道$ Gy $,稳定子群的维数有$ \dim G_x>\dim G_y=0 $。但是$ C $是DM稳定曲线,$ G_x=\mathrm{Aut}\, C $是有限群,维数为$ 0 $,矛盾!因此断言得证。

第四条断言并不能通过纯粹的GIT理论得到,而需要DM稳定曲线的约化理论和形变理论的结果。对于任意一条DM稳定曲线$ C $,首先根据引理\ref{smoothing}构造离散赋值环$ R $上的曲线族$ \mathcal{C}\to \mathrm{Spec}\,R $,使一般纤维$ \mathcal{C}_K $是光滑曲线。这样$ [\mathcal{C}_K]_m $应当在$ \mathcal{K}^{ss} $中(根据光滑曲线的稳定性,定理\ref{smstable}),进而在$ \mathcal{H}^{ss} $中。因为$ \mathcal{H}^{ss} $和$ \mathcal{H}^{ss}/G $是射影概型,并且$ \mathcal{K}^{ss} $在$ \mathcal{H}^{ss} $中闭,因此可以取到恰当的
$$ \xymatrix{
	\mathrm{Spec}\,K \ar[d] \ar[r] & \mathcal{K}^{ss} \ar[d] \\
	\mathrm{Spec}\,R \ar[r]^-{f} & \mathcal{H}^{ss}
} $$
使得$ f $的提升$ \bar{f}:\mathrm{Spec}\,R\to \mathcal{K}^{ss} $拉回得到的曲线族$ \mathcal{D}\to \mathrm{Spec}\,R $在一般纤维有同构
$$ \mathcal{D}_K\cong \mathcal{C}_K $$
根据引理\ref{DeformationforCurves}这个同构可以唯一的延拓在曲线族上
$$ \mathcal{D}\cong \mathcal{C} $$
因此特殊纤维$ C $的Hilbert点$ [C]_m $通过$ \bar{f} $映入$ \mathcal{K}^{ss} $中。

这样四条断言就全部得到证明了。

\section{Stack思路的构造}
叠思路下构造模空间,同样需要对群作用$G=\mathrm{SL}(r+1)\curvearrowright \mathcal{K} $作商,但与GIT不同的是,并不在概型的范畴内作商,而是得到作为叠的商。这一章首先讨论一类叠 $ [X/G] $,之后说明曲线模问题的函子所定义的群胚正是这种叠。在这一节中的概型未加声明时是指一般的概型,即$ \mathrm{Spec}\,\mathbb{Z} $上的概型,主要参考\cite{DM69,GeometryAlgCurvesII,NotesModuliSpaceofCurves}。
\subsection{$ [X/G] $型的叠}\label{X/G}
令$ S $为概型,$ S $上的概型的范畴记作$ \mathscr{S}=\mathfrak{Sch}_S $。令$ X,G\in \mathscr{S} $,且$ G $是$ S $上平坦的有限型群概型,作用在$ X $上。定义群胚$ [X/G] $在$ B\in \mathscr{S} $的截影是$ (\pi:E\to B,\phi:E\to X) $,或简记为$ E $,其中$ \pi:E\to B $是$ B $上的$ G $-主丛,$ G $自然的作用在$ E $上,$ \phi:E\to X $是$ G $-不变的态射;$ (\pi':E'\to B,\phi':E'\to X) $到$ (\pi:E\to B,\phi:E\to X) $的态射是$ (g:E'\to E,f:B'\to B) $,满足图表交换 
$$ \xymatrix{
	E' \ar[d]_-{\pi'} \ar[r]^-{g} & E \ar[d]^-{\pi} \\
	B' \ar[r]^-{f} & B
} $$
且$ \phi'=\phi\circ g $。$ p:[X/G]\to \mathscr{S} $是遗忘函子,$ p((\pi:E\to B,\phi:E\to X) ) =B$。对于$ \mathscr{S} $中的$ f:B'\to B $,显然可以通过拉回得到提升$ E'=E\times_B B' $。当$ G $的作用自由,并且$ X $是某个概型$ Y $的$ G $-主丛时,$ Y=X/G $可以表示$ [X/G] $,这时候每个$ [X/G](B) $的截影都是$ B\to X/G $的拉回。

但是$ [X/G] $通常不具有概型结构,在这里我们需要的是叠结构,这在$ G $光滑时成立:
\begin{theorem}
	当$ G/S $是光滑仿射概型时,$ [X/G] $是一个叠。
\end{theorem}
定理的证明需要用到许多相关的下降理论(descent theory),可以参考\cite{NotesModuliSpaceofCurves}中Proposition.2.1。当$ X $是$ S $上诺特有限型概型,$ G $是$ S $上光滑仿射有限型群概型时,对任意概型$ B/S $,给出$ \underline{B}\to [X/G] $等价于给出$ (1_B:B\to B) $在$ [X/G](B) $中的像,即一个$ B $上的$ G $-主丛$ E\to B $和一个$ G $-不变态射$ E\to X $。对$ (f:B'\to B)\in \underline{B}(B') $,有
$$ \xymatrix{
	f^*E \ar[d] \ar[r]& E \ar[d]\ar[r]&X \\
	B' \ar[r]^-{f} & B&
} $$
显然可以取平凡从$ G\times_X X $和群作用$ G\times_SX\to X $定义叠的1-态射$ \underline{X}\to [X/G] $。考虑纤维积$ \underline{B}\times_{[X/G]} \underline{X} $,对$ B' $,纤维积的截影为$ (f:B' \to B,B'\to X,\psi) $,其中$ \psi $是$ [X/G](B') $中同构,即$ G\times_X B'\cong f^*E $。通过
$$ B'\cong S\times_SB'\xrightarrow{e_G\times 1_{B'}} G\times B'\cong f^*E\to E  $$
得到$ B'\to E $,反之可以从$ B'\to E $得到一个截影,因此$ \underline{B}\times_{[X/G]} \underline{X}\cong \underline{E} $,是可表的叠,所以$ \underline{X}\to [X/G] $是可表的态射,并且是光滑的。另一方面,如果$ G $在$ X $上的作用使得$ X $的几何点的稳定子群约化且有限,可以得到$ [X/G] $的对角态射非分歧。因此利用DM叠的判定准则(定理\ref{criterionforDM}),可以得到
\begin{theorem}
	当$ X $是$ S $上诺特有限型概型,$ G $是$ S $上光滑仿射有限型群概型,且$ G $在$ X $上的作用,使得$ X $的几何点的稳定子群约化且有限,则$ [X/G] $是DM叠,且$ [X/G] $是分离的当且仅当$ G $的作用是恰当的。
\end{theorem}


\subsection{构造$ \mathscr{M}_g $}
现在说明函子$ \overline{M_g} $定义的群胚$ \overline{\mathscr{M}_g} $是一个DM叠。

在GIT的构造中假定了所有概型是代数闭域$ k $上的,有$ \mathcal{K}\subset \mathcal{H}_{d,g,r} $参数化全部亏格$ g\geqslant2 $的DM稳定曲线,记其中光滑曲线对应的开集为$ \mathcal{K}^\circ $。以$ \mathcal{K} $的情况为例,$ \mathbb{P}^r $的自同构群$G= \mathrm{PGL}(r+1)\cong \mathrm{PGL}(r+1) $作用在$ \mathcal{K} $上。下面取$ S=\mathrm{Spec}\,k $,在$ \mathscr{S}=\mathfrak{Set}_k $上通过\ref{stack}和\ref{X/G}两节中的结论证明$ \overline{\mathscr{M}_g}\cong [\mathcal{K}/G] $:

每个亏格$ g $的DM稳定曲线族$ (\pi: \mathcal{X}\to B)\in \overline{\mathscr{M}_g}(B) $,在$ B $上有局部自由层$ \pi_*\omega_{\mathcal{X}/B}^{\otimes n} $,因而有射影丛$ \mathbb{P}(\pi_*\omega_{\mathcal{X}/B}^{\otimes n} )\to B $和相伴的$ G $主丛$ E\to B $。把$ \mathcal{X} $拉回到$ \pi':\mathcal{X}\times_BE \to E $,可以得到$ E $上的射影丛$ \mathbb{P}(\pi_*\omega_{\mathcal{X}\times_BE/E}^{\otimes n}) $,这个射影丛平凡,且与$ \mathbb{P}(\pi_*\omega_{\mathcal{X}/B}^{\otimes n })\to B $在$ E $的拉回同构。这样通过$ \mathcal{K} $的泛性质($ \mathcal{H}_{d,g,r} $表示对应Hilbert函子),可以得到$ E\to \mathcal{K} $,并且显然是$ G $-不变的。因此从$ \mathcal{X}\to B $得到了$ G $-主丛$ E\to B $和$ G $-不变态射$ E\to \mathcal{K} $。而对$ {\mathscr{M}_g} $中的态射
$$ \xymatrix{
	\mathcal{X}' \ar[d]_-{\pi'} \ar[r] & \mathcal{X} \ar[d]^-{\pi} \\
	B' \ar[r]^-{f} & B
} $$
有$ f^*\pi_*\omega_{\mathcal{X}/B}\cong \pi_*'\omega_{\mathcal{X}'/B'} $,因此有交换图
$$ \xymatrix{
	E' \ar[d] \ar[r] & E \ar[d] \\
	B' \ar[r]^-{f} & B
} $$
即$ [\mathcal{K}/G] $中的态射。这样就构造了1-态射(范畴间的函子)$ \Phi:\overline{\mathscr{M}_g}\to [\mathcal{K}/G] $,并且可以验证是全忠实的。

反之,每个$ [\mathcal{K}/G] $中对象$ (E\to B,E\to \mathcal{K} ) $,简记为$ E $,把$ \mathcal{K} $的万有族从$ E\to \mathcal{K} $拉回得到亏格$ g $的DM稳定曲线族$ \pi_E:\mathcal{X}_E\to E $,并且有同构$ \mathbb{P}({\pi_E}_*\omega_{\mathcal{X}_E/E}^{\otimes n} ) \cong E\times \mathbb{P}^r$。在交换图表
$$ \xymatrix{
	\mathcal{X}_E \ar[d]^-{\pi_E} &  \\
	E \ar[r] & B
} $$
中,通过$ G $在$ E $和$ E\times \mathbb{P}^r $的作用和下降理论,可以证明此时$ E/G=B $,并且$ \mathcal{X}=\mathcal{X}_E/G $存在且满足$ \mathcal{X}\times_BE\cong \mathcal{X}_E $。这样$ \overline{\mathscr{M}_g}(B) $中的截影$ \pi:\mathcal{X}\to B $,简记为$ \pi $,在$ [\mathcal{K}/G] $中有$ \Phi(\pi) \cong E$,这就证明了两个范畴通过$ \Phi $范畴等价,也就是群胚的同构。

通过这个同构,就说明$ \overline{\mathscr{M}_g} $是一个DM叠,并且可以同样的得到$ \mathscr{M}_g $也是DM叠。这就完成了叠意义下曲线模空间的构造。

在这个构造中是通过“在叠的范畴中作商”来得到最终结果的,是一种全局(global)的描述。但通过形变理论还可以(\'etale)局部地理解这个模空间:对于域$ k $上的一条DM稳定曲线$ C $,有一个$ k $-点$ [C]\in \underline{\mathscr{M}_g} $。对$ C $的每一个形变$ \pi:\mathcal{C}\to B $,都可以通过$ \pi \in\overline{\mathscr{M}_g}(B)  $给出态射$ \underline{B}\to \overline{\mathscr{M}_g} $,这就是$ \overline{\mathscr{M}_g} $在点$ [C] $的一个局部。而$ C $的万有族$ \mathcal{X}\to \mathcal{M} $所给出的就是$ \overline{\mathscr{M}_g} $的图册,从此可以说明$ \overline{\mathscr{M}_g} $确实有维数
$$ \dim \overline{\mathscr{M}_g}=\dim k[[t_1,\ldots,t_N]]=3g-3 $$
如果形变$ \mathcal{C}\to B $的族中有同构的纤维,则还需要商去对应的自同构群$ G $,即$ B/G $(在叠的意义下)。这样$ \overline{\mathscr{M}_g} $在(\'etale)局部上就是这些$ B/G $粘贴得到的。

\section{两种“紧化”}
在这一节中讲讨论两种意义下的模空间的紧化,主要参考是\cite{DM69,GeometryAlgCurvesII,NotesModuliSpaceofCurves}。

对于光滑射影曲线的模问题$ M_g $和等价类的集合$ \mathrm{M}_g $,GIT理论和叠理论分别给出了具有不同结构的模空间$ \mathcal{M}_g $和$ \mathscr{M}_g $,它们都包含了$ \mathrm{M}_g $的信息,即它们的$ k $-点就是集合$ \mathrm{M}_g $。但是为了更好的研究不同曲线的关系,需要模空间有更好的性质,比如说“紧性”,而这两个模空间都不是紧的,需要补充它们的边界。在代数几何中,紧性是通过本征态射来刻画的,为了说明构造的$\overline{ \mathcal{M}_g} $和$ \overline{\mathscr{M}_g }$是紧的,就要说明$\overline{ \mathcal{M}_g}\to \mathrm{Spec}\,k $和$ \overline{\mathscr{M}_g }\to \mathscr{S}=\underline{\mathrm{Spec}\,k} $是本征态射。

$\overline{ \mathcal{M}_g}\to \mathrm{Spec}\,k $是本征态射非常显然:在GIT理论中,$ \mathcal{H}^{ss}/G $是域$ k $上射影概型,而$ \mathcal{K} $是$ \mathcal{H}^{ss} $的$ G $-不变闭子概型,自然得到$ \overline{ \mathcal{M}_g} $是$ k $上射影概型,因此$\overline{ \mathcal{M}_g}\to \mathrm{Spec}\,k $是本征态射。

为了说明$\overline{ \mathscr{M}_g}\to \mathrm{Spec}\,k $是本征态射,只需要用DM叠的赋值判定准则就可以了。首先验证分离性,通过定理\ref{Sepforstack},交换图表
$$ \xymatrix{
	& \overline{ \mathscr{M}_g}\ar[d]\\
	\underline{\mathrm{Spec}\,R} \ar@<1pt>[ur]^-{g_1}\ar@<-1pt>[ur]_-{g_2}\ar[r] & \underline{\mathrm{Spec}\,k} 
} $$
中的$ g_1,g_2 $即指$ \mathrm{Spec}\,R $上的两个稳定曲线族$ \mathcal{C}_1,\mathcal{C}_2 $,而$ g_1,g_2 $在一般点同构,即指两个曲线族的一般纤维同构。由引理\ref{DeformationforCurves}知道这个同构可以延拓在整个曲线族,也就是$ g_1,g_2 $的同构。

再看本征性,在定理\ref{Properforstack}的交换图表中,
$$ \xymatrix{
	\underline{\mathrm{Spec}\,K'}\ar[r]\ar[d]&\underline{\mathrm{Spec}\,K} \ar[d] \ar[r]^-{g} & \overline{ \mathscr{M}_g} \ar[d] \\
	\underline{\mathrm{Spec}\,R'}\ar[r] \ar@{-->}[urr]&\underline{\mathrm{Spec}\,R} \ar[r] & \underline{\mathrm{Spec}\,k} }$$
提升$ \underline{\mathrm{Spec}\,R'}\to \overline{ \mathscr{M}_g} $对应的恰好是DM稳定曲线的约化理论的结论,即引理\ref{reduction}。

到此为止,GIT和叠思路下和意义下的两种模空间的构造和紧化就完成了。
\section{标记曲线}
最后将曲线模问题推广,即标记曲线的模空间,主要参考\cite{GITandModuliofPointedCurves,GITandModuliofstablecurve,GeometryAlgCurvesII}。

无论GIT还是叠的构造,考虑DM稳定曲线作为被分类对象的一个重要原因就是其自同构群有限:在GIT构造中,曲线的Hilbert点稳定需要群作用的稳定子群有限,也就意味着曲线本身的自同构群有限。而在叠理论中,形如$ [X/G] $的群胚是DM叠的判定条件(定理\ref{criterionforDM})同样要求稳定子群有限。在$ g\geqslant2 $时光滑射影曲线的自同构群是有限的,但低亏格时就不成立。回顾DM稳定曲线的等价定义时可以发现,对于DM稳定曲线的有理不可约分支,它的自同构群是被其上标记点(与其他部分的交点,即结点)限制的;反之,可以在曲线上附加标记点的结构,使得低亏格时也可以限制自同构。具体来说,一个\dotuline{$ n $-标记曲线($ n $- pointed curve)}$ (C;P_1,\ldots,P_n) $是指一条结点曲线$ C $,和$ n $个不同的光滑点$ P_1,\ldots,P_n $。又是也记$ C $上的一个除子$ D=\sum_{i=1}^{n}P_i $,这时曲线记为$ (C;D) $。这些标记点的地位和结点曲线$ C $的结点相同,其间态射$ f:(C;D)\to (C':D') $必须满足$ f(P_i)=(P_i') $。用$ \omega_C(D) $代替对偶化层来定义DM稳定性,或者等价的把有理不可约分支与其他部分的交点个数是少为3(2)改为交点和标记点一共至少为3(2),这就得到了\dotuline{DM稳定标记曲线}。在稳定曲线族
$$ \mathcal{C}\to S $$
也附加上额外的截影(section)$ \sigma_i:S\to \mathcal{C} $,使得在每个纤维通过这些截影给出标记点,这就是DM稳定$ n $-标记曲线族。
$ n $-标记光滑射影曲线的模问题记为$ M_{g,n} $,而DM稳定$ n $-标记曲线的模问题记为$ \overline{M_{g,n}} $。在一般的DM稳定曲线的模空间构造中用到的大部分预备构造都有对应的结论:DM稳定$ n $-标记曲线通过$ (\omega_C(D))^{\otimes \nu} $嵌入到$ \mathbb{P}^r $中,其中$ r=(2\nu-1)(g-1)+\nu n-1 $,并且有Hilbert多项式
$$ p(t)=(2\nu t-1)(g-1)+\nu nt $$
可以构造Hilbert概型参数化$ \mathbb{P}^r $中全部有$ n $个标记点且以$ p(t) $为Hilbert多项式的曲线,记作$ \mathcal{H} $;进一步用相应的形变理论可以得到对应的$ \mathcal{K} $来参数化全部被$ (\omega_C(D))^{\otimes \nu} $嵌入的DM稳定$ n $-标记曲线,并且
$$ \dim \mathcal{K}=3g-3+n+(r+1)^2-1 $$

叠意义的模空间构造可以相对轻松地推广,群胚$ [\mathcal{K}/G] $就是$ \overline{\mathscr{M}_{g,n}} $,光滑$ n $-标记曲线的部分同理。而如果用GIT方法构造模空间,需要证明相应版本的潜稳定性定理等,来说明$ n $-标记曲线稳定性与对应Hilbert点的稳定性的关系。但这部分直到2006才被Elizabeth Baldwin补上\upcite{GITandModuliofstablecurve},其证明思路是通过归纳,将亏格$ g $且有$ n $个标记点的曲线的稳定性,与亏格$ g+n $而没有标记点的曲线的稳定性联系起来。后来David Swinarski又推广到加权曲线(weighted pointed curve)上,即标记点可以附加有理数$ 0\leqslant a_i \leqslant 1 $的权重\upcite{GITandModuliofPointedCurves}。

