\chapter{结论}
\section{曲线模空间}
在代数几何中,几何对象的同构类的分类问题通常以模问题出现,即满足一定条件的反变函子$ M\in [\mathfrak{Sch}^\circ,\mathfrak{Set}] $。在曲线的情形,首先考虑的就是光滑射影曲线的分类问题,也就是函子$ M_g $。从函子可以直接得到群胚$ \underline{M_g}=\mathscr{M}_g $,但此时它和函子本身没有区别,只是一个抽象的概念而无法返回去帮助理解曲线和曲线族。Deligne和Mumford则用叠和DM叠的概念解释群胚,使得$ \mathscr{M}_g $与概型发生了更密切的联系,更像是一个“空间”,并且可以移植许多概型的概念,例如分离、诺特、正规等。其中一条性质是本征态射,它刻画了在DM叠的意义下的紧空间。做$ \mathscr{M}_g $在这个意义下的紧化,将添加更多曲线,这就是DM稳定曲线。同时也说明,与其他奇异曲线相比,这类曲线是更接近光滑曲线、更有研究价值的。

当然想要从DM叠$ \mathscr{M}_g $和$ \overline{\mathscr{M}_g} $获取曲线和曲线族的性质,就需要对这两个DM叠的性质有更多了解。但是这个是一个新的、不容易处理的概念,它的性质“翻译”到作为概型的曲线也需要额外的过程。而GIT的构造思路就是得到与$ \mathscr{M}_g $和$ \overline{\mathscr{M}_g} $关系密切的概型,即粗模空间$ \mathcal{M}_g $和$ \overline{\mathcal{M}_g} $(虽然粗模空间是通过函子$ M_g $定义的,但是注意此时的叠和函子的信息是完全一样的),这个概型的性质过渡到叠的性质,进而过渡到曲线和曲线族的性质。

构造曲线模空间之后,就应该考虑它的一些基本性质,比如
\begin{enumerate}
	\item 模空间是否是紧的?能否紧化?是否是射影的?
	\item 模空间的维数?模空间是否连通、既约、可约?
	\item 模空间是否有奇异点?
	\item 模空间能否定义、计算上同调?是否有Chow环?
\end{enumerate}
等。这些问题使我们通过模空间来理解所分类的对象,例如模空间奇异点某种意义下描述曲线的形变在此处出现“跳跃”。除此之外,还可以把曲线模空间作为工具来研究其他问题,例如以曲线作为一般纤维的纤维化问题等。

\section{模问题与模空间}
给定一个模问题的反变函子$ M $,我们希望找到一个模空间来描述模问题所分类的对象。最好的模空间是细模空间,即$ M $是可表函子,被概型$ \mathrm{M} $表示。因为$ \mathrm{M} $是一个概型,因此可以很好的描述和研究它的性质,进而描述和研究所分类的对象。最“差”的模空间是群胚$ \underline{M} $,它和函子本身毫无区别,也难以描述几何结构,相当于没有回答问题。但是当这个群胚是叠或者DM叠时,它就接近一个“空间”,因此可以考虑它的性质了。除了这两种模空间外,还有其他模空间,比如已经见过的粗模空间,这可以定义为概型,也可以定义为代数空间(algebraic space);当群胚$ \underline{M} $是Artin叠时可以定义良好模空间(good moduli space)\upcite{GoodModuliSpace}。这些不同的模空间也有不同的构造技术,比如GIT就是常用的方法。

模空间的不同定义和性质,以及构造模空间的技术手段,又反过来让我们重新考虑被分类的对象。例如已经提到过的,曲线模空间(粗模空间和DM叠)的紧性诱导我们将具有有限自同构群的结点曲线也作为被分类的对象,也就是DM稳定曲线。如果模空间是DM叠(此时记为$ \mathscr{M} $),可以证明对角态射$ \Delta:\mathscr{M}\to \mathscr{M}\times \mathscr{M} $是非分歧的。而对角态射又与层$  Isom_B(X,Y) $密切相关,因此如果要求模空间$ \underline{M} $是DM叠,就会反映在分类对象的自同构群上。GIT构造技术出现的“稳定性”可以同样的定义在其他模问题上,比如在利用Chow簇分类高维概型时,GIT稳定性就与Chow稳定性和线性稳定性有关\upcite{StabilityofProjectiveVarieties};如果分类标记曲线的态射,GIT稳定性就与稳定态射(stable map)有关\upcite{GITModuliStableCurveandMaps};如果分类代数曲线上的向量丛时,就会出现对应的稳定向量丛的概念\upcite{stablebundle}。这些是相对较好的稳定性的概念,但也会有出现困难的情况。例如在曲面的情形,Steiner曲面是光滑曲面,但是它的Hilbert点则不稳定\upcite{ModuliofCurves},这意味着用GIT构造曲线模空间的思路,用在曲面情形时无法处理全部光滑曲面,这显然是不希望的结果,说明此时这种GIT的技巧失效了,需要其他构造方法或其他对象,例如对一般型的曲面\upcite{GlobalModuliofSurface}。

在模问题中,首先要做到的是提出一个恰当的问题,即模问题所分类的对象确实是值得研究的、性质有趣的;在定义模空间时,既要尽量完整地保持模问题的函子,又要使得空间的性质能良好的反应分类对象的性质;在构造模空间时,各种技术手段又对最初的分类对象做出限制,即是“妥协”又是指导。有时为了构造出性质良好的模空间,会对分类对象做出较多限制,但这样被分类的对象太少而没有价值;想要讨论尽量全面的问题,最后的模空间又可能难以分析,使得我们无法从中获得这些对象的信息。总之模问题是一个内容丰富且深刻的领域,值得我们继续研究。